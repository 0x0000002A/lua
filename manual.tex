% $Id: manual.tex,v 1.23 1996/11/12 16:00:16 roberto Exp $

\documentstyle[fullpage,11pt,bnf]{article}

\newcommand{\rw}[1]{{\bf #1}}
\newcommand{\see}[1]{see Section~\ref{#1}}
\newcommand{\nil}{{\bf nil}}
\newcommand{\Line}{\rule{\linewidth}{.5mm}}
\def\tecgraf{{\sf TeC\kern-.21em\lower.7ex\hbox{Graf}}}

\newcommand{\Index}[1]{#1\index{#1}}
\newcommand{\IndexVerb}[1]{{\tt #1}\index{#1}}
\newcommand{\Def}[1]{{\em #1}\index{#1}}
\newcommand{\Deffunc}[1]{\index{#1}}

\newcommand{\ff}{$\bullet$\ }

\newcommand{\Version}{2.5}

\makeindex

\begin{document}

\title{Reference Manual of the Programming Language Lua \Version}

\author{%
Roberto Ierusalimschy\quad
Luiz Henrique de Figueiredo\quad
Waldemar Celes
\vspace{1.0ex}\\
\smallskip
\small\tt lua@icad.puc-rio.br
\vspace{2.0ex}\\
%MCC 08/95 ---
\tecgraf\ --- Departamento de Inform\'atica --- PUC-Rio
}

\date{\small \verb$Date: 1996/11/12 16:00:16 $}

\maketitle

\begin{abstract}
\noindent
Lua is an extension programming language designed to be used
as a configuration language for any program that needs one.
This document describes version \Version\ of the Lua programming language and
the API that allows interaction between Lua programs and their host C programs.
The document also presents some examples of using the main
features of the system.
\end{abstract}

\vspace{4ex}
\begin{quotation}
\small
\begin{center}{\bf Sum\'ario}\end{center}
\vspace{1ex}
\noindent
Lua \'e uma linguagem de extens\~ao projetada para ser usada como
linguagem de configura\c{c}\~ao em qualquer programa que precise de
uma.
Este documento descreve a vers\~ao \Version\ da linguagem de
programa\c{c}\~ao Lua e a Interface de Programa\c{c}\~ao (API) que permite
a intera\c{c}\~ao entre programas Lua e programas C hospedeiros.
O documento tamb\'em apresenta alguns exemplos de uso das principais
ca\-racte\-r\'{\i}sticas do sistema.
\end{quotation}


\section{Introduction}

Lua is an extension programming language designed to support
general procedural programming features with data description
facilities.
It is intended to be used as a configuration language for any
program that needs one.
Lua has been designed and implemented by
W.~Celes, L.~H.~de Figueiredo and R.~Ierusalimschy.

Lua is implemented as a library, written in C.
Being an extension language, Lua has no notion of a ``main'' program:
it only works {\em embedded} in a host client,
called the {\em embedding} program.
This host program can invoke functions to execute a piece of
code in Lua, can write and read Lua variables,
and can register C functions to be called by Lua code.
Through the use of C functions, Lua can be augmented to cope with
rather different domains,
thus creating customized programming languages sharing a syntactical framework.

Lua is free-distribution software,
and provided as usual with no guarantees.
The implementation described in this manual is available
at the following URL's:
\begin{verbatim}
   http://www.inf.puc-rio.br/~roberto/lua.html
   ftp://ftp.icad.puc-rio.br/pub/lua/lua.tar.gz
\end{verbatim}


\section{Environment and Chunks}

All statements in Lua are executed in a \Def{global environment}.
This environment, which keeps all global variables and functions,
is initialized at the beginning of the embedding program and
persists until its end.

The global environment can be manipulated by Lua code or
by the embedding program,
which can read and write global variables
using functions in the library that implements Lua.

\Index{Global variables} do not need declaration.
Any variable is assumed to be global unless explicitly declared local
(\see{localvar}).
Before the first assignment, the value of a global variable is \nil.

The unit of execution of Lua is called a \Def{chunk}.
The syntax%
\footnote{As usual, \rep{{\em a}} means 0 or more {\em a\/}'s,
\opt{{\em a}} means an optional {\em a} and \oneormore{{\em a}} means
one or more {\em a\/}'s.}
for chunks is:
\begin{Produc}
\produc{chunk}{\rep{statement \Or function} \opt{ret}}
\end{Produc}%
A chunk may contain statements and function definitions,
and may be in a file or in a string inside the host program.
A chunk may optionally ends with a return statement (\see{return}).
When a chunk is executed, first all its functions and statements are compiled,
then the statements are executed in sequential order.
All modifications a chunk effects on the global environment persist
after its end.
Those include modifications to global variables and definitions
of new functions%
\footnote{Actually, a function definition is an
assignment to a global variable; \see{TypesSec}.}.

Chunks may be pre-compiled; see program \IndexVerb{luac} for details.
Text files with chunks and their binary pre-compiled forms
are interchangeable.
Lua automatically detects the file type and acts accordingly.
\index{pre-compilation}

\section{\Index{Types}} \label{TypesSec}

Lua is a dynamically typed language.
Variables do not have types; only values do.
All values carry their own type.
Therefore, there are no type definitions in the language.

There are seven \Index{basic types} in Lua: \Def{nil}, \Def{number},
\Def{string}, \Def{function}, \Def{CFunction}, \Def{userdata},
and \Def{table}.
{\em Nil} is the type of the value \nil,
whose main property is to be different from any other value.
{\em Number} represents real (floating point) numbers,
while {\em string} has the usual meaning.

Functions are considered first-class values in Lua.
This means that functions can be stored in variables,
passed as arguments to other functions and returned as results.
When a function is defined in Lua, its body is compiled and stored
in a given variable.
Lua can call (and manipulate) functions written in Lua and
functions written in C; the latter have type {\em CFunction\/}.

The type {\em userdata} is provided to allow
arbitrary \Index{C pointers} to be stored in Lua variables.
It corresponds to \verb'void*' and has no pre-defined operations in Lua,
besides assignment and equality test.
However, by using fallbacks, the programmer may define operations
for {\em userdata} values; \see{fallback}.

The type {\em table} implements \Index{associative arrays},
that is, \Index{arrays} that can be indexed not only with numbers,
but with any value (except \nil).
Therefore, this type may be used not only to represent ordinary arrays,
but also symbol tables, sets, records, etc.
To represent \Index{records}, Lua uses the field name as an index.
The language supports this representation by
providing \verb'a.name' as syntactic sugar for \verb'a["name"]'.
Tables may also carry methods.
Because functions are first class values,
table fields may contain functions.
The form \verb't:f(x)' is syntactic sugar for \verb't.f(t,x)',
which calls the method \verb'f' from the table \verb't' passing
itself as the first parameter.

It is important to notice that tables are objects, and not values.
Variables cannot contain tables, only references to them.
Assignment, parameter passing and returns always manipulate references
to tables, and do not imply any kind of copy.
Moreover, tables must be explicitly created before used
(\see{tableconstructor}).



\section{The Language}

This section describes the lexis, the syntax and the semantics of Lua.


\subsection{Lexical Conventions} \label{lexical}

Lua is a case sensitive language.
\Index{Identifiers} can be any string of letters, digits, and underscores,
not beginning with a digit.
The following words are reserved, and cannot be used as identifiers:
\index{reserved words}
\begin{verbatim}
      and       do        else      elseif
      end       function  if        local
      nil       not       or        repeat
      return    then      until     while
\end{verbatim}

The following strings denote other \Index{tokens}:
\begin{verbatim}
         ~=  <=  >=  <   >   ==  =   ..  +   -   *   /
         %   (   )   {   }   [   ]   ;   ,   .
\end{verbatim}

\Index{Literal strings} can be delimited by matching single or double quotes,
and can contain the C-like escape sequences
\verb-'\n'-, \verb-'\t'- and \verb-'\r'-.
Literal strings can also be delimited by matching \verb'[[ ... ]]'.
Literals in this bracketed form may run for several lines,
may contain nested \verb'[[ ... ]]' pairs,
and do not interpret escape sequences.

\Index{Comments} start anywhere outside a string with a
double hyphen (\verb'--') and run until the end of the line.
Moreover, if the first line of a chunk file starts with \verb'#',
this line is skipped%
\footnote{This facility allows the use of Lua as a script interpreter
in Unix systems.}.

\Index{Numerical constants} may be written with an optional decimal part,
and an optional decimal exponent.
Examples of valid numerical constants are:
\begin{verbatim}
       4     4.0     0.4     4.57e-3     0.3e12
\end{verbatim}


\subsection{\Index{Coercion}} \label{coercion}

Lua provides some automatic conversions.
Any arithmetic operation applied to a string tries to convert
that string to a number, following the usual rules.
Conversely, whenever a number is used when a string is expected,
that number is converted to a string, according to the following rule:
if the number is an integer, it is written without exponent or decimal point;
otherwise, it is formatted following the \verb'%g'
conversion specification of the \verb'printf' function in the
standard C library.



\subsection{\Index{Adjustment}} \label{adjust}

Functions in Lua can return many values.
Because there are no type declarations,
the system does not know how many values a function will return,
or how many parameters it needs.
Therefore, sometimes, a list of values must be {\em adjusted\/}, at run time,
to a given length.
If there are more values than are needed, then the last values are thrown away.
If there are more needs than values, then the list is extended with as
many  \nil's as needed.
Adjustment occurs in multiple assignment and function calls.


\subsection{Statements}

Lua supports an almost conventional set of \Index{statements}.
The conventional commands include
assignment, control structures and procedure calls.
Non-conventional commands include table constructors
(Section~\ref{tableconstructor}),
and local variable declarations (Section~\ref{localvar}).

\subsubsection{Blocks}
A \Index{block} is a list of statements, which is executed sequentially.
Any statement can be optionally followed by a semicolon:
\begin{Produc}
\produc{block}{\rep{stat sc} \opt{ret}}
\produc{sc}{\opt{\ter{;}}}
\end{Produc}%
For syntactic reasons, a \IndexVerb{return} statement can only be written
as the last statement of a block.
This restriction also avoids some ``statement not reached'' errors.

\subsubsection{\Index{Assignment}} \label{assignment}
The language allows \Index{multiple assignment}.
Therefore, the syntax defines a list of variables on the left side,
and a list of expressions on the right side.
Both lists have their elements separated by commas:
\begin{Produc}
\produc{stat}{varlist1 \ter{=} explist1}
\produc{varlist1}{var \rep{\ter{,} var}}
\end{Produc}%
This statement first evaluates all values on the right side
and eventual indices on the left side,
and then makes the assignments.
Therefore, it can be used to exchange two values, as in
\begin{verbatim}
   x, y = y, x
\end{verbatim}
Before the assignment, the list of values is {\em adjusted} to
the length of the list of variables (\see{adjust}).

A single name can denote a global or a local variable,
or a formal parameter:
\begin{Produc}
\produc{var}{name}
\end{Produc}%
Square brackets are used to index a table:
\begin{Produc}
\produc{var}{var \ter{[} exp1 \ter{]}}
\end{Produc}%
If \verb'var' results in a table value,
the field indexed by the expression value gets the assigned value.
Otherwise, the fallback {\em settable} is called,
with three parameters: the value of \verb'var',
the value of expression, and the value being assigned to it;
\see{fallback}.

The syntax \verb'var.NAME' is just syntactic sugar for
\verb'var["NAME"]'.
\begin{Produc}
\produc{var}{var \ter{.} name}
\end{Produc}%

\subsubsection{Control Structures}
The \Index{condition expression} of a control structure can return any value.
All values different from \nil\ are considered true;
\nil\ is considered false.
{\tt if}'s, {\tt while}'s and {\tt repeat}'s have the usual meaning.

\index{while-do}\index{repeat-until}\index{if-then-else}
\begin{Produc}
\produc{stat}{\rwd{while} exp1 \rwd{do} block \rwd{end} \OrNL
\rwd{repeat} block \rwd{until} exp1 \OrNL
\rwd{if} exp1 \rwd{then} block \rep{elseif}
   \opt{\rwd{else} block} \rwd{end}}
\produc{elseif}{\rwd{elseif} exp1 \rwd{then} block}
\end{Produc}

A {\tt return} is used to return values from a function or a chunk.
\label{return}
Because they may return more than one value,
the syntax for a \Index{return statement} is:
\begin{Produc}
\produc{ret}{\rwd{return} explist \opt{sc}}
\end{Produc}

\subsubsection{Function Calls as Statements} \label{funcstat}
Because of possible side-effects,
function calls can be executed as statements:
\begin{Produc}
\produc{stat}{functioncall}
\end{Produc}%
Eventual returned values are thrown away.
Function calls are explained in Section \ref{functioncall}.

\subsubsection{Local Declarations} \label{localvar}
\Index{Local variables} can be declared anywhere inside a block.
Their scope begins after the declaration and lasts until the
end of the block.
The declaration may include an initial assignment:
\begin{Produc}
\produc{stat}{\rwd{local} declist \opt{init}}
\produc{declist}{name \rep{\ter{,} name}}
\produc{init}{\ter{=} explist1}
\end{Produc}%
If present, an initial assignment has the same semantics
of a multiple assignment.
Otherwise, all variables are initialized with \nil.


\subsection{\Index{Expressions}}

\subsubsection{\Index{Simple Expressions}}
Simple expressions are:
\begin{Produc}
\produc{exp}{\ter{(} exp \ter{)}}
\produc{exp}{\rwd{nil}}
\produc{exp}{\ter{number}}
\produc{exp}{\ter{literal}}
\produc{exp}{var}
\end{Produc}%
Numbers (numerical constants) and
string literals are explained in Section~\ref{lexical}.
Variables are explained in Section~\ref{assignment}.

\subsubsection{Arithmetic Operators}
Lua supports the usual \Index{arithmetic operators}.
These operators are the binary
\verb'+', \verb'-', \verb'*', \verb'/' and \verb'^' (exponentiation),
and the unary \verb'-'.
If the operands are numbers, or strings that can be converted to
numbers, according to the rules given in Section \ref{coercion},
then all operations but exponentiation have the usual meaning.
Otherwise, the fallback ``arith'' is called (\see{fallback}).
An exponentiation always calls this fallback.
The standard mathematical library redefines this fallback,
giving the expected meaning to \Index{exponentiation}
(\see{mathlib}).

\subsubsection{Relational Operators}
Lua provides the following \Index{relational operators}:
\begin{verbatim}
       <   >   <=  >=  ~=  ==
\end{verbatim}
All these return \nil\ as false and a value different from \nil\
(actually the number 1) as true.

Equality first compares the types of its operands.
If they are different, then the result is \nil.
Otherwise, their values are compared.
Numbers and strings are compared in the usual way.
Tables, CFunctions, and functions are compared by reference,
that is, two tables are considered equal only if they are the same table.
The operator \verb'~=' is exactly the negation of equality (\verb'==').

The other operators work as follows.
If both arguments are numbers, then they are compared as such.
Otherwise, if both arguments can be converted to strings,
their values are compared using lexicographical order.
Otherwise, the ``order'' fallback is called (\see{fallback}).

\subsubsection{Logical Operators}
Like control structures, all logical operators
consider \nil\ as false and anything else as true.
The \Index{logical operators} are:
\index{and}\index{or}\index{not}
\begin{verbatim}
             and   or   not
\end{verbatim}
The operator \verb'and' returns \nil\ if its first argument is \nil;
otherwise it returns its second argument.
The operator \verb'or' returns its first argument
if it is different from \nil;
otherwise it returns its second argument.
Both \verb'and' and \verb'or' use \Index{short-cut evaluation},
that is,
the second operand is evaluated only if necessary.

\subsubsection{Concatenation}
Lua offers a string \Index{concatenation} operator,
denoted by ``\IndexVerb{..}''.
If operands are strings or numbers, then they are converted to
strings according to the rules in Section \ref{coercion}.
Otherwise, the fallback ``concat'' is called (\see{fallback}).

\subsubsection{Precedence}
\Index{Operator precedence} follows the table below,
from the lower to the higher priority:
\begin{verbatim}
             and   or
             <   >   <=  >=  ~=  ==
             ..
             +   -
             *   /
             not  - (unary)
             ^
\end{verbatim}
All binary operators are left associative,
except for \verb'^' (exponentiation),
which is right associative.

\subsubsection{Table Constructors} \label{tableconstructor}
Table \Index{constructors} are expressions that create tables;
every time a constructor is evaluated, a new table is created.
Constructors can be used to create empty tables,
or to create a table and initialize some fields.

The general syntax for constructors is:
\begin{Produc}
\produc{tableconstructor}{\ter{\{} fieldlist \ter{\}}}
\produc{fieldlist}{lfieldlist \Or ffieldlist \Or lfieldlist \ter{;} ffieldlist}
\produc{lfieldlist}{\opt{lfieldlist1}}
\produc{ffieldlist}{\opt{ffieldlist1}}
\end{Produc}

The form {\em lfieldlist1} is used to initialize lists.
\begin{Produc}
\produc{lfieldlist1}{exp \rep{\ter{,} exp} \opt{\ter{,}}}
\end{Produc}%
The expressions in the list are assigned to consecutive numerical indexes,
starting with 1.
For example:
\begin{verbatim}
   a = {"v1", "v2", 34}
\end{verbatim}
is roughly equivalent to:
\begin{verbatim}
   temp = {}
   temp[1] = "v1"
   temp[2] = "v2"
   temp[3] = 34
   a = temp
\end{verbatim}

The next form initializes named fields in a table:
\begin{Produc}
\produc{ffieldlist1}{ffield \rep{\ter{,} ffield} \opt{\ter{,}}}
\produc{ffield}{name \ter{=} exp}
\end{Produc}%
For example:
\begin{verbatim}
   a = {x = 1, y = 3}
\end{verbatim}
is roughly equivalent to:
\begin{verbatim}
   temp = {}
   temp.x = 1    -- or temp["x"] = 1
   temp.y = 3    -- or temp["y"] = 3
   a = temp
\end{verbatim}


\subsubsection{Function Calls}  \label{functioncall}
A \Index{function call} has the following syntax:
\begin{Produc}
\produc{functioncall}{var realParams}
\end{Produc}%
Here, \verb'var' can be any variable (global, local, indexed, etc).
If its value has type {\em function\/} or {\em CFunction\/},
then this function is called.
Otherwise, the ``function'' fallback is called,
having as first parameter the value of \verb'var',
and then the original call parameters.

The form:
\begin{Produc}
\produc{functioncall}{var \ter{:} name realParams}
\end{Produc}%
can be used to call ``methods''.
A call \verb'var:name(...)'
is syntactic sugar for
\begin{verbatim}
  var.name(var, ...)
\end{verbatim}
except that \verb'var' is evaluated only once.

\begin{Produc}
\produc{realParams}{\ter{(} \opt{explist1} \ter{)}}
\produc{realParams}{tableconstructor}
\produc{explist1}{exp1 \rep{\ter{,} exp1}}
\end{Produc}%
All argument expressions are evaluated before the call;
then the list of \Index{arguments} is adjusted to
the length of the list of parameters (\see{adjust});
finally, this list is assigned to the formal parameters.
A call of the form \verb'f{...}' is syntactic sugar for
\verb'f({...})', that is,
the parameter list is a single new table.

Because a function can return any number of results
(\see{return}),
the number of results must be adjusted before used.
If the function is called as a statement (\see{funcstat}),
its return list is adjusted to 0.
If the function is called in a place that needs a single value
(syntactically denoted by the non-terminal \verb'exp1'),
then its return list is adjusted to 1.
If the function is called in a place that can hold many values
(syntactically denoted by the non-terminal \verb'exp'),
then no adjustment is made.


\subsection{\Index{Function Definitions}}

Functions in Lua can be defined anywhere in the global level of a chunk.
The syntax for function definition is:
\begin{Produc}
\produc{function}{\rwd{function} var \ter{(} \opt{parlist1} \ter{)}
  block \rwd{end}}
\end{Produc}

When Lua pre-compiles a chunk,
all its function bodies are pre-compiled, too.
Then, when Lua ``executes'' the function definition,
its body is stored, with type {\em function},
into the variable \verb'var'.

Parameters act as local variables,
initialized with the argument values.
\begin{Produc}
\produc{parlist1}{name \rep{\ter{,} name}}
\end{Produc}

Results are returned using the \verb'return' statement (\see{return}).
If control reaches the end of a function without a return instruction,
then the function returns with no results.

There is a special syntax for defining \Index{methods},
that is, functions that have an extra parameter \Def{self}.
\begin{Produc}
\produc{function}{\rwd{function} var \ter{:} name \ter{(} \opt{parlist1}
  \ter{)} block \rwd{end}}
\end{Produc}%
Thus, a declaration like
\begin{verbatim}
function v:f (...)
  ...
end
\end{verbatim}
is equivalent to
\begin{verbatim}
function v.f (self, ...)
  ...
end
\end{verbatim}
that is, the function gets an extra formal parameter called \verb'self'.
Notice that
the variable \verb'v' must have been previously initialized with a table value.


\subsection{Fallbacks} \label{fallback}

Lua provides a powerful mechanism to extend its semantics,
called \Def{fallbacks}.
A fallback is a programmer defined function
that is called whenever Lua does not know how to proceed.

Lua supports the following fallbacks,
identified by the given strings:
\begin{description}
\item[``arith'':]\index{arithmetic fallback}
called when an arithmetic operation is applied to non numerical operands,
or when the binary \verb'^' operation is called.
It receives three arguments:
the two operands (the second one is nil when the operation is unary minus)
and one of the following strings describing the offended operator:
\begin{verbatim}
  add  sub  mul  div  pow  unm
\end{verbatim}
Its return value is the final result of the arithmetic operation.
The default handler issues an error.
\item[``order'':]\index{order fallback}
called when an order comparison is applied to non numerical or
non string operands.
It receives three arguments:
the two operands and
one of the following strings describing the offended operator:
\begin{verbatim}
  lt gt le ge
\end{verbatim}
Its return value is the final result of the comparison operation.
The default handler issues an error.
\item[``concat'':]\index{concatenation fallback}
called when a concatenation is applied to non string operands.
It receives the two operands as arguments.
Its return value is the final result of the concatenation operation.
The default handler issues an error.
\item[``index'':]\index{index fallback}
called when Lua tries to retrieve the value of an index
not present in a table.
It receives as arguments the table and the index.
Its return value is the final result of the indexing operation.
The default handler returns nil.
\item[``getglobal'':]\index{index getglobal}
called when Lua tries to retrieve the value of a global variable
which has a nil value (or which has not been initialized).
It receives as argument the name of the variable.
Its return value is the final result of the expression.
The default handler returns nil.
\item[``gettable'':]\index{gettable fallback}
called when Lua tries to index a non table value.
It receives as arguments the non table value and the index.
Its return value is the final result of the indexing operation.
The default handler issues an error.
\item[``settable'':]\index{settable fallback}
called when Lua tries to assign indexed a non table value.
It receives as arguments the non table value,
the index, and the assigned value.
The default handler issues an error.
\item[``function'':]\index{function fallback}
called when Lua tries to call a non function value.
It receives as arguments the non function value and the
arguments given in the original call.
Its return values are the final results of the call operation.
The default handler issues an error.
\item[``gc'':]
called during garbage collection.
It receives as argument the table being collected.
After each run of the collector this function is called with argument nil.
Because this function operates during garbage collection,
it must be used with great care,
and programmers should avoid the creation of new objects
(tables or strings) in this function.
The default handler does nothing.
\item[``error'':]\index{error fallback}
called when an error occurs.
It receives as argument a string describing the error.
The default handler prints the message on the standard error output.
\end{description}

The function \IndexVerb{setfallback} is used to change a fallback handler.
Its first argument is the name of a fallback condition,
and the second argument is the new function to be called.
It returns the old handler function for the given fallback.

Section \ref{exfallback} shows an example of the use of fallbacks.


\subsection{Error Handling} \label{error}

Because Lua is an extension language,
all Lua actions start from C code calling a function from the Lua library.
Whenever an error occurs during Lua compilation or execution,
an ``error'' fallback function is called,
and then the corresponding function from the library
(\verb'lua_dofile', \verb'lua_dostring',
\verb'lua_call', or \verb'lua_callfunction')
is terminated returning an error condition.

The only argument to the ``error'' fallback function is a string
describing the error.
The standard I/O library redefines this fallback,
using the debug facilities (\see{debugI}),
in order to print some extra information,
like the call stack.
For more information about an error,
the Lua program can include the compilation pragma \verb'$debug'.
\index{debug pragma}\label{pragma}
This pragma must be written in a line by itself.
When an error occurs in a program compiled with this option,
the error routine is able to print also the lines where the calls
(and the error) were made.
If needed, it is possible to change the ``error'' fallback handler
(\see{fallback}).

Lua code can explicitly generate an error by calling the built-in
function \verb'error' (\see{pdf-error}).


\section{The Application Program Interface}

This section describes the API for Lua, that is,
the set of C functions available to the host program to communicate
with the library.
The API functions can be classified in the following categories:
\begin{enumerate}
\item executing Lua code;
\item converting values between C and Lua;
\item manipulating (reading and writing) Lua objects;
\item calling Lua functions;
\item C functions to be called by Lua;
\item references to Lua Objects.
\end{enumerate}
All API functions are declared in the header file \verb'lua.h'.

\subsection{Executing Lua Code}
A host program can execute Lua chunks written in a file or in a string,
using the following functions:
\Deffunc{lua_dofile}\Deffunc{lua_dostring}
\begin{verbatim}
int lua_dofile   (char *filename);
int lua_dostring (char *string);
\end{verbatim}
Both functions return an error code:
0, in case of success; non zero, in case of errors.
More specifically, \verb'lua_dofile' returns 2 if for any reason
it could not open the file.
The function \verb'lua_dofile', if called with argument \verb'NULL' (0),
executes the {\tt stdin} stream.
Function \verb'lua_dofile' is also able to execute pre-compiled chunks.
It automatically detects whether the file is text or binary,
and loads it accordingly (see program \IndexVerb{luac}).

\subsection{Converting Values between C and Lua} \label{valuesCLua}
Because Lua has no static type system,
all values passed between Lua and C have type
\verb'lua_Object'\Deffunc{lua_Object},
which works like an abstract type in C that can hold any Lua value.
Values of type \verb'lua_Object' have no meaning outside Lua;
for instance,
the comparisson of two \verb"lua_Object's" is of no significance.

Because Lua has automatic memory management and garbage collection,
a \verb'lua_Object' has a limited scope,
and is only valid inside the {\em block\/} where it was created.
A C function called from Lua is a block,
and its parameters are valid only until its end.
A good programming practice is to convert Lua objects to C values
as soon as they are available,
and never to store \verb'lua_Object's in C global variables.

When C code calls Lua repeatedly, as in a loop,
objects returned by these calls accumulate,
and may create a memory problem.
To avoid this,
nested blocks can be defined with the functions:
\begin{verbatim}
void           lua_beginblock           (void);
void           lua_endblock             (void);
\end{verbatim}
After the end of the block,
all \verb'lua_Object''s created inside it are released.

To check the type of a \verb'lua_Object',
the following function is available:
\Deffunc{lua_type}
\begin{verbatim}
int            lua_type                 (lua_Object object);
\end{verbatim}
plus the following macros and functions:
\Deffunc{lua_isnil}\Deffunc{lua_isnumber}\Deffunc{lua_isstring}
\Deffunc{lua_istable}\Deffunc{lua_iscfunction}\Deffunc{lua_isuserdata}
\Deffunc{lua_isfunction}
\begin{verbatim}
int            lua_isnil                (lua_Object object);
int            lua_isnumber             (lua_Object object);
int            lua_isstring             (lua_Object object);
int            lua_istable              (lua_Object object);
int            lua_isfunction           (lua_Object object);
int            lua_iscfunction          (lua_Object object);
int            lua_isuserdata           (lua_Object object);
\end{verbatim}
All macros return 1 if the object is compatible with the given type,
and 0 otherwise.
The function \verb'lua_isnumber' accepts numbers and numerical strings,
\verb'lua_isstring' accepts strings and numbers (\see{coercion}),
and \verb'lua_isfunction' accepts Lua and C functions.
The function \verb'lua_type' can be used to distinguish between
different kinds of user data.

To translate a value from type \verb'lua_Object' to a specific C type,
the programmer can use:
\Deffunc{lua_getnumber}\Deffunc{lua_getstring}
\Deffunc{lua_getcfunction}\Deffunc{lua_getuserdata}
\begin{verbatim}
double         lua_getnumber            (lua_Object object);
char          *lua_getstring            (lua_Object object);
lua_CFunction  lua_getcfunction         (lua_Object object);
void          *lua_getuserdata          (lua_Object object);
\end{verbatim}
\verb'lua_getnumber' converts a \verb'lua_Object' to a float.
This \verb'lua_Object' must be a number or a string convertible to number
(\see{coercion}); otherwise, the function returns 0.

\verb'lua_getstring' converts a \verb'lua_Object' to a string (\verb'char *').
This \verb'lua_Object' must be a string or a number;
otherwise, the function returns 0 (the null pointer).
This function does not create a new string, but returns a pointer to
a string inside the Lua environment.
Because Lua has garbage collection, there is no guarantee that such
pointer will be valid after the block ends.

\verb'lua_getcfunction' converts a \verb'lua_Object' to a C function.
This \verb'lua_Object' must have type {\em CFunction\/};
otherwise, the function returns 0 (the null pointer).
The type \verb'lua_CFunction' is explained in Section~\ref{LuacallC}.

\verb'lua_getuserdata' converts a \verb'lua_Object' to \verb'void*'.
This \verb'lua_Object' must have type {\em userdata\/};
otherwise, the function returns 0 (the null pointer).

The reverse process, that is, passing a specific C value to Lua,
is done by using the following functions:
\Deffunc{lua_pushnumber}\Deffunc{lua_pushstring}
\Deffunc{lua_pushcfunction}\Deffunc{lua_pushusertag}
\Deffunc{lua_pushuserdata}
\begin{verbatim}
void           lua_pushnumber           (double n);
void           lua_pushstring           (char *s);
void           lua_pushcfunction        (lua_CFunction f);
void           lua_pushusertag          (void *u, int tag);
\end{verbatim}
plus the macro:
\begin{verbatim}
void           lua_pushuserdata         (void *u);
\end{verbatim}
All of them receive a C value,
convert it to a corresponding \verb'lua_Object',
and leave the result on the top of the Lua stack,
where it can be assigned to a Lua variable,
passed as parameter to a Lua function, etc. \label{pushing}

User data can have different tags,
whose semantics are defined by the host program.
Any positive integer can be used to tag a user datum.
When a user datum is retrieved,
the function \verb'lua_type' can be used to get its tag.

To complete the set,
the value \nil\ or a \verb'lua_Object' can also be pushed onto the stack,
with:
\Deffunc{lua_pushnil}\Deffunc{lua_pushobject}
\begin{verbatim}
void           lua_pushnil              (void);
void           lua_pushobject           (lua_Object object);
\end{verbatim}


\subsection{Manipulating Lua Objects}
To read the value of any global Lua variable,
one uses the function:
\Deffunc{lua_getglobal}
\begin{verbatim}
lua_Object     lua_getglobal            (char *varname);
\end{verbatim}
As in Lua, if the value of the global is \nil,
then the ``getglobal'' fallback is called.

To store a value previously pushed onto the stack in a global variable,
there is the function:
\Deffunc{lua_storeglobal}
\begin{verbatim}
void           lua_storeglobal          (char *varname);
\end{verbatim}

Tables can also be manipulated via the API.
The function
\Deffunc{lua_getsubscript}
\begin{verbatim}
lua_Object     lua_getsubscript         (void);
\end{verbatim}
expects on the stack a table and an index,
and returns the contents of the table at that index.
As in Lua, if the first object is not a table,
or the index is not present in the table,
the corresponding fallback is called.

To store a value in an index,
the program must push onto the stack the table, the index,
and the value,
and then call the function:
\Deffunc{lua_storesubscript}
\begin{verbatim}
void lua_storesubscript (void);
\end{verbatim}
Again, the corresponding fallback is called if needed.

Finally, the function
\Deffunc{lua_createtable}
\begin{verbatim}
lua_Object     lua_createtable          (void);
\end{verbatim}
creates and returns a new table.

{\em Please Notice:\/}
Most functions from the Lua library receive parameters through Lua's stack.
Because other functions also use this stack,
it is important that these
parameters be pushed just before the corresponding call,
without intermediate calls to the Lua library.
For instance, suppose the user wants the value of \verb'a[i]'.
A simplistic solution would be:
\begin{verbatim}
  /* Warning: WRONG CODE */
  lua_Object result;
  lua_pushobject(lua_getglobal("a"));  /* push table */
  lua_pushobject(lua_getglobal("i"));  /* push index */
  result = lua_getsubscript();
\end{verbatim}
However, the call \verb'lua_getglobal("i")' modifies the stack,
and invalidates the previous pushed value.
A correct solution could be:
\begin{verbatim}
  lua_Object result;
  lua_Object index = lua_getglobal("i");
  lua_pushobject(lua_getglobal("a"));  /* push table */
  lua_pushobject(index);               /* push index */
  result = lua_getsubscript();
\end{verbatim}
The functions \verb|lua_getnumber|, \verb|lua_getstring|,
 \verb|lua_getuserdata|, and \verb|lua_getcfunction|,
plus the family \verb|lua_is*|,
are safe to be called without modifying the stack.

\subsection{Calling Lua Functions}
Functions defined in Lua by a chunk executed with
\verb'dofile' or \verb'dostring' can be called from the host program.
This is done using the following protocol:
first, the arguments to the function are pushed onto the Lua stack
(\see{pushing}), in direct order, i.e., the first argument is pushed first.
Again, it is important to emphasize that, during this phase,
no other Lua function can be called.

Then, the function is called using
\Deffunc{lua_call}\Deffunc{lua_callfunction}
\begin{verbatim}
int            lua_call                 (char *functionname);
\end{verbatim}
or
\begin{verbatim}
int            lua_callfunction         (lua_Object function);
\end{verbatim}
Both functions return an error code:
0, in case of success; non zero, in case of errors.
Finally, the returned values (a Lua function may return many values)
can be retrieved with the macro
\Deffunc{lua_getresult}
\begin{verbatim}
lua_Object     lua_getresult             (int number);
\end{verbatim}
where \verb'number' is the order of the result, starting with 1.
When called with a number larger than the actual number of results,
this function returns \verb'LUA_NOOBJECT'.

Two special Lua functions have exclusive interfaces:
\verb'error' and \verb'setfallback'.
A C function can generate a Lua error calling the function
\Deffunc{lua_error}
\begin{verbatim}
void lua_error (char *message);
\end{verbatim}
This function never returns.
If the C function has been called from Lua,
the corresponding Lua execution terminates,
as if an error had occurred inside Lua code.
Otherwise, the whole program terminates.

Fallbacks can be changed with:
\Deffunc{lua_setfallback}
\begin{verbatim}
lua_Object lua_setfallback (char *name, lua_CFunction fallback);
\end{verbatim}
The first parameter is the fallback name,
and the second a CFunction to be used as the new fallback.
This function returns a \verb'lua_Object',
which is the old fallback value,
or \nil\ on fail (invalid fallback name).
This old value can be used for chaining fallbacks.

An example of C code calling a Lua function is shown in
Section~\ref{exLuacall}.


\subsection{C Functions} \label{LuacallC}
To register a C function to Lua,
there is the following macro:
\Deffunc{lua_register}
\begin{verbatim}
#define lua_register(n,f)       (lua_pushcfunction(f), lua_storeglobal(n))
/* char *n;         */
/* lua_CFunction f; */
\end{verbatim}
which receives the name the function will have in Lua,
and a pointer to the function.
This pointer must have type \verb'lua_CFunction',
which is defined as
\Deffunc{lua_CFunction}
\begin{verbatim}
typedef void (*lua_CFunction) (void);
\end{verbatim}
that is, a pointer to a function with no parameters and no results.

In order to communicate properly with Lua,
a C function must follow a protocol,
which defines the way parameters and results are passed.

To access its arguments, a C function calls:
\Deffunc{lua_getparam}
\begin{verbatim}
lua_Object     lua_getparam             (int number);
\end{verbatim}
where \verb'number' starts with 1 to get the first argument.
When called with a number larger than the actual number of arguments,
this function returns
\verb'LUA_NOOBJECT'\Deffunc{LUA_NOOBJECT}.
In this way, it is possible to write functions that work with
a variable number of parameters.

To return values, a C function just pushes them onto the stack,
in direct order (\see{valuesCLua}).
Like a Lua function, a C function called by Lua can also return
many results.

Section~\ref{exCFunction} presents an example of a CFunction.


\subsection{References to Lua Objects}

As noted in Section~\ref{LuacallC}, \verb'lua_Object's are volatile.
If the C code needs to keep a \verb'lua_Object'
outside block boundaries,
it must create a \Def{reference} to the object.
The routines to manipulate references are the following:
\Deffunc{lua_ref}\Deffunc{lua_getref}
\Deffunc{lua_pushref}\Deffunc{lua_unref}
\begin{verbatim}
int            lua_ref (int lock);
lua_Object     lua_getref  (int ref);
void           lua_pushref (int ref);
void           lua_unref (int ref);
\end{verbatim}
The function \verb'lua_ref' creates a reference
to the object that is on the top of the stack,
and returns this reference.
If \verb'lock' is true, the object is {\em locked}:
that means the object will not be garbage collected.
Notice that an unlocked reference may be garbage collected.
Whenever the referenced object is needed,
a call to \verb'lua_getref'
returns a handle to it,
whereas \verb'lua_pushref' pushes the object on the stack.
If the object has been collected,
then \verb'lua_getref' returns \verb'LUA_NOOBJECT',
and \verb'lua_pushobject' issues an error.

When a reference is no longer needed,
it can be freed with a call to \verb'lua_unref'.



\section{Predefined Functions and Libraries}

The set of \Index{predefined functions} in Lua is small but powerful.
Most of them provide features that allows some degree of
\Index{reflexivity} in the language.
Some of these features cannot be simulated with the rest of the
Language nor with the standard Lua API.
Others are just convenient interfaces to common API functions.

The libraries, on the other hand, provide useful routines
that are implemented directly through the standard API.
Therefore, they are not necessary to the language,
and are provided as separated C modules.
Currently there are three standard libraries:
\begin{itemize}
\item string manipulation;
\item mathematical functions (sin, cos, etc);
\item input and output (plus some system facilities).
\end{itemize}
In order to have access to these libraries,
the host program must call the functions
\verb-strlib_open-, \verb-mathlib_open-, and \verb-iolib_open-,
declared in \verb-lualib.h-.


\subsection{Predefined Functions}

\subsubsection*{\ff{\tt dofile (filename)}}\Deffunc{dofile}
This function receives a file name,
opens it, and executes its contents as a Lua chunk,
or as pre-compiled chunks.
When called without arguments,
it executes the contents of the standard input (\verb'stdin').
If there is any error executing the file, it returns \nil.
Otherwise, it returns the values returned by the chunk,
or a non \nil\ value if the chunk returns no values.
It issues an error when called with a non string argument.

\subsubsection*{\ff{\tt dostring (string)}}\Deffunc{dostring}
This function executes a given string as a Lua chunk.
If there is any error executing the string, it returns \nil.
Otherwise, it returns the values returned by the chunk,
or a non \nil\ value if the chunk returns no values.

\subsubsection*{\ff{\tt next (table, index)}}\Deffunc{next}
This function allows a program to traverse all fields of a table.
Its first argument is a table and its second argument
is an index in this table.
It returns the next index of the table and the
value associated with the index.
When called with \nil\ as its second argument,
the function returns the first index
of the table (and its associated value).
When called with the last index, or with \nil\ in an empty table,
it returns \nil.

In Lua there is no declaration of fields;
semantically, there is no difference between a
field not present in a table or a field with value \nil.
Therefore, the function only considers fields with non nil values.
The order the indices are enumerated is not specified,
{\em even for numeric indices}.

See Section \ref{exnext} for an example of the use of this function.

\subsubsection*{\ff{\tt nextvar (name)}}\Deffunc{nextvar}
This function is similar to the function \verb'next',
but iterates over the global variables.
Its single argument is the name of a global variable,
or \nil\ to get a first name.
Similarly to \verb'next', it returns the name of another variable
and its value,
or \nil\ if there are no more variables.
See Section \ref{exnext} for an example of the use of this function.

\subsubsection*{\ff{\tt tostring (e)}}\Deffunc{tostring}
This function receives an argument of any type and
converts it to a string in a reasonable format.

\subsubsection*{\ff{\tt print (e1, e2, ...)}}\Deffunc{print}
This function receives any number of arguments,
and prints their values in a reasonable format.
Each value is printed in a new line.
This function is not intended for formatted output,
but as a quick way to show a value,
for instance for error messages or debugging.
See Section~\ref{libio} for functions for formatted output.

\subsubsection*{\ff{\tt tonumber (e)}}\Deffunc{tonumber}
This function receives one argument,
and tries to convert it to a number.
If the argument is already a number or a string convertible
to a number (\see{coercion}), then it returns that number;
otherwise, it returns \nil.

\subsubsection*{\ff{\tt type (v)}}\Deffunc{type}
This function allows Lua to test the type of a value.
It receives one argument, and returns its type, coded as a string.
The possible results of this function are
\verb'"nil"' (a string, not the value \nil),
\verb'"number"',
\verb'"string"',
\verb'"table"',
\verb'"function"' (returned both for C functions and Lua functions),
and \verb'"userdata"'.

Besides this string, the function returns a second result,
which is the \Def{tag} of the value.
This tag can be used to distinguish between user
data with different tags,
and between C functions and Lua functions.

\subsubsection*{\ff{\tt assert (v)}}\Deffunc{assert}
This function issues an {\em ``assertion failed!''} error
when its argument is \nil.

\subsubsection*{\ff{\tt error (message)}}\Deffunc{error}\label{pdf-error}
This function issues an error message and terminates
the last called function from the library
(\verb'lua_dofile', \verb'lua_dostring', \ldots).
It never returns.

\subsubsection*{\ff{\tt setglobal (name, value)}}\Deffunc{setglobal}
This function assigns the given value to a global variable.
The string \verb'name' does not need to be a syntactically valid variable name.
Therefore, this function can set global variables with strange names like
\verb|`m v 1'| or \verb'34'.
It returns the value of its second argument.

\subsubsection*{\ff{\tt getglobal (name)}}\Deffunc{getglobal}
This function retrieves the value of a global variable.
The string \verb'name' does not need to be a syntactically valid variable name.

\subsubsection*{\ff{\tt setfallback (fallbackname, newfallback)}}
\Deffunc{setfallback}
This function sets a new fallback function to the given fallback.
It returns the old fallback function.

\subsection{String Manipulation}
This library provides generic functions for string manipulation,
such as finding and extracting substrings and pattern matching.
When indexing a string, the first character has position 1.
See Page~\pageref{pm} for an explanation about patterns,
and Section~\ref{exstring} for some examples on string manipulation
in Lua.

\subsubsection*{\ff{\tt strfind (str, pattern [, init [, plain]])}}
\Deffunc{strfind}
This function looks for the first {\em match} of
\verb-pattern- in \verb-str-.
If it finds one, it returns the indexes on \verb-str-
where this occurence starts and ends;
otherwise, it returns \nil.
If the pattern specifies captures,
the captured strings are returned as extra results.
A third optional numerical argument specifies where to start the search;
its default value is 1.
A value of 1 as a forth optional argument
turns off the pattern matching facilities,
so the function does a plain ``find substring'' operation.

\subsubsection*{\ff{\tt strlen (s)}}\Deffunc{strlen}
Receives a string and returns its length.

\subsubsection*{\ff{\tt strsub (s, i [, j])}}\Deffunc{strsub}
Returns another string, which is a substring of \verb's',
starting at \verb'i'  and runing until \verb'j'.
If \verb'j' is absent,
it is assumed to be equal to the length of \verb's'.
In particular, the call \verb'strsub(s,1,j)' returns a prefix of \verb's'
with length \verb'j',
whereas the call \verb'strsub(s,i)' returns a suffix of \verb's',
starting at \verb'i'.

\subsubsection*{\ff{\tt strlower (s)}}\Deffunc{strlower}
Receives a string and returns a copy of that string with all
upper case letters changed to lower case.
All other characters are left unchanged.

\subsubsection*{\ff{\tt strupper (s)}}\Deffunc{strupper}
Receives a string and returns a copy of that string with all
lower case letters changed to upper case.
All other characters are left unchanged.

\subsubsection*{\ff{\tt strrep (s, n)}}\Deffunc{strrep}
Returns a string which is the concatenation of \verb-n- copies of 
the string \verb-s-.

\subsubsection*{\ff{\tt ascii (s [, i])}}\Deffunc{ascii}
Returns the ascii code of the character \verb's[i]'.
If \verb'i' is absent, then it is assumed to be 1.

\subsubsection*{\ff{\tt format (formatstring, e1, e2, \ldots)}}\Deffunc{format}
\label{format}
This function returns a formated version of its variable number of arguments
following the description given in its first argument (which must be a string). 
The format string follows the same rules as the \verb'printf' family of
standard C functions.
The only differences are that the options/modifiers
\verb'*', \verb'l', \verb'L', \verb'n', \verb'p',
and \verb'h' are not supported,
and there is an extra option, \verb'q'.
This option formats a string in a form suitable to be safely read
back by the Lua interpreter;
that is,
the string is written between double quotes,
and all double quotes, returns and backslashes in the string
are correctly escaped when written.
For instance, the call
\begin{verbatim}
format('%q', 'a string with "quotes" and \n new line')
\end{verbatim}
will produce the string:
\begin{verbatim}
"a string with \"quotes\" and \
 new line"
\end{verbatim}

The options \verb'c', \verb'd', \verb'E', \verb'e', \verb'f',
\verb'g' \verb'i', \verb'o', \verb'u', \verb'X', and \verb'x' all
expect a number as argument,
whereas \verb'q' and \verb's' expect a string.

\subsubsection*{\ff{\tt gsub (s, pat, repl [, n])}}\Deffunc{gsub}
Returns a copy of \verb-s-,
where all occurrences of the pattern \verb-pat- have been
replaced by a replacement string specified by \verb-repl-.
This function also returns, as a second value,
the total number of substitutions made.

If \verb-repl- is a string, its value is used for replacement.
Any sequence in \verb-repl- of the form \verb-%n-
with \verb-n- between 1 and 9
stands for the value of the n-th captured substring.

If \verb-repl- is a function, this function is called every time a
match occurs, with all captured substrings as parameters.
If the value returned by this function is a string,
it is used as the replacement string;
otherwise, the replacement string is the empty string.

An optional parameter \verb-n- limits 
the maximum number of substitutions to occur.
For instance, when \verb-n- is 1 only the first occurrence of
\verb-pat- is replaced.

As an example, in the following expression each occurrence of the form
\verb-$name$- calls the function \verb|getenv|,
passing \verb|name| as argument
(because only this part of the pattern is captured).
The value returned by \verb|getenv| will replace the pattern.
Therefore, the whole expression:
\begin{verbatim}
  gsub("home = $HOME$, user = $USER$", "$(%w%w*)$", getenv)
\end{verbatim}
may return the string:
\begin{verbatim}
home = /home/roberto, user = roberto
\end{verbatim}

\subsubsection*{Patterns} \label{pm}

\paragraph{Character Class:}
a \Def{character class} is used to represent a set of characters.
The following combinations are allowed in describing a character class:
\begin{description}
\item[{\em x}] (where {\em x} is any character not in the list \verb'()%.[*?')
--- represents the character {\em x} itself.
\item[{\tt .}] --- represents all characters.
\item[{\tt \%a}] --- represents all letters.
\item[{\tt \%A}] --- represents all non letter characters.
\item[{\tt \%d}] --- represents all digits.
\item[{\tt \%D}] --- represents all non digits.
\item[{\tt \%l}] --- represents all lower case letters.
\item[{\tt \%L}] --- represents all non lower case letter characters.
\item[{\tt \%s}] --- represents all space characters.
\item[{\tt \%S}] --- represents all non space characters.
\item[{\tt \%u}] --- represents all upper case letters.
\item[{\tt \%U}] --- represents all non upper case letter characters.
\item[{\tt \%w}] --- represents all alphanumeric characters.
\item[{\tt \%W}] --- represents all non alphanumeric characters.
\item[{\tt \%\em x}] (where {\em x} is any non alphanumeric character)  ---
represents the character {\em x}.
\item[{\tt [char-set]}] --- 
Represents the class which is the union of all
characters in char-set.
To include a \verb']' in char-set, it must be the first character.
A range of characters may be specified by
separating the end characters of the range with a \verb'-';
e.g., \verb'A-Z' specifies the upper case characters.
If \verb'-' appears as the first or last character of char-set,
then it represents itself.
All classes \verb'%'{\em x} described above can also be used as
components in a char-set.
All other characters in char-set represent themselves.
\item[{\tt [\^{ }char-set]}] ---
represents the complement of char-set,
where char-set is interpreted as above.
\end{description}

\paragraph{Pattern Item:}
a \Def{pattern item} may be a single character class,
or a character class followed by \verb'*' or by \verb'?'.
A single character class matches any single character in the class.
A character class followed by \verb'*' matches 0 or more repetitions
of characters in the class.
A character class followed by \verb'?' matches 0 or one occurrence
of a character in the class.
A pattern item may also has the form \verb'%n',
for \verb-n- between 1 and 9;
such item matches a sub-string equal to the n-th captured string.

\paragraph{Pattern:}
a \Def{pattern} is a sequence of pattern items.
Any repetition item (\verb'*') inside a pattern will always
match the longest possible sequence.
A \verb'^' at the beginning of a pattern anchors the match at the
beginning of the subject string.
A \verb'$' at the end of a pattern anchors the match at the
end of the subject string.

A pattern may contain sub-patterns enclosed in parentheses,
that describe \Def{captures}.
When a match succeeds, the sub-strings of the subject string
that match captures are {\em captured} for future use.
Captures are numbered according to their left parentheses.

\subsection{Mathematical Functions} \label{mathlib}

This library is an interface to some functions of the standard C math library.
Moreover, it registers a fallback for the binary operator \verb'^' which,
when applied to numbers \verb'x^y', returns $x^y$.

The library provides the following functions:
\Deffunc{abs}\Deffunc{acos}\Deffunc{asin}\Deffunc{atan}
\Deffunc{atan2}\Deffunc{ceil}\Deffunc{cos}\Deffunc{floor}
\Deffunc{log}\Deffunc{log10}\Deffunc{max}\Deffunc{min}
\Deffunc{mod}\Deffunc{sin}\Deffunc{sqrt}\Deffunc{tan}
\Deffunc{random}\Deffunc{randomseed}
\begin{verbatim}
abs acos asin atan atan2 ceil cos floor log log10
max min  mod  sin  sqrt tan random randomseed
\end{verbatim}
Most of them
are only interfaces to the homonymous functions in the C library,
except that, for the trigonometric functions,
all angles are expressed in degrees, not radians.

The function \verb'max' returns the maximum
value of its numeric arguments.
Similarly, \verb'min' computes the minimum.
Both can be used with an unlimited number of arguments.

The functions \verb'random' and \verb'randomseed' are interfaces to
the simple random generator functions \verb'rand' and \verb'srand',
provided by ANSI C.
The function \verb'random' returns pseudo-random numbers in the range
$[0,1)$.


\subsection{I/O Facilities} \label{libio}

All I/O operations in Lua are done over two {\em current} files:
one for reading and one for writing.
Initially, the current input file is \verb'stdin',
and the current output file is \verb'stdout'.

Unless otherwise stated,
all I/O functions return \nil\ on failure and
some value different from \nil\ on success.

\subsubsection*{\ff{\tt readfrom (filename)}}\Deffunc{readfrom}

This function may be called in three ways.
When called with a file name,
it opens the named file,
sets it as the {\em current} input file,
and returns a {\em handle} to the file
(this handle is a user data containing the file stream \verb|FILE *|).
When called with a file handle, returned by a previous call,
it restores the file as the current input.
When called without parameters,
it closes the current input file,
and restores \verb'stdin' as the current input file.

If this function fails, it returns \nil,
plus a string describing the error.

{\em System dependent:} if \verb'filename' starts with a \verb'|',
then a \Index{piped input} is open, via function \IndexVerb{popen}.

\subsubsection*{\ff{\tt writeto (filename)}}\Deffunc{writeto}

This function may be called in three ways.
When called with a file name,
it opens the named file,
sets it as the {\em current} output file,
and returns a {\em handle} to the file
(this handle is a user data containing the file stream \verb|FILE *|).
Notice that, if the file already exists,
it will be {\em completely erased} with this operation.
When called with a file handle, returned by a previous call,
it restores the file as the current output.
When called without parameters,
this function closes the current output file,
and restores \verb'stdout' as the current output file.
\index{closing a file}

If this function fails, it returns \nil,
plus a string describing the error.

{\em System dependent:} if \verb'filename' starts with a \verb'|',
then a \Index{piped output} is open, via function \IndexVerb{popen}.

\subsubsection*{\ff{\tt appendto (filename)}}\Deffunc{appendto}

This function opens a file named \verb'filename' and sets it as the
{\em current} output file.
It returns the file handle,
or \nil\ in case of error.
Unlike the \verb'writeto' operation,
this function does not erase any previous content of the file.
If this function fails, it returns \nil,
plus a string describing the error.

Notice that function \verb|writeto| is available to close a file.

\subsubsection*{\ff{\tt remove (filename)}}\Deffunc{remove}

This function deletes the file with the given name.
If this function fails, it returns \nil,
plus a string describing the error.

\subsubsection*{\ff{\tt rename (name1, name2)}}\Deffunc{rename}

This function renames file \verb'name1' to \verb'name2'.
If this function fails, it returns \nil,
plus a string describing the error.

\subsubsection*{\ff{\tt tmpname ()}}\Deffunc{tmpname}

This function returns a string with a file name that can safely
be used for a temporary file.

\subsubsection*{\ff{\tt read ([readpattern])}}\Deffunc{read}

This function reads the current input
according to a read pattern, that specifies how much to read;
characters are read from the current input file until
the read pattern fails or ends.
The function \verb|read| returns a string with the characters read,
or \nil\ if the read pattern fails {\em and\/}
the result string would be empty.
When called without parameters,
it uses a default pattern that reads the next line
(see below).

A \Def{read pattern} is a sequence of read pattern items.
An item may be a single character class
or a character class followed by \verb'?' or by \verb'*'.
A single character class reads the next character from the input
if it belongs to the class, otherwise it fails.
A character class followed by \verb'?' reads the next character
from the input if it belongs to the class;
it never fails.
A character class followed by \verb'*' reads until a character that
does not belong to the class, or end of file;
since it can match a sequence of zero characteres, it never fails.%
\footnote{
Notice that this behaviour is different from regular pattern matching,
where a \verb'*' expands to the maximum length {\em such that\/}
the rest of the pattern does not fail.}

A pattern item may contain sub-patterns enclosed in curly brackets,
that describe \Def{skips}.
Characters matching a skip are read,
but are not included in the resulting string.

Following are some examples of read patterns and their meanings:
\begin{itemize}
\item \verb|"."| returns the next character, or \nil\ on end of file.
\item \verb|".*"| reads the whole file.
\item \verb|"[^\n]*{\n}"| returns the next line
(skipping the end of line), or \nil\ on end of file.
This is the default pattern.
\item \verb|"{%s*}%S%S*"| returns the next word
(maximal sequence of non white-space characters),
or \nil\ on end of file.
\item \verb|"{%s*}[+-]?%d%d*"| returns the next integer
or \nil\ if the next characters do not conform to an integer format.
\end{itemize}

\subsubsection*{\ff{\tt write (value1, ...)}}\Deffunc{write}

This function writes the value of each of its arguments to the
current output file.
The arguments must be strings or numbers.
If this function fails, it returns \nil,
plus a string describing the error.

\subsubsection*{\ff{\tt date ([format])}}\Deffunc{date}

This function returns a string containing date and time
formatted according to the given string \verb'format',
following the same rules of the ANSI C function \verb'strftime'.
When called without arguments,
it returns a reasonable date and time representation.

\subsubsection*{\ff{\tt exit ([code])}}\Deffunc{exit}

This function calls the C function \verb-exit-,
with an optional \verb-code-,
to terminate the program.

\subsubsection*{\ff{\tt getenv (varname)}}\Deffunc{getenv}

Returns the value of the environment variable \verb|varname|,
or \nil\ if the variable is not defined.

\subsubsection*{\ff{\tt execute (command)}}\Deffunc{execute}

This function is equivalent to the C function \verb|system|.
It passes \verb|command| to be executed by an Operating System Shell.
It returns an error code, which is implementation-defined.


\section{The Debugger Interface} \label{debugI}

Lua has no built-in debugger facilities.
Instead, it offers a special interface,
by means of functions and {\em hooks},
which allows the construction of different
kinds of debuggers, profilers, and other tools
that need ``inside information'' from the interpreter.
This interface is declared in the header file \verb'luadebug.h'.

\subsection{Stack and Function Information}

The main function to get information about the interpreter stack
is
\begin{verbatim}
lua_Function lua_stackedfunction (int level);
\end{verbatim}
It returns a handle (\verb'lua_Function') to the {\em activation record\/}
of the function executing at a given level.
Level 0 is the current running function,
while level $n+1$ is the function that has called level $n$.
When called with a level greater than the stack depth,
\verb'lua_stackedfunction' returns \verb'LUA_NOOBJECT'.

The type \verb'lua_Function' is just another name
to \verb'lua_Object'.
Although, in this library,
a \verb'lua_Function' can be used wherever a \verb'lua_Object' is required,
a parameter \verb'lua_Function' accepts only a handle returned by
\verb'lua_stackedfunction'.

Three other functions produce extra information about a function:
\begin{verbatim}
void lua_funcinfo (lua_Object func, char **filename, int *linedefined);
int lua_currentline (lua_Function func);
char *lua_getobjname (lua_Object o, char **name);
\end{verbatim}
\verb'lua_funcinfo' gives the file name and the line where the
given function has been defined.
If the ``function'' is in fact the main code of a chunk,
then \verb'linedefined' is 0.
If the function is a C function,
then \verb'linedefined' is -1, and \verb'filename' is \verb'"(C)"'.

The function \verb'lua_currentline' gives the current line where
a given function is executing.
It only works if the function has been pre-compiled with debug
information (\see{pragma}).
When no line information is available, it returns -1.

Function \verb'lua_getobjname' tries to find a reasonable name for
a given function.
Because functions in Lua are first class values,
they do not have a fixed name.
Some functions may be the value of many global variables,
while others may be stored only in a table field.
Function \verb'lua_getobjname' first checks whether the given
function is a fallback.
If so, it returns the string \verb'"fallback"',
and \verb'name' is set to point to the fallback name.
Otherwise, if the given function is the value of a global variable,
then \verb'lua_getobjname' returns the string \verb'"global"',
while \verb'name' points to the variable name.
If the given function is neither a fallback nor a global variable,
then \verb'lua_getobjname' returns the empty string,
and \verb'name' is set to \verb'NULL'.

\subsection{Manipulating Local Variables}

The following functions allow the manipulation of the
local variables of a given activation record.
They only work if the function has been pre-compiled with debug
information (\see{pragma}).
\begin{verbatim}
lua_Object lua_getlocal (lua_Function func, int local_number, char **name);
int lua_setlocal (lua_Function func, int local_number);
\end{verbatim}
The first one returns the value of a local variable,
and sets \verb'name' to point to the variable name.
\verb'local_number' is an index for local variables.
The first parameter has index 1, and so on, until the
last active local variable.
When called with a \verb'local_number' greater than the
number of active local variables,
or if the activation record has no debug information,
\verb'lua_getlocal' returns \verb'LUA_NOOBJECT'.

The function \verb'lua_setlocal' sets the local variable
\verb'local_number' to the value previously pushed on the stack
(\see{valuesCLua}).
If the function succeeds, then it returns 1.
If \verb'local_number' is greater than the number
of active local variables,
or if the activation record has no debug information,
then this function fails and returns 0.

\subsection{Hooks}

The Lua interpreter offers two hooks for debugging purposes:
\begin{verbatim}
typedef void (*lua_CHFunction) (lua_Function func, char *file, int line);
extern lua_CHFunction lua_callhook;

typedef void (*lua_LHFunction) (int line);
extern lua_LHFunction lua_linehook;
\end{verbatim}
The first one is called whenever the interpreter enters or leaves a
function.
When entering a function,
its parameters are a handle to the function activation record,
plus the file and the line where the function is defined (the same
information which is provided by \verb'lua_funcinfo');
when leaving a function, \verb'func' is \verb'LUA_NOOBJECT',
\verb'file' is \verb'"(return)"', and \verb'line' is 0.

The other hook is called every time the interpreter changes
the line of code it is executing.
Its only parameter is the line number
(the same information which is provided by the call
\verb'lua_currentline(lua_stackedfunction(0))').
This second hook is only called if the active function
has been pre-compiled with debug information (\see{pragma}).

A hook is disabled when its value is NULL (0),
which is the initial value of both hooks.


\section{Some Examples}

This section gives examples showing some features of Lua.
It does not intend to cover the whole language,
but only to illustrate some interesting uses of the system.


\subsection{\Index{Data Structures}}
Tables are a strong unifying data constructor.
They directly implement a multitude of data types,
like ordinary arrays, records, sets, bags, and lists.

Arrays need no explanations.
In Lua, it is conventional to start indices from 1,
but this is only a convention.
Arrays can be indexed by 0, negative numbers, or any other value (except \nil).
Records are also trivially implemented by the syntactic sugar
\verb'a.x'.

The best way to implement a set is to store
its elements as indices of a table.
The statement \verb's = {}' creates an empty set \verb's'. 
The statement \verb's[x] = 1' inserts the value of \verb'x' into
the set \verb's'.
The expression \verb's[x]' is true if and only if
\verb'x' belongs to \verb's'.
Finally, the statement \verb's[x] = nil' removes \verb'x' from \verb's'.

Bags can be implemented similarly to sets,
but using the value associated to an element as its counter.
So, to insert an element, 
the following code is enough:
\begin{verbatim}
if s[x] then s[x] = s[x]+1 else s[x] = 1 end
\end{verbatim}
and to remove an element:
\begin{verbatim}
if s[x] then s[x] = s[x]-1 end
if s[x] == 0 then s[x] = nil end
\end{verbatim}

Lisp-like lists also have an easy implementation.
The ``cons'' of two elements \verb'x' and \verb'y' can be
created with the code \verb'l = {car=x, cdr=y}'.
The expression \verb'l.car' extracts the header, 
while \verb'l.cdr' extracts the tail.
An alternative way is to create the list directly with \verb'l={x,y}',
and then to extract the header with \verb'l[1]' and
the tail with \verb'l[2]'.

\subsection{The Functions {\tt next} and {\tt nextvar}} \label{exnext}
\Deffunc{next}\Deffunc{nextvar}
This example shows how to use the function \verb'next' to iterate
over the fields of a table.
Function \IndexVerb{clone} receives any table and returns a clone of it.
\begin{verbatim}
function clone (t)           -- t is a table
  local new_t = {}           -- create a new table
  local i, v = next(t, nil)  -- i is an index of t, v = t[i]
  while i do
    new_t[i] = v
    i, v = next(t, i)        -- get next index
  end
  return new_t
end
\end{verbatim}

The next example prints the names of all global variables
in the system with non nil values:
\begin{verbatim}
function printGlobalVariables ()
  local i, v = nextvar(nil)
  while i do
    print(i)
    i, v = nextvar(i)
  end
end
\end{verbatim}


\subsection{String Manipulation} \label{exstring}

The first example is a function to trim extra white-spaces at the beginning
and end of a string.
\begin{verbatim}
function trim(s)
  local _, i = strfind(s, '^ *')
  local f, __ = strfind(s, ' *$')
  return strsub(s, i+1, f-1)
end
\end{verbatim}

The second example shows a function that eliminates all spaces
of a string.
\begin{verbatim}
function remove_blanks (s)
  return gsub(s, "%s%s*", "")
end
\end{verbatim}


\subsection{\Index{Variable number of arguments}}
Lua does not provide any explicit mechanism to deal with
variable number of arguments in function calls.
However, one can use table constructors to simulate this mechanism.
As an example, suppose a function to concatenate all its arguments.
It could be written like 
\begin{verbatim}
function concat (o)
  local i = 1
  local s = ''
  while o[i] do
    s = s .. o[i]
    i = i+1
  end
  return s
end
\end{verbatim}
To call it, one uses a table constructor to join all arguments:
\begin{verbatim}
  x = concat{"hello ", "john", " and ", "mary"}
\end{verbatim}

\subsection{\Index{Persistence}}
Because of its reflexive facilities,
persistence in Lua can be achieved within the language.
This section shows some ways to store and retrieve values in Lua,
using a text file written in the language itself as the storage media.

To store a single value with a name,
the following code is enough:
\begin{verbatim}
function store (name, value)
  write(format('\n%s =', name))
  write_value(value)
end
\end{verbatim}
\begin{verbatim}
function write_value (value)
  local t = type(value)
      if t == 'nil'    then write('nil')
  elseif t == 'number' then write(value)
  elseif t == 'string' then write(value, 'q')
  end
end
\end{verbatim}
In order to restore this value, a \verb'lua_dofile' suffices.

Storing tables is a little more complex.
Assuming that the table is a tree,
and that all indices are identifiers
(that is, the tables are being used as records),
then its value can be written directly with table constructors.
First, the function \verb'write_value' is changed to
\begin{verbatim}
function write_value (value)
  local t = type(value)
      if t == 'nil'    then write('nil')
  elseif t == 'number' then write(value)
  elseif t == 'string' then write(value, 'q')
  elseif t == 'table'  then write_record(value)
  end
end
\end{verbatim}
The function \verb'write_record' is:
\begin{verbatim}
function write_record(t)
  local i, v = next(t, nil)
  write('{')  -- starts constructor
  while i do
    store(i, v)
    write(', ')
    i, v = next(t, i)
  end
  write('}')  -- closes constructor
end
\end{verbatim}


\subsection{Inheritance} \label{exfallback}
The fallback for absent indices can be used to implement many
kinds of \Index{inheritance} in Lua.
As an example,
the following code implements single inheritance:
\begin{verbatim}
function Index (t,f)
  if f == 'parent' then  -- to avoid loop
    return OldIndex(t,f)
  end
  local p = t.parent
  if type(p) == 'table' then
    return p[f]
  else
    return OldIndex(t,f)
  end
end

OldIndex = setfallback("index", Index)
\end{verbatim}
Whenever Lua attempts to access an absent field in a table,
it calls the fallback function \verb'Index'.
If the table has a field \verb'parent' with a table value,
then Lua attempts to access the desired field in this parent object.
This process is repeated ``upwards'' until a value
for the field is found or the object has no parent.
In the latter case, the previous fallback is called to supply a value
for the field.

When better performance is needed,
the same fallback may be implemented in C,
as illustrated in Figure~\ref{Cinher}.
\begin{figure}
\Line
\begin{verbatim}
#include "lua.h"

int lockedParentName;  /* lock index for the string "parent" */
int lockedOldIndex;    /* previous fallback function */

void callOldFallback (lua_Object table, lua_Object index)
{
  lua_Object oldIndex = lua_getref(lockedOldIndex);
  lua_pushobject(table);
  lua_pushobject(index);
  lua_callfunction(oldIndex);
}

void Index (void)
{
  lua_Object table = lua_getparam(1);
  lua_Object index = lua_getparam(2);
  lua_Object parent;
  if (lua_isstring(index) && strcmp(lua_getstring(index), "parent") == 0)
  {
    callOldFallback(table, index);
    return;
  }
  lua_pushobject(table);
  lua_pushref(lockedParentName);
  parent = lua_getsubscript();
  if (lua_istable(parent))
  {
    lua_pushobject(parent);
    lua_pushobject(index);
    /* return result from getsubscript */
    lua_pushobject(lua_getsubscript());
  }
  else
    callOldFallback(table, index);
}
\end{verbatim}
\caption{Inheritance in C.\label{Cinher}}
\Line
\end{figure}
This code must be registered with:
\begin{verbatim}
  lua_pushstring("parent");
  lockedParentName = lua_ref(1);
  lua_pushobject(lua_setfallback("index", Index));
  lockedOldIndex = lua_ref(1);
\end{verbatim}
Notice how the string \verb'"parent"' is kept
locked in Lua for optimal performance.

\subsection{\Index{Programming with Classes}}
There are many different ways to do object-oriented programming in Lua.
This section presents one possible way to
implement classes,
using the inheritance mechanism presented above.
{\em Please notice: the following examples only work
with the index fallback redefined according to
Section~\ref{exfallback}}.

As one could expect, a good way to represent a class is
with a table.
This table will contain all instance methods of the class,
plus optional default values for instance variables.
An instance of a class has its \verb'parent' field pointing to
the class,
and so it ``inherits'' all methods.

For instance, a class \verb'Point' can be described as in
Figure~\ref{Point}.
Function \verb'create' helps the creation of new points,
adding the parent field.
Function \verb'move' is an example of an instance method.
\begin{figure}
\Line
\begin{verbatim}
Point = {x = 0, y = 0}

function Point:create (o)
  o.parent = self
  return o
end

function Point:move (p)
  self.x = self.x + p.x
  self.y = self.y + p.y
end

...

--
-- creating points
--
p1 = Point:create{x = 10, y = 20}
p2 = Point:create{x = 10}  -- y will be inherited until it is set

--
-- example of a method invocation
--
p1:move(p2)
\end{verbatim}
\caption{A Class {\tt Point}.\label{Point}}
\Line
\end{figure}
Finally, a subclass can be created as a new table,
with the \verb'parent' field pointing to its superclass.
It is interesting to notice how the use of \verb'self' in
method \verb'create' allows this method to work properly even
when inherited by a subclass.
As usual, a subclass may overwrite any inherited method with
its own version.

\subsection{\Index{Modules}}
Here we explain one possible way to simulate modules in Lua.
The main idea is to use a table to store the module functions.

A module should be written as a separate chunk, starting with:
\begin{verbatim}
if modulename then return end  -- avoid loading twice the same module
modulename = {}  -- create a table to represent the module
\end{verbatim}
After that, functions can be directly defined with the syntax
\begin{verbatim}
function modulename.foo (...)
  ...
end
\end{verbatim}

Any code that needs this module has only to execute
\verb'dofile("filename")', where \verb'filename' is the file
where the module is written.
After this, any function can be called with
\begin{verbatim}
modulename.foo(...)
\end{verbatim}

If a module function is going to be used many times,
the program can give a local name to it.
Because functions are values, it is enough to write
\begin{verbatim}
localname = modulename.foo
\end{verbatim}
Finally, a module may be {\em opened},
giving direct access to all its functions,
as shown in the code in Figure~\ref{openmod}.
\begin{figure}
\Line
\begin{verbatim}
function open (mod)
  local n, f = next(mod, nil)
  while n do
    setglobal(n, f)
    n, f = next(mod, n)
  end
end
\end{verbatim}
\caption{Opening a module.\label{openmod}}
\Line
\end{figure}

\subsection{A CFunction} \label{exCFunction}\index{functions in C}
A CFunction to compute the maximum of a variable number of arguments
is shown in Figure~\ref{Cmax}.
\begin{figure}
\Line
\begin{verbatim}
void math_max (void)
{
 int i=1;   /* number of arguments */
 double d, dmax;
 lua_Object o;
 /* the function must get at least one argument */
 if ((o = lua_getparam(i++)) == LUA_NOOBJECT)
   lua_error ("too few arguments to function `max'");
 /* and this argument must be a number */
 if (!lua_isnumber(o))
   lua_error ("incorrect argument to function `max'");
 dmax = lua_getnumber (o);
 /* loops until there is no more arguments */
 while ((o = lua_getparam(i++)) != LUA_NOOBJECT)
 {
  if (!lua_isnumber(o))
    lua_error ("incorrect argument to function `max'");
  d = lua_getnumber (o);
  if (d > dmax) dmax = d;
 }
 /* push the result to be returned */
 lua_pushnumber (dmax);
}
\end{verbatim}
\caption{C function {\tt math\_max}.\label{Cmax}}
\Line
\end{figure}
After registered with
\begin{verbatim}
lua_register ("max", math_max);
\end{verbatim}
this function is available in Lua, as follows:
\begin{verbatim}
i = max(4, 5, 10, -34)  -- i receives 10
\end{verbatim}


\subsection{Calling Lua Functions} \label{exLuacall}

This example illustrates how a C function can call the Lua function
\verb'remove_blanks' presented in Section~\ref{exstring}.
\begin{verbatim}
void remove_blanks (char *s)
{
  lua_pushstring(s);  /* prepare parameter */
  lua_call("remove_blanks");  /* call Lua function */
  strcpy(s, lua_getstring(lua_getresult(1)));  /* copy result back to 's' */
}
\end{verbatim}


\section{\Index{Lua Stand-alone}}

Although Lua has been designed as an extension language,
the language can also be used as a stand-alone interpreter.
An implementation of such an interpreter,
called simply \verb|lua|,
is provided with the standard distribution.
This program can be called with any sequence of the following arguments:
\begin{description}
\item[{\tt -v}] prints version information.
\item[{\tt -}] runs interactively, accepting commands from standard input
until an \verb|EOF|.
\item[{\tt -e stat}] executes \verb|stat| as a Lua chunk.
\item[{\tt var=exp}] executes \verb|var=exp| as a Lua chunk.
\item[{\tt filename}] executes file \verb|filename| as a Lua chunk.
\end{description}
All arguments are handle in order.
For instance, an invocation like
\begin{verbatim}
$ lua - a=1 prog.lua
\end{verbatim}
will first interact with the user until an \verb|EOF|,
then will set \verb'a' to 1,
and finally will run file \verb'prog.lua'.

Please notice that the interaction with the shell may lead to
unintended results.
For instance, a call like
\begin{verbatim}
$ lua a="name" prog.lua
\end{verbatim}
will {\em not} set \verb|a| to the string \verb|"name"|.
Instead, the quotes will be handled by the shell,
lua will get only \verb'a=name' to run,
and \verb'a' will finish with \nil.
Instead, one should write
\begin{verbatim}
$ lua 'a="name"' prog.lua
\end{verbatim}

\section*{Acknowledgments}

The authors would like to thank CENPES/PETROBR\'AS which,
jointly with \tecgraf, used extensively early versions of
this system and gave valuable comments.
The authors would also like to thank Carlos Henrique Levy,
who found the name of the game.
Lua means {\em moon} in Portuguese.



\appendix

\section*{Incompatibilities with Previous Versions}

Although great care has been taken to avoid incompatibilities with
the previous public versions of Lua,
some differences had to be introduced.
Here is a list of all these incompatibilities.

\subsection*{Incompatibilities with \Index{version 2.4}}
The whole I/O facilities have been rewritten.
We strongly encourage programmers to addapt their code
to this new version.
However, we are keeping the old version of the libraries
in the distribution,
to allow a smooth transition.
The incompatibilities between the new and the old libraries are:
\begin{itemize}
\item The format facility of function \verb'write' has been supersed by
function \verb'format';
therefore this facility has been dropped.
\item Function \verb'read' now uses {\em read patterns} to specify
what to read;
this is incompatible with the old format options.
\item Function \verb'strfind' now accepts patterns,
so it may have a different behavior when the pattern includes
special characteres.
\end{itemize}

\subsection*{Incompatibilities with \Index{version 2.2}}
\begin{itemize}
\item
Functions \verb'date' and \verb'time' (from \verb'iolib')
have been superseded by the new, more powerful version of function \verb'date'.
\item
Function \verb'append' (from \verb'iolib') now returns 1 whenever it succeeds,
whether the file is new or not.
\item
Function \verb'int2str' (from \verb'strlib') has been superseded by new
function \verb'format', with parameter \verb'"%c"'.
\item
The API lock mechanism has been superseded by the reference mechanism.
However, \verb-lua.h- provides compatibility macros,
so there is no need to change programs.
\item
The API function \verb'lua_pushliteral' now is just a macro to
\verb'lua_pushstring'.
\end{itemize}

\subsection*{Incompatibilities with \Index{version 2.1}}
\begin{itemize}
\item
The function \verb'type' now returns the string \verb'"function"'
both for C and Lua functions.
Because Lua functions and C functions are compatible,
this behavior is usually more useful.
When needed, the second result of function {\tt type} may be used
to distinguish between Lua and C functions.
\item
A function definition only assigns the function value to the
given variable at execution time.
\end{itemize}

\subsection*{Incompatibilities with \Index{version 1.1}}
\begin{itemize}
\item
The equality test operator now is denoted by \verb'==',
instead of \verb'='.
\item
The syntax for table construction has been greatly simplified.
The old \verb'@(size)' has been substituted by \verb'{}'.
The list constructor (formerly \verb'@[...]') and the record
constructor (formerly \verb'@{...}') now are both coded like
\verb'{...}'.
When the construction involves a function call,
like in \verb'@func{...}',
the new syntax does not use the \verb'@'.
More important, {\em a construction function must now
explicitly return the constructed table}.
\item
The function \verb'lua_call' no longer has the parameter \verb'nparam'.
\item
The function \verb'lua_pop' is no longer available,
since it could lead to strange behavior.
In particular,
to access results returned from a Lua function,
the new macro \verb'lua_getresult' should be used.
\item
The old functions \verb'lua_storefield' and \verb'lua_storeindexed'
have been replaced by
\begin{verbatim}
int lua_storesubscript (void);
\end{verbatim}
with the parameters explicitly pushed on the stack.
\item
The functionality of the function \verb'lua_errorfunction' has been
replaced by the {\em fallback} mechanism (\see{error}).
\item
When calling a function from the Lua library,
parameters passed through the stack
must be pushed just before the corresponding call,
with no intermediate calls to Lua.
Special care should be taken with macros like
\verb'lua_getindexed' and \verb'lua_getfield'.
\end{itemize}

\newcommand{\indexentry}[2]{\item {#1} #2}
%\catcode`\_=12
\begin{theindex}
% $Id: manual.tex,v 1.36 2000/04/17 19:23:48 roberto Exp roberto $

\documentclass[11pt]{article}
\usepackage{fullpage,bnf}
\usepackage{graphicx}
%\usepackage{times}

\catcode`\_=12

\newcommand{\See}[1]{Section~\ref{#1}}
\newcommand{\see}[1]{(see \See{#1})}
\newcommand{\M}[1]{\rm\emph{#1}}
\newcommand{\T}[1]{{\tt #1}}
\newcommand{\Math}[1]{$#1$}
\newcommand{\nil}{{\bf nil}}
\def\tecgraf{{\sf TeC\kern-.21em\lower.7ex\hbox{Graf}}}

\newcommand{\Index}[1]{#1\index{#1}}
\newcommand{\IndexVerb}[1]{\T{#1}\index{#1}}
\newcommand{\IndexEmph}[1]{\emph{#1}\index{#1}}
\newcommand{\Def}[1]{\emph{#1}\index{#1}}
\newcommand{\Deffunc}[1]{\index{#1}}

\newcommand{\ff}{$\bullet$\ }

\newcommand{\Version}{4.0}

% LHF
\renewcommand{\ter}[1]{{\rm`{\tt#1}'}}
\newcommand{\NOTE}{\par\noindent\emph{NOTE}: }

\makeindex

\begin{document}

%{===============================================================
\thispagestyle{empty}
\pagestyle{empty}

{
\parindent=0pt
\vglue1.5in
{\LARGE\bf
The Programming Language Lua}
\hfill
\vskip4pt \hrule height 4pt width \hsize \vskip4pt
\hfill
Reference Manual for Lua version \Version
\\
\null
\hfill
Last revised on \today
\\
\vfill
\centering
\includegraphics[width=0.7\textwidth]{nolabel.ps}
\vfill
\vskip4pt \hrule height 2pt width \hsize
}

\newpage
\begin{quotation}
\parskip=10pt
\footnotesize
\null\vfill

\noindent
Copyright \copyright\ 1994--2000 TeCGraf, PUC-Rio.  All rights reserved.

\noindent
Permission is hereby granted, without written agreement and without license
or royalty fees, to use, copy, modify, and distribute this software and its
documentation for any purpose, including commercial applications, subject to
the following conditions:
\begin{itemize}
\item The above copyright notice and this permission notice shall appear in all
   copies or substantial portions of this software.

\item The origin of this software must not be misrepresented; you must not
   claim that you wrote the original software. If you use this software in a
   product, an acknowledgment in the product documentation would be greatly
   appreciated (but it is not required).

\item Altered source versions must be plainly marked as such, and must not be
   misrepresented as being the original software.
\end{itemize}
The authors specifically disclaim any warranties, including, but not limited
to, the implied warranties of merchantability and fitness for a particular
purpose.  The software provided hereunder is on an ``as is'' basis, and the
authors have no obligation to provide maintenance, support, updates,
enhancements, or modifications.  In no event shall TeCGraf, PUC-Rio, or the
authors be held liable to any party for direct, indirect, special,
incidental, or consequential damages arising out of the use of this software
and its documentation.

\noindent
The Lua language and this implementation have been entirely designed and
written by Waldemar Celes, Roberto Ierusalimschy and Luiz Henrique de
Figueiredo at TeCGraf, PUC-Rio.

\noindent
This implementation contains no third-party code.

\noindent
Copies of this manual can be obtained at
\verb|http://www.tecgraf.puc-rio.br/lua/|.
\end{quotation}
%}===============================================================
\newpage

\title{Reference Manual of the Programming Language Lua \Version}

\author{%
Roberto Ierusalimschy\quad
Luiz Henrique de Figueiredo\quad
Waldemar Celes
\vspace{1.0ex}\\
\smallskip
\small\tt lua@tecgraf.puc-rio.br
\vspace{2.0ex}\\
%MCC 08/95 ---
\tecgraf\ --- Computer Science Department --- PUC-Rio
}

\date{{\small \tt\$Date: 2000/04/17 19:23:48 $ $}}

\maketitle

\thispagestyle{empty}
\pagestyle{empty}

\begin{abstract}
\noindent
Lua is a powerful, light-weight programming language
designed for extending applications.
Lua is also frequently used as a general-purpose, stand-alone language.
Lua combines simple procedural syntax
(similar to Pascal)
with
powerful data description constructs
based on associative arrays and extensible semantics.
Lua is
dynamically typed,
interpreted from bytecodes,
and has automatic memory management with garbage collection,
making it ideal for
configuration,
scripting,
and
rapid prototyping.

This document describes version \Version\ of the Lua programming language
and the API that allows interaction between Lua programs and their
host C programs.
\end{abstract}

\def\abstractname{Resumo}
\begin{abstract}
\noindent
Lua \'e uma linguagem de programa\c{c}\~ao
poderosa e leve,
projetada para extender aplica\c{c}\~oes.
Lua tamb\'em \'e frequentemente usada como uma linguagem de prop\'osito geral.
Lua combina programa\c{c}\~ao procedural
(com sintaxe semelhante \`a de Pascal)
com
poderosas constru\c{c}\~oes para descri\c{c}\~ao de dados,
baseadas em tabelas associativas e sem\^antica extens\'\i vel.
Lua \'e
tipada dinamicamente,
interpretada a partir de \emph{bytecodes},
e tem gerenciamento autom\'atico de mem\'oria com coleta de lixo.
Essas caracter\'{\i}sticas fazem de Lua uma linguagem ideal para
configura\c{c}\~ao,
automa\c{c}\~ao (\emph{scripting})
e prototipagem r\'apida.

Este documento descreve a vers\~ao \Version\ da linguagem de
programa\c{c}\~ao Lua e a Interface de Programa\c{c}\~ao (API) que permite
a intera\c{c}\~ao entre programas Lua e programas C hospedeiros.
\end{abstract}

\newpage
\null
\newpage
\tableofcontents

\newpage
\setcounter{page}{1}
\pagestyle{plain}


\section{Introduction}

Lua is an extension programming language designed to support
general procedural programming with data description
facilities.
Lua is intended to be used as a powerful, light-weight
configuration language for any program that needs one.

Lua is implemented as a library, written in C.
Being an extension language, Lua has no notion of a ``main'' program:
it only works \emph{embedded} in a host client,
called the \emph{embedding} program.
This host program can invoke functions to execute a piece of
code in Lua, can write and read Lua variables,
and can register C~functions to be called by Lua code.
Through the use of C~functions, Lua can be augmented to cope with
a wide range of different domains,
thus creating customized programming languages sharing a syntactical framework.

Lua is free-distribution software,
and provided as usual with no guarantees,
as stated in the copyright notice.
The implementation described in this manual is available
at the following URL's:
\begin{verbatim}
   http://www.tecgraf.puc-rio.br/lua/
   ftp://ftp.tecgraf.puc-rio.br/pub/lua/
\end{verbatim}

Like any other reference manual,
this document is dry in places.
For a discussion of the decisions behind the design of Lua,
see the papers below,
which are available at the web site above.
\begin{itemize}
\item
R.~Ierusalimschy, L.~H.~de Figueiredo, and W.~Celes.
Lua---an extensible extension language.
\emph{Software: Practice \& Experience} {\bf 26} \#6 (1996) 635--652.
\item
L.~H.~de Figueiredo, R.~Ierusalimschy, and W.~Celes.
The design and implementation of a language for extending applications.
\emph{Proceedings of XXI Brazilian Seminar on Software and Hardware} (1994) 273--283.
\item
L.~H.~de Figueiredo, R.~Ierusalimschy, and W.~Celes.
Lua: an extensible embedded language.
\emph{Dr. Dobb's Journal} {\bf  21} \#12 (Dec 1996) 26--33.
\end{itemize}

\section{Environment and Chunks}

All statements in Lua are executed in a \Def{global environment}.
This environment, which keeps all global variables,
is initialized with a call from the embedding program to
\verb|lua_newstate| and
persists until a call to \verb|lua_close|,
or the end of the embedding program.
Optionally, a user can create multiple independent global
environments, and freely switch between them \see{mangstate}.

The global environment can be manipulated by Lua code or
by the embedding program,
which can read and write global variables
using API functions from the library that implements Lua.

\Index{Global variables} do not need declaration.
Any variable is assumed to be global unless explicitly declared local
\see{localvar}.
Before the first assignment, the value of a global variable is \nil;
this default can be changed \see{tag-method}.

The unit of execution of Lua is called a \Def{chunk}.
A chunk is simply a sequence of statements:
\begin{Produc}
\produc{chunk}{\rep{stat} \opt{ret}}
\end{Produc}%
Statements are described in \See{stats}.
(The notation above is the usual extended BNF,
in which
\rep{\emph{a}} means 0 or more \emph{a}'s,
\opt{\emph{a}} means an optional \emph{a}, and
\oneormore{\emph{a}} means one or more \emph{a}'s.)

A chunk may be in a file or in a string inside the host program.
A chunk may optionally end with a \verb|return| statement \see{return}.
When a chunk is executed, first all its code is pre-compiled,
and then the statements are executed in sequential order.
All modifications a chunk effects on the global environment persist
after the chunk ends.

Chunks may also be pre-compiled into binary form;
see program \IndexVerb{luac} for details.
Text files with chunks and their binary pre-compiled forms
are interchangeable.
Lua automatically detects the file type and acts accordingly.
\index{pre-compilation}

\section{\Index{Types and Tags}} \label{TypesSec}

Lua is a \emph{dynamically typed language}.
This means that
variables do not have types; only values do.
Therefore, there are no type definitions in the language.
All values carry their own type.
Besides a type, all values also have a \IndexEmph{tag}.

There are six \Index{basic types} in Lua: \Def{nil}, \Def{number},
\Def{string}, \Def{function}, \Def{userdata}, and \Def{table}.
\emph{Nil} is the type of the value \nil,
whose main property is to be different from any other value.
\emph{Number} represents real (double-precision floating-point) numbers,
while \emph{string} has the usual meaning.
Lua is \Index{eight-bit clean},
and so strings may contain any 8-bit character,
\emph{including} embedded zeros (\verb|'\0'|) \see{lexical}.
The \verb|type| function returns a string describing the type
of a given value \see{pdf-type}.

Functions are considered \emph{first-class values} in Lua.
This means that functions can be stored in variables,
passed as arguments to other functions, and returned as results.
Lua can call (and manipulate) functions written in Lua and
functions written in C.
The kinds of functions can be distinguished by their tags:
all Lua functions have the same tag,
and all C~functions have the same tag,
which is different from the tag of Lua functions.
The \verb|tag| function returns the tag
of a given value \see{pdf-tag}.

The type \emph{userdata} is provided to allow
arbitrary \Index{C pointers} to be stored in Lua variables.
It corresponds to a \verb|void*| and has no pre-defined operations in Lua,
besides assignment and equality test.
However, by using \emph{tag methods},
the programmer can define operations for \emph{userdata} values
\see{tag-method}.

The type \emph{table} implements \Index{associative arrays},
that is, \Index{arrays} that can be indexed not only with numbers,
but with any value (except \nil).
Therefore, this type may be used not only to represent ordinary arrays,
but also symbol tables, sets, records, etc.
Tables are the main data structuring mechanism in Lua.
To represent \Index{records}, Lua uses the field name as an index.
The language supports this representation by
providing \verb|a.name| as syntactic sugar for \verb|a["name"]|.
Tables may also carry \emph{methods}:
Because functions are first class values,
table fields may contain functions.
The form \verb|t:f(x)| is syntactic sugar for \verb|t.f(t,x)|,
which calls the method \verb|f| from the table \verb|t| passing
itself as the first parameter \see{func-def}.

Note that tables are \emph{objects}, and not values.
Variables cannot contain tables, only \emph{references} to them.
Assignment, parameter passing, and returns always manipulate references
to tables, and do not imply any kind of copy.
Moreover, tables must be explicitly created before used
\see{tableconstructor}.

Tags are mainly used to select \emph{tag methods} when
some events occur.
Tag methods are the main mechanism for extending the
semantics of Lua \see{tag-method}.
Each of the types \M{nil}, \M{number}, and \M{string} has a different tag.
All values of each of these types have the same pre-defined tag.
Values of type \M{function} can have two different tags,
depending on whether they are Lua functions or C~functions.
Finally,
values of type \M{userdata} and \M{table} have
variable tags, assigned by the program \see{tag-method}.
Tags are created with the function \verb|newtag|,
and the function \verb|tag| returns the tag of a given value.
To change the tag of a given table,
there is the function \verb|settag| \see{pdf-newtag}.


\section{The Language}

This section describes the lexis, the syntax, and the semantics of Lua.


\subsection{Lexical Conventions} \label{lexical}

\IndexEmph{Identifiers} in Lua can be any string of letters,
digits, and underscores,
not beginning with a digit.
This coincides with the definition of identifiers in most languages,
except that
the definition of letter depends on the current locale:
Any character considered alphabetic by the current locale
can be used in an identifier.
The following words are \emph{reserved}, and cannot be used as identifiers:
\index{reserved words}
\begin{verbatim}
   and       break     do        else
   elseif    end       for       function
   if        local     nil       not
   or        repeat    return    then
   until     while
\end{verbatim}
Lua is a case-sensitive language:
\T{and} is a reserved word, but \T{And} and \T{\'and}
(if the locale permits) are two different, valid identifiers.
As a convention, identifiers starting with underscore followed by
uppercase letters (such as \verb|_INPUT|)
are reserved for internal variables.

The following strings denote other \Index{tokens}:
\begin{verbatim}
   ~=  <=  >=  <   >   ==  =   +   -   *   /   %
   (   )   {   }   [   ]   ;   ,   .   ..  ...
\end{verbatim}

\IndexEmph{Literal strings} can be delimited by matching single or double quotes,
and can contain the C-like escape sequences
\verb|'\a'| (bell),
\verb|'\b'| (backspace),
\verb|'\f'| (form feed),
\verb|'\n'| (newline),
\verb|'\r'| (carriage return),
\verb|'\t'| (horizontal tab),
\verb|'\v'| (vertical tab),
\verb|'\\'|, (backslash),
\verb|'\"'|, (double quote),
\verb|'\''| (single quote),
and \verb|'\|\emph{newline}\verb|'| (that is, a backslash followed by a real newline,
which  results in a newline in the string).
A character in a string may also be specified by its numerical value,
through the escape sequence \verb|'\ddd'|,
where \verb|ddd| is a sequence of up to three \emph{decimal} digits.
Strings in Lua may contain any 8-bit value, including embedded zeros,
which can be specified as \verb|'\000'|.

Literal strings can also be delimited by matching \verb|[[| \dots\ \verb|]]|.
Literals in this bracketed form may run for several lines,
may contain nested \verb|[[ ... ]]| pairs,
and do not interpret escape sequences.
This form is specially convenient for
writing strings that contain program pieces or
other quoted strings.
As an example, in a system using ASCII,
the following three literals are equivalent:
\begin{verbatim}
1) "alo\n123\""
2) '\97lo\10\04923"'
3) [[alo
   123"]]
\end{verbatim}


\Index{Comments} start anywhere outside a string with a
double hyphen (\verb|--|) and run until the end of the line.
Moreover,
the first line of a chunk is skipped if it starts with \verb|#|.
This facility allows the use of Lua as a script interpreter
in Unix systems \see{lua-sa}.

\Index{Numerical constants} may be written with an optional decimal part,
and an optional decimal exponent.
Examples of valid numerical constants are
\begin{verbatim}
   3     3.0     3.1416  314.16e-2   0.31416E1
\end{verbatim}

\subsection{The \Index{Pre-processor}} \label{pre-processor}

All lines that start with a \verb|$| sign are handled by a pre-processor.
The following directives are understood by the pre-processor:
\begin{description}
\item[\T{\$debug}] --- turn on debugging facilities \see{pragma}.
\item[\T{\$nodebug}] --- turn off debugging facilities \see{pragma}.
\item[\T{\$if \M{cond}}] --- start a conditional part.
If \M{cond} is false, then this part is skipped by the lexical analyzer.
\item[\T{\$ifnot \M{cond}}] --- start a conditional part.
If \M{cond} is true, then this part is skipped by the lexical analyzer.
\item[\T{\$end}] --- end a conditional part.
\item[\T{\$else}] --- start an ``else'' conditional part,
flipping the ``skip'' status.
\item[\T{\$endinput}] --- end the lexical parse of the chunk.
For all purposes,
it is as if the chunk physically ended at this point.
\end{description}

Directives may be freely nested.
In particular, a \verb|$endinput| may occur inside a \verb|$if|;
in that case, even the matching \verb|$end| is not parsed.

A \M{cond} part may be
\begin{description}
\item[\T{nil}] --- always false.
\item[\T{1}] --- always true.
\item[\T{\M{name}}] --- true if the value of the
global variable \M{name} is different from \nil.
Note that \M{name} is evaluated \emph{before} the chunk starts its execution.
Therefore, actions in a chunk do not affect its own conditional directives.
\end{description}

\subsection{\Index{Coercion}} \label{coercion}

Lua provides some automatic conversions between values at run time.
Any arithmetic operation applied to a string tries to convert
that string to a number, following the usual rules.
Conversely, whenever a number is used when a string is expected,
that number is converted to a string, in a reasonable format.
The format is chosen so that
a conversion from number to string then back to number
reproduces the original number \emph{exactly}.
Thus,
the conversion does not necessarily produces nice-looking text for some numbers.
For complete control on how numbers are converted to strings,
use the \verb|format| function \see{format}.


\subsection{\Index{Adjustment}} \label{adjust}

Functions in Lua can return many values.
Because there are no type declarations,
when a function is called
the system does not know how many values the function will return,
or how many parameters it needs.
Therefore, sometimes, a list of values must be \emph{adjusted}, at run time,
to a given length.
If there are more values than are needed,
then the excess values are thrown away.
If there are less values than are needed,
then the list is extended with as many  \nil's as needed.
This adjustment occurs in multiple assignments \see{assignment}
and function calls \see{functioncall}.


\subsection{Statements}\label{stats}

Lua supports an almost conventional set of \Index{statements},
similar to those in Pascal or C.
The conventional commands include
assignment, control structures, and procedure calls.
Non-conventional commands include table constructors
\see{tableconstructor}
and local variable declarations \see{localvar}.

\subsubsection{Blocks}
A \Index{block} is a list of statements, which are executed sequentially.
A statement may be have an optional \Index{label},
which is syntactically an identifier,
and can be optionally followed by a semicolon:
\begin{Produc}
\produc{block}{\opt{label} \rep{stat \opt{\ter{;}}}}
\produc{label}{\ter{$\vert$} name \ter{$\vert$}}
\end{Produc}%
\NOTE
For syntactic reasons, the \rwd{return} and
\rwd{break} statements can only be written
as the last statement of a block.

A block may be explicitly delimited:
\begin{Produc}
\produc{stat}{\rwd{do} block \rwd{end}}
\end{Produc}%
This is useful to control the scope of local variables \see{localvar},
and to add a \rwd{return} or \rwd{break} statement in the middle
of another block:
\begin{verbatim}
  do return end        -- return is the last statement in this block
\end{verbatim}

\subsubsection{\Index{Assignment}} \label{assignment}
The language allows \Index{multiple assignment}.
Therefore, the syntax for assignment
defines a list of variables on the left side
and a list of expressions on the right side.
Both lists have their elements separated by commas:
\begin{Produc}
\produc{stat}{varlist1 \ter{=} explist1}
\produc{varlist1}{var \rep{\ter{,} var}}
\end{Produc}%
This statement first evaluates all values on the right side
and eventual indices on the left side,
and then makes the assignments.
So
\begin{verbatim}
   i = 3
   i, a[i] = 4, 20
\end{verbatim}
sets \verb|a[3]| to 20, but does not affect \verb|a[4]|.

Multiple assignment can be used to exchange two values, as in
\begin{verbatim}
   x, y = y, x
\end{verbatim}

The two lists in a multiple assignment may have different lengths.
Before the assignment, the list of values is adjusted to
the length of the list of variables \see{adjust}.

A single name can denote a global variable, a local variable,
or a formal parameter:
\begin{Produc}
\produc{var}{name}
\end{Produc}%
Square brackets are used to index a table:
\begin{Produc}
\produc{var}{simpleexp \ter{[} exp1 \ter{]}}
\end{Produc}%
The \M{simpleexp} should result in a table value,
from where the field indexed by the expression \M{exp1}
value gets the assigned value.

The syntax \verb|var.NAME| is just syntactic sugar for
\verb|var["NAME"]|:
\begin{Produc}
\produc{var}{simpleexp \ter{.} name}
\end{Produc}%

The meaning of assignments and evaluations of global variables and
indexed variables can be changed by tag methods \see{tag-method}.
Actually,
an assignment \verb|x = val|, where \verb|x| is a global variable,
is equivalent to a call \verb|setglobal("x", val)|;
an assignment \verb|t[i] = val| is equivalent to
\verb|settable_event(t,i,val)|.
See \See{tag-method} for a complete description of these functions.
(The function \verb|setglobal| is pre-defined in Lua.
The function \T{settable\_event} is used only for explanatory purposes.)

\subsubsection{Control Structures}
The control structures 
\index{while-do}\index{repeat-until}\index{if-then-else}%
\T{if}, \T{while}, and \T{repeat} have the usual meaning and
familiar syntax:
\begin{Produc}
\produc{stat}{\rwd{while} exp1 \rwd{do} block \rwd{end}}
\produc{stat}{\rwd{repeat} block \rwd{until} exp1}
\produc{stat}{\rwd{if} exp1 \rwd{then} block
  \rep{\rwd{elseif} exp1 \rwd{then} block}
   \opt{\rwd{else} block} \rwd{end}}
\end{Produc}%
The \Index{condition expression} \M{exp1} of a control structure may return any value.
All values different from \nil\ are considered true;
only \nil\ is considered false.

\index{return}
The \rwd{return} statement is used to return values from a function or from a chunk.
\label{return}
Because functions or chunks may return more than one value,
the syntax for a \Index{return statement} is
\begin{Produc}
\produc{stat}{\rwd{return} \opt{explist1}}
\end{Produc}%

\index{break}
The \rwd{break} statement can be used to terminate the execution of a block,
skipping to the next statement after the block:
\begin{Produc}
\produc{stat}{\rwd{break} \opt{name}}
\end{Produc}%
A \rwd{break} without a label ends the innermost enclosing loop
(while, repeat, or for).
A \rwd{break} with a label breaks the innermost enclosing
statement with that label.
Thus,
labels do not have to be unique.

For syntactic reasons, \rwd{return} and \rwd{break}
statements can only be written as the last statement of a block.

\subsubsection{For Statement} \label{for}\index{for}

The \rwd{for} statement has the following syntax:
\begin{Produc}
\produc{stat}{\rwd{for} name \ter{=} exp1 \ter{,} exp1 \opt{\ter{,} exp1}
                    \rwd{do} block \rwd{end}}
\end{Produc}%
A \rwd{for} statement like
\begin{verbatim}
   for var=e1,e2,e3 do block end
\end{verbatim}
is equivalent to the following code:
\begin{verbatim}
   do
     local var, _limit, _step = tonumber(e1), tonumber(e2), tonumber(e3)
     if not (var and _limit and _step) then error() end
     while (_step>0 and var<=_limit) or (_step<=0 and var>=_limit) do
       block
       var = var+_step
     end
   end
\end{verbatim}
Notice the following:
\begin{itemize}\itemsep=0pt
\item \verb|_limit| and \verb|_step| are invisible variables.
The names are here for explanatory purposes only.
\item The behavior is \emph{undefined} if you assign to \verb|var| inside
the block.
\item If the third expression (the step) is absent, then a step of 1 is used.
\item Both the limit and the step are evaluated only once,
before the loop starts.
\item The variable \verb|var| is local to the statement;
you cannot use its value after the \rwd{for} ends.
\item You can use \rwd{break} to exit a \rwd{for}.
If you need the value of the index,
then assign it to another variable before breaking.
\end{itemize}

\subsubsection{Function Calls as Statements} \label{funcstat}
Because of possible side-effects,
function calls can be executed as statements:
\begin{Produc}
\produc{stat}{functioncall}
\end{Produc}%
In this case, all returned values are thrown away.
Function calls are explained in \See{functioncall}.

\subsubsection{Local Declarations} \label{localvar}
\Index{Local variables} may be declared anywhere inside a block.
The declaration may include an initial assignment:
\begin{Produc}
\produc{stat}{\rwd{local} declist \opt{init}}
\produc{declist}{name \rep{\ter{,} name}}
\produc{init}{\ter{=} explist1}
\end{Produc}%
If present, an initial assignment has the same semantics
of a multiple assignment.
Otherwise, all variables are initialized with \nil.

The scope of local variables begins \emph{after}
the declaration and lasts until the end of the block.
Thus, the code
\verb|local print=print|
creates a local variable called \verb|print| whose
initial value is that of the \emph{global} variable of the same name.


\subsection{\Index{Expressions}}

\subsubsection{\Index{Basic Expressions}}
The basic expressions in Lua are
\begin{Produc}
\produc{exp}{\ter{(} exp \ter{)}}
\produc{exp}{\rwd{nil}}
\produc{exp}{number}
\produc{exp}{literal}
\produc{exp}{function}
\produc{exp}{simpleexp}
\end{Produc}%
\begin{Produc}
\produc{simpleexp}{var}
\produc{simpleexp}{upvalue}
\produc{simpleexp}{functioncall}
\produc{simpleexp}{tableconstructor}
\end{Produc}%

Numbers (numerical constants) and
literal strings are explained in \See{lexical};
variables are explained in \See{assignment};
upvalues are explained in \See{upvalue};
function definitions (\M{function}) are explained in \See{func-def};
function calls are explained in \See{functioncall}.
Table constructors are explained in \See{tableconstructor}.

An access to a global variable \verb|x| is equivalent to a
call \verb|getglobal("x")|;
an access to an indexed variable \verb|t[i]| is equivalent to
a call \verb|gettable_event(t,i)|.
See \See{tag-method} for a description of these functions.
(Function \verb|getglobal| is pre-defined in Lua.
Function \T{gettable\_event} is used only for explanatory purposes.)

The non-terminal \M{exp1} is used to indicate that the values
returned by an expression must be adjusted to one single value:
\begin{Produc}
\produc{exp1}{exp}
\end{Produc}%

\subsubsection{Arithmetic Operators}
Lua supports the usual \Index{arithmetic operators}:
the binary \verb|+| (addition),
\verb|-| (subtraction), \verb|*| (multiplication),
\verb|/| (division) and \verb|^| (exponentiation),
and unary \verb|-| (negation).
If the operands are numbers, or strings that can be converted to
numbers (according to the rules given in \See{coercion}),
then all operations except exponentiation have the usual meaning.
Otherwise, an appropriate tag method is called \see{tag-method}.
An exponentiation always calls a tag method.
The standard mathematical library redefines this method for numbers,
giving the expected meaning to \Index{exponentiation}
\see{mathlib}.

\subsubsection{Relational Operators}
Lua provides the following \Index{relational operators}:
\begin{verbatim}
   ==  ~=  <   >   <=  >=
\end{verbatim}
All these return \nil\ as false and a value different from \nil\ as true.

Equality first compares the tags of its operands.
If they are different, then the result is \nil.
Otherwise, their values are compared.
Numbers and strings are compared in the usual way.
Tables, userdata, and functions are compared by reference,
that is, two tables are considered equal only if they are the \emph{same} table.
The operator \verb|~=| is exactly the negation of equality (\verb|==|).

\NOTE
The conversion rules of \See{coercion}
\emph{do not} apply to equality comparisons.
Thus, \verb|"0"==0| evaluates to \emph{false},
and \verb|t[0]| and \verb|t["0"]| denote different
entries in a table.

The order operators work as follows.
If both arguments are numbers, then they are compared as such.
Otherwise, if both arguments are strings,
then their values are compared using lexicographical order.
Otherwise, the ``lt'' tag method is called \see{tag-method}.

\subsubsection{Logical Operators}
The \Index{logical operators} are
\index{and}\index{or}\index{not}
\begin{verbatim}
   and   or   not
\end{verbatim}
Like control structures, all logical operators
consider \nil\ as false and anything else as true.
The conjunction operator \verb|and| returns \nil\ if its first argument is \nil;
otherwise, it returns its second argument.
The disjunction operator \verb|or| returns its first argument
if it is different from \nil;
otherwise, it returns its second argument.
Both \verb|and| and \verb|or| use \Index{short-cut evaluation},
that is,
the second operand is evaluated only when necessary.

There are two useful Lua idioms with logical operators.
The first idiom is \verb|x = x or v|,
which is equivalent to
\begin{verbatim}
      if x == nil then x = v end
\end{verbatim}
i.e., it sets \verb|x| to a default value \verb|v| when
\verb|x| is not set.
The other idiom is \verb|x = a and b or c|,
which should be read as \verb|x = a and (b or c)|,
is equivalent to
\begin{verbatim}
   if a then x = b else x = c end
\end{verbatim}
provided that \verb|b| is not \nil.

\subsubsection{Concatenation}
The string \Index{concatenation} operator in Lua is
denoted by ``\IndexVerb{..}''.
If both operands are strings or numbers, they are converted to
strings according to the rules in \See{coercion}.
Otherwise, the ``concat'' tag method is called \see{tag-method}.

\subsubsection{Precedence}
\Index{Operator precedence} follows the table below,
from the lower to the higher priority:
\begin{verbatim}
   and   or
   <   >   <=  >=  ~=  ==
   ..
   +   -
   *   /
   not  - (unary)
   ^
\end{verbatim}
All binary operators are left associative,
except for \verb|^| (exponentiation),
which is right associative.
\NOTE
The pre-compiler may rearrange the order of evaluation of
associative operators (such as~\verb|..| or~\verb|+|),
as long as these optimizations do not change normal results.
However, these optimizations may change some results
if you define non-associative
tag methods for these operators.

\subsubsection{Table Constructors} \label{tableconstructor}
Table \Index{constructors} are expressions that create tables;
every time a constructor is evaluated, a new table is created.
Constructors can be used to create empty tables,
or to create a table and initialize some fields.
The general syntax for constructors is
\begin{Produc}
\produc{tableconstructor}{\ter{\{} fieldlist \ter{\}}}
\produc{fieldlist}{lfieldlist \Or ffieldlist \Or lfieldlist \ter{;} ffieldlist
	\Or ffieldlist \ter{;} lfieldlist}
\produc{lfieldlist}{\opt{lfieldlist1}}
\produc{ffieldlist}{\opt{ffieldlist1}}
\end{Produc}%

The form \emph{lfieldlist1} is used to initialize lists:
\begin{Produc}
\produc{lfieldlist1}{exp \rep{\ter{,} exp} \opt{\ter{,}}}
\end{Produc}%
The expressions in the list are assigned to consecutive numerical indices,
starting with 1.
For example,
\begin{verbatim}
   a = {"v1", "v2", 34}
\end{verbatim}
is equivalent to
\begin{verbatim}
  do
    local temp = {}
    temp[1] = "v1"
    temp[2] = "v2"
    temp[3] = 34
    a = temp
  end
\end{verbatim}

The form \emph{ffieldlist1} initializes other fields in a table:
\begin{Produc}
\produc{ffieldlist1}{ffield \rep{\ter{,} ffield} \opt{\ter{,}}}
\produc{ffield}{\ter{[} exp \ter{]} \ter{=} exp \Or name \ter{=} exp}
\end{Produc}%
For example,
\begin{verbatim}
   a = {[f(k)] = g(y), x = 1, y = 3, [0] = b+c}
\end{verbatim}
is equivalent to
\begin{verbatim}
  do
    local temp = {}
    temp[f(k)] = g(y)
    temp.x = 1    -- or temp["x"] = 1
    temp.y = 3    -- or temp["y"] = 3
    temp[0] = b+c
    a = temp
  end
\end{verbatim}
An expression like \verb|{x = 1, y = 4}| is
in fact syntactic sugar for \verb|{["x"] = 1, ["y"] = 4}|.

Both forms may have an optional trailing comma,
and can be used in the same constructor separated by
a semi-collon.
For example, all forms below are correct.
\begin{verbatim}
   x = {;}
   x = {"a", "b",}
   x = {type="list"; "a", "b"}
   x = {f(0), f(1), f(2),; n=3,}
\end{verbatim}

\subsubsection{Function Calls}  \label{functioncall}
A \Index{function call} has the following syntax:
\begin{Produc}
\produc{functioncall}{simpleexp args}
\end{Produc}%
First, \M{simpleexp} is evaluated.
If its value has type \emph{function},
then this function is called,
with the given arguments.
Otherwise, the ``function'' tag method is called,
having as first parameter the value of \M{simpleexp},
and then the original call arguments.

The form
\begin{Produc}
\produc{functioncall}{simpleexp \ter{:} name args}
\end{Produc}%
can be used to call ``methods''.
A call \verb|simpleexp:name(...)|
is syntactic sugar for
\begin{verbatim}
  simpleexp.name(simpleexp, ...)
\end{verbatim}
except that \verb|simpleexp| is evaluated only once.

Arguments have the following syntax:
\begin{Produc}
\produc{args}{\ter{(} \opt{explist1} \ter{)}}
\produc{args}{tableconstructor}
\produc{args}{\ter{literal}}
\produc{explist1}{\rep{exp1 \ter{,}} exp}
\end{Produc}%
All argument expressions are evaluated before the call.
A call of the form \verb|f{...}| is syntactic sugar for
\verb|f({...})|, that is,
the parameter list is a single new table.
A call of the form \verb|f'...'|
(or \verb|f"..."| or \verb|f[[...]]|) is syntactic sugar for
\verb|f('...')|, that is,
the parameter list is a single literal string.

Because a function can return any number of results
\see{return},
the number of results must be adjusted before used.
If the function is called as a statement \see{funcstat},
then its return list is adjusted to~0,
thus discarding all returned values.
If the function is called in a place that needs a single value
(syntactically denoted by the non-terminal \M{exp1}),
then its return list is adjusted to~1,
thus discarding all returned values but the first one.
If the function is called in a place that can hold many values
(syntactically denoted by the non-terminal \M{exp}),
then no adjustment is made.
The only places that can hold many values
is the last (or the only) expression in an assignment,
in an argument list, or in a return statement.
Here are some examples.
\begin{verbatim}
   f();               -- adjusted to 0
   g(f(), x);         -- f() is adjusted to 1 result
   g(x, f());         -- g gets x plus all values returned by f()
   a,b,c = f(), x;    -- f() is adjusted to 1 result (and c gets nil)
   a,b,c = x, f();    -- f() is adjusted to 2
   a,b,c = f();       -- f() is adjusted to 3
   return f();        -- returns all values returned by f()
   return x,y,f();    -- returns a, b, and all values returned by f()
\end{verbatim}

\subsubsection{\Index{Function Definitions}} \label{func-def}

The syntax for function definition is
\begin{Produc}
\produc{function}{\rwd{function} \ter{(} \opt{parlist1} \ter{)}
  block \rwd{end}}
\produc{stat}{\rwd{function} funcname \ter{(} \opt{parlist1} \ter{)}
  block \rwd{end}}
\produc{funcname}{name \Or name \ter{.} name \Or name \ter{:} name}
\end{Produc}%
The statement
\begin{verbatim}
      function f ()
        ...
      end
\end{verbatim}
is just syntactic sugar for
\begin{verbatim}
      f = function ()
            ...
          end
\end{verbatim}
and the statement
\begin{verbatim}
      function o.f ()
        ...
      end
\end{verbatim}
is syntactic sugar for
\begin{verbatim}
      o.f = function ()
              ...
            end
\end{verbatim}

A function definition is an executable expression,
whose value has type \emph{function}.
When Lua pre-compiles a chunk,
all its function bodies are pre-compiled, too.
Then, whenever Lua executes the function definition,
its upvalues are fixed \see{upvalue},
and the function is \emph{instantiated} (or \emph{closed}).
This function instance (or \emph{closure})
is the final value of the expression.
Different instances of the same function
may have different upvalues.

Parameters act as local variables,
initialized with the argument values:
\begin{Produc}
\produc{parlist1}{\ter{\ldots}}
\produc{parlist1}{name \rep{\ter{,} name} \opt{\ter{,} \ter{\ldots}}}
\end{Produc}%
\label{vararg}
When a function is called,
the list of \Index{arguments} is adjusted to
the length of the list of parameters \see{adjust},
unless the function is a \Def{vararg} function,
which is
indicated by the dots (\ldots) at the end of its parameter list.
A vararg function does not adjust its argument list;
instead, it collects all extra arguments into an implicit parameter,
called \IndexVerb{arg}.
The value of \verb|arg| is a table,
with a field~\verb|n| whose value is the number of extra arguments,
and the extra arguments at positions 1,~2,~\ldots,\M{n}.

As an example, consider the following definitions:
\begin{verbatim}
   function f(a, b) end
   function g(a, b, ...) end
   function r() return 1,2,3 end
\end{verbatim}
Then, we have the following mapping from arguments to parameters:
\begin{verbatim}
   CALL            PARAMETERS

   f(3)             a=3, b=nil
   f(3, 4)          a=3, b=4
   f(3, 4, 5)       a=3, b=4
   f(r(), 10)       a=1, b=10
   f(r())           a=1, b=2

   g(3)             a=3, b=nil, arg={n=0}
   g(3, 4)          a=3, b=4, arg={n=0}
   g(3, 4, 5, 8)    a=3, b=4, arg={5, 8; n=2}
   g(5, r())        a=5, b=1, arg={2, 3; n=2}
\end{verbatim}

Results are returned using the \verb|return| statement \see{return}.
If control reaches the end of a function
without encountering a \rwd{return} statement,
then the function returns with no results.

The syntax
\begin{Produc}
\produc{funcname}{name \ter{:} name}
\end{Produc}%
is used for defining \Index{methods},
that is, functions that have an implicit extra parameter \IndexVerb{self}:
Thus, the statement
\begin{verbatim}
      function v:f (...)
        ...
      end
\end{verbatim}
is equivalent to
\begin{verbatim}
      v.f = function (self, ...)
        ...
      end
\end{verbatim}
that is, the function gets an extra formal parameter called \verb|self|.
Note that the variable \verb|v| must have been
previously initialized with a table value.


\subsection{Visibility and Upvalues} \label{upvalue}
\index{Visibility} \index{Upvalues}

A function body may refer to its own local variables
(which include its parameters) and to global variables,
as long as they are not \emph{shadowed} by local
variables from enclosing functions.
A function \emph{cannot} access a local
variable from an enclosing function,
since such variables may no longer exist when the function is called.
However, a function may access the \emph{value} of a local variable
from an enclosing function, using \emph{upvalues},
whose syntax is
\begin{Produc}
\produc{upvalue}{\ter{\%} name}
\end{Produc}%
An upvalue is somewhat similar to a variable expression,
but whose value is \emph{frozen} when the function wherein it
appears is instantiated.
The name used in an upvalue may be the name of any variable visible
at the point where the function is defined,
that is
global variables and local variables from the immediately enclosing function.

Here are some examples:
\begin{verbatim}
      a,b,c = 1,2,3   -- global variables
      local d
      function f (x)
        local b       -- x and b are local to f; b shadows the global b
        local g = function (a)
          local y     -- a and y are local to g
          p = a       -- OK, access local 'a'
          p = c       -- OK, access global 'c'
          p = b       -- ERROR: cannot access a variable in outer scope
          p = %b      -- OK, access frozen value of 'b' (local to 'f')
          p = %c      -- OK, access frozen value of global 'c'
          p = %y      -- ERROR: 'y' is not visible where 'g' is defined
          p = %d      -- ERROR: 'd' is not visible where 'g' is defined
        end           -- g
      end             -- f
\end{verbatim}


\subsection{Error Handling} \label{error}

Because Lua is an extension language,
all Lua actions start from C~code in the host program
calling a function from the Lua library.
Whenever an error occurs during Lua compilation or execution,
the function \verb|_ERRORMESSAGE| is called \Deffunc{_ERRORMESSAGE}
(provided it is different from \nil),
and then the corresponding function from the library
(\verb|lua_dofile|, \verb|lua_dostring|,
\verb|lua_dobuffer|, or \verb|lua_callfunction|)
is terminated, returning an error condition.

The only argument to \verb|_ERRORMESSAGE| is a string
describing the error.
The default definition for
this function calls \verb|_ALERT|, \Deffunc{_ALERT}
which prints the message to \verb|stderr| \see{alert}.
The standard I/O library redefines \verb|_ERRORMESSAGE|,
and uses the debug facilities \see{debugI}
to print some extra information,
such as a call stack traceback.

To provide more information about errors,
Lua programs should include the compilation pragma \verb|$debug|,
\index{debug pragma}\label{pragma}
or be loaded from the host after calling \verb|lua_setdebug(1)|
\see{debugI}.
When an error occurs in a chunk compiled with this option,
the I/O error-message routine is able to print the number of the
lines where the calls (and the error) were made.

Lua code can explicitly generate an error by calling the built-in
function \verb|error| \see{pdf-error}.
Lua code can ``catch'' an error using the built-in function
\verb|call| \see{pdf-call}.


\subsection{Tag Methods} \label{tag-method}

Lua provides a powerful mechanism to extend its semantics,
called \Def{tag methods}.
A tag method is a programmer-defined function
that is called at specific key points during the evaluation of a program,
allowing the programmer to change the standard Lua behavior at these points.
Each of these points is called an \Def{event}.

The tag method called for any specific event is selected
according to the tag of the values involved
in the event \see{TypesSec}.
The function \IndexVerb{settagmethod} changes the tag method
associated with a given pair \M{(tag, event)}.
Its first parameter is the tag, the second parameter is the event name
(a string; see below),
and the third parameter is the new method (a function),
or \nil\ to restore the default behavior for the pair.
The \verb|settagmethod| function returns the previous tag method for that pair.
Another function, \IndexVerb{gettagmethod},
receives a tag and an event name and returns the
current method associated with the pair.

Tag methods are called in the following events,
identified by the given names.
The semantics of tag methods is better explained by a Lua function
describing the behavior of the interpreter at each event.
This function not only shows when a tag method is called,
but also its arguments, its results, and the default behavior.
The code shown here is only \emph{illustrative};
the real behavior is hard coded in the interpreter,
and it is much more efficient than this simulation.
All functions used in these descriptions
(\verb|rawgetglobal|, \verb|tonumber|, \verb|call|, etc.)
are described in \See{predefined}.

\begin{description}

\item[``add'':]\index{add event}
called when a \verb|+| operation is applied to non numerical operands.

The function \verb|getbinmethod| defines how Lua chooses a tag method
for a binary operation.
First, Lua tries the first operand.
If its tag does not define a tag method for the operation,
then Lua tries the second operand.
If it also fails, then it gets a tag method from tag~0:
\begin{verbatim}
      function getbinmethod (op1, op2, event)
        return gettagmethod(tag(op1), event) or
               gettagmethod(tag(op2), event) or
               gettagmethod(0, event)
      end
\end{verbatim}
Using this function,
the tag method for the ``add' event is
\begin{verbatim}
      function add_event (op1, op2)
        local o1, o2 = tonumber(op1), tonumber(op2)
        if o1 and o2 then  -- both operands are numeric
          return o1+o2  -- '+' here is the primitive 'add'
        else  -- at least one of the operands is not numeric
          local tm = getbinmethod(op1, op2, "add")
          if tm then
            -- call the method with both operands and an extra
            -- argument with the event name
            return tm(op1, op2, "add")
          else  -- no tag method available: default behavior
            error("unexpected type at arithmetic operation")
          end
        end
      end
\end{verbatim}

\item[``sub'':]\index{sub event}
called when a \verb|-| operation is applied to non numerical operands.
Behavior similar to the ``add'' event.

\item[``mul'':]\index{mul event}
called when a \verb|*| operation is applied to non numerical operands.
Behavior similar to the ``add'' event.

\item[``div'':]\index{div event}
called when a \verb|/| operation is applied to non numerical operands.
Behavior similar to the ``add'' event.

\item[``pow'':]\index{pow event}
called when a \verb|^| operation (exponentiation) is applied.
\begin{verbatim}
      function pow_event (op1, op2)
        local tm = getbinmethod(op1, op2, "pow")
        if tm then
          -- call the method with both operands and an extra
          -- argument with the event name
          return tm(op1, op2, "pow")
        else  -- no tag method available: default behavior
          error("unexpected type at arithmetic operation")
        end
      end
\end{verbatim}

\item[``unm'':]\index{unm event}
called when a unary \verb|-| operation is applied to a non numerical operand.
\begin{verbatim}
      function unm_event (op)
        local o = tonumber(op)
        if o then  -- operand is numeric
          return -o  -- '-' here is the primitive 'unm'
        else  -- the operand is not numeric.
          -- Try to get a tag method from the operand;
          --  if it does not have one, try a "global" one (tag 0)
          local tm = gettagmethod(tag(op), "unm") or
                     gettagmethod(0, "unm")
          if tm then
            -- call the method with the operand, nil, and an extra
            -- argument with the event name
            return tm(op, nil, "unm")
          else  -- no tag method available: default behavior
            error("unexpected type at arithmetic operation")
          end
        end
      end
\end{verbatim}

\item[``lt'':]\index{lt event}
called when an order operation is applied to non-numerical
or non-string operands.
It corresponds to the \verb|<| operator.
\begin{verbatim}
      function lt_event (op1, op2)
        if type(op1) == "number" and type(op2) == "number" then
          return op1 < op2   -- numeric comparison
        elseif type(op1) == "string" and type(op2) == "string" then
          return op1 < op2   -- lexicographic comparison
        else
          local tm = getbinmethod(op1, op2, "lt")
          if tm then
            return tm(op1, op2, "lt")
          else
            error("unexpected type at comparison");
          end
        end
      end
\end{verbatim}
The other order operators use this tag method according to the
usual equivalences:
\begin{verbatim}
   a>b    <=>  b<a
   a<=b   <=>  not (b<a)
   a>=b   <=>  not (a<b)
\end{verbatim}

\item[``concat'':]\index{concatenation event}
called when a concatenation is applied to non string operands.
\begin{verbatim}
      function concat_event (op1, op2)
        if (type(op1) == "string" or type(op1) == "number") and
           (type(op2) == "string" or type(op2) == "number") then
          return op1..op2  -- primitive string concatenation
        else
          local tm = getbinmethod(op1, op2, "concat")
          if tm then
            return tm(op1, op2, "concat")
          else
            error("unexpected type for concatenation")
          end
        end
      end
\end{verbatim}

\item[``index'':]\index{index event}
called when Lua tries to retrieve the value of an index
not present in a table.
See event ``gettable'' for its semantics.

\item[``getglobal'':]\index{getglobal event}
called whenever Lua needs the value of a global variable.
This method can only be set for \nil\ and for tags
created by \verb|newtag|.
Note that
the tag is that of the \emph{current value} of the global variable.
\begin{verbatim}
      function getglobal (varname)
        local value = rawgetglobal(varname)
        local tm = gettagmethod(tag(value), "getglobal")
        if not tm then
          return value
        else
          return tm(varname, value)
        end
      end
\end{verbatim}
The function \verb|getglobal| is pre-defined in Lua \see{predefined}.

\item[``setglobal'':]\index{setglobal event}
called whenever Lua assigns to a global variable.
This method cannot be set for numbers, strings, and tables and
userdata with default tags.
\begin{verbatim}
      function setglobal (varname, newvalue)
        local oldvalue = rawgetglobal(varname)
        local tm = gettagmethod(tag(oldvalue), "setglobal")
        if not tm then
          rawsetglobal(varname, newvalue)
        else
          tm(varname, oldvalue, newvalue)
        end
      end
\end{verbatim}
The function \verb|setglobal| is pre-defined in Lua \see{predefined}.

\item[``gettable'':]\index{gettable event}
called whenever Lua accesses an indexed variable.
This method cannot be set for tables with default tag.
\begin{verbatim}
      function gettable_event (table, index)
        local tm = gettagmethod(tag(table), "gettable")
        if tm then
          return tm(table, index)
        elseif type(table) ~= "table" then
          error("indexed expression not a table");
        else
          local v = rawgettable(table, index)
          tm = gettagmethod(tag(table), "index")
          if v == nil and tm then
            return tm(table, index)
          else
            return v
          end
        end
      end
\end{verbatim}

\item[``settable'':]\index{settable event}
called when Lua assigns to an indexed variable.
This method cannot be set for tables with default tag.
\begin{verbatim}
      function settable_event (table, index, value)
        local tm = gettagmethod(tag(table), "settable")
        if tm then
          tm(table, index, value)
        elseif type(table) ~= "table" then
          error("indexed expression not a table")
        else
          rawsettable(table, index, value)
        end
      end
\end{verbatim}

\item[``function'':]\index{function event}
called when Lua tries to call a non function value.
\begin{verbatim}
      function function_event (func, ...)
        if type(func) == "function" then
          return call(func, arg)
        else
          local tm = gettagmethod(tag(func), "function")
          if tm then
            for i=arg.n,1,-1 do
              arg[i+1] = arg[i]
            end
            arg.n = arg.n+1
            arg[1] = func
            return call(tm, arg)
          else
            error("call expression not a function")
          end
        end
      end
\end{verbatim}

\item[``gc'':]\index{gc event}
called when Lua is ``garbage collecting'' a userdata.
This tag method can be set only from~C,
and cannot be set for a userdata with default tag.
For each userdata to be collected,
Lua does the equivalent of the following function:
\begin{verbatim}
      function gc_event (obj)
        local tm = gettagmethod(tag(obj), "gc")
        if tm then
          tm(obj)
        end
      end
\end{verbatim}
Moreover, at the end of a garbage collection cycle,
Lua does the equivalent of the call \verb|gc_event(nil)|.

\end{description}




\section{The Application Program Interface}

This section describes the API for Lua, that is,
the set of C~functions available to the host program to communicate
with the Lua library.
The API functions can be classified into the following categories:
\begin{enumerate}
\item managing states;
\item exchanging values between C and Lua;
\item executing Lua code;
\item manipulating (reading and writing) Lua objects;
\item calling Lua functions;
\item defining C~functions to be called by Lua;
\item manipulating references to Lua Objects.
\end{enumerate}
All API functions and related types and constants
are declared in the header file \verb|lua.h|.

\NOTE
Even when we use the term \emph{function},
\emph{any facility in the API may be provided as a macro instead}.
All such macros use each of its arguments exactly once,
and so do not generate hidden side-effects.


\subsection{States} \label{mangstate}

The Lua library is reentrant:
it does not have any global variable.
The whole state of the Lua interpreter
(global variables, stack, tag methods, etc.)
is stored in a dynamic structure; \Deffunc{lua_State}
this state must be passed as the first argument to almost
every function in the library.

Before calling any API function,
you must create a state.
This is done by calling\Deffunc{lua_newstate}
\begin{verbatim}
lua_State *lua_newstate (const char *s, ...);
\end{verbatim}
The arguments to this function form a list of name-value options,
terminated with \verb|NULL|.
Currently, the function accepts the following options:
\begin{itemize}
\item \verb|"stack"| --- the stack size.
Each function call needs one stack position for each local variable
and temporary variables, plus one position for book-keeping.
The stack must also have at least ten extra positions available.
For very small implementations, without recursive functions,
a stack size of 100 should be enough.
The default stack size is 1024.

\item \verb|"builtin"| --- the value is a boolean (0 or 1) that
indicates whether the predefined functions should be loaded or not.
The default is to load those functions.
\end{itemize}
For instance, the call
\begin{verbatim}
lua_State *L = lua_newstate(NULL);
\end{verbatim}
creates a new state with a stack of 1024 positions
and with the predefined functions loaded;
the call
\begin{verbatim}
lua_State *L = lua_newstate("builtin", 0, "stack", 100, NULL);
\end{verbatim}
creates a new state with a stack of 100 positions,
without the predefined functions.

To release a state, you call
\begin{verbatim}
void lua_close (lua_State *L);
\end{verbatim}
This function destroys all objects in the current Lua environment
(calling the corresponding garbage collection tag methods)
and frees all dynamic memory used by the state.
Usually, you do not need to call this function,
because all resources are naturally released when the program ends.
On the other hand,
long-running programs ---
like a daemon or web server, for example ---
might need to release states as soon as they are not needed,
to avoid growing too big.

With the exception of \verb|lua_newstate|,
all functions in the API need a state as their first argument.
However, most applications use a single state.
To avoid the burden of passing this only state explicitly to all
functions, and also to keep compatibility with old versions of Lua,
the API provides a set of macros and one global variable that
take care of this state argument for single-state applications:
\begin{verbatim}
#ifndef LUA_REENTRANT
\end{verbatim}
\begin{verbatim}
extern lua_State *lua_state;
\end{verbatim}
\begin{verbatim}
#define lua_close()             (lua_close)(lua_state)
#define lua_dofile(filename)    (lua_dofile)(lua_state, filename)
#define lua_dostring(str)       (lua_dostring)(lua_state, str)
   ...
\end{verbatim}
\begin{verbatim}
#endif
\end{verbatim}
For each function in the API, there is a macro with the same name
that supplies \verb|lua_state| as the first argument to the call.
(The parentheses around the function name avoid it being expanded
again as a macro.)
The only exception is \verb|lua_newstate|;
in this case, the corresponding macro is
\begin{verbatim}
#define lua_open()      ((void)(lua_state?0:(lua_state=lua_newstate(0))))
\end{verbatim}
This code checks whether the global state has been initialized;
if not, it creates a new state with default settings and
assigns it to \verb|lua_newstate|.

By default, the single-state macros are all active.
If you need to use multiple states,
and therefore will provide the state argument explicitly in each call,
you should define \IndexVerb{LUA_REENTRANT} before
including \verb|lua.h| in your code:
\begin{verbatim}
#define LUA_REENTRANT
#include "lua.h"
\end{verbatim}

In the sequel, we will show all functions in the single-state form
(therefore, they are actually macros).
When you define \verb|LUA_REENTRANT|,
all of them get a state as the first parameter.


\subsection{Exchanging Values between C and Lua} \label{valuesCLua}
Because Lua has no static type system,
all values passed between Lua and C have type
\verb|lua_Object|\Deffunc{lua_Object},
which works like an abstract type in C that can hold any Lua value.
Values of type \verb|lua_Object| have no meaning outside Lua;
for instance,
you cannot compare two \verb|lua_Object's| directly.
Instead, you should use the following function:
\Deffunc{lua_equal}
\begin{verbatim}
int lua_equal       (lua_Object o1, lua_Object o2);
\end{verbatim}

To check the type of a \verb|lua_Object|,
the following functions are available:
\Deffunc{lua_isnil}\Deffunc{lua_isnumber}\Deffunc{lua_isstring}
\Deffunc{lua_istable}\Deffunc{lua_iscfunction}\Deffunc{lua_isuserdata}
\Deffunc{lua_isfunction}
\Deffunc{lua_type}
\begin{verbatim}
int lua_isnil        (lua_Object object);
int lua_isnumber     (lua_Object object);
int lua_isstring     (lua_Object object);
int lua_istable      (lua_Object object);
int lua_isfunction   (lua_Object object);
int lua_iscfunction  (lua_Object object);
int lua_isuserdata   (lua_Object object);
const char *lua_type (lua_Object object);
\end{verbatim}
The \verb|lua_is*| functions return 1 if the object is compatible
with the given type, and 0 otherwise.
The function \verb|lua_isnumber| accepts numbers and numerical strings,
\verb|lua_isstring| accepts strings and numbers \see{coercion},
and \verb|lua_isfunction| accepts Lua functions and C~functions.
To distinguish between Lua functions and C~functions,
you should use \verb|lua_iscfunction|.
To distinguish between numbers and numerical strings,
you can use \verb|lua_type|.
The \verb|lua_type| returns one of the following strings,
describing the type of the given object:
\verb|"nil"|, \verb|"number"|, \verb|"string"|, \verb|"table"|,
\verb|"function"|, \verb|"userdata"|, or \verb|"NOOBJECT"|.

To get the tag of a \verb|lua_Object|,
use the following function:
\Deffunc{lua_tag}
\begin{verbatim}
int lua_tag (lua_Object object);
\end{verbatim}

To translate a value from type \verb|lua_Object| to a specific C type,
you can use the following conversion functions:
\Deffunc{lua_getnumber}\Deffunc{lua_getstring}\Deffunc{lua_strlen}
\Deffunc{lua_getcfunction}\Deffunc{lua_getuserdata}
\begin{verbatim}
double         lua_getnumber    (lua_Object object);
const char    *lua_getstring    (lua_Object object);
long           lua_strlen       (lua_Object object);
lua_CFunction  lua_getcfunction (lua_Object object);
void          *lua_getuserdata  (lua_Object object);
\end{verbatim}

\verb|lua_getnumber| converts a \verb|lua_Object| to a floating-point number.
This \verb|lua_Object| must be a number or a string convertible to number
\see{coercion}; otherwise, \verb|lua_getnumber| returns~0.

\verb|lua_getstring| converts a \verb|lua_Object| to a string
(\verb|const char*|).
This \verb|lua_Object| must be a string or a number;
otherwise, the function returns \verb|NULL|.
This function does not create a new string,
but returns a pointer to a string inside the Lua environment.
Those strings always have a 0 after their last character (like in C),
but may contain other zeros in their body.
If you do not know whether a string may contain zeros,
you can use \verb|lua_strlen| to get the actual length.
Because Lua has garbage collection,
there is no guarantee that the pointer returned by \verb|lua_getstring|
will be valid after the block ends
\see{GC}.
So,
if you need the string later on,
you should duplicate it with something like
\verb|memcpy(malloc(lua_strlen(o),lua_getstring(o)))|.

\verb|lua_getcfunction| converts a \verb|lua_Object| to a C~function.
This \verb|lua_Object| must be a C~function;
otherwise, \verb|lua_getcfunction| returns \verb|NULL|.
The type \verb|lua_CFunction| is explained in \See{LuacallC}.

\verb|lua_getuserdata| converts a \verb|lua_Object| to \verb|void*|.
This \verb|lua_Object| must have type \emph{userdata};
otherwise, \verb|lua_getuserdata| returns \verb|NULL|.

\subsection{Communication between Lua and C}\label{Lua-C-protocol}

All communication between Lua and C is done through two
abstract data types, called \Def{lua2C} and \Def{C2lua}.
The first one, as the name implies, is used to pass values
from Lua to C:
parameters when Lua calls C and results when C calls Lua.
The structure C2lua is used in the reverse direction:
parameters when C calls Lua and results when Lua calls C.

The structure lua2C is an \emph{abstract array}
that can be indexed with the function:
\Deffunc{lua_lua2C}
\begin{verbatim}
lua_Object lua_lua2C (int number);
\end{verbatim}
where \verb|number| starts with 1.
When called with a number larger than the array size,
this function returns \verb|LUA_NOOBJECT|\Deffunc{LUA_NOOBJECT}.
In this way, it is possible to write C~functions that receive
a variable number of parameters,
and to call Lua functions that return a variable number of results.
Note that the structure lua2C cannot be directly modified by C code.

The structure C2lua is an \emph{abstract stack}.
Pushing elements into this stack
is done with the following functions:
\Deffunc{lua_pushnumber}\Deffunc{lua_pushlstring}\Deffunc{lua_pushstring}
\Deffunc{lua_pushcfunction}\Deffunc{lua_pushusertag}
\Deffunc{lua_pushnil}\Deffunc{lua_pushobject}
\Deffunc{lua_pushuserdata}\label{pushing}
\begin{verbatim}
void lua_pushnumber    (double n);
void lua_pushlstring   (const char *s, long len);
void lua_pushstring    (const char *s);
void lua_pushusertag   (void *u, int tag);
void lua_pushnil       (void);
void lua_pushobject    (lua_Object object);
void lua_pushcfunction (lua_CFunction f);  /* macro */
\end{verbatim}
All of them receive a C value,
convert it to a corresponding \verb|lua_Object|,
and leave the result on the top of C2lua.
In particular, functions \verb|lua_pushlstring| and \verb|lua_pushstring|
make an internal copy of the given string.
Function \verb|lua_pushstring| can only be used to push proper C strings
(that is, strings that end with a zero and do not contain embedded zeros);
otherwise you should use the more general \verb|lua_pushlstring|.
The function
\Deffunc{lua_pop}
\begin{verbatim}
lua_Object lua_pop (void);
\end{verbatim}
returns a reference to the object at the top of the C2lua stack,
and pops it.

As a general rule, all API functions pop from the stack
all elements they use.

When C code calls Lua repeatedly, as in a loop,
objects returned by these calls can accumulate,
and may cause a stack overflow.
To avoid this,
nested blocks can be defined with the functions
\begin{verbatim}
void lua_beginblock (void);
void lua_endblock   (void);
\end{verbatim}
After the end of the block,
all \verb|lua_Object|'s created inside it are released.
The use of explicit nested blocks is good programming practice
and is strongly encouraged.

\subsection{Garbage Collection}\label{GC}
Because Lua has automatic memory management and garbage collection,
a \verb|lua_Object| has a limited scope,
and is only valid inside the \emph{block} where it has been created.
A C~function called from Lua is a block,
and its parameters are valid only until its end.
It is good programming practice to convert Lua objects to C values
as soon as they are available,
and never to store \verb|lua_Object|s in C global variables.

A garbage collection cycle can be forced by:
\Deffunc{lua_collectgarbage}
\begin{verbatim}
long lua_collectgarbage (long limit);
\end{verbatim}
This function returns the number of objects collected.
The argument \verb|limit| makes the next cycle occur only
after that number of new objects have been created.
If \verb|limit| is 0,
then Lua uses an adaptive heuristics to set this limit.


\subsection{Userdata and Tags}\label{C-tags}

Because userdata are objects,
the function \verb|lua_pushusertag| may create a new userdata.
If Lua has a userdata with the given value (\verb|void*|) and tag,
then that userdata is pushed.
Otherwise, a new userdata is created, with the given value and tag.
If this function is called with
\verb|tag| equal to \verb|LUA_ANYTAG|\Deffunc{LUA_ANYTAG},
then Lua will try to find any userdata with the given value,
regardless of its tag.
If there is no userdata with that value, then a new one is created,
with tag equal to 0.

Userdata can have different tags,
whose semantics are only known to the host program.
Tags are created with the function
\Deffunc{lua_newtag}
\begin{verbatim}
int lua_newtag (void);
\end{verbatim}
The function \verb|lua_settag| changes the tag of
the object on the top of C2lua (and pops it);
the object must be a userdata or a table:
\Deffunc{lua_settag}
\begin{verbatim}
void lua_settag (int tag);
\end{verbatim}
The given \verb|tag| must be a value created with \verb|lua_newtag|.

\subsection{Executing Lua Code}
A host program can execute Lua chunks written in a file or in a string
using the following functions:%
\Deffunc{lua_dofile}\Deffunc{lua_dostring}\Deffunc{lua_dobuffer}
\begin{verbatim}
int lua_dofile   (const char *filename);
int lua_dostring (const char *string);
int lua_dobuffer (const char *buff, int size, const char *name);
\end{verbatim}
All these functions return an error code:
0, in case of success; non zero, in case of errors.
More specifically, \verb|lua_dofile| returns 2 if for any reason
it could not open the file.
(In this case,
you may want to
check \verb|errno|,
call \verb|strerror|,
or call \verb|perror| to tell the user what went wrong.)
When called with argument \verb|NULL|,
\verb|lua_dofile| executes the \verb|stdin| stream.
Functions \verb|lua_dofile| and \verb|lua_dobuffer|
are both able to execute pre-compiled chunks.
They automatically detect whether the chunk is text or binary,
and load it accordingly (see program \IndexVerb{luac}).
Function \verb|lua_dostring| executes only source code,
given in textual form.

The third parameter to \verb|lua_dobuffer| (\verb|name|)
is the ``name of the chunk'',
used in error messages and debug information.
If \verb|name| is \verb|NULL|,
then Lua gives a default name to the chunk.

These functions return, in structure lua2C,
any values eventually returned by the chunks.
They also empty the stack C2lua.


\subsection{Manipulating Lua Objects}
To read the value of any global Lua variable,
one uses the function
\Deffunc{lua_getglobal}
\begin{verbatim}
lua_Object lua_getglobal (const char *varname);
\end{verbatim}
As in Lua, this function may trigger a tag method
for the ``getglobal'' event.
To read the real value of any global variable,
without invoking any tag method,
use the \emph{raw} version:
\Deffunc{lua_rawgetglobal}
\begin{verbatim}
lua_Object lua_rawgetglobal (const char *varname);
\end{verbatim}

To store a value previously pushed onto C2lua in a global variable,
there is the function
\Deffunc{lua_setglobal}
\begin{verbatim}
void lua_setglobal (const char *varname);
\end{verbatim}
As in Lua, this function may trigger a tag method
for the ``setglobal'' event.
To set the real value of any global variable,
without invoking any tag method,
use the \emph{raw} version:
\Deffunc{lua_rawgetglobal}
\begin{verbatim}
void lua_rawsetglobal (const char *varname);
\end{verbatim}

Tables can also be manipulated via the API.
The function
\Deffunc{lua_gettable}
\begin{verbatim}
lua_Object lua_gettable (void);
\end{verbatim}
pops a table and an index from the stack C2lua,
and returns the contents of the table at that index.
As in Lua, this operation may trigger a tag method
for the ``gettable'' event.
To get the real value of any table index,
without invoking any tag method,
use the \emph{raw} version:
\Deffunc{lua_rawgetglobal}
\begin{verbatim}
lua_Object lua_rawgettable (void);
\end{verbatim}

To store a value in an index,
the program must push the table, the index, and the value onto C2lua
(in this order),
and then call the function
\Deffunc{lua_settable}
\begin{verbatim}
void lua_settable (void);
\end{verbatim}
As in Lua, this operation may trigger a tag method
for the ``settable'' event.
To set the real value of any table index,
without invoking any tag method,
use the \emph{raw} version:
\Deffunc{lua_rawsettable}
\begin{verbatim}
void lua_rawsettable (void);
\end{verbatim}

Finally, the function
\Deffunc{lua_createtable}
\begin{verbatim}
lua_Object lua_createtable (void);
\end{verbatim}
creates and returns a new, empty table.


\subsection{Calling Lua Functions}
Functions defined in Lua by a chunk
can be called from the host program.
This is done using the following protocol:
first, the arguments to the function are pushed onto C2lua
\see{pushing}, in direct order, i.e., the first argument is pushed first.
Then, the function is called using
\Deffunc{lua_callfunction}
\begin{verbatim}
int lua_callfunction (lua_Object function);
\end{verbatim}
This function returns an error code:
0, in case of success; non zero, in case of errors.
Finally, the results are returned in structure lua2C
(recall that a Lua function may return many values),
and can be retrieved with the macro \verb|lua_getresult|,
\Deffunc{lua_getresult}
which is just another name for the function \verb|lua_lua2C|.
Note that \verb|lua_callfunction|
pops all elements from the C2lua stack.

The following example shows how the host program may do the
equivalent to the Lua code:
\begin{verbatim}
      a,b = f("how", t.x, 4)
\end{verbatim}
\begin{verbatim}
  lua_pushstring("how");                               /* 1st argument */
  lua_pushobject(lua_getglobal("t"));      /* push value of global 't' */
  lua_pushstring("x");                          /* push the string 'x' */
  lua_pushobject(lua_gettable());      /* push result of t.x (2nd arg) */
  lua_pushnumber(4);                                   /* 3rd argument */
  lua_callfunction(lua_getglobal("f"));           /* call Lua function */
  lua_pushobject(lua_getresult(1));   /* push first result of the call */
  lua_setglobal("a");                       /* set global variable 'a' */
  lua_pushobject(lua_getresult(2));  /* push second result of the call */
  lua_setglobal("b");                       /* set global variable 'b' */
\end{verbatim}

Some special Lua functions have exclusive interfaces.
The host program can generate a Lua error calling the function
\Deffunc{lua_error}
\begin{verbatim}
void lua_error (const char *message);
\end{verbatim}
This function never returns.
If \verb|lua_error| is called from a C~function that has been called from Lua,
then the corresponding Lua execution terminates,
as if an error had occurred inside Lua code.
Otherwise, the whole host program terminates with a call to \verb|exit(1)|.
Before terminating execution,
the \verb|message| is passed to the error handler function,
\verb|_ERRORMESSAGE| \see{error}.
If \verb|message| is \verb|NULL|,
then \verb|_ERRORMESSAGE| is not called.

Tag methods can be changed with: \Deffunc{lua_settagmethod}
\begin{verbatim}
lua_Object lua_settagmethod (int tag, const char *event);
\end{verbatim}
The first parameter is the tag,
and the second is the event name \see{tag-method};
the new method is pushed from C2lua.
This function returns a \verb|lua_Object|,
which is the old tag method value.
To get just the current value of a tag method,
use the function \Deffunc{lua_gettagmethod}
\begin{verbatim}
lua_Object lua_gettagmethod (int tag, const char *event);
\end{verbatim}

It is also possible to copy all tag methods from one tag
to another: \Deffunc{lua_copytagmethods}
\begin{verbatim}
int lua_copytagmethods (int tagto, int tagfrom);
\end{verbatim}
This function returns \verb|tagto|.

You can traverse a table with the function \Deffunc{lua_next}
\begin{verbatim}
int lua_next (lua_Object t, int i);
\end{verbatim}
Its first argument is the table to be traversed,
and the second is a \emph{cursor};
this cursor starts in 0,
and for each call the function returns a value to
be used in the next call,
or 0 to signal the end of the traversal.
The function also returns, in the Lua2C array,
a key-value pair from the table.
A typical traversal looks like the following code:
\begin{verbatim}
  int i;
  lua_Object t;
  ...   /* gets the table at `t' */
  i = 0;
  lua_beginblock();
  while ((i = lua_next(t, i)) != 0) {
    lua_Object key = lua_getresult(1);
    lua_Object value = lua_getresult(2);
    ...  /* uses `key' and `value' */
    lua_endblock();
    lua_beginblock();  /* reopens a block */
  }
  lua_endblock();
\end{verbatim}
The pairs of \verb|lua_beginblock|/\verb|lua_endblock| remove the
results of each iteration from the stack.
Without them, a traversal of a large table may overflow the stack.

To traverse the global variables, use \Deffunc{lua_nextvar}
\begin{verbatim}
const char *lua_nextvar (const char *varname);
\end{verbatim}
Here, the cursor is a string;
in the first call you set it to \verb|NULL|;
for each call the function returns the name of a global variable,
to be used in the next call,
or \verb|NULL| to signal the end of the traverse.
The function also returns, in the Lua2C array,
the name (again) and the value of the global variable.
A typical traversal looks like the following code:
\begin{verbatim}
  const char *name = NULL;
  lua_beginblock();
  while ((name = lua_nextvar(name)) != NULL) {
    lua_Object value = lua_getresult(2);
    ...  /* uses `name' and `value' */
    lua_endblock();
    lua_beginblock();  /* reopens a block */
  }
  lua_endblock();
\end{verbatim}


\subsection{Defining C Functions} \label{LuacallC}
To register a C~function to Lua,
there is the following convenience macro:
\Deffunc{lua_register}
\begin{verbatim}
#define lua_register(n,f)       (lua_pushcfunction(f), lua_setglobal(n))
/* const char *n;   */
/* lua_CFunction f; */
\end{verbatim}
which receives the name the function will have in Lua,
and a pointer to the function.
This pointer must have type \verb|lua_CFunction|,
which is defined as
\Deffunc{lua_CFunction}
\begin{verbatim}
typedef void (*lua_CFunction) (void);
\end{verbatim}
that is, a pointer to a function with no parameters and no results.

In order to communicate properly with Lua,
a C~function must follow a protocol,
which defines the way parameters and results are passed.

A C~function receives its arguments in structure lua2C;
to access them, it uses the macro \verb|lua_getparam|, \Deffunc{lua_getparam}
again just another name for \verb|lua_lua2C|.
To return values, a C~function just pushes them onto the stack C2lua,
in direct order \see{valuesCLua}.
Like a Lua function, a C~function called by Lua can also return
many results.

When a C~function is created,
it is possible to associate some \emph{upvalues} to it
\see{upvalue},
thus creating a C closure;
these values are passed to the function whenever it is called,
as common arguments.
To associate upvalues to a C~function,
first these values must be pushed on C2lua.
Then the function \Deffunc{lua_pushcclosure}
\begin{verbatim}
void lua_pushcclosure (lua_CFunction fn, int n);
\end{verbatim}
is used to put the C~function on C2lua,
with the argument \verb|n| telling how many upvalues must be
associated with the function;
in fact, the macro \verb|lua_pushcfunction| is defined as
\verb|lua_pushcclosure| with \verb|n| set to 0.
Then, whenever the C~function is called,
these upvalues are inserted as the first arguments \M{n} to the function,
before the actual arguments provided in the call.

For some examples of C~functions, see files \verb|lstrlib.c|,
\verb|liolib.c| and \verb|lmathlib.c| in the official Lua distribution.
In particular,
\verb|liolib.c| defines C~closures with file handles are upvalues.

\subsection{References to Lua Objects}

As noted in \See{GC}, \verb|lua_Object|s are volatile.
If the C code needs to keep a \verb|lua_Object|
outside block boundaries,
then it must create a \Def{reference} to the object.
The routines to manipulate references are the following:
\Deffunc{lua_ref}\Deffunc{lua_getref}
\Deffunc{lua_unref}
\begin{verbatim}
int        lua_ref    (int lock);
lua_Object lua_getref (int ref);
void       lua_unref  (int ref);
\end{verbatim}
The function \verb|lua_ref| creates a reference
to the object that is on the top of the stack,
and returns this reference.
For a \nil\ object,
the reference is always \verb|LUA_REFNIL|;\Deffunc{LUA_REFNIL}
otherwise, it is a non-negative integer.
The constant \verb|LUA_NOREF| \Deffunc{LUA_NOREF}
is different from any valid reference.
If \verb|lock| is true, then the object is \emph{locked}:
this means the object will not be garbage collected.
\emph{Unlocked references may be garbage collected}.
Whenever the referenced object is needed in~C,
a call to \verb|lua_getref|
returns a handle to it;
if the object has been collected,
\verb|lua_getref| returns \verb|LUA_NOOBJECT|.

When a reference is no longer needed,
it can be released with a call to \verb|lua_unref|.



\section{Predefined Functions and Libraries}

The set of \Index{predefined functions} in Lua is small but powerful.
Most of them provide features that allow some degree of
\Index{reflexivity} in the language.
Some of these features cannot be simulated with the rest of the
language nor with the standard Lua API.
Others are just convenient interfaces to common API functions.

The libraries, on the other hand, provide useful routines
that are implemented directly through the standard API.
Therefore, they are not necessary to the language,
and are provided as separate C modules.
Currently, there are three standard libraries:
\begin{itemize}
\item string manipulation;
\item mathematical functions (sin, log, etc);
\item input and output (plus some system facilities).
\end{itemize}
To have access to these libraries,
the C host program must call the functions
\verb|lua_strlibopen|, \verb|lua_mathlibopen|,
and \verb|lua_iolibopen|, declared in \verb|lualib.h|.
\Deffunc{lua_strlibopen}\Deffunc{lua_mathlibopen}\Deffunc{lua_iolibopen}


\subsection{Predefined Functions} \label{predefined}

\subsubsection*{\ff \T{_ALERT (message)}}\Deffunc{alert}\label{alert}
Prints its only string argument to \IndexVerb{stderr}.
All error messages in Lua are printed through the function stored
in the \verb|_ALERT| global variable
\see{error}.
Therefore, a program may assign another function to this variable
to change the way such messages are shown
(for instance, for systems without \verb|stderr|).

\subsubsection*{\ff \T{assert (v [, message])}}\Deffunc{assert}
Issues an \emph{``assertion failed!''} error
when its argument \verb|v| is \nil.
This function is equivalent to the following Lua function:
\begin{verbatim}
      function assert (v, m)
        if not v then
          m = m or ""
          error("assertion failed!  " .. m)
        end
      end
\end{verbatim}

\subsubsection*{\ff \T{call (func, arg [, mode [, errhandler]])}}\Deffunc{call}
\label{pdf-call}
Calls function \verb|func| with
the arguments given by the table \verb|arg|.
The call is equivalent to
\begin{verbatim}
      func(arg[1], arg[2], ..., arg[n])
\end{verbatim}
where \verb|n| is the result of \verb|getn(arg)| \see{getn}.

By default,
all results from \verb|func| are simply returned by \verb|call|.
If the string \verb|mode| contains \verb|"p"|,
then the results are \emph{packed} in a single table.\index{packed results}
That is, \verb|call| returns just one table;
at index \verb|n|, the table has the total number of results
from the call;
the first result is at index 1, etc.
For instance, the following calls produce the following results:
\begin{verbatim}
   a = call(sin, {5})                --> a = 0.0871557 = sin(5)
   a = call(max, {1,4,5; n=2})       --> a = 4 (only 1 and 4 are arguments)
   a = call(max, {1,4,5; n=2}, "p")  --> a = {4; n=1}
   t = {x=1}
   a = call(next, {t,nil;n=2}, "p")  --> a={"x", 1; n=2}
\end{verbatim}

By default,
if an error occurs during the call to \verb|func|,
the error is propagated.
If the string \verb|mode| contains \verb|"x"|,
then the call is \emph{protected}.\index{protected calls}
In this mode, function \verb|call| does not propagate an error,
regardless of what happens during the call.
Instead, it returns \nil\ to signal the error
(besides calling the appropriated error handler).

If \verb|errhandler| is provided,
the error function \verb|_ERRORMESSAGE| is temporarily set \verb|errhandler|,
while \verb|func| runs.
In particular, if \verb|errhandler| is \nil,
no error messages will be issued during the execution of the called function.

\subsubsection*{\ff \T{collectgarbage ([limit])}}\Deffunc{collectgarbage}
Forces a garbage collection cycle.
Returns the number of objects collected.
The optional argument \verb|limit| is a number that
makes the next cycle occur only after that number of new
objects have been created.
If \verb|limit| is absent or equal to 0,
then Lua uses an adaptive algorithm to set this limit.
\verb|collectgarbage| is equivalent to
the API function \verb|lua_collectgarbage|.

\subsubsection*{\ff \T{copytagmethods (tagto, tagfrom)}}
\Deffunc{copytagmethods}
Copies all tag methods from one tag to another;
it returns \verb|tagto|.

\subsubsection*{\ff \T{dofile (filename)}}\Deffunc{dofile}
Receives a file name,
opens the named file, and executes its contents as a Lua chunk,
or as pre-compiled chunks.
When called without arguments,
\verb|dofile| executes the contents of the standard input (\verb|stdin|).
If there is any error executing the file,
then \verb|dofile| returns \nil.
Otherwise, it returns the values returned by the chunk,
or a non \nil\ value if the chunk returns no values.
It issues an error when called with a non string argument.
\verb|dofile| is equivalent to the API function \verb|lua_dofile|.

\subsubsection*{\ff \T{dostring (string [, chunkname])}}\Deffunc{dostring}
Executes a given string as a Lua chunk.
If there is any error executing the string,
then \verb|dostring| returns \nil.
Otherwise, it returns the values returned by the chunk,
or a non \nil\ value if the chunk returns no values.
The optional parameter \verb|chunkname|
is the ``name of the chunk'',
used in error messages and debug information.
\verb|dostring| is equivalent to the API function \verb|lua_dostring|.

\subsubsection*{\ff \T{error (message)}}\Deffunc{error}\label{pdf-error}
Calls the error handler \see{error} and then terminates
the last protected function called
(in~C: \verb|lua_dofile|, \verb|lua_dostring|,
\verb|lua_dobuffer|, or \verb|lua_callfunction|;
in Lua: \verb|dofile|, \verb|dostring|, or \verb|call| in protected mode).
If \verb|message| is \nil, then the error handler is not called.
Function \verb|error| never returns.
\verb|error| is equivalent to the API function \verb|lua_error|.

\subsubsection*{\ff \T{foreach (table, function)}}\Deffunc{foreach}
Executes the given \verb|function| over all elements of \verb|table|.
For each element, the function is called with the index and
respective value as arguments.
If the function returns any non-\nil\ value,
then the loop is broken, and this value is returned
as the final value of \verb|foreach|.

This function could be defined in Lua:
\begin{verbatim}
      function foreach (t, f)
        local i, v = nil
        while 1 do
          i, v = next(t, i)
          if not i then break end
          local res = f(i, v)
          if res then return res end
        end
      end
\end{verbatim}

You may change the \emph{values} of existing fields in the table during the traversal,
but
if you create new indices,
then
the semantics of \verb|foreach| is undefined.


\subsubsection*{\ff \T{foreachi (table, function)}}\Deffunc{foreachi}
Executes the given \verb|function| over the
numerical indices of \verb|table|.
For each index, the function is called with the index and
respective value as arguments.
Indices are visited in sequential order,
from 1 to \verb|n|,
where \verb|n| is the result of \verb|getn(table)| \see{getn}.
If the function returns any non-\nil\ value,
then the loop is broken, and this value is returned
as the final value of \verb|foreachi|.

This function could be defined in Lua:
\begin{verbatim}
      function foreachi (t, f)
        for i=1,getn(t) do
          local res = f(i, t[i])
          if res then return res end
        end
      end
\end{verbatim}

You may change the \emph{values} of existing fields in the table during the traversal,
but
if you create new indices (even non-numeric),
then
the semantics of \verb|foreachi| is undefined.

\subsubsection*{\ff \T{foreachvar (function)}}\Deffunc{foreachvar}
Executes \verb|function| over all global variables.
For each variable,
the function is called with its name and its value as arguments.
If the function returns any non-nil value,
then the loop is broken, and this value is returned
as the final value of \verb|foreachvar|.

This function could be defined in Lua:
\begin{verbatim}
      function foreachvar (f)
        local n, v = nil
        while 1 do
          n, v = nextvar(n)
          if not n then break end
          local res = f(n, v)
          if res then return res end
        end
      end
\end{verbatim}

You may change the values of existing global variables during the traversal,
but
if you create new global variables,
then
the semantics of \verb|foreachvar| is undefined.


\subsubsection*{\ff \T{getglobal (name)}}\Deffunc{getglobal}
Gets the value of a global variable,
or calls a tag method for ``getgloball''.
Its full semantics is explained in \See{tag-method}.
The string \verb|name| does not need to be a
syntactically valid variable name.

\subsubsection*{\ff \T{getn (table)}}\Deffunc{getn}\label{getn}
Returns the ``size'' of a table, when seen as a list.
If the table has an \verb|n| field with a numeric value,
this value is its ``size''.
Otherwise, the size is the largest numerical index with a non-nil
value in the table.
This function could be defined in Lua:
\begin{verbatim}
      function getn (t)
        if type(t.n) == 'number' then return t.n end
        local max, i = 0, nil
        while 1 do
          i = next(t, i)
          if not i then break end
          if type(i) == 'number' and i>max then max=i end
        end
        return max
      end
\end{verbatim}

\subsubsection*{\ff \T{gettagmethod (tag, event)}}
\Deffunc{gettagmethod}
Returns the current tag method
for a given pair \M{(tag, event)}.

\subsubsection*{\ff \T{newtag ()}}\Deffunc{newtag}\label{pdf-newtag}
Returns a new tag.
\verb|newtag| is equivalent to the API function \verb|lua_newtag|.

\subsubsection*{\ff \T{next (table, [index])}}\Deffunc{next}
Allows a program to traverse all fields of a table.
Its first argument is a table and its second argument
is an index in this table.
It returns the next index of the table and the
value associated with the index.
When called with \nil\ as its second argument,
\verb|next| returns the first index
of the table and its associated value.
When called with the last index,
or with \nil\ in an empty table,
it returns \nil.
If the second argument is absent, then it is interpreted as \nil.

Lua has no declaration of fields;
semantically, there is no difference between a
field not present in a table or a field with value \nil.
Therefore, \verb|next| only considers fields with non \nil\ values.
The order in which the indices are enumerated is not specified,
\emph{even for numeric indices}
(to traverse a table in numeric order,
use a counter or the function \verb|foreachi|).

You may change the \emph{values} of existing fields in the table during the traversal,
but
if you create new indices,
then
the semantics of \verb|next| is undefined.

\subsubsection*{\ff \T{nextvar (name)}}\Deffunc{nextvar}
This function is similar to the function \verb|next|,
but iterates instead over the global variables.
Its single argument is the name of a global variable,
or \nil\ to get a first name.
If this argument is absent, then it is interpreted as \nil.
Like \verb|next|, \verb|nextvar| returns the name of another variable
and its value,
or \nil\ if there are no more variables.

You may change the \emph{values} of existing global variables during the traversal,
but
if you create new global variables,
then
the semantics of \verb|nextvar| is undefined.

\subsubsection*{\ff \T{print (e1, e2, ...)}}\Deffunc{print}
Receives any number of arguments,
and prints their values using the strings returned by \verb|tostring|.
This function is not intended for formatted output,
but only as a quick way to show a value,
for instance for debugging.
See \See{libio} for functions for formatted output.

\subsubsection*{\ff \T{rawgetglobal (name)}}\Deffunc{rawgetglobal}
Gets the value of a global variable,
without invoking any tag method.
The string \verb|name| does not need to be a
syntactically valid variable name.

\subsubsection*{\ff \T{rawgettable (table, index)}}\Deffunc{rawgettable}
Gets the real value of \verb|table[index]|,
without invoking any tag method.
\verb|table| must be a table,
and \verb|index| is any value different from \nil.

\subsubsection*{\ff \T{rawsetglobal (name, value)}}\Deffunc{rawsetglobal}
Sets the named global variable to the given value,
without invoking any tag method.
The string \verb|name| does not need to be a
syntactically valid variable name.
Therefore,
this function can be used to set global variables with strange names like
\verb|"m v 1"| or \verb|"34"|.

\subsubsection*{\ff \T{rawsettable (table, index, value)}}\Deffunc{rawsettable}
Sets the real value of \verb|table[index]| to \verb|value|,
without invoking any tag method.
\verb|table| must be a table,
\verb|index| is any value different from \nil,
and \verb|value| is any Lua value.

\subsubsection*{\ff \T{setglobal (name, value)}}\Deffunc{setglobal}
Sets the named global variable to the given value,
or calls a tag method for ``setgloball''.
Its full semantics is explained in \See{tag-method}.
The string \verb|name| does not need to be a
syntactically valid variable name.

\subsubsection*{\ff \T{settag (t, tag)}}\Deffunc{settag}
Sets the tag of a given table \see{TypesSec}.
\verb|tag| must be a value created with \verb|newtag|
\see{pdf-newtag}.
It returns the value of its first argument (the table).
For the safety of host programs,
it is impossible to change the tag of a userdata from Lua.

\subsubsection*{\ff \T{settagmethod (tag, event, newmethod)}}
\Deffunc{settagmethod}
Sets a new tag method to the given pair \M{(tag, event)}.
It returns the old method.
If \verb|newmethod| is \nil,
then \verb|settagmethod| restores the default behavior for the given event.

\subsubsection*{\ff \T{sort (table [, comp])}}\Deffunc{sort}
Sorts table elements in a given order, \emph{in-place},
from \verb|table[1]| to \verb|table[n]|,
where \verb|n| is the result of \verb|getn(table)| \see{getn}.
If \verb|comp| is given,
it must be a function that receives two table elements,
and returns true when the first is less than the second
(so that \verb|not comp(a[i+1], a[i])| will be true after the sort).
If \verb|comp| is not given,
the standard Lua operator \verb|<| is used instead.

\subsubsection*{\ff \T{tag (v)}}\Deffunc{tag}\label{pdf-tag}
Allows Lua programs to test the tag of a value \see{TypesSec}.
It receives one argument, and returns its tag (a number).
\verb|tag| is equivalent to the API function \verb|lua_tag|.

\subsubsection*{\ff \T{tonumber (e [, base])}}\Deffunc{tonumber}
Receives one argument,
and tries to convert it to a number.
If the argument is already a number or a string convertible
to a number, then \verb|tonumber| returns that number;
otherwise, it returns \nil.

An optional argument specifies the base to interpret the numeral.
The base may be any integer between 2 and 36, inclusive.
In bases above~10, the letter `A' (either upper or lower case)
represents~10, `B' represents~11, and so forth, with `Z' representing 35.

In base 10 (the default), the number may have a decimal part,
as well as an optional exponent part \see{coercion}.
In other bases, only unsigned integers are accepted.

\subsubsection*{\ff \T{tostring (e)}}\Deffunc{tostring}
Receives an argument of any type and
converts it to a string in a reasonable format.
For complete control on how numbers are converted,
use function \verb|format|.



\subsubsection*{\ff \T{tinsert (table [, pos] , value)}}\Deffunc{tinsert}

Inserts element \verb|value| at table position \verb|pos|,
shifting other elements to open space, if necessary.
The default value for \verb|pos| is \verb|n+1|,
where \verb|n| is the result of \verb|getn(table)| \see{getn},
so that a call \verb|tinsert(t,x)| inserts \verb|x| at the end
of table \verb|t|.
This function also sets or increments the field \verb|n| of the table
to \verb|n+1|.

This function is equivalent to the following Lua function,
except that the table accesses are all \emph{raw} (that is, without tag methods):
\begin{verbatim}
      function tinsert (t, ...)
        local pos, value
        local n = getn(t)
        if arg.n == 1 then
          pos, value = n+1, arg[1]
        else
          pos, value = arg[1], arg[2]
        end
        t.n = n+1;
        for i=n,pos,-1 do
          t[i+1] = t[i]
        end
        t[pos] = value
      end
\end{verbatim}

\subsubsection*{\ff \T{tremove (table [, pos])}}\Deffunc{tremove}

Removes from \verb|table| the element at position \verb|pos|,
shifting other elements to close the space, if necessary.
Returns the value of the removed element.
The default value for \verb|pos| is \verb|n|,
where \verb|n| is the result of \verb|getn(table)| \see{getn},
so that a call \verb|tremove(t)| removes the last element
of table \verb|t|.
This function also sets or decrements the field \verb|n| of the table
to \verb|n-1|.

This function is equivalent to the following Lua function,
except that the table accesses are all \emph{raw} (that is, without tag methods):
\begin{verbatim}
      function tremove (t, pos)
        local n = getn(t)
        if n<=0 then return end
        pos = pos or n
        local value = t[pos]
        for i=pos,n-1 do
          t[i] = t[i+1]
        end
        t[n] = nil
        t.n = n-1
        return value
      end
\end{verbatim}

\subsubsection*{\ff \T{type (v)}}\Deffunc{type}\label{pdf-type}
Allows Lua programs to test the type of a value.
It receives one argument, and returns its type, coded as a string.
The possible results of this function are
\verb|"nil"| (a string, not the value \nil),
\verb|"number"|,
\verb|"string"|,
\verb|"table"|,
\verb|"function"|,
and \verb|"userdata"|.
\verb|type| is equivalent to the API function \verb|lua_type|.


\subsection{String Manipulation}
This library provides generic functions for string manipulation,
such as finding and extracting substrings and pattern matching.
When indexing a string, the first character is at position~1
(not at~0, as in C).

\subsubsection*{\ff \T{strbyte (s [, i])}}\Deffunc{strbyte}
Returns the internal numerical code of the character \verb|s[i]|.
If \verb|i| is absent, then it is assumed to be 1.
If \verb|i| is negative,
it is replaced by the length of the string minus its
absolute value plus 1.
Therefore, \Math{-1} points to the last character of \verb|s|.

\NOTE
\emph{numerical codes are not necessarily portable across platforms}.

\subsubsection*{\ff \T{strchar (i1, i2, \ldots)}}\Deffunc{strchar}
Receives 0 or more integers.
Returns a string with length equal to the number of arguments,
wherein each character has the internal numerical code equal
to its correspondent argument.

\NOTE
\emph{numerical codes are not necessarily portable across platforms}.

\subsubsection*{\ff \T{strfind (str, pattern [, init [, plain]])}}
\Deffunc{strfind}
Looks for the first \emph{match} of
\verb|pattern| in \verb|str|.
If it finds one, then it returns the indices on \verb|str|
where this occurrence starts and ends;
otherwise, it returns \nil.
If the pattern specifies captures (see \verb|gsub| below),
the captured strings are returned as extra results.
A third optional numerical argument specifies where to start the search;
its default value is 1.
If \verb|init| is negative,
it is replaced by the length of the string minus its
absolute value plus 1.
Therefore, \Math{-1} points to the last character of \verb|str|.
A value of 1 as a fourth optional argument
turns off the pattern matching facilities,
so the function does a plain ``find substring'' operation,
with no characters in \verb|pattern| being considered ``magic''.

\subsubsection*{\ff \T{strlen (s)}}\Deffunc{strlen}
Receives a string and returns its length.
The empty string \verb|""| has length 0.
Embedded zeros are counted.

\subsubsection*{\ff \T{strlower (s)}}\Deffunc{strlower}
Receives a string and returns a copy of that string with all
upper case letters changed to lower case.
All other characters are left unchanged.
The definition of what is an upper-case
letter depends on the current locale.

\subsubsection*{\ff \T{strrep (s, n)}}\Deffunc{strrep}
Returns a string that is the concatenation of \verb|n| copies of
the string \verb|s|.

\subsubsection*{\ff \T{strsub (s, i [, j])}}\Deffunc{strsub}
Returns another string, which is a substring of \verb|s|,
starting at \verb|i|  and running until \verb|j|.
If \verb|i| or \verb|j| are negative,
they are replaced by the length of the string minus their
absolute value plus 1.
Therefore, \Math{-1} points to the last character of \verb|s|
and \Math{-2} to the previous one.
If \verb|j| is absent, it is assumed to be equal to \Math{-1}
(which is the same as the string length).
In particular,
the call \verb|strsub(s,1,j)| returns a prefix of \verb|s|
with length \verb|j|,
and the call \verb|strsub(s, -i)| returns a suffix of \verb|s|
with length \verb|i|.

\subsubsection*{\ff \T{strupper (s)}}\Deffunc{strupper}
Receives a string and returns a copy of that string with all
lower case letters changed to upper case.
All other characters are left unchanged.
The definition of what is a lower case
letter depends on the current locale.

\subsubsection*{\ff \T{format (formatstring, e1, e2, \ldots)}}\Deffunc{format}
\label{format}
Returns a formatted version of its variable number of arguments
following the description given in its first argument (which must be a string).
The format string follows the same rules as the \verb|printf| family of
standard C~functions.
The only differences are that the options/modifiers
\verb|*|, \verb|l|, \verb|L|, \verb|n|, \verb|p|,
and \verb|h| are not supported,
and there is an extra option, \verb|q|.
The \verb|q| option formats a string in a form suitable to be safely read
back by the Lua interpreter:
The string is written between double quotes,
and all double quotes, returns, and backslashes in the string
are correctly escaped when written.
For instance, the call
\begin{verbatim}
format('%q', 'a string with "quotes" and \n new line')
\end{verbatim}
will produce the string:
\begin{verbatim}
"a string with \"quotes\" and \
 new line"
\end{verbatim}

Conversions can be applied to the \M{n}-th argument in the argument list,
rather than the next unused argument.
In this case, the conversion character \verb|%| is replaced
by the sequence \verb|%d$|, where \verb|d| is a
decimal digit in the range [1,9],
giving the position of the argument in the argument list.
For instance, the call \verb|format("%2$d -> %1$03d", 1, 34)| will
result in \verb|"34 -> 001"|.
The same argument can be used in more than one conversion.

The options \verb|c|, \verb|d|, \verb|E|, \verb|e|, \verb|f|,
\verb|g|, \verb|G|, \verb|i|, \verb|o|, \verb|u|, \verb|X|, and \verb|x| all
expect a number as argument,
whereas \verb|q| and \verb|s| expect a string.
The \verb|*| modifier can be simulated by building
the appropriate format string.
For example, \verb|"%*g"| can be simulated with
\verb|"%"..width.."g"|.

\NOTE
\emph{Neither the format string nor the string values to be formatted with
\T{format} can contain embedded zeros.}

\subsubsection*{\ff \T{gsub (s, pat, repl [, n])}}
\Deffunc{gsub}
Returns a copy of \verb|s|,
in which all occurrences of the pattern \verb|pat| have been
replaced by a replacement string specified by \verb|repl|.
This function also returns, as a second value,
the total number of substitutions made.

If \verb|repl| is a string, then its value is used for replacement.
Any sequence in \verb|repl| of the form \verb|%n|
with \verb|n| between 1 and 9
stands for the value of the \M{n}-th captured substring.

If \verb|repl| is a function, then this function is called every time a
match occurs, with all captured substrings passed as arguments,
in order (see below).
If the value returned by this function is a string,
then it is used as the replacement string;
otherwise, the replacement string is the empty string.

The last, optional parameter \verb|n| limits
the maximum number of substitutions to occur.
For instance, when \verb|n| is 1 only the first occurrence of
\verb|pat| is replaced.

Here are some examples:
\begin{verbatim}
  x = gsub("hello world", "(%w+)", "%1 %1")
  --> x="hello hello world world"

  x = gsub("hello world", "(%w+)", "%1 %1", 1)
  --> x="hello hello world"

  x = gsub("hello world from Lua", "(%w+)%s*(%w+)", "%2 %1")
  --> x="world hello Lua from"

  x = gsub("home = $HOME, user = $USER", "%$(%w+)", getenv)
  --> x="home = /home/roberto, user = roberto"  (for instance)

  x = gsub("4+5 = $return 4+5$", "%$(.-)%$", dostring)
  --> x="4+5 = 9"

  local t = {name="lua", version="3.2"}
  x = gsub("$name - $version", "%$(%w+)", function (v) return %t[v] end)
  --> x="lua - 3.2"

  t = {n=0}
  gsub("first second word", "(%w+)", function (w) tinsert(%t, w) end)
  --> t={"first", "second", "word"; n=3}
\end{verbatim}


\subsubsection*{Patterns} \label{pm}

\paragraph{Character Class:}
a \Def{character class} is used to represent a set of characters.
The following combinations are allowed in describing a character class:
\begin{description}
\item[\emph{x}] (where \emph{x} is any character not in the list
\verb|^$()%.[]*+-?|)
--- represents the character \emph{x} itself.
\item[\T{.}] --- (a dot) represents all characters.
\item[\T{\%a}] --- represents all letters.
\item[\T{\%c}] --- represents all control characters.
\item[\T{\%d}] --- represents all digits.
\item[\T{\%l}] --- represents all lower case letters.
\item[\T{\%p}] --- represents all punctuation characters.
\item[\T{\%s}] --- represents all space characters.
\item[\T{\%u}] --- represents all upper case letters.
\item[\T{\%w}] --- represents all alphanumeric characters.
\item[\T{\%x}] --- represents all hexadecimal digits.
\item[\T{\%z}] --- represents the character with representation 0.
\item[\T{\%\M{x}}] (where \M{x} is any non alphanumeric character)  ---
represents the character \M{x}.
This is the standard way to escape the magic characters \verb|()%.[]*-?|.
It is strongly recommended that any control character (even the non magic)
should be preceded by a \verb|%|
when used to represent itself in a pattern,

\item[\T{[char-set]}] ---
represents the class which is the union of all
characters in char-set.
A range of characters may be specified by
separating the end characters of the range with a \verb|-|.
All classes \verb|%|\emph{x} described above may also be used as
components in a char-set.
All other characters in char-set represent themselves.
For example, \verb|[%w_]| (or \verb|[_%w]|)
represents all alphanumeric characters plus the underscore,
\verb|[0-7]| represents the octal digits,
and \verb|[0-7%l%-]| represents the octal digits plus
the lower case letters plus the \verb|-| character.

The interaction between ranges and classes is not defined.
Therefore, patterns like \verb|[%a-z]| or \verb|[a-%%]|
have no meaning.

\item[\T{[\^{ }char-set]}] ---
represents the complement of char-set,
where char-set is interpreted as above.
\end{description}
For all classes represented by single letters (\verb|%a|, \verb|%c|, \ldots),
the corresponding upper-case letter represents the complement of the class.
For instance, \verb|%S| represents all non-space characters.

The definitions of letter, space, etc. depend on the current locale.
In particular, the class \verb|[a-z]| may not be equivalent to \verb|%l|.
The second form should be preferred for portability.

\paragraph{Pattern Item:}
a \Def{pattern item} may be
\begin{itemize}
\item
a single character class,
which matches any single character in the class;
\item
a single character class followed by \verb|*|,
which matches 0 or more repetitions of characters in the class.
These repetition items will always match the longest possible sequence;
\item
a single character class followed by \verb|+|,
which matches 1 or more repetitions of characters in the class.
These repetition items will always match the longest possible sequence;
\item
a single character class followed by \verb|-|,
which also matches 0 or more repetitions of characters in the class.
Unlike \verb|*|,
these repetition items will always match the shortest possible sequence;
\item
a single character class followed by \verb|?|,
which matches 0 or 1 occurrence of a character in the class;
\item
\T{\%\M{n}}, for \M{n} between 1 and 9;
such item matches a sub-string equal to the \M{n}-th captured string
(see below);
\item
\T{\%b\M{xy}}, where \M{x} and \M{y} are two distinct characters;
such item matches strings that start with~\M{x}, end with~\M{y},
and where the \M{x} and \M{y} are \emph{balanced}.
This means that, if one reads the string from left to right,
counting \Math{+1} for an \M{x} and \Math{-1} for a \M{y},
the ending \M{y} is the first where the count reaches 0.
For instance, the item \verb|%b()| matches expressions with
balanced parentheses.
\end{itemize}

\paragraph{Pattern:}
a \Def{pattern} is a sequence of pattern items.
A \verb|^| at the beginning of a pattern anchors the match at the
beginning of the subject string.
A \verb|$| at the end of a pattern anchors the match at the
end of the subject string.
At other positions,
\verb|^| and \verb|$| have no special meaning and represent themselves.

\paragraph{Captures:}
A pattern may contain sub-patterns enclosed in parentheses,
that describe \Def{captures}.
When a match succeeds, the sub-strings of the subject string
that match captures are stored (\emph{captured}) for future use.
Captures are numbered according to their left parentheses.
For instance, in the pattern \verb|"(a*(.)%w(%s*))"|,
the part of the string matching \verb|"a*(.)%w(%s*)"| is
stored as the first capture (and therefore has number~1);
the character matching \verb|.| is captured with number~2,
and the part matching \verb|%s*| has number~3.

\NOTE
{\em A pattern cannot contain embedded zeros.
Use \verb|%z| instead.}


\subsection{Mathematical Functions} \label{mathlib}

This library is an interface to some functions of the standard C math library.
In addition, it registers a tag method for the binary operator \verb|^| that
returns \Math{x^y} when applied to numbers \verb|x^y|.

The library provides the following functions:
\Deffunc{abs}\Deffunc{acos}\Deffunc{asin}\Deffunc{atan}
\Deffunc{atan2}\Deffunc{ceil}\Deffunc{cos}\Deffunc{floor}
\Deffunc{log}\Deffunc{log10}\Deffunc{max}\Deffunc{min}
\Deffunc{mod}\Deffunc{sin}\Deffunc{sqrt}\Deffunc{tan}
\Deffunc{frexp}\Deffunc{ldexp}
\Deffunc{random}\Deffunc{randomseed}
\begin{verbatim}
   abs  acos  asin  atan  atan2  ceil  cos  deg     floor  log  log10
   max  min   mod   rad   sin    sqrt  tan  frexp   ldexp
   random     randomseed
\end{verbatim}
plus a global variable \IndexVerb{PI}.
Most of them
are only interfaces to the homonymous functions in the C~library,
except that, for the trigonometric functions,
all angles are expressed in \emph{degrees}, not radians.
Functions \IndexVerb{deg} and \IndexVerb{rad} can be used to convert
between radians and degrees.

The function \verb|max| returns the maximum
value of its numeric arguments.
Similarly, \verb|min| computes the minimum.
Both can be used with 1, 2, or more arguments.

The functions \verb|random| and \verb|randomseed| are interfaces to
the simple random generator functions \verb|rand| and \verb|srand|,
provided by ANSI C.
(No guarantees can be given for their statistical properties.)
The function \verb|random|, when called without arguments,
returns a pseudo-random real number in the range \Math{[0,1)}.
When called with a number \Math{n},
\verb|random| returns a pseudo-random integer in the range \Math{[1,n]}.
When called with two arguments, \Math{l} and \Math{u},
\verb|random| returns a pseudo-random integer in the range \Math{[l,u]}.


\subsection{I/O Facilities} \label{libio}

All input and output operations in Lua are done, by default,
over two \Def{file handles}, one for reading and one for writing.
These handles are stored in two Lua global variables,
called \verb|_INPUT| and \verb|_OUTPUT|.
The global variables
\verb|_STDIN|, \verb|_STDOUT|, and \verb|_STDERR|
are initialized with file descriptors for
\verb|stdin|, \verb|stdout| and \verb|stderr|.
Initially, \verb|_INPUT=_STDIN| and \verb|_OUTPUT=_STDOUT|.
\Deffunc{_INPUT}\Deffunc{_OUTPUT}
\Deffunc{_STDIN}\Deffunc{_STDOUT}\Deffunc{_STDERR}

A file handle is a userdata containing the file stream \verb|FILE*|,
and with a distinctive tag created by the I/O library.

Unless otherwise stated,
all I/O functions return \nil\ on failure and
some value different from \nil\ on success.

\subsubsection*{\ff \T{openfile (filename, mode)}}\Deffunc{openfile}

This function opens a file,
in the mode specified in the string \verb|mode|.
It returns a new file handle,
or, in case of errors, \nil\ plus a string describing the error.
This function does not modify either \verb|_INPUT| or \verb|_OUTPUT|.

The \verb|mode| string can be any of the following:
\begin{description}
\item[``r''] read mode;
\item[``w''] write mode;
\item[``a''] append mode;
\item[``r+''] update mode, all previous data is preserved;
\item[``w+''] update mode, all previous data is erased;
\item[``a+''] append update mode, previous data is preserved,
  writing is only allowed at the end of file.
\end{description}
The \verb|mode| string may also have a \verb|b| at the end,
which is needed in some systems to open the file in binary mode.
This string is exactlty what is used in the standard~C function \verb|fopen|.

\subsubsection*{\ff \T{closefile (handle)}}\Deffunc{closefile}

This function closes the given file.
It does not modify either \verb|_INPUT| or \verb|_OUTPUT|.

\subsubsection*{\ff \T{readfrom (filename)}}\Deffunc{readfrom}

This function may be called in two ways.
When called with a file name, it opens the named file,
sets its handle as the value of \verb|_INPUT|,
and returns this value.
It does not close the current input file.
When called without parameters,
it closes the \verb|_INPUT| file,
and restores \verb|stdin| as the value of \verb|_INPUT|.

If this function fails, it returns \nil,
plus a string describing the error.

\begin{quotation}
\noindent
\emph{System dependent}: if \verb|filename| starts with a \verb-|-,
then a \Index{piped input} is opened, via function \IndexVerb{popen}.
Not all systems implement pipes.
Moreover,
the number of files that can be open at the same time is
usually limited and depends on the system.
\end{quotation}

\subsubsection*{\ff \T{writeto (filename)}}\Deffunc{writeto}

This function may be called in two ways.
When called with a file name,
it opens the named file,
sets its handle as the value of \verb|_OUTPUT|,
and returns this value.
It does not close the current output file.
Note that, if the file already exists,
then it will be \emph{completely erased} with this operation.
When called without parameters,
this function closes the \verb|_OUTPUT| file,
and restores \verb|stdout| as the value of \verb|_OUTPUT|.
\index{closing a file}

If this function fails, it returns \nil,
plus a string describing the error.

\begin{quotation}
\noindent
\emph{System dependent}: if \verb|filename| starts with a \verb-|-,
then a \Index{piped output} is opened, via function \IndexVerb{popen}.
Not all systems implement pipes.
Moreover,
the number of files that can be open at the same time is
usually limited and depends on the system.
\end{quotation}

\subsubsection*{\ff \T{appendto (filename)}}\Deffunc{appendto}

Opens a file named \verb|filename| and sets it as the
value of \verb|_OUTPUT|.
Unlike the \verb|writeto| operation,
this function does not erase any previous contents of the file.
If this function fails, it returns \nil,
plus a string describing the error.

\subsubsection*{\ff \T{remove (filename)}}\Deffunc{remove}

Deletes the file with the given name.
If this function fails, it returns \nil,
plus a string describing the error.

\subsubsection*{\ff \T{rename (name1, name2)}}\Deffunc{rename}

Renames file named \verb|name1| to \verb|name2|.
If this function fails, it returns \nil,
plus a string describing the error.

\subsubsection*{\ff \T{flush ([filehandle])}}\Deffunc{flush}

Saves any written data to the given file.
If \verb|filehandle| is not specified,
then \verb|flush| flushes all open files.
If this function fails, it returns \nil,
plus a string describing the error.

\subsubsection*{\ff \T{seek (filehandle [, whence] [, offset])}}\Deffunc{seek}

Sets and gets the file position,
measured in bytes from the beginning of the file,
to the position given by \verb|offset| plus a base
specified by the string \verb|whence|, as follows:
\begin{description}
\item[``set''] base is position 0 (beginning of the file);
\item[``cur''] base is current position;
\item[``end''] base is end of file;
\end{description}
In case of success, function \verb|seek| returns the final file position,
measured in bytes from the beginning of the file.
If the call fails, it returns \nil,
plus a string describing the error.

The default value for \verb|whence| is \verb|"cur"|,
and for \verb|offset| is 0.
Therefore, the call \verb|seek(file)| returns the current
file position, without changing it;
the call \verb|seek(file, "set")| sets the position to the
beginning of the file (and returns 0);
and the call \verb|seek(file, "end")| sets the position to the
end of the file, and returns its size.

\subsubsection*{\ff \T{tmpname ()}}\Deffunc{tmpname}

Returns a string with a file name that can safely
be used for a temporary file.
The file must be explicitly opened before its use
and removed when no longer needed.

\subsubsection*{\ff \T{read ([filehandle,] format1, ...)}}\Deffunc{read}

Reads file \verb|_INPUT|,
or \verb|filehandle| if this argument is given,
according to the given formats, which specify what to read.
For each format,
the function returns a string (or a number) with the characters read,
or \nil\ if it cannot read data with the specified format.
When called without formats,
it uses a default format that reads the next line
(see below).

The available formats are
\begin{description}
\item[``*n''] reads a number;
this is the only format that returns a number instead of a string.
\item[``*l''] reads the next line
(skipping the end of line), or \nil\ on end of file.
This is the default format.
\item[``*a''] reads the whole file, starting at the current position.
On end of file, it returns the empty string.
\item[``*w''] reads the next word
(maximal sequence of non white-space characters),
skipping spaces if necessary, or \nil\ on end of file.
\item[\emph{number}] reads a string with up to that number of characters,
or \nil\ on end of file.
\end{description}

\subsubsection*{\ff \T{write ([filehandle, ] value1, ...)}}\Deffunc{write}

Writes the value of each of its arguments to
file \verb|_OUTPUT|,
or to \verb|filehandle| if this argument is given.
The arguments must be strings or numbers.
To write other values,
use \verb|tostring| or \verb|format| before \verb|write|.
If this function fails, it returns \nil,
plus a string describing the error.

\subsubsection*{\ff \T{date ([format])}}\Deffunc{date}

Returns a string containing date and time
formatted according to the given string \verb|format|,
following the same rules of the ANSI~C function \verb|strftime|.
When called without arguments,
it returns a reasonable date and time representation that depends on
the host system and on the current locale.

\subsubsection*{\ff \T{clock ()}}\Deffunc{clock}

Returns an approximation of the amount of CPU time
used by the program, in seconds.

\subsubsection*{\ff \T{exit ([code])}}\Deffunc{exit}

Calls the C~function \verb|exit|,
with an optional \verb|code|,
to terminate the program.
The default value for \verb|code| is the success code.

\subsubsection*{\ff \T{getenv (varname)}}\Deffunc{getenv}

Returns the value of the process environment variable \verb|varname|,
or \nil\ if the variable is not defined.

\subsubsection*{\ff \T{execute (command)}}\Deffunc{execute}

This function is equivalent to the C~function \verb|system|.
It passes \verb|command| to be executed by an operating system shell.
It returns a status code, which is system-dependent.

\subsubsection*{\ff \T{setlocale (locale [, category])}}\Deffunc{setlocale}

This function is an interface to the ANSI~C function \verb|setlocale|.
\verb|locale| is a string specifying a locale;
\verb|category| is an optional string describing which category to change:
\verb|"all"|, \verb|"collate"|, \verb|"ctype"|,
\verb|"monetary"|, \verb|"numeric"|, or \verb|"time"|;
the default category is \verb|"all"|.
The function returns the name of the new locale,
or \nil\ if the request cannot be honored.


\section{The Debug Interface} \label{debugI}

Lua has no built-in debugging facilities.
Instead, it offers a special interface,
by means of functions and \emph{hooks},
which allows the construction of different
kinds of debuggers, profilers, and other tools
that need ``inside information'' from the interpreter.
This interface is declared in the header file \verb|luadebug.h|,
and has \emph{no} single-state variant.

\subsection{Stack and Function Information}

\Deffunc{lua_getstack}
The main function to get information about the interpreter stack is
\begin{verbatim}
int lua_getstack (lua_State *L, int level, lua_Debug *ar);
\end{verbatim}
It fills parts of a \verb|lua_Debug| structure with
an identification of the \emph{activation record}
of the function executing at a given level.
Level~0 is the current running function,
whereas level \Math{n+1} is the function that has called level \Math{n}.
Usually, \verb|lua_getstack| returns 1;
when called with a level greater than the stack depth,
it returns 0.

\Deffunc{lua_Debug}
The structure \verb|lua_Debug| is used to carry different pieces of information
about an active function:
\begin{verbatim}
struct lua_Debug {
  const char *event;     /* "call", "return" */
  const char *source;    /* (S) */
  int linedefined;       /* (S) */
  const char *what;      /* (S) "Lua" function, "C" function, Lua "main" */
  int currentline;       /* (l) */
  const char *name;      /* (n) */
  const char *namewhat;  /* (n) global, tag method, local, field */
  int nups;              /* (u) number of upvalues */
  lua_Object func;       /* (f) function being executed */
  /* private part */
  ...
};
\end{verbatim}
The \verb|lua_getstack| function fills only the private part
of this structure, for future use.
To fill in the other fields of \verb|lua_Debug| with useful information,
call \Deffunc{lua_getinfo}
\begin{verbatim}
int lua_getinfo (lua_State *L, const char *what, lua_Debug *ar);
\end{verbatim}
This function returns 0 on error
(e.g., an invalid option in \verb|what|).
Each character in the string \verb|what|
selects some fields of \verb|ar| to be filled,
as indicated by the letter in parentheses in the definition of \verb|lua_Debug|;
that is, an \verb|S| fills the fields \verb|source| and \verb|linedefined|,
and \verb|l| fills the field \verb|currentline|, etc.
We describe each field below:
\begin{description}

\item[source]
If the function was defined in a string,
\verb|source| is that string;
if the function was defined in a file,
\verb|source| starts with a \verb|@| followed by the file name.

\item[linedefined]
the line number where starts the definition of the function.

\item[what] the string \verb|"Lua"| if this is a Lua function,
\verb|"C"| if this is a C~function,
or \verb|"main"| if this is the main part of a chunk.

\item[currentline]
the current line where the given function is executing.
It only works if the function has been compiled with debug
information.
When no line information is available,
\verb|currentline| is set to \Math{-1}.

\item[name]
a reasonable name for the given function.
Because functions in Lua are first class values,
they do not have a fixed name:
Some functions may be the value of many global variables,
while others may be stored only in a table field.
The \verb|lua_getinfo| function checks whether the given
function is a tag method or the value of a global variable.
If the given function is a tag method,
then \verb|name| points to the event name.
If the given function is the value of a global variable,
then \verb|name| points to the variable name.
If the given function is neither a tag method nor a global variable,
then \verb|name| is set to \verb|NULL|.

\item[namewhat]
Explains the previous field.
If the function is a global variable,
\verb|namewhat| is \verb|"global"|;
if the function is a tag method,
\verb|namewhat| is \verb|"tag-method"|;
otherwise \verb|namewhat| is \verb|""| (the empty string).

\item[nups]
Number of upvalues of a C~function.
If the function is not a C~function,
\verb|nups| is set to 0.

\item[func]
The function being executed, as a \verb|lua_Object|.

\end{description}

The generation of debug information is controlled by an internal flag,
which can be switched with
\begin{verbatim}
int lua_setdebug (lua_State *L, int debug);
\end{verbatim}
This function sets the flag and returns its previous value.
This flag can also be set from Lua~\see{pragma}.
Setting the flag using \verb|lua_setdebug| affects all chunks that are
compiled afterwards.
Individual functions may still control the generation of debug information
by using \verb|$debug| or \verb|$nodebug|.

\subsection{Manipulating Local Variables}

For the manipulation of local variables,
\verb|luadebug.h| defines the following record:
\begin{verbatim}
struct lua_Localvar {
  int index;
  const char *name;
  lua_Object value;
};
\end{verbatim}
where \verb|index| is an index for local variables
(the first parameter has index 1, and so on,
until the last active local variable).

\Deffunc{lua_getlocal}\Deffunc{lua_setlocal}
The following functions allow the manipulation of the
local variables of a given activation record.
They only work if the function has been compiled with debug
information \see{pragma}.
For these functions, a local variable becomes
visible in the line after its definition.
\begin{verbatim}
int lua_getlocal (lua_State *L, const lua_Debug *ar, lua_Localvar *v);
int lua_setlocal (lua_State *L, const lua_Debug *ar, lua_Localvar *v);
\end{verbatim}
The parameter \verb|ar| must be a valid activation record,
filled by a previous call to \verb|lua_getstack| or
given as argument to a hook \see{sub-hooks}.
To use \verb|lua_getlocal|,
you fill the \verb|index| field of \verb|v| with the index
of a local variable; then the function fills the fields
\verb|name| and \verb|value| with the name and the current
value of that variable.
For \verb|lua_setlocal|,
you fill the \verb|index| and the \verb|value| fields of \verb|v|,
and the function assigns that value to the variable.
Both functions return 0 on failure, that happens
if the index is greater than the number of active local variables,
or if the activation record has no debug information.

As an example, the following function lists the names of all
local variables for a function in a given level of the stack:
\begin{verbatim}
int listvars (lua_State *L, int level) {
  lua_Debug ar;
  int i;
  if (lua_getstack(L, level, &ar) == 0)
    return 0;  /* failure: no such level on the stack */
  for (i=1; ; i++) {
    lua_Localvar v;
    v.index = i;
    if (lua_getlocal(L, &ar, &v) == 0)
      return 1;  /* no more locals, or no debug information */
    printf("%s\n", v.name);
  }
}
\end{verbatim}


\subsection{Hooks}\label{sub-hooks}

The Lua interpreter offers two hooks for debugging purposes:
a \emph{call} hook and a \emph{line} hook.
Both have the same type,
\begin{verbatim}
typedef void (*lua_Hook) (lua_State *L, lua_Debug *ar);
\end{verbatim}
and you can set them with the following functions:
\Deffunc{lua_Hook}\Deffunc{lua_setcallhook}\Deffunc{lua_setlinehook}
\begin{verbatim}
lua_Hook lua_setcallhook (lua_State *L, lua_Hook func);
lua_Hook lua_setlinehook (lua_State *L, lua_Hook func);
\end{verbatim}
A hook is disabled when its value is \verb|NULL|,
which is the initial value of both hooks.
The functions \verb|lua_setcallhook| and \verb|lua_setlinehook|
set their corresponding hooks and return their previous values.

The call hook is called whenever the
interpreter enters or leaves a function.
The \verb|event| field of \verb|ar| has the strings \verb|"call"|
or \verb|"return"|.
This \verb|ar| can then be used in calls to \verb|lua_getinfo|,
\verb|lua_getlocal|, and \verb|lua_setlocal|,
to get more information about the function and to manipulate its
local variables.

The line hook is called every time the interpreter changes
the line of code it is executing.
The \verb|event| field of \verb|ar| has the string \verb|"line"|,
and the \verb|currentline| field has the line number.
Again, you can use this \verb|ar| in other calls to the debug API.
This hook is called only if the active function
has been compiled with debug information~\see{pragma}.

While Lua is running a hook, it disables other calls to hooks.
Therefore, if a hook calls Lua to execute a function or a chunk,
this execution ocurrs without any calls to hooks.

A hook cannot call \T{lua_error}.
It must return to Lua through a regular return.
(There is no problem if the error is inside a chunk or a Lua function
called by the hook, because those errors are protected;
the control returns to the hook anyway.)


\subsection{The Reflexive Debug Interface}

The library \verb|ldblib| provides
the functionality of the debug interface to Lua programs.
If you want to use this library,
your host application must open it,
by calling \verb|lua_dblibopen|.

You should exert great care when using this library.
The functions provided here should be used exclusively for debugging
and similar tasks (e.g., profiling).
Please resist the temptation to use them as a
usual programming tool.
They are slow and violate some (otherwise) secure aspects of the
language (e.g., privacy of local variables).
As a general rule, if your program does not need this library,
do not open it.


\subsubsection*{\ff \T{getstack (level, [what])}}\Deffunc{getstack}

This function returns a table with information about the function
running at level \verb|level| of the stack.
Level 0 is the current function (\verb|getstack| itself);
level 1 is the function that called \verb|getstack|.
If \verb|level| is larger than the number of active functions,
the function returns \nil.
The table contains all the fields returned by \verb|lua_getinfo|,
with the string \verb|what| describing what to get.
The default for \rerb|what| is to get all information available.

For instance, the expression \verb|getstack(1,"n").name| returns
the name of the current function,
if a reasonable name can be found.


\subsubsection*{\ff \T{getlocal (level, local)}}\Deffunc{getlocal}

This function returns the name and the value of the local variable
with index \verb|local| of the function at level \verb|level| of the stack.
(The first parameter has index 1, and so on,
until the last active local variable.)
The function returns \nil\ if there is no local
variable with the given index,
and raises an error when called with a \verb|level| out of range.
(You can call \verb|getstack| to check wheter the level is valid.)

\subsubsection*{\ff \T{setlocal (level, local, value)}}\Deffunc{setlocal}

This function assigns the value \verb|value| to the local variable
with index \verb|local| of the function at level \verb|level| of the stack.
The function returns \nil\ if there is no local
variable with the given index,
and raises an error when called with a \verb|level| out of range.

\subsubsection*{\ff \T{setcallhook (hook)}}\Deffunc{setcallhook}

Sets the function \verb|hook| as the call hook;
this hook will be called every time the interpreter starts and
exits the execution of a function.
The only argument to this hook is the event name (\verb|"call"| or
\verb|"return"|).
You can call \verb|getstack| with level 2 to get more information about
the function being called or returning
(level 0 is the \verb|getstack| function,
and level 1 is the hook function).

When called without arguments,
this function turns off call hooks.

\subsubsection*{\ff \T{setlinehook (hook)}}\Deffunc{setlinehook}

Sets the function \verb|hook| as the line hook;
this hook will be called every time the interpreter changes
the line of code it is executing.
The only argument to the hook is the line number the interpreter
is about to execute.
This hook is called only if the active function
has been compiled with debug information~\see{pragma}.

When called without arguments,
this function turns off line hooks.


\section{\Index{Lua Stand-alone}} \label{lua-sa}

Although Lua has been designed as an extension language,
the language is frequently used as a stand-alone interpreter.
An implementation of such an interpreter,
called simply \verb|lua|,
is provided with the standard distribution.

This program can be called with any sequence of the following arguments:
\begin{description}
\item[\T{-}] executes \verb|stdin| as a file;
\item[\T{-c}] calls \verb|lua_close| after running all arguments;
\item[\T{-d}] turns on debug information;
\item[\T{-e} \rm\emph{stat}] executes string \verb|stat|;
\item[\T{-f filename}] executes file \verb|filename| with the
remaining arguments in table \verb|arg|;
\item[\T{-i}] enters interactive mode with prompt;
\item[\T{-q}] enters interactive mode without prompt;
\item[\T{-v}] prints version information;
\item[\T{var=value}] sets global \verb|var| to string \verb|"value"|;
\item[\T{filename}] executes file \verb|filename|.
\end{description}
When called without arguments,
Lua behaves as \verb|lua -v -i| when \verb|stdin| is a terminal,
and as \verb|lua -| otherwise.

All arguments are handled in order.
For instance, an invocation like
\begin{verbatim}
$ lua -i a=test prog.lua
\end{verbatim}
will first interact with the user until an \verb|EOF| in \verb|stdin|,
then will set \verb|a| to \verb|"test"|,
and finally will run the file \verb|prog.lua|.

When the option \T{-f filename} is used,
all following arguments from the command line
are passed to the Lua program in a table called \verb|arg|.
The field \verb|n| gets the index of the last argument,
and the field 0 gets the \T{filename}.
For instance, in the call
\begin{verbatim}
$ lua a.lua -f b.lua t1 t3
\end{verbatim}
the interpreter first runs the file \T{a.lua},
then creates a table \T{arg},
\begin{verbatim}
  arg = {"t1", "t3";  n = 2, [0] = "b.lua"}
\end{verbatim}
and then runs the file \T{b.lua}.
The stand-alone interpreter also provides a \verb|getarg| function that
can be used to access \emph{all} command line arguments.

In interactive mode,
a multi-line statement can be written finishing intermediate
lines with a backslash (\verb|\|).
If the global variable \verb|_PROMPT| is defined as a string,
its value is used as the prompt. \index{_PROMPT}
Therefore, the prompt can be changed like below:
\begin{verbatim}
$ lua _PROMPT='myprompt> ' -i
\end{verbatim}

In Unix systems, Lua scripts can be made into executable programs
by using \verb|chmod +x| and the~\verb|#!| form,
as in \verb|#!/usr/local/bin/lua|,
or \verb|#!/usr/local/bin/lua -f| to get other arguments.


\section*{Acknowledgments}

The authors would like to thank CENPES/PETROBRAS which,
jointly with \tecgraf, used extensively early versions of
this system and gave valuable comments.
The authors would also like to thank Carlos Henrique Levy,
who found the name of the game.
Lua means \emph{moon} in Portuguese.


\appendix

\section*{Incompatibilities with Previous Versions}

Although great care has been taken to avoid incompatibilities with
the previous public versions of Lua,
some differences had to be introduced.
Here is a list of all these incompatibilities.

\subsection*{Incompatibilities with \Index{version 3.2}}
\begin{itemize}

\item
General read patterns are now deprecated.
\item
Garbage-collection tag methods for tables is now deprecated.
\item
\verb|setglobal|, \verb|rawsetglobal|, and \verb|sort| no longer return a value;
\verb|type| no longer return a second value.
\item
In nested function calls like \verb|f(g(x))|
\emph{all} return values from \verb|g| are passed as arguments to \verb|f|.
(This only happens when \verb|g| is the last
[or the only] argument to \verb|f|.)
\item
There is now only one tag method for order operators.
\item
The debug API has been completely rewritten.
\item
The pre-compiler may use the fact that some operators are associative,
for optimizations.
This may cause problems if these operators
have non-associative tag methods.
\item
All functions from the old API are now macros.
\item
A \verb|const| qualifier has been added to \verb|char *|
in all API functions that handle C~strings.
\item
\verb|luaL_openlib| no longer automatically calls \verb|lua_open|.
So,
you must now explicitly call \verb|lua_open| before opening
the standard libraries.
\item
\verb|lua_type| now returns a string describing the type,
and is no longer a synonym for \verb|lua_tag|.
\item Old pre-compiled code is obsolete, and must be re-compiled.

\end{itemize}

%{===============================================================
\section*{The complete syntax of Lua}

\renewenvironment{Produc}{\vspace{0.8ex}\par\noindent\hspace{3ex}\it\begin{tabular}{rrl}}{\end{tabular}\vspace{0.8ex}\par\noindent}

\renewcommand{\OrNL}{\\ & \Or & }

\begin{Produc}

\produc{chunk}{\rep{stat} \opt{ret}}

\produc{block}{\opt{label} \rep{stat \opt{\ter{;}}}}

\produc{label}{\ter{$\vert$} name \ter{$\vert$}}

\produc{stat}{%
	varlist1 \ter{=} explist1
\OrNL	functioncall
\OrNL	\rwd{do} block \rwd{end}
\OrNL	\rwd{while} exp1 \rwd{do} block \rwd{end}
\OrNL	\rwd{repeat} block \rwd{until} exp1
\OrNL	\rwd{if} exp1 \rwd{then} block
	\rep{\rwd{elseif} exp1 \rwd{then} block}
	\opt{\rwd{else} block} \rwd{end}
\OrNL	\rwd{return} \opt{explist1}
\OrNL	\rwd{break} \opt{name}
\OrNL	\rwd{for} name \ter{=} exp1 \ter{,} exp1 \opt{\ter{,} exp1}
	\rwd{do} block \rwd{end}
\OrNL	\rwd{function} funcname \ter{(} \opt{parlist1} \ter{)} block \rwd{end}
\OrNL	\rwd{local} declist \opt{init}
}

\produc{var}{%
	name
\OrNL	simpleexp \ter{[} exp1 \ter{]}
\OrNL	simpleexp \ter{.} name
}

\produc{varlist1}{var \rep{\ter{,} var}}

\produc{declist}{name \rep{\ter{,} name}}

\produc{init}{\ter{=} explist1}

\produc{exp}{%
	\rwd{nil}
\Or	number
\Or	literal
\Or	function
\Or	simpleexp
\Or	\ter{(} exp \ter{)}
}

\produc{exp1}{exp}

\produc{explist1}{\rep{exp1 \ter{,}} exp}

\produc{simpleexp}{%
	var
\Or	upvalue
\Or	functioncall
\Or	tableconstructor
}

\produc{tableconstructor}{\ter{\{} fieldlist \ter{\}}}
\produc{fieldlist}{%
	lfieldlist
\Or	ffieldlist
\Or	lfieldlist \ter{;} ffieldlist
\Or	ffieldlist \ter{;} lfieldlist
}
\produc{lfieldlist}{\opt{lfieldlist1}}
\produc{ffieldlist}{\opt{ffieldlist1}}
\produc{lfieldlist1}{exp \rep{\ter{,} exp} \opt{\ter{,}}}
\produc{ffieldlist1}{ffield \rep{\ter{,} ffield} \opt{\ter{,}}}
\produc{ffield}{%
	\ter{[} exp \ter{]} \ter{=} exp
\Or	name \ter{=} exp
}

\produc{functioncall}{%
	simpleexp args
\Or	simpleexp \ter{:} name args
}

\produc{args}{%
	\ter{(} \opt{explist1} \ter{)}
\Or	tableconstructor
\Or	\ter{literal}
}

\produc{function}{\rwd{function} \ter{(} \opt{parlist1} \ter{)} block \rwd{end}}

\produc{funcname}{%
	name
\OrNL	name \ter{.} name
\OrNL	name \ter{:} name
}

\produc{parlist1} name}

\end{Produc}
%}===============================================================

% restore underscore to usual meaning
\catcode`\_=8

\newcommand{\indexentry}[2]{\item {#1} #2}
\begin{theindex}
% $Id: manual.tex,v 1.36 2000/04/17 19:23:48 roberto Exp roberto $

\documentclass[11pt]{article}
\usepackage{fullpage,bnf}
\usepackage{graphicx}
%\usepackage{times}

\catcode`\_=12

\newcommand{\See}[1]{Section~\ref{#1}}
\newcommand{\see}[1]{(see \See{#1})}
\newcommand{\M}[1]{\rm\emph{#1}}
\newcommand{\T}[1]{{\tt #1}}
\newcommand{\Math}[1]{$#1$}
\newcommand{\nil}{{\bf nil}}
\def\tecgraf{{\sf TeC\kern-.21em\lower.7ex\hbox{Graf}}}

\newcommand{\Index}[1]{#1\index{#1}}
\newcommand{\IndexVerb}[1]{\T{#1}\index{#1}}
\newcommand{\IndexEmph}[1]{\emph{#1}\index{#1}}
\newcommand{\Def}[1]{\emph{#1}\index{#1}}
\newcommand{\Deffunc}[1]{\index{#1}}

\newcommand{\ff}{$\bullet$\ }

\newcommand{\Version}{4.0}

% LHF
\renewcommand{\ter}[1]{{\rm`{\tt#1}'}}
\newcommand{\NOTE}{\par\noindent\emph{NOTE}: }

\makeindex

\begin{document}

%{===============================================================
\thispagestyle{empty}
\pagestyle{empty}

{
\parindent=0pt
\vglue1.5in
{\LARGE\bf
The Programming Language Lua}
\hfill
\vskip4pt \hrule height 4pt width \hsize \vskip4pt
\hfill
Reference Manual for Lua version \Version
\\
\null
\hfill
Last revised on \today
\\
\vfill
\centering
\includegraphics[width=0.7\textwidth]{nolabel.ps}
\vfill
\vskip4pt \hrule height 2pt width \hsize
}

\newpage
\begin{quotation}
\parskip=10pt
\footnotesize
\null\vfill

\noindent
Copyright \copyright\ 1994--2000 TeCGraf, PUC-Rio.  All rights reserved.

\noindent
Permission is hereby granted, without written agreement and without license
or royalty fees, to use, copy, modify, and distribute this software and its
documentation for any purpose, including commercial applications, subject to
the following conditions:
\begin{itemize}
\item The above copyright notice and this permission notice shall appear in all
   copies or substantial portions of this software.

\item The origin of this software must not be misrepresented; you must not
   claim that you wrote the original software. If you use this software in a
   product, an acknowledgment in the product documentation would be greatly
   appreciated (but it is not required).

\item Altered source versions must be plainly marked as such, and must not be
   misrepresented as being the original software.
\end{itemize}
The authors specifically disclaim any warranties, including, but not limited
to, the implied warranties of merchantability and fitness for a particular
purpose.  The software provided hereunder is on an ``as is'' basis, and the
authors have no obligation to provide maintenance, support, updates,
enhancements, or modifications.  In no event shall TeCGraf, PUC-Rio, or the
authors be held liable to any party for direct, indirect, special,
incidental, or consequential damages arising out of the use of this software
and its documentation.

\noindent
The Lua language and this implementation have been entirely designed and
written by Waldemar Celes, Roberto Ierusalimschy and Luiz Henrique de
Figueiredo at TeCGraf, PUC-Rio.

\noindent
This implementation contains no third-party code.

\noindent
Copies of this manual can be obtained at
\verb|http://www.tecgraf.puc-rio.br/lua/|.
\end{quotation}
%}===============================================================
\newpage

\title{Reference Manual of the Programming Language Lua \Version}

\author{%
Roberto Ierusalimschy\quad
Luiz Henrique de Figueiredo\quad
Waldemar Celes
\vspace{1.0ex}\\
\smallskip
\small\tt lua@tecgraf.puc-rio.br
\vspace{2.0ex}\\
%MCC 08/95 ---
\tecgraf\ --- Computer Science Department --- PUC-Rio
}

\date{{\small \tt\$Date: 2000/04/17 19:23:48 $ $}}

\maketitle

\thispagestyle{empty}
\pagestyle{empty}

\begin{abstract}
\noindent
Lua is a powerful, light-weight programming language
designed for extending applications.
Lua is also frequently used as a general-purpose, stand-alone language.
Lua combines simple procedural syntax
(similar to Pascal)
with
powerful data description constructs
based on associative arrays and extensible semantics.
Lua is
dynamically typed,
interpreted from bytecodes,
and has automatic memory management with garbage collection,
making it ideal for
configuration,
scripting,
and
rapid prototyping.

This document describes version \Version\ of the Lua programming language
and the API that allows interaction between Lua programs and their
host C programs.
\end{abstract}

\def\abstractname{Resumo}
\begin{abstract}
\noindent
Lua \'e uma linguagem de programa\c{c}\~ao
poderosa e leve,
projetada para extender aplica\c{c}\~oes.
Lua tamb\'em \'e frequentemente usada como uma linguagem de prop\'osito geral.
Lua combina programa\c{c}\~ao procedural
(com sintaxe semelhante \`a de Pascal)
com
poderosas constru\c{c}\~oes para descri\c{c}\~ao de dados,
baseadas em tabelas associativas e sem\^antica extens\'\i vel.
Lua \'e
tipada dinamicamente,
interpretada a partir de \emph{bytecodes},
e tem gerenciamento autom\'atico de mem\'oria com coleta de lixo.
Essas caracter\'{\i}sticas fazem de Lua uma linguagem ideal para
configura\c{c}\~ao,
automa\c{c}\~ao (\emph{scripting})
e prototipagem r\'apida.

Este documento descreve a vers\~ao \Version\ da linguagem de
programa\c{c}\~ao Lua e a Interface de Programa\c{c}\~ao (API) que permite
a intera\c{c}\~ao entre programas Lua e programas C hospedeiros.
\end{abstract}

\newpage
\null
\newpage
\tableofcontents

\newpage
\setcounter{page}{1}
\pagestyle{plain}


\section{Introduction}

Lua is an extension programming language designed to support
general procedural programming with data description
facilities.
Lua is intended to be used as a powerful, light-weight
configuration language for any program that needs one.

Lua is implemented as a library, written in C.
Being an extension language, Lua has no notion of a ``main'' program:
it only works \emph{embedded} in a host client,
called the \emph{embedding} program.
This host program can invoke functions to execute a piece of
code in Lua, can write and read Lua variables,
and can register C~functions to be called by Lua code.
Through the use of C~functions, Lua can be augmented to cope with
a wide range of different domains,
thus creating customized programming languages sharing a syntactical framework.

Lua is free-distribution software,
and provided as usual with no guarantees,
as stated in the copyright notice.
The implementation described in this manual is available
at the following URL's:
\begin{verbatim}
   http://www.tecgraf.puc-rio.br/lua/
   ftp://ftp.tecgraf.puc-rio.br/pub/lua/
\end{verbatim}

Like any other reference manual,
this document is dry in places.
For a discussion of the decisions behind the design of Lua,
see the papers below,
which are available at the web site above.
\begin{itemize}
\item
R.~Ierusalimschy, L.~H.~de Figueiredo, and W.~Celes.
Lua---an extensible extension language.
\emph{Software: Practice \& Experience} {\bf 26} \#6 (1996) 635--652.
\item
L.~H.~de Figueiredo, R.~Ierusalimschy, and W.~Celes.
The design and implementation of a language for extending applications.
\emph{Proceedings of XXI Brazilian Seminar on Software and Hardware} (1994) 273--283.
\item
L.~H.~de Figueiredo, R.~Ierusalimschy, and W.~Celes.
Lua: an extensible embedded language.
\emph{Dr. Dobb's Journal} {\bf  21} \#12 (Dec 1996) 26--33.
\end{itemize}

\section{Environment and Chunks}

All statements in Lua are executed in a \Def{global environment}.
This environment, which keeps all global variables,
is initialized with a call from the embedding program to
\verb|lua_newstate| and
persists until a call to \verb|lua_close|,
or the end of the embedding program.
Optionally, a user can create multiple independent global
environments, and freely switch between them \see{mangstate}.

The global environment can be manipulated by Lua code or
by the embedding program,
which can read and write global variables
using API functions from the library that implements Lua.

\Index{Global variables} do not need declaration.
Any variable is assumed to be global unless explicitly declared local
\see{localvar}.
Before the first assignment, the value of a global variable is \nil;
this default can be changed \see{tag-method}.

The unit of execution of Lua is called a \Def{chunk}.
A chunk is simply a sequence of statements:
\begin{Produc}
\produc{chunk}{\rep{stat} \opt{ret}}
\end{Produc}%
Statements are described in \See{stats}.
(The notation above is the usual extended BNF,
in which
\rep{\emph{a}} means 0 or more \emph{a}'s,
\opt{\emph{a}} means an optional \emph{a}, and
\oneormore{\emph{a}} means one or more \emph{a}'s.)

A chunk may be in a file or in a string inside the host program.
A chunk may optionally end with a \verb|return| statement \see{return}.
When a chunk is executed, first all its code is pre-compiled,
and then the statements are executed in sequential order.
All modifications a chunk effects on the global environment persist
after the chunk ends.

Chunks may also be pre-compiled into binary form;
see program \IndexVerb{luac} for details.
Text files with chunks and their binary pre-compiled forms
are interchangeable.
Lua automatically detects the file type and acts accordingly.
\index{pre-compilation}

\section{\Index{Types and Tags}} \label{TypesSec}

Lua is a \emph{dynamically typed language}.
This means that
variables do not have types; only values do.
Therefore, there are no type definitions in the language.
All values carry their own type.
Besides a type, all values also have a \IndexEmph{tag}.

There are six \Index{basic types} in Lua: \Def{nil}, \Def{number},
\Def{string}, \Def{function}, \Def{userdata}, and \Def{table}.
\emph{Nil} is the type of the value \nil,
whose main property is to be different from any other value.
\emph{Number} represents real (double-precision floating-point) numbers,
while \emph{string} has the usual meaning.
Lua is \Index{eight-bit clean},
and so strings may contain any 8-bit character,
\emph{including} embedded zeros (\verb|'\0'|) \see{lexical}.
The \verb|type| function returns a string describing the type
of a given value \see{pdf-type}.

Functions are considered \emph{first-class values} in Lua.
This means that functions can be stored in variables,
passed as arguments to other functions, and returned as results.
Lua can call (and manipulate) functions written in Lua and
functions written in C.
The kinds of functions can be distinguished by their tags:
all Lua functions have the same tag,
and all C~functions have the same tag,
which is different from the tag of Lua functions.
The \verb|tag| function returns the tag
of a given value \see{pdf-tag}.

The type \emph{userdata} is provided to allow
arbitrary \Index{C pointers} to be stored in Lua variables.
It corresponds to a \verb|void*| and has no pre-defined operations in Lua,
besides assignment and equality test.
However, by using \emph{tag methods},
the programmer can define operations for \emph{userdata} values
\see{tag-method}.

The type \emph{table} implements \Index{associative arrays},
that is, \Index{arrays} that can be indexed not only with numbers,
but with any value (except \nil).
Therefore, this type may be used not only to represent ordinary arrays,
but also symbol tables, sets, records, etc.
Tables are the main data structuring mechanism in Lua.
To represent \Index{records}, Lua uses the field name as an index.
The language supports this representation by
providing \verb|a.name| as syntactic sugar for \verb|a["name"]|.
Tables may also carry \emph{methods}:
Because functions are first class values,
table fields may contain functions.
The form \verb|t:f(x)| is syntactic sugar for \verb|t.f(t,x)|,
which calls the method \verb|f| from the table \verb|t| passing
itself as the first parameter \see{func-def}.

Note that tables are \emph{objects}, and not values.
Variables cannot contain tables, only \emph{references} to them.
Assignment, parameter passing, and returns always manipulate references
to tables, and do not imply any kind of copy.
Moreover, tables must be explicitly created before used
\see{tableconstructor}.

Tags are mainly used to select \emph{tag methods} when
some events occur.
Tag methods are the main mechanism for extending the
semantics of Lua \see{tag-method}.
Each of the types \M{nil}, \M{number}, and \M{string} has a different tag.
All values of each of these types have the same pre-defined tag.
Values of type \M{function} can have two different tags,
depending on whether they are Lua functions or C~functions.
Finally,
values of type \M{userdata} and \M{table} have
variable tags, assigned by the program \see{tag-method}.
Tags are created with the function \verb|newtag|,
and the function \verb|tag| returns the tag of a given value.
To change the tag of a given table,
there is the function \verb|settag| \see{pdf-newtag}.


\section{The Language}

This section describes the lexis, the syntax, and the semantics of Lua.


\subsection{Lexical Conventions} \label{lexical}

\IndexEmph{Identifiers} in Lua can be any string of letters,
digits, and underscores,
not beginning with a digit.
This coincides with the definition of identifiers in most languages,
except that
the definition of letter depends on the current locale:
Any character considered alphabetic by the current locale
can be used in an identifier.
The following words are \emph{reserved}, and cannot be used as identifiers:
\index{reserved words}
\begin{verbatim}
   and       break     do        else
   elseif    end       for       function
   if        local     nil       not
   or        repeat    return    then
   until     while
\end{verbatim}
Lua is a case-sensitive language:
\T{and} is a reserved word, but \T{And} and \T{\'and}
(if the locale permits) are two different, valid identifiers.
As a convention, identifiers starting with underscore followed by
uppercase letters (such as \verb|_INPUT|)
are reserved for internal variables.

The following strings denote other \Index{tokens}:
\begin{verbatim}
   ~=  <=  >=  <   >   ==  =   +   -   *   /   %
   (   )   {   }   [   ]   ;   ,   .   ..  ...
\end{verbatim}

\IndexEmph{Literal strings} can be delimited by matching single or double quotes,
and can contain the C-like escape sequences
\verb|'\a'| (bell),
\verb|'\b'| (backspace),
\verb|'\f'| (form feed),
\verb|'\n'| (newline),
\verb|'\r'| (carriage return),
\verb|'\t'| (horizontal tab),
\verb|'\v'| (vertical tab),
\verb|'\\'|, (backslash),
\verb|'\"'|, (double quote),
\verb|'\''| (single quote),
and \verb|'\|\emph{newline}\verb|'| (that is, a backslash followed by a real newline,
which  results in a newline in the string).
A character in a string may also be specified by its numerical value,
through the escape sequence \verb|'\ddd'|,
where \verb|ddd| is a sequence of up to three \emph{decimal} digits.
Strings in Lua may contain any 8-bit value, including embedded zeros,
which can be specified as \verb|'\000'|.

Literal strings can also be delimited by matching \verb|[[| \dots\ \verb|]]|.
Literals in this bracketed form may run for several lines,
may contain nested \verb|[[ ... ]]| pairs,
and do not interpret escape sequences.
This form is specially convenient for
writing strings that contain program pieces or
other quoted strings.
As an example, in a system using ASCII,
the following three literals are equivalent:
\begin{verbatim}
1) "alo\n123\""
2) '\97lo\10\04923"'
3) [[alo
   123"]]
\end{verbatim}


\Index{Comments} start anywhere outside a string with a
double hyphen (\verb|--|) and run until the end of the line.
Moreover,
the first line of a chunk is skipped if it starts with \verb|#|.
This facility allows the use of Lua as a script interpreter
in Unix systems \see{lua-sa}.

\Index{Numerical constants} may be written with an optional decimal part,
and an optional decimal exponent.
Examples of valid numerical constants are
\begin{verbatim}
   3     3.0     3.1416  314.16e-2   0.31416E1
\end{verbatim}

\subsection{The \Index{Pre-processor}} \label{pre-processor}

All lines that start with a \verb|$| sign are handled by a pre-processor.
The following directives are understood by the pre-processor:
\begin{description}
\item[\T{\$debug}] --- turn on debugging facilities \see{pragma}.
\item[\T{\$nodebug}] --- turn off debugging facilities \see{pragma}.
\item[\T{\$if \M{cond}}] --- start a conditional part.
If \M{cond} is false, then this part is skipped by the lexical analyzer.
\item[\T{\$ifnot \M{cond}}] --- start a conditional part.
If \M{cond} is true, then this part is skipped by the lexical analyzer.
\item[\T{\$end}] --- end a conditional part.
\item[\T{\$else}] --- start an ``else'' conditional part,
flipping the ``skip'' status.
\item[\T{\$endinput}] --- end the lexical parse of the chunk.
For all purposes,
it is as if the chunk physically ended at this point.
\end{description}

Directives may be freely nested.
In particular, a \verb|$endinput| may occur inside a \verb|$if|;
in that case, even the matching \verb|$end| is not parsed.

A \M{cond} part may be
\begin{description}
\item[\T{nil}] --- always false.
\item[\T{1}] --- always true.
\item[\T{\M{name}}] --- true if the value of the
global variable \M{name} is different from \nil.
Note that \M{name} is evaluated \emph{before} the chunk starts its execution.
Therefore, actions in a chunk do not affect its own conditional directives.
\end{description}

\subsection{\Index{Coercion}} \label{coercion}

Lua provides some automatic conversions between values at run time.
Any arithmetic operation applied to a string tries to convert
that string to a number, following the usual rules.
Conversely, whenever a number is used when a string is expected,
that number is converted to a string, in a reasonable format.
The format is chosen so that
a conversion from number to string then back to number
reproduces the original number \emph{exactly}.
Thus,
the conversion does not necessarily produces nice-looking text for some numbers.
For complete control on how numbers are converted to strings,
use the \verb|format| function \see{format}.


\subsection{\Index{Adjustment}} \label{adjust}

Functions in Lua can return many values.
Because there are no type declarations,
when a function is called
the system does not know how many values the function will return,
or how many parameters it needs.
Therefore, sometimes, a list of values must be \emph{adjusted}, at run time,
to a given length.
If there are more values than are needed,
then the excess values are thrown away.
If there are less values than are needed,
then the list is extended with as many  \nil's as needed.
This adjustment occurs in multiple assignments \see{assignment}
and function calls \see{functioncall}.


\subsection{Statements}\label{stats}

Lua supports an almost conventional set of \Index{statements},
similar to those in Pascal or C.
The conventional commands include
assignment, control structures, and procedure calls.
Non-conventional commands include table constructors
\see{tableconstructor}
and local variable declarations \see{localvar}.

\subsubsection{Blocks}
A \Index{block} is a list of statements, which are executed sequentially.
A statement may be have an optional \Index{label},
which is syntactically an identifier,
and can be optionally followed by a semicolon:
\begin{Produc}
\produc{block}{\opt{label} \rep{stat \opt{\ter{;}}}}
\produc{label}{\ter{$\vert$} name \ter{$\vert$}}
\end{Produc}%
\NOTE
For syntactic reasons, the \rwd{return} and
\rwd{break} statements can only be written
as the last statement of a block.

A block may be explicitly delimited:
\begin{Produc}
\produc{stat}{\rwd{do} block \rwd{end}}
\end{Produc}%
This is useful to control the scope of local variables \see{localvar},
and to add a \rwd{return} or \rwd{break} statement in the middle
of another block:
\begin{verbatim}
  do return end        -- return is the last statement in this block
\end{verbatim}

\subsubsection{\Index{Assignment}} \label{assignment}
The language allows \Index{multiple assignment}.
Therefore, the syntax for assignment
defines a list of variables on the left side
and a list of expressions on the right side.
Both lists have their elements separated by commas:
\begin{Produc}
\produc{stat}{varlist1 \ter{=} explist1}
\produc{varlist1}{var \rep{\ter{,} var}}
\end{Produc}%
This statement first evaluates all values on the right side
and eventual indices on the left side,
and then makes the assignments.
So
\begin{verbatim}
   i = 3
   i, a[i] = 4, 20
\end{verbatim}
sets \verb|a[3]| to 20, but does not affect \verb|a[4]|.

Multiple assignment can be used to exchange two values, as in
\begin{verbatim}
   x, y = y, x
\end{verbatim}

The two lists in a multiple assignment may have different lengths.
Before the assignment, the list of values is adjusted to
the length of the list of variables \see{adjust}.

A single name can denote a global variable, a local variable,
or a formal parameter:
\begin{Produc}
\produc{var}{name}
\end{Produc}%
Square brackets are used to index a table:
\begin{Produc}
\produc{var}{simpleexp \ter{[} exp1 \ter{]}}
\end{Produc}%
The \M{simpleexp} should result in a table value,
from where the field indexed by the expression \M{exp1}
value gets the assigned value.

The syntax \verb|var.NAME| is just syntactic sugar for
\verb|var["NAME"]|:
\begin{Produc}
\produc{var}{simpleexp \ter{.} name}
\end{Produc}%

The meaning of assignments and evaluations of global variables and
indexed variables can be changed by tag methods \see{tag-method}.
Actually,
an assignment \verb|x = val|, where \verb|x| is a global variable,
is equivalent to a call \verb|setglobal("x", val)|;
an assignment \verb|t[i] = val| is equivalent to
\verb|settable_event(t,i,val)|.
See \See{tag-method} for a complete description of these functions.
(The function \verb|setglobal| is pre-defined in Lua.
The function \T{settable\_event} is used only for explanatory purposes.)

\subsubsection{Control Structures}
The control structures 
\index{while-do}\index{repeat-until}\index{if-then-else}%
\T{if}, \T{while}, and \T{repeat} have the usual meaning and
familiar syntax:
\begin{Produc}
\produc{stat}{\rwd{while} exp1 \rwd{do} block \rwd{end}}
\produc{stat}{\rwd{repeat} block \rwd{until} exp1}
\produc{stat}{\rwd{if} exp1 \rwd{then} block
  \rep{\rwd{elseif} exp1 \rwd{then} block}
   \opt{\rwd{else} block} \rwd{end}}
\end{Produc}%
The \Index{condition expression} \M{exp1} of a control structure may return any value.
All values different from \nil\ are considered true;
only \nil\ is considered false.

\index{return}
The \rwd{return} statement is used to return values from a function or from a chunk.
\label{return}
Because functions or chunks may return more than one value,
the syntax for a \Index{return statement} is
\begin{Produc}
\produc{stat}{\rwd{return} \opt{explist1}}
\end{Produc}%

\index{break}
The \rwd{break} statement can be used to terminate the execution of a block,
skipping to the next statement after the block:
\begin{Produc}
\produc{stat}{\rwd{break} \opt{name}}
\end{Produc}%
A \rwd{break} without a label ends the innermost enclosing loop
(while, repeat, or for).
A \rwd{break} with a label breaks the innermost enclosing
statement with that label.
Thus,
labels do not have to be unique.

For syntactic reasons, \rwd{return} and \rwd{break}
statements can only be written as the last statement of a block.

\subsubsection{For Statement} \label{for}\index{for}

The \rwd{for} statement has the following syntax:
\begin{Produc}
\produc{stat}{\rwd{for} name \ter{=} exp1 \ter{,} exp1 \opt{\ter{,} exp1}
                    \rwd{do} block \rwd{end}}
\end{Produc}%
A \rwd{for} statement like
\begin{verbatim}
   for var=e1,e2,e3 do block end
\end{verbatim}
is equivalent to the following code:
\begin{verbatim}
   do
     local var, _limit, _step = tonumber(e1), tonumber(e2), tonumber(e3)
     if not (var and _limit and _step) then error() end
     while (_step>0 and var<=_limit) or (_step<=0 and var>=_limit) do
       block
       var = var+_step
     end
   end
\end{verbatim}
Notice the following:
\begin{itemize}\itemsep=0pt
\item \verb|_limit| and \verb|_step| are invisible variables.
The names are here for explanatory purposes only.
\item The behavior is \emph{undefined} if you assign to \verb|var| inside
the block.
\item If the third expression (the step) is absent, then a step of 1 is used.
\item Both the limit and the step are evaluated only once,
before the loop starts.
\item The variable \verb|var| is local to the statement;
you cannot use its value after the \rwd{for} ends.
\item You can use \rwd{break} to exit a \rwd{for}.
If you need the value of the index,
then assign it to another variable before breaking.
\end{itemize}

\subsubsection{Function Calls as Statements} \label{funcstat}
Because of possible side-effects,
function calls can be executed as statements:
\begin{Produc}
\produc{stat}{functioncall}
\end{Produc}%
In this case, all returned values are thrown away.
Function calls are explained in \See{functioncall}.

\subsubsection{Local Declarations} \label{localvar}
\Index{Local variables} may be declared anywhere inside a block.
The declaration may include an initial assignment:
\begin{Produc}
\produc{stat}{\rwd{local} declist \opt{init}}
\produc{declist}{name \rep{\ter{,} name}}
\produc{init}{\ter{=} explist1}
\end{Produc}%
If present, an initial assignment has the same semantics
of a multiple assignment.
Otherwise, all variables are initialized with \nil.

The scope of local variables begins \emph{after}
the declaration and lasts until the end of the block.
Thus, the code
\verb|local print=print|
creates a local variable called \verb|print| whose
initial value is that of the \emph{global} variable of the same name.


\subsection{\Index{Expressions}}

\subsubsection{\Index{Basic Expressions}}
The basic expressions in Lua are
\begin{Produc}
\produc{exp}{\ter{(} exp \ter{)}}
\produc{exp}{\rwd{nil}}
\produc{exp}{number}
\produc{exp}{literal}
\produc{exp}{function}
\produc{exp}{simpleexp}
\end{Produc}%
\begin{Produc}
\produc{simpleexp}{var}
\produc{simpleexp}{upvalue}
\produc{simpleexp}{functioncall}
\produc{simpleexp}{tableconstructor}
\end{Produc}%

Numbers (numerical constants) and
literal strings are explained in \See{lexical};
variables are explained in \See{assignment};
upvalues are explained in \See{upvalue};
function definitions (\M{function}) are explained in \See{func-def};
function calls are explained in \See{functioncall}.
Table constructors are explained in \See{tableconstructor}.

An access to a global variable \verb|x| is equivalent to a
call \verb|getglobal("x")|;
an access to an indexed variable \verb|t[i]| is equivalent to
a call \verb|gettable_event(t,i)|.
See \See{tag-method} for a description of these functions.
(Function \verb|getglobal| is pre-defined in Lua.
Function \T{gettable\_event} is used only for explanatory purposes.)

The non-terminal \M{exp1} is used to indicate that the values
returned by an expression must be adjusted to one single value:
\begin{Produc}
\produc{exp1}{exp}
\end{Produc}%

\subsubsection{Arithmetic Operators}
Lua supports the usual \Index{arithmetic operators}:
the binary \verb|+| (addition),
\verb|-| (subtraction), \verb|*| (multiplication),
\verb|/| (division) and \verb|^| (exponentiation),
and unary \verb|-| (negation).
If the operands are numbers, or strings that can be converted to
numbers (according to the rules given in \See{coercion}),
then all operations except exponentiation have the usual meaning.
Otherwise, an appropriate tag method is called \see{tag-method}.
An exponentiation always calls a tag method.
The standard mathematical library redefines this method for numbers,
giving the expected meaning to \Index{exponentiation}
\see{mathlib}.

\subsubsection{Relational Operators}
Lua provides the following \Index{relational operators}:
\begin{verbatim}
   ==  ~=  <   >   <=  >=
\end{verbatim}
All these return \nil\ as false and a value different from \nil\ as true.

Equality first compares the tags of its operands.
If they are different, then the result is \nil.
Otherwise, their values are compared.
Numbers and strings are compared in the usual way.
Tables, userdata, and functions are compared by reference,
that is, two tables are considered equal only if they are the \emph{same} table.
The operator \verb|~=| is exactly the negation of equality (\verb|==|).

\NOTE
The conversion rules of \See{coercion}
\emph{do not} apply to equality comparisons.
Thus, \verb|"0"==0| evaluates to \emph{false},
and \verb|t[0]| and \verb|t["0"]| denote different
entries in a table.

The order operators work as follows.
If both arguments are numbers, then they are compared as such.
Otherwise, if both arguments are strings,
then their values are compared using lexicographical order.
Otherwise, the ``lt'' tag method is called \see{tag-method}.

\subsubsection{Logical Operators}
The \Index{logical operators} are
\index{and}\index{or}\index{not}
\begin{verbatim}
   and   or   not
\end{verbatim}
Like control structures, all logical operators
consider \nil\ as false and anything else as true.
The conjunction operator \verb|and| returns \nil\ if its first argument is \nil;
otherwise, it returns its second argument.
The disjunction operator \verb|or| returns its first argument
if it is different from \nil;
otherwise, it returns its second argument.
Both \verb|and| and \verb|or| use \Index{short-cut evaluation},
that is,
the second operand is evaluated only when necessary.

There are two useful Lua idioms with logical operators.
The first idiom is \verb|x = x or v|,
which is equivalent to
\begin{verbatim}
      if x == nil then x = v end
\end{verbatim}
i.e., it sets \verb|x| to a default value \verb|v| when
\verb|x| is not set.
The other idiom is \verb|x = a and b or c|,
which should be read as \verb|x = a and (b or c)|,
is equivalent to
\begin{verbatim}
   if a then x = b else x = c end
\end{verbatim}
provided that \verb|b| is not \nil.

\subsubsection{Concatenation}
The string \Index{concatenation} operator in Lua is
denoted by ``\IndexVerb{..}''.
If both operands are strings or numbers, they are converted to
strings according to the rules in \See{coercion}.
Otherwise, the ``concat'' tag method is called \see{tag-method}.

\subsubsection{Precedence}
\Index{Operator precedence} follows the table below,
from the lower to the higher priority:
\begin{verbatim}
   and   or
   <   >   <=  >=  ~=  ==
   ..
   +   -
   *   /
   not  - (unary)
   ^
\end{verbatim}
All binary operators are left associative,
except for \verb|^| (exponentiation),
which is right associative.
\NOTE
The pre-compiler may rearrange the order of evaluation of
associative operators (such as~\verb|..| or~\verb|+|),
as long as these optimizations do not change normal results.
However, these optimizations may change some results
if you define non-associative
tag methods for these operators.

\subsubsection{Table Constructors} \label{tableconstructor}
Table \Index{constructors} are expressions that create tables;
every time a constructor is evaluated, a new table is created.
Constructors can be used to create empty tables,
or to create a table and initialize some fields.
The general syntax for constructors is
\begin{Produc}
\produc{tableconstructor}{\ter{\{} fieldlist \ter{\}}}
\produc{fieldlist}{lfieldlist \Or ffieldlist \Or lfieldlist \ter{;} ffieldlist
	\Or ffieldlist \ter{;} lfieldlist}
\produc{lfieldlist}{\opt{lfieldlist1}}
\produc{ffieldlist}{\opt{ffieldlist1}}
\end{Produc}%

The form \emph{lfieldlist1} is used to initialize lists:
\begin{Produc}
\produc{lfieldlist1}{exp \rep{\ter{,} exp} \opt{\ter{,}}}
\end{Produc}%
The expressions in the list are assigned to consecutive numerical indices,
starting with 1.
For example,
\begin{verbatim}
   a = {"v1", "v2", 34}
\end{verbatim}
is equivalent to
\begin{verbatim}
  do
    local temp = {}
    temp[1] = "v1"
    temp[2] = "v2"
    temp[3] = 34
    a = temp
  end
\end{verbatim}

The form \emph{ffieldlist1} initializes other fields in a table:
\begin{Produc}
\produc{ffieldlist1}{ffield \rep{\ter{,} ffield} \opt{\ter{,}}}
\produc{ffield}{\ter{[} exp \ter{]} \ter{=} exp \Or name \ter{=} exp}
\end{Produc}%
For example,
\begin{verbatim}
   a = {[f(k)] = g(y), x = 1, y = 3, [0] = b+c}
\end{verbatim}
is equivalent to
\begin{verbatim}
  do
    local temp = {}
    temp[f(k)] = g(y)
    temp.x = 1    -- or temp["x"] = 1
    temp.y = 3    -- or temp["y"] = 3
    temp[0] = b+c
    a = temp
  end
\end{verbatim}
An expression like \verb|{x = 1, y = 4}| is
in fact syntactic sugar for \verb|{["x"] = 1, ["y"] = 4}|.

Both forms may have an optional trailing comma,
and can be used in the same constructor separated by
a semi-collon.
For example, all forms below are correct.
\begin{verbatim}
   x = {;}
   x = {"a", "b",}
   x = {type="list"; "a", "b"}
   x = {f(0), f(1), f(2),; n=3,}
\end{verbatim}

\subsubsection{Function Calls}  \label{functioncall}
A \Index{function call} has the following syntax:
\begin{Produc}
\produc{functioncall}{simpleexp args}
\end{Produc}%
First, \M{simpleexp} is evaluated.
If its value has type \emph{function},
then this function is called,
with the given arguments.
Otherwise, the ``function'' tag method is called,
having as first parameter the value of \M{simpleexp},
and then the original call arguments.

The form
\begin{Produc}
\produc{functioncall}{simpleexp \ter{:} name args}
\end{Produc}%
can be used to call ``methods''.
A call \verb|simpleexp:name(...)|
is syntactic sugar for
\begin{verbatim}
  simpleexp.name(simpleexp, ...)
\end{verbatim}
except that \verb|simpleexp| is evaluated only once.

Arguments have the following syntax:
\begin{Produc}
\produc{args}{\ter{(} \opt{explist1} \ter{)}}
\produc{args}{tableconstructor}
\produc{args}{\ter{literal}}
\produc{explist1}{\rep{exp1 \ter{,}} exp}
\end{Produc}%
All argument expressions are evaluated before the call.
A call of the form \verb|f{...}| is syntactic sugar for
\verb|f({...})|, that is,
the parameter list is a single new table.
A call of the form \verb|f'...'|
(or \verb|f"..."| or \verb|f[[...]]|) is syntactic sugar for
\verb|f('...')|, that is,
the parameter list is a single literal string.

Because a function can return any number of results
\see{return},
the number of results must be adjusted before used.
If the function is called as a statement \see{funcstat},
then its return list is adjusted to~0,
thus discarding all returned values.
If the function is called in a place that needs a single value
(syntactically denoted by the non-terminal \M{exp1}),
then its return list is adjusted to~1,
thus discarding all returned values but the first one.
If the function is called in a place that can hold many values
(syntactically denoted by the non-terminal \M{exp}),
then no adjustment is made.
The only places that can hold many values
is the last (or the only) expression in an assignment,
in an argument list, or in a return statement.
Here are some examples.
\begin{verbatim}
   f();               -- adjusted to 0
   g(f(), x);         -- f() is adjusted to 1 result
   g(x, f());         -- g gets x plus all values returned by f()
   a,b,c = f(), x;    -- f() is adjusted to 1 result (and c gets nil)
   a,b,c = x, f();    -- f() is adjusted to 2
   a,b,c = f();       -- f() is adjusted to 3
   return f();        -- returns all values returned by f()
   return x,y,f();    -- returns a, b, and all values returned by f()
\end{verbatim}

\subsubsection{\Index{Function Definitions}} \label{func-def}

The syntax for function definition is
\begin{Produc}
\produc{function}{\rwd{function} \ter{(} \opt{parlist1} \ter{)}
  block \rwd{end}}
\produc{stat}{\rwd{function} funcname \ter{(} \opt{parlist1} \ter{)}
  block \rwd{end}}
\produc{funcname}{name \Or name \ter{.} name \Or name \ter{:} name}
\end{Produc}%
The statement
\begin{verbatim}
      function f ()
        ...
      end
\end{verbatim}
is just syntactic sugar for
\begin{verbatim}
      f = function ()
            ...
          end
\end{verbatim}
and the statement
\begin{verbatim}
      function o.f ()
        ...
      end
\end{verbatim}
is syntactic sugar for
\begin{verbatim}
      o.f = function ()
              ...
            end
\end{verbatim}

A function definition is an executable expression,
whose value has type \emph{function}.
When Lua pre-compiles a chunk,
all its function bodies are pre-compiled, too.
Then, whenever Lua executes the function definition,
its upvalues are fixed \see{upvalue},
and the function is \emph{instantiated} (or \emph{closed}).
This function instance (or \emph{closure})
is the final value of the expression.
Different instances of the same function
may have different upvalues.

Parameters act as local variables,
initialized with the argument values:
\begin{Produc}
\produc{parlist1}{\ter{\ldots}}
\produc{parlist1}{name \rep{\ter{,} name} \opt{\ter{,} \ter{\ldots}}}
\end{Produc}%
\label{vararg}
When a function is called,
the list of \Index{arguments} is adjusted to
the length of the list of parameters \see{adjust},
unless the function is a \Def{vararg} function,
which is
indicated by the dots (\ldots) at the end of its parameter list.
A vararg function does not adjust its argument list;
instead, it collects all extra arguments into an implicit parameter,
called \IndexVerb{arg}.
The value of \verb|arg| is a table,
with a field~\verb|n| whose value is the number of extra arguments,
and the extra arguments at positions 1,~2,~\ldots,\M{n}.

As an example, consider the following definitions:
\begin{verbatim}
   function f(a, b) end
   function g(a, b, ...) end
   function r() return 1,2,3 end
\end{verbatim}
Then, we have the following mapping from arguments to parameters:
\begin{verbatim}
   CALL            PARAMETERS

   f(3)             a=3, b=nil
   f(3, 4)          a=3, b=4
   f(3, 4, 5)       a=3, b=4
   f(r(), 10)       a=1, b=10
   f(r())           a=1, b=2

   g(3)             a=3, b=nil, arg={n=0}
   g(3, 4)          a=3, b=4, arg={n=0}
   g(3, 4, 5, 8)    a=3, b=4, arg={5, 8; n=2}
   g(5, r())        a=5, b=1, arg={2, 3; n=2}
\end{verbatim}

Results are returned using the \verb|return| statement \see{return}.
If control reaches the end of a function
without encountering a \rwd{return} statement,
then the function returns with no results.

The syntax
\begin{Produc}
\produc{funcname}{name \ter{:} name}
\end{Produc}%
is used for defining \Index{methods},
that is, functions that have an implicit extra parameter \IndexVerb{self}:
Thus, the statement
\begin{verbatim}
      function v:f (...)
        ...
      end
\end{verbatim}
is equivalent to
\begin{verbatim}
      v.f = function (self, ...)
        ...
      end
\end{verbatim}
that is, the function gets an extra formal parameter called \verb|self|.
Note that the variable \verb|v| must have been
previously initialized with a table value.


\subsection{Visibility and Upvalues} \label{upvalue}
\index{Visibility} \index{Upvalues}

A function body may refer to its own local variables
(which include its parameters) and to global variables,
as long as they are not \emph{shadowed} by local
variables from enclosing functions.
A function \emph{cannot} access a local
variable from an enclosing function,
since such variables may no longer exist when the function is called.
However, a function may access the \emph{value} of a local variable
from an enclosing function, using \emph{upvalues},
whose syntax is
\begin{Produc}
\produc{upvalue}{\ter{\%} name}
\end{Produc}%
An upvalue is somewhat similar to a variable expression,
but whose value is \emph{frozen} when the function wherein it
appears is instantiated.
The name used in an upvalue may be the name of any variable visible
at the point where the function is defined,
that is
global variables and local variables from the immediately enclosing function.

Here are some examples:
\begin{verbatim}
      a,b,c = 1,2,3   -- global variables
      local d
      function f (x)
        local b       -- x and b are local to f; b shadows the global b
        local g = function (a)
          local y     -- a and y are local to g
          p = a       -- OK, access local 'a'
          p = c       -- OK, access global 'c'
          p = b       -- ERROR: cannot access a variable in outer scope
          p = %b      -- OK, access frozen value of 'b' (local to 'f')
          p = %c      -- OK, access frozen value of global 'c'
          p = %y      -- ERROR: 'y' is not visible where 'g' is defined
          p = %d      -- ERROR: 'd' is not visible where 'g' is defined
        end           -- g
      end             -- f
\end{verbatim}


\subsection{Error Handling} \label{error}

Because Lua is an extension language,
all Lua actions start from C~code in the host program
calling a function from the Lua library.
Whenever an error occurs during Lua compilation or execution,
the function \verb|_ERRORMESSAGE| is called \Deffunc{_ERRORMESSAGE}
(provided it is different from \nil),
and then the corresponding function from the library
(\verb|lua_dofile|, \verb|lua_dostring|,
\verb|lua_dobuffer|, or \verb|lua_callfunction|)
is terminated, returning an error condition.

The only argument to \verb|_ERRORMESSAGE| is a string
describing the error.
The default definition for
this function calls \verb|_ALERT|, \Deffunc{_ALERT}
which prints the message to \verb|stderr| \see{alert}.
The standard I/O library redefines \verb|_ERRORMESSAGE|,
and uses the debug facilities \see{debugI}
to print some extra information,
such as a call stack traceback.

To provide more information about errors,
Lua programs should include the compilation pragma \verb|$debug|,
\index{debug pragma}\label{pragma}
or be loaded from the host after calling \verb|lua_setdebug(1)|
\see{debugI}.
When an error occurs in a chunk compiled with this option,
the I/O error-message routine is able to print the number of the
lines where the calls (and the error) were made.

Lua code can explicitly generate an error by calling the built-in
function \verb|error| \see{pdf-error}.
Lua code can ``catch'' an error using the built-in function
\verb|call| \see{pdf-call}.


\subsection{Tag Methods} \label{tag-method}

Lua provides a powerful mechanism to extend its semantics,
called \Def{tag methods}.
A tag method is a programmer-defined function
that is called at specific key points during the evaluation of a program,
allowing the programmer to change the standard Lua behavior at these points.
Each of these points is called an \Def{event}.

The tag method called for any specific event is selected
according to the tag of the values involved
in the event \see{TypesSec}.
The function \IndexVerb{settagmethod} changes the tag method
associated with a given pair \M{(tag, event)}.
Its first parameter is the tag, the second parameter is the event name
(a string; see below),
and the third parameter is the new method (a function),
or \nil\ to restore the default behavior for the pair.
The \verb|settagmethod| function returns the previous tag method for that pair.
Another function, \IndexVerb{gettagmethod},
receives a tag and an event name and returns the
current method associated with the pair.

Tag methods are called in the following events,
identified by the given names.
The semantics of tag methods is better explained by a Lua function
describing the behavior of the interpreter at each event.
This function not only shows when a tag method is called,
but also its arguments, its results, and the default behavior.
The code shown here is only \emph{illustrative};
the real behavior is hard coded in the interpreter,
and it is much more efficient than this simulation.
All functions used in these descriptions
(\verb|rawgetglobal|, \verb|tonumber|, \verb|call|, etc.)
are described in \See{predefined}.

\begin{description}

\item[``add'':]\index{add event}
called when a \verb|+| operation is applied to non numerical operands.

The function \verb|getbinmethod| defines how Lua chooses a tag method
for a binary operation.
First, Lua tries the first operand.
If its tag does not define a tag method for the operation,
then Lua tries the second operand.
If it also fails, then it gets a tag method from tag~0:
\begin{verbatim}
      function getbinmethod (op1, op2, event)
        return gettagmethod(tag(op1), event) or
               gettagmethod(tag(op2), event) or
               gettagmethod(0, event)
      end
\end{verbatim}
Using this function,
the tag method for the ``add' event is
\begin{verbatim}
      function add_event (op1, op2)
        local o1, o2 = tonumber(op1), tonumber(op2)
        if o1 and o2 then  -- both operands are numeric
          return o1+o2  -- '+' here is the primitive 'add'
        else  -- at least one of the operands is not numeric
          local tm = getbinmethod(op1, op2, "add")
          if tm then
            -- call the method with both operands and an extra
            -- argument with the event name
            return tm(op1, op2, "add")
          else  -- no tag method available: default behavior
            error("unexpected type at arithmetic operation")
          end
        end
      end
\end{verbatim}

\item[``sub'':]\index{sub event}
called when a \verb|-| operation is applied to non numerical operands.
Behavior similar to the ``add'' event.

\item[``mul'':]\index{mul event}
called when a \verb|*| operation is applied to non numerical operands.
Behavior similar to the ``add'' event.

\item[``div'':]\index{div event}
called when a \verb|/| operation is applied to non numerical operands.
Behavior similar to the ``add'' event.

\item[``pow'':]\index{pow event}
called when a \verb|^| operation (exponentiation) is applied.
\begin{verbatim}
      function pow_event (op1, op2)
        local tm = getbinmethod(op1, op2, "pow")
        if tm then
          -- call the method with both operands and an extra
          -- argument with the event name
          return tm(op1, op2, "pow")
        else  -- no tag method available: default behavior
          error("unexpected type at arithmetic operation")
        end
      end
\end{verbatim}

\item[``unm'':]\index{unm event}
called when a unary \verb|-| operation is applied to a non numerical operand.
\begin{verbatim}
      function unm_event (op)
        local o = tonumber(op)
        if o then  -- operand is numeric
          return -o  -- '-' here is the primitive 'unm'
        else  -- the operand is not numeric.
          -- Try to get a tag method from the operand;
          --  if it does not have one, try a "global" one (tag 0)
          local tm = gettagmethod(tag(op), "unm") or
                     gettagmethod(0, "unm")
          if tm then
            -- call the method with the operand, nil, and an extra
            -- argument with the event name
            return tm(op, nil, "unm")
          else  -- no tag method available: default behavior
            error("unexpected type at arithmetic operation")
          end
        end
      end
\end{verbatim}

\item[``lt'':]\index{lt event}
called when an order operation is applied to non-numerical
or non-string operands.
It corresponds to the \verb|<| operator.
\begin{verbatim}
      function lt_event (op1, op2)
        if type(op1) == "number" and type(op2) == "number" then
          return op1 < op2   -- numeric comparison
        elseif type(op1) == "string" and type(op2) == "string" then
          return op1 < op2   -- lexicographic comparison
        else
          local tm = getbinmethod(op1, op2, "lt")
          if tm then
            return tm(op1, op2, "lt")
          else
            error("unexpected type at comparison");
          end
        end
      end
\end{verbatim}
The other order operators use this tag method according to the
usual equivalences:
\begin{verbatim}
   a>b    <=>  b<a
   a<=b   <=>  not (b<a)
   a>=b   <=>  not (a<b)
\end{verbatim}

\item[``concat'':]\index{concatenation event}
called when a concatenation is applied to non string operands.
\begin{verbatim}
      function concat_event (op1, op2)
        if (type(op1) == "string" or type(op1) == "number") and
           (type(op2) == "string" or type(op2) == "number") then
          return op1..op2  -- primitive string concatenation
        else
          local tm = getbinmethod(op1, op2, "concat")
          if tm then
            return tm(op1, op2, "concat")
          else
            error("unexpected type for concatenation")
          end
        end
      end
\end{verbatim}

\item[``index'':]\index{index event}
called when Lua tries to retrieve the value of an index
not present in a table.
See event ``gettable'' for its semantics.

\item[``getglobal'':]\index{getglobal event}
called whenever Lua needs the value of a global variable.
This method can only be set for \nil\ and for tags
created by \verb|newtag|.
Note that
the tag is that of the \emph{current value} of the global variable.
\begin{verbatim}
      function getglobal (varname)
        local value = rawgetglobal(varname)
        local tm = gettagmethod(tag(value), "getglobal")
        if not tm then
          return value
        else
          return tm(varname, value)
        end
      end
\end{verbatim}
The function \verb|getglobal| is pre-defined in Lua \see{predefined}.

\item[``setglobal'':]\index{setglobal event}
called whenever Lua assigns to a global variable.
This method cannot be set for numbers, strings, and tables and
userdata with default tags.
\begin{verbatim}
      function setglobal (varname, newvalue)
        local oldvalue = rawgetglobal(varname)
        local tm = gettagmethod(tag(oldvalue), "setglobal")
        if not tm then
          rawsetglobal(varname, newvalue)
        else
          tm(varname, oldvalue, newvalue)
        end
      end
\end{verbatim}
The function \verb|setglobal| is pre-defined in Lua \see{predefined}.

\item[``gettable'':]\index{gettable event}
called whenever Lua accesses an indexed variable.
This method cannot be set for tables with default tag.
\begin{verbatim}
      function gettable_event (table, index)
        local tm = gettagmethod(tag(table), "gettable")
        if tm then
          return tm(table, index)
        elseif type(table) ~= "table" then
          error("indexed expression not a table");
        else
          local v = rawgettable(table, index)
          tm = gettagmethod(tag(table), "index")
          if v == nil and tm then
            return tm(table, index)
          else
            return v
          end
        end
      end
\end{verbatim}

\item[``settable'':]\index{settable event}
called when Lua assigns to an indexed variable.
This method cannot be set for tables with default tag.
\begin{verbatim}
      function settable_event (table, index, value)
        local tm = gettagmethod(tag(table), "settable")
        if tm then
          tm(table, index, value)
        elseif type(table) ~= "table" then
          error("indexed expression not a table")
        else
          rawsettable(table, index, value)
        end
      end
\end{verbatim}

\item[``function'':]\index{function event}
called when Lua tries to call a non function value.
\begin{verbatim}
      function function_event (func, ...)
        if type(func) == "function" then
          return call(func, arg)
        else
          local tm = gettagmethod(tag(func), "function")
          if tm then
            for i=arg.n,1,-1 do
              arg[i+1] = arg[i]
            end
            arg.n = arg.n+1
            arg[1] = func
            return call(tm, arg)
          else
            error("call expression not a function")
          end
        end
      end
\end{verbatim}

\item[``gc'':]\index{gc event}
called when Lua is ``garbage collecting'' a userdata.
This tag method can be set only from~C,
and cannot be set for a userdata with default tag.
For each userdata to be collected,
Lua does the equivalent of the following function:
\begin{verbatim}
      function gc_event (obj)
        local tm = gettagmethod(tag(obj), "gc")
        if tm then
          tm(obj)
        end
      end
\end{verbatim}
Moreover, at the end of a garbage collection cycle,
Lua does the equivalent of the call \verb|gc_event(nil)|.

\end{description}




\section{The Application Program Interface}

This section describes the API for Lua, that is,
the set of C~functions available to the host program to communicate
with the Lua library.
The API functions can be classified into the following categories:
\begin{enumerate}
\item managing states;
\item exchanging values between C and Lua;
\item executing Lua code;
\item manipulating (reading and writing) Lua objects;
\item calling Lua functions;
\item defining C~functions to be called by Lua;
\item manipulating references to Lua Objects.
\end{enumerate}
All API functions and related types and constants
are declared in the header file \verb|lua.h|.

\NOTE
Even when we use the term \emph{function},
\emph{any facility in the API may be provided as a macro instead}.
All such macros use each of its arguments exactly once,
and so do not generate hidden side-effects.


\subsection{States} \label{mangstate}

The Lua library is reentrant:
it does not have any global variable.
The whole state of the Lua interpreter
(global variables, stack, tag methods, etc.)
is stored in a dynamic structure; \Deffunc{lua_State}
this state must be passed as the first argument to almost
every function in the library.

Before calling any API function,
you must create a state.
This is done by calling\Deffunc{lua_newstate}
\begin{verbatim}
lua_State *lua_newstate (const char *s, ...);
\end{verbatim}
The arguments to this function form a list of name-value options,
terminated with \verb|NULL|.
Currently, the function accepts the following options:
\begin{itemize}
\item \verb|"stack"| --- the stack size.
Each function call needs one stack position for each local variable
and temporary variables, plus one position for book-keeping.
The stack must also have at least ten extra positions available.
For very small implementations, without recursive functions,
a stack size of 100 should be enough.
The default stack size is 1024.

\item \verb|"builtin"| --- the value is a boolean (0 or 1) that
indicates whether the predefined functions should be loaded or not.
The default is to load those functions.
\end{itemize}
For instance, the call
\begin{verbatim}
lua_State *L = lua_newstate(NULL);
\end{verbatim}
creates a new state with a stack of 1024 positions
and with the predefined functions loaded;
the call
\begin{verbatim}
lua_State *L = lua_newstate("builtin", 0, "stack", 100, NULL);
\end{verbatim}
creates a new state with a stack of 100 positions,
without the predefined functions.

To release a state, you call
\begin{verbatim}
void lua_close (lua_State *L);
\end{verbatim}
This function destroys all objects in the current Lua environment
(calling the corresponding garbage collection tag methods)
and frees all dynamic memory used by the state.
Usually, you do not need to call this function,
because all resources are naturally released when the program ends.
On the other hand,
long-running programs ---
like a daemon or web server, for example ---
might need to release states as soon as they are not needed,
to avoid growing too big.

With the exception of \verb|lua_newstate|,
all functions in the API need a state as their first argument.
However, most applications use a single state.
To avoid the burden of passing this only state explicitly to all
functions, and also to keep compatibility with old versions of Lua,
the API provides a set of macros and one global variable that
take care of this state argument for single-state applications:
\begin{verbatim}
#ifndef LUA_REENTRANT
\end{verbatim}
\begin{verbatim}
extern lua_State *lua_state;
\end{verbatim}
\begin{verbatim}
#define lua_close()             (lua_close)(lua_state)
#define lua_dofile(filename)    (lua_dofile)(lua_state, filename)
#define lua_dostring(str)       (lua_dostring)(lua_state, str)
   ...
\end{verbatim}
\begin{verbatim}
#endif
\end{verbatim}
For each function in the API, there is a macro with the same name
that supplies \verb|lua_state| as the first argument to the call.
(The parentheses around the function name avoid it being expanded
again as a macro.)
The only exception is \verb|lua_newstate|;
in this case, the corresponding macro is
\begin{verbatim}
#define lua_open()      ((void)(lua_state?0:(lua_state=lua_newstate(0))))
\end{verbatim}
This code checks whether the global state has been initialized;
if not, it creates a new state with default settings and
assigns it to \verb|lua_newstate|.

By default, the single-state macros are all active.
If you need to use multiple states,
and therefore will provide the state argument explicitly in each call,
you should define \IndexVerb{LUA_REENTRANT} before
including \verb|lua.h| in your code:
\begin{verbatim}
#define LUA_REENTRANT
#include "lua.h"
\end{verbatim}

In the sequel, we will show all functions in the single-state form
(therefore, they are actually macros).
When you define \verb|LUA_REENTRANT|,
all of them get a state as the first parameter.


\subsection{Exchanging Values between C and Lua} \label{valuesCLua}
Because Lua has no static type system,
all values passed between Lua and C have type
\verb|lua_Object|\Deffunc{lua_Object},
which works like an abstract type in C that can hold any Lua value.
Values of type \verb|lua_Object| have no meaning outside Lua;
for instance,
you cannot compare two \verb|lua_Object's| directly.
Instead, you should use the following function:
\Deffunc{lua_equal}
\begin{verbatim}
int lua_equal       (lua_Object o1, lua_Object o2);
\end{verbatim}

To check the type of a \verb|lua_Object|,
the following functions are available:
\Deffunc{lua_isnil}\Deffunc{lua_isnumber}\Deffunc{lua_isstring}
\Deffunc{lua_istable}\Deffunc{lua_iscfunction}\Deffunc{lua_isuserdata}
\Deffunc{lua_isfunction}
\Deffunc{lua_type}
\begin{verbatim}
int lua_isnil        (lua_Object object);
int lua_isnumber     (lua_Object object);
int lua_isstring     (lua_Object object);
int lua_istable      (lua_Object object);
int lua_isfunction   (lua_Object object);
int lua_iscfunction  (lua_Object object);
int lua_isuserdata   (lua_Object object);
const char *lua_type (lua_Object object);
\end{verbatim}
The \verb|lua_is*| functions return 1 if the object is compatible
with the given type, and 0 otherwise.
The function \verb|lua_isnumber| accepts numbers and numerical strings,
\verb|lua_isstring| accepts strings and numbers \see{coercion},
and \verb|lua_isfunction| accepts Lua functions and C~functions.
To distinguish between Lua functions and C~functions,
you should use \verb|lua_iscfunction|.
To distinguish between numbers and numerical strings,
you can use \verb|lua_type|.
The \verb|lua_type| returns one of the following strings,
describing the type of the given object:
\verb|"nil"|, \verb|"number"|, \verb|"string"|, \verb|"table"|,
\verb|"function"|, \verb|"userdata"|, or \verb|"NOOBJECT"|.

To get the tag of a \verb|lua_Object|,
use the following function:
\Deffunc{lua_tag}
\begin{verbatim}
int lua_tag (lua_Object object);
\end{verbatim}

To translate a value from type \verb|lua_Object| to a specific C type,
you can use the following conversion functions:
\Deffunc{lua_getnumber}\Deffunc{lua_getstring}\Deffunc{lua_strlen}
\Deffunc{lua_getcfunction}\Deffunc{lua_getuserdata}
\begin{verbatim}
double         lua_getnumber    (lua_Object object);
const char    *lua_getstring    (lua_Object object);
long           lua_strlen       (lua_Object object);
lua_CFunction  lua_getcfunction (lua_Object object);
void          *lua_getuserdata  (lua_Object object);
\end{verbatim}

\verb|lua_getnumber| converts a \verb|lua_Object| to a floating-point number.
This \verb|lua_Object| must be a number or a string convertible to number
\see{coercion}; otherwise, \verb|lua_getnumber| returns~0.

\verb|lua_getstring| converts a \verb|lua_Object| to a string
(\verb|const char*|).
This \verb|lua_Object| must be a string or a number;
otherwise, the function returns \verb|NULL|.
This function does not create a new string,
but returns a pointer to a string inside the Lua environment.
Those strings always have a 0 after their last character (like in C),
but may contain other zeros in their body.
If you do not know whether a string may contain zeros,
you can use \verb|lua_strlen| to get the actual length.
Because Lua has garbage collection,
there is no guarantee that the pointer returned by \verb|lua_getstring|
will be valid after the block ends
\see{GC}.
So,
if you need the string later on,
you should duplicate it with something like
\verb|memcpy(malloc(lua_strlen(o),lua_getstring(o)))|.

\verb|lua_getcfunction| converts a \verb|lua_Object| to a C~function.
This \verb|lua_Object| must be a C~function;
otherwise, \verb|lua_getcfunction| returns \verb|NULL|.
The type \verb|lua_CFunction| is explained in \See{LuacallC}.

\verb|lua_getuserdata| converts a \verb|lua_Object| to \verb|void*|.
This \verb|lua_Object| must have type \emph{userdata};
otherwise, \verb|lua_getuserdata| returns \verb|NULL|.

\subsection{Communication between Lua and C}\label{Lua-C-protocol}

All communication between Lua and C is done through two
abstract data types, called \Def{lua2C} and \Def{C2lua}.
The first one, as the name implies, is used to pass values
from Lua to C:
parameters when Lua calls C and results when C calls Lua.
The structure C2lua is used in the reverse direction:
parameters when C calls Lua and results when Lua calls C.

The structure lua2C is an \emph{abstract array}
that can be indexed with the function:
\Deffunc{lua_lua2C}
\begin{verbatim}
lua_Object lua_lua2C (int number);
\end{verbatim}
where \verb|number| starts with 1.
When called with a number larger than the array size,
this function returns \verb|LUA_NOOBJECT|\Deffunc{LUA_NOOBJECT}.
In this way, it is possible to write C~functions that receive
a variable number of parameters,
and to call Lua functions that return a variable number of results.
Note that the structure lua2C cannot be directly modified by C code.

The structure C2lua is an \emph{abstract stack}.
Pushing elements into this stack
is done with the following functions:
\Deffunc{lua_pushnumber}\Deffunc{lua_pushlstring}\Deffunc{lua_pushstring}
\Deffunc{lua_pushcfunction}\Deffunc{lua_pushusertag}
\Deffunc{lua_pushnil}\Deffunc{lua_pushobject}
\Deffunc{lua_pushuserdata}\label{pushing}
\begin{verbatim}
void lua_pushnumber    (double n);
void lua_pushlstring   (const char *s, long len);
void lua_pushstring    (const char *s);
void lua_pushusertag   (void *u, int tag);
void lua_pushnil       (void);
void lua_pushobject    (lua_Object object);
void lua_pushcfunction (lua_CFunction f);  /* macro */
\end{verbatim}
All of them receive a C value,
convert it to a corresponding \verb|lua_Object|,
and leave the result on the top of C2lua.
In particular, functions \verb|lua_pushlstring| and \verb|lua_pushstring|
make an internal copy of the given string.
Function \verb|lua_pushstring| can only be used to push proper C strings
(that is, strings that end with a zero and do not contain embedded zeros);
otherwise you should use the more general \verb|lua_pushlstring|.
The function
\Deffunc{lua_pop}
\begin{verbatim}
lua_Object lua_pop (void);
\end{verbatim}
returns a reference to the object at the top of the C2lua stack,
and pops it.

As a general rule, all API functions pop from the stack
all elements they use.

When C code calls Lua repeatedly, as in a loop,
objects returned by these calls can accumulate,
and may cause a stack overflow.
To avoid this,
nested blocks can be defined with the functions
\begin{verbatim}
void lua_beginblock (void);
void lua_endblock   (void);
\end{verbatim}
After the end of the block,
all \verb|lua_Object|'s created inside it are released.
The use of explicit nested blocks is good programming practice
and is strongly encouraged.

\subsection{Garbage Collection}\label{GC}
Because Lua has automatic memory management and garbage collection,
a \verb|lua_Object| has a limited scope,
and is only valid inside the \emph{block} where it has been created.
A C~function called from Lua is a block,
and its parameters are valid only until its end.
It is good programming practice to convert Lua objects to C values
as soon as they are available,
and never to store \verb|lua_Object|s in C global variables.

A garbage collection cycle can be forced by:
\Deffunc{lua_collectgarbage}
\begin{verbatim}
long lua_collectgarbage (long limit);
\end{verbatim}
This function returns the number of objects collected.
The argument \verb|limit| makes the next cycle occur only
after that number of new objects have been created.
If \verb|limit| is 0,
then Lua uses an adaptive heuristics to set this limit.


\subsection{Userdata and Tags}\label{C-tags}

Because userdata are objects,
the function \verb|lua_pushusertag| may create a new userdata.
If Lua has a userdata with the given value (\verb|void*|) and tag,
then that userdata is pushed.
Otherwise, a new userdata is created, with the given value and tag.
If this function is called with
\verb|tag| equal to \verb|LUA_ANYTAG|\Deffunc{LUA_ANYTAG},
then Lua will try to find any userdata with the given value,
regardless of its tag.
If there is no userdata with that value, then a new one is created,
with tag equal to 0.

Userdata can have different tags,
whose semantics are only known to the host program.
Tags are created with the function
\Deffunc{lua_newtag}
\begin{verbatim}
int lua_newtag (void);
\end{verbatim}
The function \verb|lua_settag| changes the tag of
the object on the top of C2lua (and pops it);
the object must be a userdata or a table:
\Deffunc{lua_settag}
\begin{verbatim}
void lua_settag (int tag);
\end{verbatim}
The given \verb|tag| must be a value created with \verb|lua_newtag|.

\subsection{Executing Lua Code}
A host program can execute Lua chunks written in a file or in a string
using the following functions:%
\Deffunc{lua_dofile}\Deffunc{lua_dostring}\Deffunc{lua_dobuffer}
\begin{verbatim}
int lua_dofile   (const char *filename);
int lua_dostring (const char *string);
int lua_dobuffer (const char *buff, int size, const char *name);
\end{verbatim}
All these functions return an error code:
0, in case of success; non zero, in case of errors.
More specifically, \verb|lua_dofile| returns 2 if for any reason
it could not open the file.
(In this case,
you may want to
check \verb|errno|,
call \verb|strerror|,
or call \verb|perror| to tell the user what went wrong.)
When called with argument \verb|NULL|,
\verb|lua_dofile| executes the \verb|stdin| stream.
Functions \verb|lua_dofile| and \verb|lua_dobuffer|
are both able to execute pre-compiled chunks.
They automatically detect whether the chunk is text or binary,
and load it accordingly (see program \IndexVerb{luac}).
Function \verb|lua_dostring| executes only source code,
given in textual form.

The third parameter to \verb|lua_dobuffer| (\verb|name|)
is the ``name of the chunk'',
used in error messages and debug information.
If \verb|name| is \verb|NULL|,
then Lua gives a default name to the chunk.

These functions return, in structure lua2C,
any values eventually returned by the chunks.
They also empty the stack C2lua.


\subsection{Manipulating Lua Objects}
To read the value of any global Lua variable,
one uses the function
\Deffunc{lua_getglobal}
\begin{verbatim}
lua_Object lua_getglobal (const char *varname);
\end{verbatim}
As in Lua, this function may trigger a tag method
for the ``getglobal'' event.
To read the real value of any global variable,
without invoking any tag method,
use the \emph{raw} version:
\Deffunc{lua_rawgetglobal}
\begin{verbatim}
lua_Object lua_rawgetglobal (const char *varname);
\end{verbatim}

To store a value previously pushed onto C2lua in a global variable,
there is the function
\Deffunc{lua_setglobal}
\begin{verbatim}
void lua_setglobal (const char *varname);
\end{verbatim}
As in Lua, this function may trigger a tag method
for the ``setglobal'' event.
To set the real value of any global variable,
without invoking any tag method,
use the \emph{raw} version:
\Deffunc{lua_rawgetglobal}
\begin{verbatim}
void lua_rawsetglobal (const char *varname);
\end{verbatim}

Tables can also be manipulated via the API.
The function
\Deffunc{lua_gettable}
\begin{verbatim}
lua_Object lua_gettable (void);
\end{verbatim}
pops a table and an index from the stack C2lua,
and returns the contents of the table at that index.
As in Lua, this operation may trigger a tag method
for the ``gettable'' event.
To get the real value of any table index,
without invoking any tag method,
use the \emph{raw} version:
\Deffunc{lua_rawgetglobal}
\begin{verbatim}
lua_Object lua_rawgettable (void);
\end{verbatim}

To store a value in an index,
the program must push the table, the index, and the value onto C2lua
(in this order),
and then call the function
\Deffunc{lua_settable}
\begin{verbatim}
void lua_settable (void);
\end{verbatim}
As in Lua, this operation may trigger a tag method
for the ``settable'' event.
To set the real value of any table index,
without invoking any tag method,
use the \emph{raw} version:
\Deffunc{lua_rawsettable}
\begin{verbatim}
void lua_rawsettable (void);
\end{verbatim}

Finally, the function
\Deffunc{lua_createtable}
\begin{verbatim}
lua_Object lua_createtable (void);
\end{verbatim}
creates and returns a new, empty table.


\subsection{Calling Lua Functions}
Functions defined in Lua by a chunk
can be called from the host program.
This is done using the following protocol:
first, the arguments to the function are pushed onto C2lua
\see{pushing}, in direct order, i.e., the first argument is pushed first.
Then, the function is called using
\Deffunc{lua_callfunction}
\begin{verbatim}
int lua_callfunction (lua_Object function);
\end{verbatim}
This function returns an error code:
0, in case of success; non zero, in case of errors.
Finally, the results are returned in structure lua2C
(recall that a Lua function may return many values),
and can be retrieved with the macro \verb|lua_getresult|,
\Deffunc{lua_getresult}
which is just another name for the function \verb|lua_lua2C|.
Note that \verb|lua_callfunction|
pops all elements from the C2lua stack.

The following example shows how the host program may do the
equivalent to the Lua code:
\begin{verbatim}
      a,b = f("how", t.x, 4)
\end{verbatim}
\begin{verbatim}
  lua_pushstring("how");                               /* 1st argument */
  lua_pushobject(lua_getglobal("t"));      /* push value of global 't' */
  lua_pushstring("x");                          /* push the string 'x' */
  lua_pushobject(lua_gettable());      /* push result of t.x (2nd arg) */
  lua_pushnumber(4);                                   /* 3rd argument */
  lua_callfunction(lua_getglobal("f"));           /* call Lua function */
  lua_pushobject(lua_getresult(1));   /* push first result of the call */
  lua_setglobal("a");                       /* set global variable 'a' */
  lua_pushobject(lua_getresult(2));  /* push second result of the call */
  lua_setglobal("b");                       /* set global variable 'b' */
\end{verbatim}

Some special Lua functions have exclusive interfaces.
The host program can generate a Lua error calling the function
\Deffunc{lua_error}
\begin{verbatim}
void lua_error (const char *message);
\end{verbatim}
This function never returns.
If \verb|lua_error| is called from a C~function that has been called from Lua,
then the corresponding Lua execution terminates,
as if an error had occurred inside Lua code.
Otherwise, the whole host program terminates with a call to \verb|exit(1)|.
Before terminating execution,
the \verb|message| is passed to the error handler function,
\verb|_ERRORMESSAGE| \see{error}.
If \verb|message| is \verb|NULL|,
then \verb|_ERRORMESSAGE| is not called.

Tag methods can be changed with: \Deffunc{lua_settagmethod}
\begin{verbatim}
lua_Object lua_settagmethod (int tag, const char *event);
\end{verbatim}
The first parameter is the tag,
and the second is the event name \see{tag-method};
the new method is pushed from C2lua.
This function returns a \verb|lua_Object|,
which is the old tag method value.
To get just the current value of a tag method,
use the function \Deffunc{lua_gettagmethod}
\begin{verbatim}
lua_Object lua_gettagmethod (int tag, const char *event);
\end{verbatim}

It is also possible to copy all tag methods from one tag
to another: \Deffunc{lua_copytagmethods}
\begin{verbatim}
int lua_copytagmethods (int tagto, int tagfrom);
\end{verbatim}
This function returns \verb|tagto|.

You can traverse a table with the function \Deffunc{lua_next}
\begin{verbatim}
int lua_next (lua_Object t, int i);
\end{verbatim}
Its first argument is the table to be traversed,
and the second is a \emph{cursor};
this cursor starts in 0,
and for each call the function returns a value to
be used in the next call,
or 0 to signal the end of the traversal.
The function also returns, in the Lua2C array,
a key-value pair from the table.
A typical traversal looks like the following code:
\begin{verbatim}
  int i;
  lua_Object t;
  ...   /* gets the table at `t' */
  i = 0;
  lua_beginblock();
  while ((i = lua_next(t, i)) != 0) {
    lua_Object key = lua_getresult(1);
    lua_Object value = lua_getresult(2);
    ...  /* uses `key' and `value' */
    lua_endblock();
    lua_beginblock();  /* reopens a block */
  }
  lua_endblock();
\end{verbatim}
The pairs of \verb|lua_beginblock|/\verb|lua_endblock| remove the
results of each iteration from the stack.
Without them, a traversal of a large table may overflow the stack.

To traverse the global variables, use \Deffunc{lua_nextvar}
\begin{verbatim}
const char *lua_nextvar (const char *varname);
\end{verbatim}
Here, the cursor is a string;
in the first call you set it to \verb|NULL|;
for each call the function returns the name of a global variable,
to be used in the next call,
or \verb|NULL| to signal the end of the traverse.
The function also returns, in the Lua2C array,
the name (again) and the value of the global variable.
A typical traversal looks like the following code:
\begin{verbatim}
  const char *name = NULL;
  lua_beginblock();
  while ((name = lua_nextvar(name)) != NULL) {
    lua_Object value = lua_getresult(2);
    ...  /* uses `name' and `value' */
    lua_endblock();
    lua_beginblock();  /* reopens a block */
  }
  lua_endblock();
\end{verbatim}


\subsection{Defining C Functions} \label{LuacallC}
To register a C~function to Lua,
there is the following convenience macro:
\Deffunc{lua_register}
\begin{verbatim}
#define lua_register(n,f)       (lua_pushcfunction(f), lua_setglobal(n))
/* const char *n;   */
/* lua_CFunction f; */
\end{verbatim}
which receives the name the function will have in Lua,
and a pointer to the function.
This pointer must have type \verb|lua_CFunction|,
which is defined as
\Deffunc{lua_CFunction}
\begin{verbatim}
typedef void (*lua_CFunction) (void);
\end{verbatim}
that is, a pointer to a function with no parameters and no results.

In order to communicate properly with Lua,
a C~function must follow a protocol,
which defines the way parameters and results are passed.

A C~function receives its arguments in structure lua2C;
to access them, it uses the macro \verb|lua_getparam|, \Deffunc{lua_getparam}
again just another name for \verb|lua_lua2C|.
To return values, a C~function just pushes them onto the stack C2lua,
in direct order \see{valuesCLua}.
Like a Lua function, a C~function called by Lua can also return
many results.

When a C~function is created,
it is possible to associate some \emph{upvalues} to it
\see{upvalue},
thus creating a C closure;
these values are passed to the function whenever it is called,
as common arguments.
To associate upvalues to a C~function,
first these values must be pushed on C2lua.
Then the function \Deffunc{lua_pushcclosure}
\begin{verbatim}
void lua_pushcclosure (lua_CFunction fn, int n);
\end{verbatim}
is used to put the C~function on C2lua,
with the argument \verb|n| telling how many upvalues must be
associated with the function;
in fact, the macro \verb|lua_pushcfunction| is defined as
\verb|lua_pushcclosure| with \verb|n| set to 0.
Then, whenever the C~function is called,
these upvalues are inserted as the first arguments \M{n} to the function,
before the actual arguments provided in the call.

For some examples of C~functions, see files \verb|lstrlib.c|,
\verb|liolib.c| and \verb|lmathlib.c| in the official Lua distribution.
In particular,
\verb|liolib.c| defines C~closures with file handles are upvalues.

\subsection{References to Lua Objects}

As noted in \See{GC}, \verb|lua_Object|s are volatile.
If the C code needs to keep a \verb|lua_Object|
outside block boundaries,
then it must create a \Def{reference} to the object.
The routines to manipulate references are the following:
\Deffunc{lua_ref}\Deffunc{lua_getref}
\Deffunc{lua_unref}
\begin{verbatim}
int        lua_ref    (int lock);
lua_Object lua_getref (int ref);
void       lua_unref  (int ref);
\end{verbatim}
The function \verb|lua_ref| creates a reference
to the object that is on the top of the stack,
and returns this reference.
For a \nil\ object,
the reference is always \verb|LUA_REFNIL|;\Deffunc{LUA_REFNIL}
otherwise, it is a non-negative integer.
The constant \verb|LUA_NOREF| \Deffunc{LUA_NOREF}
is different from any valid reference.
If \verb|lock| is true, then the object is \emph{locked}:
this means the object will not be garbage collected.
\emph{Unlocked references may be garbage collected}.
Whenever the referenced object is needed in~C,
a call to \verb|lua_getref|
returns a handle to it;
if the object has been collected,
\verb|lua_getref| returns \verb|LUA_NOOBJECT|.

When a reference is no longer needed,
it can be released with a call to \verb|lua_unref|.



\section{Predefined Functions and Libraries}

The set of \Index{predefined functions} in Lua is small but powerful.
Most of them provide features that allow some degree of
\Index{reflexivity} in the language.
Some of these features cannot be simulated with the rest of the
language nor with the standard Lua API.
Others are just convenient interfaces to common API functions.

The libraries, on the other hand, provide useful routines
that are implemented directly through the standard API.
Therefore, they are not necessary to the language,
and are provided as separate C modules.
Currently, there are three standard libraries:
\begin{itemize}
\item string manipulation;
\item mathematical functions (sin, log, etc);
\item input and output (plus some system facilities).
\end{itemize}
To have access to these libraries,
the C host program must call the functions
\verb|lua_strlibopen|, \verb|lua_mathlibopen|,
and \verb|lua_iolibopen|, declared in \verb|lualib.h|.
\Deffunc{lua_strlibopen}\Deffunc{lua_mathlibopen}\Deffunc{lua_iolibopen}


\subsection{Predefined Functions} \label{predefined}

\subsubsection*{\ff \T{_ALERT (message)}}\Deffunc{alert}\label{alert}
Prints its only string argument to \IndexVerb{stderr}.
All error messages in Lua are printed through the function stored
in the \verb|_ALERT| global variable
\see{error}.
Therefore, a program may assign another function to this variable
to change the way such messages are shown
(for instance, for systems without \verb|stderr|).

\subsubsection*{\ff \T{assert (v [, message])}}\Deffunc{assert}
Issues an \emph{``assertion failed!''} error
when its argument \verb|v| is \nil.
This function is equivalent to the following Lua function:
\begin{verbatim}
      function assert (v, m)
        if not v then
          m = m or ""
          error("assertion failed!  " .. m)
        end
      end
\end{verbatim}

\subsubsection*{\ff \T{call (func, arg [, mode [, errhandler]])}}\Deffunc{call}
\label{pdf-call}
Calls function \verb|func| with
the arguments given by the table \verb|arg|.
The call is equivalent to
\begin{verbatim}
      func(arg[1], arg[2], ..., arg[n])
\end{verbatim}
where \verb|n| is the result of \verb|getn(arg)| \see{getn}.

By default,
all results from \verb|func| are simply returned by \verb|call|.
If the string \verb|mode| contains \verb|"p"|,
then the results are \emph{packed} in a single table.\index{packed results}
That is, \verb|call| returns just one table;
at index \verb|n|, the table has the total number of results
from the call;
the first result is at index 1, etc.
For instance, the following calls produce the following results:
\begin{verbatim}
   a = call(sin, {5})                --> a = 0.0871557 = sin(5)
   a = call(max, {1,4,5; n=2})       --> a = 4 (only 1 and 4 are arguments)
   a = call(max, {1,4,5; n=2}, "p")  --> a = {4; n=1}
   t = {x=1}
   a = call(next, {t,nil;n=2}, "p")  --> a={"x", 1; n=2}
\end{verbatim}

By default,
if an error occurs during the call to \verb|func|,
the error is propagated.
If the string \verb|mode| contains \verb|"x"|,
then the call is \emph{protected}.\index{protected calls}
In this mode, function \verb|call| does not propagate an error,
regardless of what happens during the call.
Instead, it returns \nil\ to signal the error
(besides calling the appropriated error handler).

If \verb|errhandler| is provided,
the error function \verb|_ERRORMESSAGE| is temporarily set \verb|errhandler|,
while \verb|func| runs.
In particular, if \verb|errhandler| is \nil,
no error messages will be issued during the execution of the called function.

\subsubsection*{\ff \T{collectgarbage ([limit])}}\Deffunc{collectgarbage}
Forces a garbage collection cycle.
Returns the number of objects collected.
The optional argument \verb|limit| is a number that
makes the next cycle occur only after that number of new
objects have been created.
If \verb|limit| is absent or equal to 0,
then Lua uses an adaptive algorithm to set this limit.
\verb|collectgarbage| is equivalent to
the API function \verb|lua_collectgarbage|.

\subsubsection*{\ff \T{copytagmethods (tagto, tagfrom)}}
\Deffunc{copytagmethods}
Copies all tag methods from one tag to another;
it returns \verb|tagto|.

\subsubsection*{\ff \T{dofile (filename)}}\Deffunc{dofile}
Receives a file name,
opens the named file, and executes its contents as a Lua chunk,
or as pre-compiled chunks.
When called without arguments,
\verb|dofile| executes the contents of the standard input (\verb|stdin|).
If there is any error executing the file,
then \verb|dofile| returns \nil.
Otherwise, it returns the values returned by the chunk,
or a non \nil\ value if the chunk returns no values.
It issues an error when called with a non string argument.
\verb|dofile| is equivalent to the API function \verb|lua_dofile|.

\subsubsection*{\ff \T{dostring (string [, chunkname])}}\Deffunc{dostring}
Executes a given string as a Lua chunk.
If there is any error executing the string,
then \verb|dostring| returns \nil.
Otherwise, it returns the values returned by the chunk,
or a non \nil\ value if the chunk returns no values.
The optional parameter \verb|chunkname|
is the ``name of the chunk'',
used in error messages and debug information.
\verb|dostring| is equivalent to the API function \verb|lua_dostring|.

\subsubsection*{\ff \T{error (message)}}\Deffunc{error}\label{pdf-error}
Calls the error handler \see{error} and then terminates
the last protected function called
(in~C: \verb|lua_dofile|, \verb|lua_dostring|,
\verb|lua_dobuffer|, or \verb|lua_callfunction|;
in Lua: \verb|dofile|, \verb|dostring|, or \verb|call| in protected mode).
If \verb|message| is \nil, then the error handler is not called.
Function \verb|error| never returns.
\verb|error| is equivalent to the API function \verb|lua_error|.

\subsubsection*{\ff \T{foreach (table, function)}}\Deffunc{foreach}
Executes the given \verb|function| over all elements of \verb|table|.
For each element, the function is called with the index and
respective value as arguments.
If the function returns any non-\nil\ value,
then the loop is broken, and this value is returned
as the final value of \verb|foreach|.

This function could be defined in Lua:
\begin{verbatim}
      function foreach (t, f)
        local i, v = nil
        while 1 do
          i, v = next(t, i)
          if not i then break end
          local res = f(i, v)
          if res then return res end
        end
      end
\end{verbatim}

You may change the \emph{values} of existing fields in the table during the traversal,
but
if you create new indices,
then
the semantics of \verb|foreach| is undefined.


\subsubsection*{\ff \T{foreachi (table, function)}}\Deffunc{foreachi}
Executes the given \verb|function| over the
numerical indices of \verb|table|.
For each index, the function is called with the index and
respective value as arguments.
Indices are visited in sequential order,
from 1 to \verb|n|,
where \verb|n| is the result of \verb|getn(table)| \see{getn}.
If the function returns any non-\nil\ value,
then the loop is broken, and this value is returned
as the final value of \verb|foreachi|.

This function could be defined in Lua:
\begin{verbatim}
      function foreachi (t, f)
        for i=1,getn(t) do
          local res = f(i, t[i])
          if res then return res end
        end
      end
\end{verbatim}

You may change the \emph{values} of existing fields in the table during the traversal,
but
if you create new indices (even non-numeric),
then
the semantics of \verb|foreachi| is undefined.

\subsubsection*{\ff \T{foreachvar (function)}}\Deffunc{foreachvar}
Executes \verb|function| over all global variables.
For each variable,
the function is called with its name and its value as arguments.
If the function returns any non-nil value,
then the loop is broken, and this value is returned
as the final value of \verb|foreachvar|.

This function could be defined in Lua:
\begin{verbatim}
      function foreachvar (f)
        local n, v = nil
        while 1 do
          n, v = nextvar(n)
          if not n then break end
          local res = f(n, v)
          if res then return res end
        end
      end
\end{verbatim}

You may change the values of existing global variables during the traversal,
but
if you create new global variables,
then
the semantics of \verb|foreachvar| is undefined.


\subsubsection*{\ff \T{getglobal (name)}}\Deffunc{getglobal}
Gets the value of a global variable,
or calls a tag method for ``getgloball''.
Its full semantics is explained in \See{tag-method}.
The string \verb|name| does not need to be a
syntactically valid variable name.

\subsubsection*{\ff \T{getn (table)}}\Deffunc{getn}\label{getn}
Returns the ``size'' of a table, when seen as a list.
If the table has an \verb|n| field with a numeric value,
this value is its ``size''.
Otherwise, the size is the largest numerical index with a non-nil
value in the table.
This function could be defined in Lua:
\begin{verbatim}
      function getn (t)
        if type(t.n) == 'number' then return t.n end
        local max, i = 0, nil
        while 1 do
          i = next(t, i)
          if not i then break end
          if type(i) == 'number' and i>max then max=i end
        end
        return max
      end
\end{verbatim}

\subsubsection*{\ff \T{gettagmethod (tag, event)}}
\Deffunc{gettagmethod}
Returns the current tag method
for a given pair \M{(tag, event)}.

\subsubsection*{\ff \T{newtag ()}}\Deffunc{newtag}\label{pdf-newtag}
Returns a new tag.
\verb|newtag| is equivalent to the API function \verb|lua_newtag|.

\subsubsection*{\ff \T{next (table, [index])}}\Deffunc{next}
Allows a program to traverse all fields of a table.
Its first argument is a table and its second argument
is an index in this table.
It returns the next index of the table and the
value associated with the index.
When called with \nil\ as its second argument,
\verb|next| returns the first index
of the table and its associated value.
When called with the last index,
or with \nil\ in an empty table,
it returns \nil.
If the second argument is absent, then it is interpreted as \nil.

Lua has no declaration of fields;
semantically, there is no difference between a
field not present in a table or a field with value \nil.
Therefore, \verb|next| only considers fields with non \nil\ values.
The order in which the indices are enumerated is not specified,
\emph{even for numeric indices}
(to traverse a table in numeric order,
use a counter or the function \verb|foreachi|).

You may change the \emph{values} of existing fields in the table during the traversal,
but
if you create new indices,
then
the semantics of \verb|next| is undefined.

\subsubsection*{\ff \T{nextvar (name)}}\Deffunc{nextvar}
This function is similar to the function \verb|next|,
but iterates instead over the global variables.
Its single argument is the name of a global variable,
or \nil\ to get a first name.
If this argument is absent, then it is interpreted as \nil.
Like \verb|next|, \verb|nextvar| returns the name of another variable
and its value,
or \nil\ if there are no more variables.

You may change the \emph{values} of existing global variables during the traversal,
but
if you create new global variables,
then
the semantics of \verb|nextvar| is undefined.

\subsubsection*{\ff \T{print (e1, e2, ...)}}\Deffunc{print}
Receives any number of arguments,
and prints their values using the strings returned by \verb|tostring|.
This function is not intended for formatted output,
but only as a quick way to show a value,
for instance for debugging.
See \See{libio} for functions for formatted output.

\subsubsection*{\ff \T{rawgetglobal (name)}}\Deffunc{rawgetglobal}
Gets the value of a global variable,
without invoking any tag method.
The string \verb|name| does not need to be a
syntactically valid variable name.

\subsubsection*{\ff \T{rawgettable (table, index)}}\Deffunc{rawgettable}
Gets the real value of \verb|table[index]|,
without invoking any tag method.
\verb|table| must be a table,
and \verb|index| is any value different from \nil.

\subsubsection*{\ff \T{rawsetglobal (name, value)}}\Deffunc{rawsetglobal}
Sets the named global variable to the given value,
without invoking any tag method.
The string \verb|name| does not need to be a
syntactically valid variable name.
Therefore,
this function can be used to set global variables with strange names like
\verb|"m v 1"| or \verb|"34"|.

\subsubsection*{\ff \T{rawsettable (table, index, value)}}\Deffunc{rawsettable}
Sets the real value of \verb|table[index]| to \verb|value|,
without invoking any tag method.
\verb|table| must be a table,
\verb|index| is any value different from \nil,
and \verb|value| is any Lua value.

\subsubsection*{\ff \T{setglobal (name, value)}}\Deffunc{setglobal}
Sets the named global variable to the given value,
or calls a tag method for ``setgloball''.
Its full semantics is explained in \See{tag-method}.
The string \verb|name| does not need to be a
syntactically valid variable name.

\subsubsection*{\ff \T{settag (t, tag)}}\Deffunc{settag}
Sets the tag of a given table \see{TypesSec}.
\verb|tag| must be a value created with \verb|newtag|
\see{pdf-newtag}.
It returns the value of its first argument (the table).
For the safety of host programs,
it is impossible to change the tag of a userdata from Lua.

\subsubsection*{\ff \T{settagmethod (tag, event, newmethod)}}
\Deffunc{settagmethod}
Sets a new tag method to the given pair \M{(tag, event)}.
It returns the old method.
If \verb|newmethod| is \nil,
then \verb|settagmethod| restores the default behavior for the given event.

\subsubsection*{\ff \T{sort (table [, comp])}}\Deffunc{sort}
Sorts table elements in a given order, \emph{in-place},
from \verb|table[1]| to \verb|table[n]|,
where \verb|n| is the result of \verb|getn(table)| \see{getn}.
If \verb|comp| is given,
it must be a function that receives two table elements,
and returns true when the first is less than the second
(so that \verb|not comp(a[i+1], a[i])| will be true after the sort).
If \verb|comp| is not given,
the standard Lua operator \verb|<| is used instead.

\subsubsection*{\ff \T{tag (v)}}\Deffunc{tag}\label{pdf-tag}
Allows Lua programs to test the tag of a value \see{TypesSec}.
It receives one argument, and returns its tag (a number).
\verb|tag| is equivalent to the API function \verb|lua_tag|.

\subsubsection*{\ff \T{tonumber (e [, base])}}\Deffunc{tonumber}
Receives one argument,
and tries to convert it to a number.
If the argument is already a number or a string convertible
to a number, then \verb|tonumber| returns that number;
otherwise, it returns \nil.

An optional argument specifies the base to interpret the numeral.
The base may be any integer between 2 and 36, inclusive.
In bases above~10, the letter `A' (either upper or lower case)
represents~10, `B' represents~11, and so forth, with `Z' representing 35.

In base 10 (the default), the number may have a decimal part,
as well as an optional exponent part \see{coercion}.
In other bases, only unsigned integers are accepted.

\subsubsection*{\ff \T{tostring (e)}}\Deffunc{tostring}
Receives an argument of any type and
converts it to a string in a reasonable format.
For complete control on how numbers are converted,
use function \verb|format|.



\subsubsection*{\ff \T{tinsert (table [, pos] , value)}}\Deffunc{tinsert}

Inserts element \verb|value| at table position \verb|pos|,
shifting other elements to open space, if necessary.
The default value for \verb|pos| is \verb|n+1|,
where \verb|n| is the result of \verb|getn(table)| \see{getn},
so that a call \verb|tinsert(t,x)| inserts \verb|x| at the end
of table \verb|t|.
This function also sets or increments the field \verb|n| of the table
to \verb|n+1|.

This function is equivalent to the following Lua function,
except that the table accesses are all \emph{raw} (that is, without tag methods):
\begin{verbatim}
      function tinsert (t, ...)
        local pos, value
        local n = getn(t)
        if arg.n == 1 then
          pos, value = n+1, arg[1]
        else
          pos, value = arg[1], arg[2]
        end
        t.n = n+1;
        for i=n,pos,-1 do
          t[i+1] = t[i]
        end
        t[pos] = value
      end
\end{verbatim}

\subsubsection*{\ff \T{tremove (table [, pos])}}\Deffunc{tremove}

Removes from \verb|table| the element at position \verb|pos|,
shifting other elements to close the space, if necessary.
Returns the value of the removed element.
The default value for \verb|pos| is \verb|n|,
where \verb|n| is the result of \verb|getn(table)| \see{getn},
so that a call \verb|tremove(t)| removes the last element
of table \verb|t|.
This function also sets or decrements the field \verb|n| of the table
to \verb|n-1|.

This function is equivalent to the following Lua function,
except that the table accesses are all \emph{raw} (that is, without tag methods):
\begin{verbatim}
      function tremove (t, pos)
        local n = getn(t)
        if n<=0 then return end
        pos = pos or n
        local value = t[pos]
        for i=pos,n-1 do
          t[i] = t[i+1]
        end
        t[n] = nil
        t.n = n-1
        return value
      end
\end{verbatim}

\subsubsection*{\ff \T{type (v)}}\Deffunc{type}\label{pdf-type}
Allows Lua programs to test the type of a value.
It receives one argument, and returns its type, coded as a string.
The possible results of this function are
\verb|"nil"| (a string, not the value \nil),
\verb|"number"|,
\verb|"string"|,
\verb|"table"|,
\verb|"function"|,
and \verb|"userdata"|.
\verb|type| is equivalent to the API function \verb|lua_type|.


\subsection{String Manipulation}
This library provides generic functions for string manipulation,
such as finding and extracting substrings and pattern matching.
When indexing a string, the first character is at position~1
(not at~0, as in C).

\subsubsection*{\ff \T{strbyte (s [, i])}}\Deffunc{strbyte}
Returns the internal numerical code of the character \verb|s[i]|.
If \verb|i| is absent, then it is assumed to be 1.
If \verb|i| is negative,
it is replaced by the length of the string minus its
absolute value plus 1.
Therefore, \Math{-1} points to the last character of \verb|s|.

\NOTE
\emph{numerical codes are not necessarily portable across platforms}.

\subsubsection*{\ff \T{strchar (i1, i2, \ldots)}}\Deffunc{strchar}
Receives 0 or more integers.
Returns a string with length equal to the number of arguments,
wherein each character has the internal numerical code equal
to its correspondent argument.

\NOTE
\emph{numerical codes are not necessarily portable across platforms}.

\subsubsection*{\ff \T{strfind (str, pattern [, init [, plain]])}}
\Deffunc{strfind}
Looks for the first \emph{match} of
\verb|pattern| in \verb|str|.
If it finds one, then it returns the indices on \verb|str|
where this occurrence starts and ends;
otherwise, it returns \nil.
If the pattern specifies captures (see \verb|gsub| below),
the captured strings are returned as extra results.
A third optional numerical argument specifies where to start the search;
its default value is 1.
If \verb|init| is negative,
it is replaced by the length of the string minus its
absolute value plus 1.
Therefore, \Math{-1} points to the last character of \verb|str|.
A value of 1 as a fourth optional argument
turns off the pattern matching facilities,
so the function does a plain ``find substring'' operation,
with no characters in \verb|pattern| being considered ``magic''.

\subsubsection*{\ff \T{strlen (s)}}\Deffunc{strlen}
Receives a string and returns its length.
The empty string \verb|""| has length 0.
Embedded zeros are counted.

\subsubsection*{\ff \T{strlower (s)}}\Deffunc{strlower}
Receives a string and returns a copy of that string with all
upper case letters changed to lower case.
All other characters are left unchanged.
The definition of what is an upper-case
letter depends on the current locale.

\subsubsection*{\ff \T{strrep (s, n)}}\Deffunc{strrep}
Returns a string that is the concatenation of \verb|n| copies of
the string \verb|s|.

\subsubsection*{\ff \T{strsub (s, i [, j])}}\Deffunc{strsub}
Returns another string, which is a substring of \verb|s|,
starting at \verb|i|  and running until \verb|j|.
If \verb|i| or \verb|j| are negative,
they are replaced by the length of the string minus their
absolute value plus 1.
Therefore, \Math{-1} points to the last character of \verb|s|
and \Math{-2} to the previous one.
If \verb|j| is absent, it is assumed to be equal to \Math{-1}
(which is the same as the string length).
In particular,
the call \verb|strsub(s,1,j)| returns a prefix of \verb|s|
with length \verb|j|,
and the call \verb|strsub(s, -i)| returns a suffix of \verb|s|
with length \verb|i|.

\subsubsection*{\ff \T{strupper (s)}}\Deffunc{strupper}
Receives a string and returns a copy of that string with all
lower case letters changed to upper case.
All other characters are left unchanged.
The definition of what is a lower case
letter depends on the current locale.

\subsubsection*{\ff \T{format (formatstring, e1, e2, \ldots)}}\Deffunc{format}
\label{format}
Returns a formatted version of its variable number of arguments
following the description given in its first argument (which must be a string).
The format string follows the same rules as the \verb|printf| family of
standard C~functions.
The only differences are that the options/modifiers
\verb|*|, \verb|l|, \verb|L|, \verb|n|, \verb|p|,
and \verb|h| are not supported,
and there is an extra option, \verb|q|.
The \verb|q| option formats a string in a form suitable to be safely read
back by the Lua interpreter:
The string is written between double quotes,
and all double quotes, returns, and backslashes in the string
are correctly escaped when written.
For instance, the call
\begin{verbatim}
format('%q', 'a string with "quotes" and \n new line')
\end{verbatim}
will produce the string:
\begin{verbatim}
"a string with \"quotes\" and \
 new line"
\end{verbatim}

Conversions can be applied to the \M{n}-th argument in the argument list,
rather than the next unused argument.
In this case, the conversion character \verb|%| is replaced
by the sequence \verb|%d$|, where \verb|d| is a
decimal digit in the range [1,9],
giving the position of the argument in the argument list.
For instance, the call \verb|format("%2$d -> %1$03d", 1, 34)| will
result in \verb|"34 -> 001"|.
The same argument can be used in more than one conversion.

The options \verb|c|, \verb|d|, \verb|E|, \verb|e|, \verb|f|,
\verb|g|, \verb|G|, \verb|i|, \verb|o|, \verb|u|, \verb|X|, and \verb|x| all
expect a number as argument,
whereas \verb|q| and \verb|s| expect a string.
The \verb|*| modifier can be simulated by building
the appropriate format string.
For example, \verb|"%*g"| can be simulated with
\verb|"%"..width.."g"|.

\NOTE
\emph{Neither the format string nor the string values to be formatted with
\T{format} can contain embedded zeros.}

\subsubsection*{\ff \T{gsub (s, pat, repl [, n])}}
\Deffunc{gsub}
Returns a copy of \verb|s|,
in which all occurrences of the pattern \verb|pat| have been
replaced by a replacement string specified by \verb|repl|.
This function also returns, as a second value,
the total number of substitutions made.

If \verb|repl| is a string, then its value is used for replacement.
Any sequence in \verb|repl| of the form \verb|%n|
with \verb|n| between 1 and 9
stands for the value of the \M{n}-th captured substring.

If \verb|repl| is a function, then this function is called every time a
match occurs, with all captured substrings passed as arguments,
in order (see below).
If the value returned by this function is a string,
then it is used as the replacement string;
otherwise, the replacement string is the empty string.

The last, optional parameter \verb|n| limits
the maximum number of substitutions to occur.
For instance, when \verb|n| is 1 only the first occurrence of
\verb|pat| is replaced.

Here are some examples:
\begin{verbatim}
  x = gsub("hello world", "(%w+)", "%1 %1")
  --> x="hello hello world world"

  x = gsub("hello world", "(%w+)", "%1 %1", 1)
  --> x="hello hello world"

  x = gsub("hello world from Lua", "(%w+)%s*(%w+)", "%2 %1")
  --> x="world hello Lua from"

  x = gsub("home = $HOME, user = $USER", "%$(%w+)", getenv)
  --> x="home = /home/roberto, user = roberto"  (for instance)

  x = gsub("4+5 = $return 4+5$", "%$(.-)%$", dostring)
  --> x="4+5 = 9"

  local t = {name="lua", version="3.2"}
  x = gsub("$name - $version", "%$(%w+)", function (v) return %t[v] end)
  --> x="lua - 3.2"

  t = {n=0}
  gsub("first second word", "(%w+)", function (w) tinsert(%t, w) end)
  --> t={"first", "second", "word"; n=3}
\end{verbatim}


\subsubsection*{Patterns} \label{pm}

\paragraph{Character Class:}
a \Def{character class} is used to represent a set of characters.
The following combinations are allowed in describing a character class:
\begin{description}
\item[\emph{x}] (where \emph{x} is any character not in the list
\verb|^$()%.[]*+-?|)
--- represents the character \emph{x} itself.
\item[\T{.}] --- (a dot) represents all characters.
\item[\T{\%a}] --- represents all letters.
\item[\T{\%c}] --- represents all control characters.
\item[\T{\%d}] --- represents all digits.
\item[\T{\%l}] --- represents all lower case letters.
\item[\T{\%p}] --- represents all punctuation characters.
\item[\T{\%s}] --- represents all space characters.
\item[\T{\%u}] --- represents all upper case letters.
\item[\T{\%w}] --- represents all alphanumeric characters.
\item[\T{\%x}] --- represents all hexadecimal digits.
\item[\T{\%z}] --- represents the character with representation 0.
\item[\T{\%\M{x}}] (where \M{x} is any non alphanumeric character)  ---
represents the character \M{x}.
This is the standard way to escape the magic characters \verb|()%.[]*-?|.
It is strongly recommended that any control character (even the non magic)
should be preceded by a \verb|%|
when used to represent itself in a pattern,

\item[\T{[char-set]}] ---
represents the class which is the union of all
characters in char-set.
A range of characters may be specified by
separating the end characters of the range with a \verb|-|.
All classes \verb|%|\emph{x} described above may also be used as
components in a char-set.
All other characters in char-set represent themselves.
For example, \verb|[%w_]| (or \verb|[_%w]|)
represents all alphanumeric characters plus the underscore,
\verb|[0-7]| represents the octal digits,
and \verb|[0-7%l%-]| represents the octal digits plus
the lower case letters plus the \verb|-| character.

The interaction between ranges and classes is not defined.
Therefore, patterns like \verb|[%a-z]| or \verb|[a-%%]|
have no meaning.

\item[\T{[\^{ }char-set]}] ---
represents the complement of char-set,
where char-set is interpreted as above.
\end{description}
For all classes represented by single letters (\verb|%a|, \verb|%c|, \ldots),
the corresponding upper-case letter represents the complement of the class.
For instance, \verb|%S| represents all non-space characters.

The definitions of letter, space, etc. depend on the current locale.
In particular, the class \verb|[a-z]| may not be equivalent to \verb|%l|.
The second form should be preferred for portability.

\paragraph{Pattern Item:}
a \Def{pattern item} may be
\begin{itemize}
\item
a single character class,
which matches any single character in the class;
\item
a single character class followed by \verb|*|,
which matches 0 or more repetitions of characters in the class.
These repetition items will always match the longest possible sequence;
\item
a single character class followed by \verb|+|,
which matches 1 or more repetitions of characters in the class.
These repetition items will always match the longest possible sequence;
\item
a single character class followed by \verb|-|,
which also matches 0 or more repetitions of characters in the class.
Unlike \verb|*|,
these repetition items will always match the shortest possible sequence;
\item
a single character class followed by \verb|?|,
which matches 0 or 1 occurrence of a character in the class;
\item
\T{\%\M{n}}, for \M{n} between 1 and 9;
such item matches a sub-string equal to the \M{n}-th captured string
(see below);
\item
\T{\%b\M{xy}}, where \M{x} and \M{y} are two distinct characters;
such item matches strings that start with~\M{x}, end with~\M{y},
and where the \M{x} and \M{y} are \emph{balanced}.
This means that, if one reads the string from left to right,
counting \Math{+1} for an \M{x} and \Math{-1} for a \M{y},
the ending \M{y} is the first where the count reaches 0.
For instance, the item \verb|%b()| matches expressions with
balanced parentheses.
\end{itemize}

\paragraph{Pattern:}
a \Def{pattern} is a sequence of pattern items.
A \verb|^| at the beginning of a pattern anchors the match at the
beginning of the subject string.
A \verb|$| at the end of a pattern anchors the match at the
end of the subject string.
At other positions,
\verb|^| and \verb|$| have no special meaning and represent themselves.

\paragraph{Captures:}
A pattern may contain sub-patterns enclosed in parentheses,
that describe \Def{captures}.
When a match succeeds, the sub-strings of the subject string
that match captures are stored (\emph{captured}) for future use.
Captures are numbered according to their left parentheses.
For instance, in the pattern \verb|"(a*(.)%w(%s*))"|,
the part of the string matching \verb|"a*(.)%w(%s*)"| is
stored as the first capture (and therefore has number~1);
the character matching \verb|.| is captured with number~2,
and the part matching \verb|%s*| has number~3.

\NOTE
{\em A pattern cannot contain embedded zeros.
Use \verb|%z| instead.}


\subsection{Mathematical Functions} \label{mathlib}

This library is an interface to some functions of the standard C math library.
In addition, it registers a tag method for the binary operator \verb|^| that
returns \Math{x^y} when applied to numbers \verb|x^y|.

The library provides the following functions:
\Deffunc{abs}\Deffunc{acos}\Deffunc{asin}\Deffunc{atan}
\Deffunc{atan2}\Deffunc{ceil}\Deffunc{cos}\Deffunc{floor}
\Deffunc{log}\Deffunc{log10}\Deffunc{max}\Deffunc{min}
\Deffunc{mod}\Deffunc{sin}\Deffunc{sqrt}\Deffunc{tan}
\Deffunc{frexp}\Deffunc{ldexp}
\Deffunc{random}\Deffunc{randomseed}
\begin{verbatim}
   abs  acos  asin  atan  atan2  ceil  cos  deg     floor  log  log10
   max  min   mod   rad   sin    sqrt  tan  frexp   ldexp
   random     randomseed
\end{verbatim}
plus a global variable \IndexVerb{PI}.
Most of them
are only interfaces to the homonymous functions in the C~library,
except that, for the trigonometric functions,
all angles are expressed in \emph{degrees}, not radians.
Functions \IndexVerb{deg} and \IndexVerb{rad} can be used to convert
between radians and degrees.

The function \verb|max| returns the maximum
value of its numeric arguments.
Similarly, \verb|min| computes the minimum.
Both can be used with 1, 2, or more arguments.

The functions \verb|random| and \verb|randomseed| are interfaces to
the simple random generator functions \verb|rand| and \verb|srand|,
provided by ANSI C.
(No guarantees can be given for their statistical properties.)
The function \verb|random|, when called without arguments,
returns a pseudo-random real number in the range \Math{[0,1)}.
When called with a number \Math{n},
\verb|random| returns a pseudo-random integer in the range \Math{[1,n]}.
When called with two arguments, \Math{l} and \Math{u},
\verb|random| returns a pseudo-random integer in the range \Math{[l,u]}.


\subsection{I/O Facilities} \label{libio}

All input and output operations in Lua are done, by default,
over two \Def{file handles}, one for reading and one for writing.
These handles are stored in two Lua global variables,
called \verb|_INPUT| and \verb|_OUTPUT|.
The global variables
\verb|_STDIN|, \verb|_STDOUT|, and \verb|_STDERR|
are initialized with file descriptors for
\verb|stdin|, \verb|stdout| and \verb|stderr|.
Initially, \verb|_INPUT=_STDIN| and \verb|_OUTPUT=_STDOUT|.
\Deffunc{_INPUT}\Deffunc{_OUTPUT}
\Deffunc{_STDIN}\Deffunc{_STDOUT}\Deffunc{_STDERR}

A file handle is a userdata containing the file stream \verb|FILE*|,
and with a distinctive tag created by the I/O library.

Unless otherwise stated,
all I/O functions return \nil\ on failure and
some value different from \nil\ on success.

\subsubsection*{\ff \T{openfile (filename, mode)}}\Deffunc{openfile}

This function opens a file,
in the mode specified in the string \verb|mode|.
It returns a new file handle,
or, in case of errors, \nil\ plus a string describing the error.
This function does not modify either \verb|_INPUT| or \verb|_OUTPUT|.

The \verb|mode| string can be any of the following:
\begin{description}
\item[``r''] read mode;
\item[``w''] write mode;
\item[``a''] append mode;
\item[``r+''] update mode, all previous data is preserved;
\item[``w+''] update mode, all previous data is erased;
\item[``a+''] append update mode, previous data is preserved,
  writing is only allowed at the end of file.
\end{description}
The \verb|mode| string may also have a \verb|b| at the end,
which is needed in some systems to open the file in binary mode.
This string is exactlty what is used in the standard~C function \verb|fopen|.

\subsubsection*{\ff \T{closefile (handle)}}\Deffunc{closefile}

This function closes the given file.
It does not modify either \verb|_INPUT| or \verb|_OUTPUT|.

\subsubsection*{\ff \T{readfrom (filename)}}\Deffunc{readfrom}

This function may be called in two ways.
When called with a file name, it opens the named file,
sets its handle as the value of \verb|_INPUT|,
and returns this value.
It does not close the current input file.
When called without parameters,
it closes the \verb|_INPUT| file,
and restores \verb|stdin| as the value of \verb|_INPUT|.

If this function fails, it returns \nil,
plus a string describing the error.

\begin{quotation}
\noindent
\emph{System dependent}: if \verb|filename| starts with a \verb-|-,
then a \Index{piped input} is opened, via function \IndexVerb{popen}.
Not all systems implement pipes.
Moreover,
the number of files that can be open at the same time is
usually limited and depends on the system.
\end{quotation}

\subsubsection*{\ff \T{writeto (filename)}}\Deffunc{writeto}

This function may be called in two ways.
When called with a file name,
it opens the named file,
sets its handle as the value of \verb|_OUTPUT|,
and returns this value.
It does not close the current output file.
Note that, if the file already exists,
then it will be \emph{completely erased} with this operation.
When called without parameters,
this function closes the \verb|_OUTPUT| file,
and restores \verb|stdout| as the value of \verb|_OUTPUT|.
\index{closing a file}

If this function fails, it returns \nil,
plus a string describing the error.

\begin{quotation}
\noindent
\emph{System dependent}: if \verb|filename| starts with a \verb-|-,
then a \Index{piped output} is opened, via function \IndexVerb{popen}.
Not all systems implement pipes.
Moreover,
the number of files that can be open at the same time is
usually limited and depends on the system.
\end{quotation}

\subsubsection*{\ff \T{appendto (filename)}}\Deffunc{appendto}

Opens a file named \verb|filename| and sets it as the
value of \verb|_OUTPUT|.
Unlike the \verb|writeto| operation,
this function does not erase any previous contents of the file.
If this function fails, it returns \nil,
plus a string describing the error.

\subsubsection*{\ff \T{remove (filename)}}\Deffunc{remove}

Deletes the file with the given name.
If this function fails, it returns \nil,
plus a string describing the error.

\subsubsection*{\ff \T{rename (name1, name2)}}\Deffunc{rename}

Renames file named \verb|name1| to \verb|name2|.
If this function fails, it returns \nil,
plus a string describing the error.

\subsubsection*{\ff \T{flush ([filehandle])}}\Deffunc{flush}

Saves any written data to the given file.
If \verb|filehandle| is not specified,
then \verb|flush| flushes all open files.
If this function fails, it returns \nil,
plus a string describing the error.

\subsubsection*{\ff \T{seek (filehandle [, whence] [, offset])}}\Deffunc{seek}

Sets and gets the file position,
measured in bytes from the beginning of the file,
to the position given by \verb|offset| plus a base
specified by the string \verb|whence|, as follows:
\begin{description}
\item[``set''] base is position 0 (beginning of the file);
\item[``cur''] base is current position;
\item[``end''] base is end of file;
\end{description}
In case of success, function \verb|seek| returns the final file position,
measured in bytes from the beginning of the file.
If the call fails, it returns \nil,
plus a string describing the error.

The default value for \verb|whence| is \verb|"cur"|,
and for \verb|offset| is 0.
Therefore, the call \verb|seek(file)| returns the current
file position, without changing it;
the call \verb|seek(file, "set")| sets the position to the
beginning of the file (and returns 0);
and the call \verb|seek(file, "end")| sets the position to the
end of the file, and returns its size.

\subsubsection*{\ff \T{tmpname ()}}\Deffunc{tmpname}

Returns a string with a file name that can safely
be used for a temporary file.
The file must be explicitly opened before its use
and removed when no longer needed.

\subsubsection*{\ff \T{read ([filehandle,] format1, ...)}}\Deffunc{read}

Reads file \verb|_INPUT|,
or \verb|filehandle| if this argument is given,
according to the given formats, which specify what to read.
For each format,
the function returns a string (or a number) with the characters read,
or \nil\ if it cannot read data with the specified format.
When called without formats,
it uses a default format that reads the next line
(see below).

The available formats are
\begin{description}
\item[``*n''] reads a number;
this is the only format that returns a number instead of a string.
\item[``*l''] reads the next line
(skipping the end of line), or \nil\ on end of file.
This is the default format.
\item[``*a''] reads the whole file, starting at the current position.
On end of file, it returns the empty string.
\item[``*w''] reads the next word
(maximal sequence of non white-space characters),
skipping spaces if necessary, or \nil\ on end of file.
\item[\emph{number}] reads a string with up to that number of characters,
or \nil\ on end of file.
\end{description}

\subsubsection*{\ff \T{write ([filehandle, ] value1, ...)}}\Deffunc{write}

Writes the value of each of its arguments to
file \verb|_OUTPUT|,
or to \verb|filehandle| if this argument is given.
The arguments must be strings or numbers.
To write other values,
use \verb|tostring| or \verb|format| before \verb|write|.
If this function fails, it returns \nil,
plus a string describing the error.

\subsubsection*{\ff \T{date ([format])}}\Deffunc{date}

Returns a string containing date and time
formatted according to the given string \verb|format|,
following the same rules of the ANSI~C function \verb|strftime|.
When called without arguments,
it returns a reasonable date and time representation that depends on
the host system and on the current locale.

\subsubsection*{\ff \T{clock ()}}\Deffunc{clock}

Returns an approximation of the amount of CPU time
used by the program, in seconds.

\subsubsection*{\ff \T{exit ([code])}}\Deffunc{exit}

Calls the C~function \verb|exit|,
with an optional \verb|code|,
to terminate the program.
The default value for \verb|code| is the success code.

\subsubsection*{\ff \T{getenv (varname)}}\Deffunc{getenv}

Returns the value of the process environment variable \verb|varname|,
or \nil\ if the variable is not defined.

\subsubsection*{\ff \T{execute (command)}}\Deffunc{execute}

This function is equivalent to the C~function \verb|system|.
It passes \verb|command| to be executed by an operating system shell.
It returns a status code, which is system-dependent.

\subsubsection*{\ff \T{setlocale (locale [, category])}}\Deffunc{setlocale}

This function is an interface to the ANSI~C function \verb|setlocale|.
\verb|locale| is a string specifying a locale;
\verb|category| is an optional string describing which category to change:
\verb|"all"|, \verb|"collate"|, \verb|"ctype"|,
\verb|"monetary"|, \verb|"numeric"|, or \verb|"time"|;
the default category is \verb|"all"|.
The function returns the name of the new locale,
or \nil\ if the request cannot be honored.


\section{The Debug Interface} \label{debugI}

Lua has no built-in debugging facilities.
Instead, it offers a special interface,
by means of functions and \emph{hooks},
which allows the construction of different
kinds of debuggers, profilers, and other tools
that need ``inside information'' from the interpreter.
This interface is declared in the header file \verb|luadebug.h|,
and has \emph{no} single-state variant.

\subsection{Stack and Function Information}

\Deffunc{lua_getstack}
The main function to get information about the interpreter stack is
\begin{verbatim}
int lua_getstack (lua_State *L, int level, lua_Debug *ar);
\end{verbatim}
It fills parts of a \verb|lua_Debug| structure with
an identification of the \emph{activation record}
of the function executing at a given level.
Level~0 is the current running function,
whereas level \Math{n+1} is the function that has called level \Math{n}.
Usually, \verb|lua_getstack| returns 1;
when called with a level greater than the stack depth,
it returns 0.

\Deffunc{lua_Debug}
The structure \verb|lua_Debug| is used to carry different pieces of information
about an active function:
\begin{verbatim}
struct lua_Debug {
  const char *event;     /* "call", "return" */
  const char *source;    /* (S) */
  int linedefined;       /* (S) */
  const char *what;      /* (S) "Lua" function, "C" function, Lua "main" */
  int currentline;       /* (l) */
  const char *name;      /* (n) */
  const char *namewhat;  /* (n) global, tag method, local, field */
  int nups;              /* (u) number of upvalues */
  lua_Object func;       /* (f) function being executed */
  /* private part */
  ...
};
\end{verbatim}
The \verb|lua_getstack| function fills only the private part
of this structure, for future use.
To fill in the other fields of \verb|lua_Debug| with useful information,
call \Deffunc{lua_getinfo}
\begin{verbatim}
int lua_getinfo (lua_State *L, const char *what, lua_Debug *ar);
\end{verbatim}
This function returns 0 on error
(e.g., an invalid option in \verb|what|).
Each character in the string \verb|what|
selects some fields of \verb|ar| to be filled,
as indicated by the letter in parentheses in the definition of \verb|lua_Debug|;
that is, an \verb|S| fills the fields \verb|source| and \verb|linedefined|,
and \verb|l| fills the field \verb|currentline|, etc.
We describe each field below:
\begin{description}

\item[source]
If the function was defined in a string,
\verb|source| is that string;
if the function was defined in a file,
\verb|source| starts with a \verb|@| followed by the file name.

\item[linedefined]
the line number where starts the definition of the function.

\item[what] the string \verb|"Lua"| if this is a Lua function,
\verb|"C"| if this is a C~function,
or \verb|"main"| if this is the main part of a chunk.

\item[currentline]
the current line where the given function is executing.
It only works if the function has been compiled with debug
information.
When no line information is available,
\verb|currentline| is set to \Math{-1}.

\item[name]
a reasonable name for the given function.
Because functions in Lua are first class values,
they do not have a fixed name:
Some functions may be the value of many global variables,
while others may be stored only in a table field.
The \verb|lua_getinfo| function checks whether the given
function is a tag method or the value of a global variable.
If the given function is a tag method,
then \verb|name| points to the event name.
If the given function is the value of a global variable,
then \verb|name| points to the variable name.
If the given function is neither a tag method nor a global variable,
then \verb|name| is set to \verb|NULL|.

\item[namewhat]
Explains the previous field.
If the function is a global variable,
\verb|namewhat| is \verb|"global"|;
if the function is a tag method,
\verb|namewhat| is \verb|"tag-method"|;
otherwise \verb|namewhat| is \verb|""| (the empty string).

\item[nups]
Number of upvalues of a C~function.
If the function is not a C~function,
\verb|nups| is set to 0.

\item[func]
The function being executed, as a \verb|lua_Object|.

\end{description}

The generation of debug information is controlled by an internal flag,
which can be switched with
\begin{verbatim}
int lua_setdebug (lua_State *L, int debug);
\end{verbatim}
This function sets the flag and returns its previous value.
This flag can also be set from Lua~\see{pragma}.
Setting the flag using \verb|lua_setdebug| affects all chunks that are
compiled afterwards.
Individual functions may still control the generation of debug information
by using \verb|$debug| or \verb|$nodebug|.

\subsection{Manipulating Local Variables}

For the manipulation of local variables,
\verb|luadebug.h| defines the following record:
\begin{verbatim}
struct lua_Localvar {
  int index;
  const char *name;
  lua_Object value;
};
\end{verbatim}
where \verb|index| is an index for local variables
(the first parameter has index 1, and so on,
until the last active local variable).

\Deffunc{lua_getlocal}\Deffunc{lua_setlocal}
The following functions allow the manipulation of the
local variables of a given activation record.
They only work if the function has been compiled with debug
information \see{pragma}.
For these functions, a local variable becomes
visible in the line after its definition.
\begin{verbatim}
int lua_getlocal (lua_State *L, const lua_Debug *ar, lua_Localvar *v);
int lua_setlocal (lua_State *L, const lua_Debug *ar, lua_Localvar *v);
\end{verbatim}
The parameter \verb|ar| must be a valid activation record,
filled by a previous call to \verb|lua_getstack| or
given as argument to a hook \see{sub-hooks}.
To use \verb|lua_getlocal|,
you fill the \verb|index| field of \verb|v| with the index
of a local variable; then the function fills the fields
\verb|name| and \verb|value| with the name and the current
value of that variable.
For \verb|lua_setlocal|,
you fill the \verb|index| and the \verb|value| fields of \verb|v|,
and the function assigns that value to the variable.
Both functions return 0 on failure, that happens
if the index is greater than the number of active local variables,
or if the activation record has no debug information.

As an example, the following function lists the names of all
local variables for a function in a given level of the stack:
\begin{verbatim}
int listvars (lua_State *L, int level) {
  lua_Debug ar;
  int i;
  if (lua_getstack(L, level, &ar) == 0)
    return 0;  /* failure: no such level on the stack */
  for (i=1; ; i++) {
    lua_Localvar v;
    v.index = i;
    if (lua_getlocal(L, &ar, &v) == 0)
      return 1;  /* no more locals, or no debug information */
    printf("%s\n", v.name);
  }
}
\end{verbatim}


\subsection{Hooks}\label{sub-hooks}

The Lua interpreter offers two hooks for debugging purposes:
a \emph{call} hook and a \emph{line} hook.
Both have the same type,
\begin{verbatim}
typedef void (*lua_Hook) (lua_State *L, lua_Debug *ar);
\end{verbatim}
and you can set them with the following functions:
\Deffunc{lua_Hook}\Deffunc{lua_setcallhook}\Deffunc{lua_setlinehook}
\begin{verbatim}
lua_Hook lua_setcallhook (lua_State *L, lua_Hook func);
lua_Hook lua_setlinehook (lua_State *L, lua_Hook func);
\end{verbatim}
A hook is disabled when its value is \verb|NULL|,
which is the initial value of both hooks.
The functions \verb|lua_setcallhook| and \verb|lua_setlinehook|
set their corresponding hooks and return their previous values.

The call hook is called whenever the
interpreter enters or leaves a function.
The \verb|event| field of \verb|ar| has the strings \verb|"call"|
or \verb|"return"|.
This \verb|ar| can then be used in calls to \verb|lua_getinfo|,
\verb|lua_getlocal|, and \verb|lua_setlocal|,
to get more information about the function and to manipulate its
local variables.

The line hook is called every time the interpreter changes
the line of code it is executing.
The \verb|event| field of \verb|ar| has the string \verb|"line"|,
and the \verb|currentline| field has the line number.
Again, you can use this \verb|ar| in other calls to the debug API.
This hook is called only if the active function
has been compiled with debug information~\see{pragma}.

While Lua is running a hook, it disables other calls to hooks.
Therefore, if a hook calls Lua to execute a function or a chunk,
this execution ocurrs without any calls to hooks.

A hook cannot call \T{lua_error}.
It must return to Lua through a regular return.
(There is no problem if the error is inside a chunk or a Lua function
called by the hook, because those errors are protected;
the control returns to the hook anyway.)


\subsection{The Reflexive Debug Interface}

The library \verb|ldblib| provides
the functionality of the debug interface to Lua programs.
If you want to use this library,
your host application must open it,
by calling \verb|lua_dblibopen|.

You should exert great care when using this library.
The functions provided here should be used exclusively for debugging
and similar tasks (e.g., profiling).
Please resist the temptation to use them as a
usual programming tool.
They are slow and violate some (otherwise) secure aspects of the
language (e.g., privacy of local variables).
As a general rule, if your program does not need this library,
do not open it.


\subsubsection*{\ff \T{getstack (level, [what])}}\Deffunc{getstack}

This function returns a table with information about the function
running at level \verb|level| of the stack.
Level 0 is the current function (\verb|getstack| itself);
level 1 is the function that called \verb|getstack|.
If \verb|level| is larger than the number of active functions,
the function returns \nil.
The table contains all the fields returned by \verb|lua_getinfo|,
with the string \verb|what| describing what to get.
The default for \rerb|what| is to get all information available.

For instance, the expression \verb|getstack(1,"n").name| returns
the name of the current function,
if a reasonable name can be found.


\subsubsection*{\ff \T{getlocal (level, local)}}\Deffunc{getlocal}

This function returns the name and the value of the local variable
with index \verb|local| of the function at level \verb|level| of the stack.
(The first parameter has index 1, and so on,
until the last active local variable.)
The function returns \nil\ if there is no local
variable with the given index,
and raises an error when called with a \verb|level| out of range.
(You can call \verb|getstack| to check wheter the level is valid.)

\subsubsection*{\ff \T{setlocal (level, local, value)}}\Deffunc{setlocal}

This function assigns the value \verb|value| to the local variable
with index \verb|local| of the function at level \verb|level| of the stack.
The function returns \nil\ if there is no local
variable with the given index,
and raises an error when called with a \verb|level| out of range.

\subsubsection*{\ff \T{setcallhook (hook)}}\Deffunc{setcallhook}

Sets the function \verb|hook| as the call hook;
this hook will be called every time the interpreter starts and
exits the execution of a function.
The only argument to this hook is the event name (\verb|"call"| or
\verb|"return"|).
You can call \verb|getstack| with level 2 to get more information about
the function being called or returning
(level 0 is the \verb|getstack| function,
and level 1 is the hook function).

When called without arguments,
this function turns off call hooks.

\subsubsection*{\ff \T{setlinehook (hook)}}\Deffunc{setlinehook}

Sets the function \verb|hook| as the line hook;
this hook will be called every time the interpreter changes
the line of code it is executing.
The only argument to the hook is the line number the interpreter
is about to execute.
This hook is called only if the active function
has been compiled with debug information~\see{pragma}.

When called without arguments,
this function turns off line hooks.


\section{\Index{Lua Stand-alone}} \label{lua-sa}

Although Lua has been designed as an extension language,
the language is frequently used as a stand-alone interpreter.
An implementation of such an interpreter,
called simply \verb|lua|,
is provided with the standard distribution.

This program can be called with any sequence of the following arguments:
\begin{description}
\item[\T{-}] executes \verb|stdin| as a file;
\item[\T{-c}] calls \verb|lua_close| after running all arguments;
\item[\T{-d}] turns on debug information;
\item[\T{-e} \rm\emph{stat}] executes string \verb|stat|;
\item[\T{-f filename}] executes file \verb|filename| with the
remaining arguments in table \verb|arg|;
\item[\T{-i}] enters interactive mode with prompt;
\item[\T{-q}] enters interactive mode without prompt;
\item[\T{-v}] prints version information;
\item[\T{var=value}] sets global \verb|var| to string \verb|"value"|;
\item[\T{filename}] executes file \verb|filename|.
\end{description}
When called without arguments,
Lua behaves as \verb|lua -v -i| when \verb|stdin| is a terminal,
and as \verb|lua -| otherwise.

All arguments are handled in order.
For instance, an invocation like
\begin{verbatim}
$ lua -i a=test prog.lua
\end{verbatim}
will first interact with the user until an \verb|EOF| in \verb|stdin|,
then will set \verb|a| to \verb|"test"|,
and finally will run the file \verb|prog.lua|.

When the option \T{-f filename} is used,
all following arguments from the command line
are passed to the Lua program in a table called \verb|arg|.
The field \verb|n| gets the index of the last argument,
and the field 0 gets the \T{filename}.
For instance, in the call
\begin{verbatim}
$ lua a.lua -f b.lua t1 t3
\end{verbatim}
the interpreter first runs the file \T{a.lua},
then creates a table \T{arg},
\begin{verbatim}
  arg = {"t1", "t3";  n = 2, [0] = "b.lua"}
\end{verbatim}
and then runs the file \T{b.lua}.
The stand-alone interpreter also provides a \verb|getarg| function that
can be used to access \emph{all} command line arguments.

In interactive mode,
a multi-line statement can be written finishing intermediate
lines with a backslash (\verb|\|).
If the global variable \verb|_PROMPT| is defined as a string,
its value is used as the prompt. \index{_PROMPT}
Therefore, the prompt can be changed like below:
\begin{verbatim}
$ lua _PROMPT='myprompt> ' -i
\end{verbatim}

In Unix systems, Lua scripts can be made into executable programs
by using \verb|chmod +x| and the~\verb|#!| form,
as in \verb|#!/usr/local/bin/lua|,
or \verb|#!/usr/local/bin/lua -f| to get other arguments.


\section*{Acknowledgments}

The authors would like to thank CENPES/PETROBRAS which,
jointly with \tecgraf, used extensively early versions of
this system and gave valuable comments.
The authors would also like to thank Carlos Henrique Levy,
who found the name of the game.
Lua means \emph{moon} in Portuguese.


\appendix

\section*{Incompatibilities with Previous Versions}

Although great care has been taken to avoid incompatibilities with
the previous public versions of Lua,
some differences had to be introduced.
Here is a list of all these incompatibilities.

\subsection*{Incompatibilities with \Index{version 3.2}}
\begin{itemize}

\item
General read patterns are now deprecated.
\item
Garbage-collection tag methods for tables is now deprecated.
\item
\verb|setglobal|, \verb|rawsetglobal|, and \verb|sort| no longer return a value;
\verb|type| no longer return a second value.
\item
In nested function calls like \verb|f(g(x))|
\emph{all} return values from \verb|g| are passed as arguments to \verb|f|.
(This only happens when \verb|g| is the last
[or the only] argument to \verb|f|.)
\item
There is now only one tag method for order operators.
\item
The debug API has been completely rewritten.
\item
The pre-compiler may use the fact that some operators are associative,
for optimizations.
This may cause problems if these operators
have non-associative tag methods.
\item
All functions from the old API are now macros.
\item
A \verb|const| qualifier has been added to \verb|char *|
in all API functions that handle C~strings.
\item
\verb|luaL_openlib| no longer automatically calls \verb|lua_open|.
So,
you must now explicitly call \verb|lua_open| before opening
the standard libraries.
\item
\verb|lua_type| now returns a string describing the type,
and is no longer a synonym for \verb|lua_tag|.
\item Old pre-compiled code is obsolete, and must be re-compiled.

\end{itemize}

%{===============================================================
\section*{The complete syntax of Lua}

\renewenvironment{Produc}{\vspace{0.8ex}\par\noindent\hspace{3ex}\it\begin{tabular}{rrl}}{\end{tabular}\vspace{0.8ex}\par\noindent}

\renewcommand{\OrNL}{\\ & \Or & }

\begin{Produc}

\produc{chunk}{\rep{stat} \opt{ret}}

\produc{block}{\opt{label} \rep{stat \opt{\ter{;}}}}

\produc{label}{\ter{$\vert$} name \ter{$\vert$}}

\produc{stat}{%
	varlist1 \ter{=} explist1
\OrNL	functioncall
\OrNL	\rwd{do} block \rwd{end}
\OrNL	\rwd{while} exp1 \rwd{do} block \rwd{end}
\OrNL	\rwd{repeat} block \rwd{until} exp1
\OrNL	\rwd{if} exp1 \rwd{then} block
	\rep{\rwd{elseif} exp1 \rwd{then} block}
	\opt{\rwd{else} block} \rwd{end}
\OrNL	\rwd{return} \opt{explist1}
\OrNL	\rwd{break} \opt{name}
\OrNL	\rwd{for} name \ter{=} exp1 \ter{,} exp1 \opt{\ter{,} exp1}
	\rwd{do} block \rwd{end}
\OrNL	\rwd{function} funcname \ter{(} \opt{parlist1} \ter{)} block \rwd{end}
\OrNL	\rwd{local} declist \opt{init}
}

\produc{var}{%
	name
\OrNL	simpleexp \ter{[} exp1 \ter{]}
\OrNL	simpleexp \ter{.} name
}

\produc{varlist1}{var \rep{\ter{,} var}}

\produc{declist}{name \rep{\ter{,} name}}

\produc{init}{\ter{=} explist1}

\produc{exp}{%
	\rwd{nil}
\Or	number
\Or	literal
\Or	function
\Or	simpleexp
\Or	\ter{(} exp \ter{)}
}

\produc{exp1}{exp}

\produc{explist1}{\rep{exp1 \ter{,}} exp}

\produc{simpleexp}{%
	var
\Or	upvalue
\Or	functioncall
\Or	tableconstructor
}

\produc{tableconstructor}{\ter{\{} fieldlist \ter{\}}}
\produc{fieldlist}{%
	lfieldlist
\Or	ffieldlist
\Or	lfieldlist \ter{;} ffieldlist
\Or	ffieldlist \ter{;} lfieldlist
}
\produc{lfieldlist}{\opt{lfieldlist1}}
\produc{ffieldlist}{\opt{ffieldlist1}}
\produc{lfieldlist1}{exp \rep{\ter{,} exp} \opt{\ter{,}}}
\produc{ffieldlist1}{ffield \rep{\ter{,} ffield} \opt{\ter{,}}}
\produc{ffield}{%
	\ter{[} exp \ter{]} \ter{=} exp
\Or	name \ter{=} exp
}

\produc{functioncall}{%
	simpleexp args
\Or	simpleexp \ter{:} name args
}

\produc{args}{%
	\ter{(} \opt{explist1} \ter{)}
\Or	tableconstructor
\Or	\ter{literal}
}

\produc{function}{\rwd{function} \ter{(} \opt{parlist1} \ter{)} block \rwd{end}}

\produc{funcname}{%
	name
\OrNL	name \ter{.} name
\OrNL	name \ter{:} name
}

\produc{parlist1} name}

\end{Produc}
%}===============================================================

% restore underscore to usual meaning
\catcode`\_=8

\newcommand{\indexentry}[2]{\item {#1} #2}
\begin{theindex}
% $Id: manual.tex,v 1.36 2000/04/17 19:23:48 roberto Exp roberto $

\documentclass[11pt]{article}
\usepackage{fullpage,bnf}
\usepackage{graphicx}
%\usepackage{times}

\catcode`\_=12

\newcommand{\See}[1]{Section~\ref{#1}}
\newcommand{\see}[1]{(see \See{#1})}
\newcommand{\M}[1]{\rm\emph{#1}}
\newcommand{\T}[1]{{\tt #1}}
\newcommand{\Math}[1]{$#1$}
\newcommand{\nil}{{\bf nil}}
\def\tecgraf{{\sf TeC\kern-.21em\lower.7ex\hbox{Graf}}}

\newcommand{\Index}[1]{#1\index{#1}}
\newcommand{\IndexVerb}[1]{\T{#1}\index{#1}}
\newcommand{\IndexEmph}[1]{\emph{#1}\index{#1}}
\newcommand{\Def}[1]{\emph{#1}\index{#1}}
\newcommand{\Deffunc}[1]{\index{#1}}

\newcommand{\ff}{$\bullet$\ }

\newcommand{\Version}{4.0}

% LHF
\renewcommand{\ter}[1]{{\rm`{\tt#1}'}}
\newcommand{\NOTE}{\par\noindent\emph{NOTE}: }

\makeindex

\begin{document}

%{===============================================================
\thispagestyle{empty}
\pagestyle{empty}

{
\parindent=0pt
\vglue1.5in
{\LARGE\bf
The Programming Language Lua}
\hfill
\vskip4pt \hrule height 4pt width \hsize \vskip4pt
\hfill
Reference Manual for Lua version \Version
\\
\null
\hfill
Last revised on \today
\\
\vfill
\centering
\includegraphics[width=0.7\textwidth]{nolabel.ps}
\vfill
\vskip4pt \hrule height 2pt width \hsize
}

\newpage
\begin{quotation}
\parskip=10pt
\footnotesize
\null\vfill

\noindent
Copyright \copyright\ 1994--2000 TeCGraf, PUC-Rio.  All rights reserved.

\noindent
Permission is hereby granted, without written agreement and without license
or royalty fees, to use, copy, modify, and distribute this software and its
documentation for any purpose, including commercial applications, subject to
the following conditions:
\begin{itemize}
\item The above copyright notice and this permission notice shall appear in all
   copies or substantial portions of this software.

\item The origin of this software must not be misrepresented; you must not
   claim that you wrote the original software. If you use this software in a
   product, an acknowledgment in the product documentation would be greatly
   appreciated (but it is not required).

\item Altered source versions must be plainly marked as such, and must not be
   misrepresented as being the original software.
\end{itemize}
The authors specifically disclaim any warranties, including, but not limited
to, the implied warranties of merchantability and fitness for a particular
purpose.  The software provided hereunder is on an ``as is'' basis, and the
authors have no obligation to provide maintenance, support, updates,
enhancements, or modifications.  In no event shall TeCGraf, PUC-Rio, or the
authors be held liable to any party for direct, indirect, special,
incidental, or consequential damages arising out of the use of this software
and its documentation.

\noindent
The Lua language and this implementation have been entirely designed and
written by Waldemar Celes, Roberto Ierusalimschy and Luiz Henrique de
Figueiredo at TeCGraf, PUC-Rio.

\noindent
This implementation contains no third-party code.

\noindent
Copies of this manual can be obtained at
\verb|http://www.tecgraf.puc-rio.br/lua/|.
\end{quotation}
%}===============================================================
\newpage

\title{Reference Manual of the Programming Language Lua \Version}

\author{%
Roberto Ierusalimschy\quad
Luiz Henrique de Figueiredo\quad
Waldemar Celes
\vspace{1.0ex}\\
\smallskip
\small\tt lua@tecgraf.puc-rio.br
\vspace{2.0ex}\\
%MCC 08/95 ---
\tecgraf\ --- Computer Science Department --- PUC-Rio
}

\date{{\small \tt\$Date: 2000/04/17 19:23:48 $ $}}

\maketitle

\thispagestyle{empty}
\pagestyle{empty}

\begin{abstract}
\noindent
Lua is a powerful, light-weight programming language
designed for extending applications.
Lua is also frequently used as a general-purpose, stand-alone language.
Lua combines simple procedural syntax
(similar to Pascal)
with
powerful data description constructs
based on associative arrays and extensible semantics.
Lua is
dynamically typed,
interpreted from bytecodes,
and has automatic memory management with garbage collection,
making it ideal for
configuration,
scripting,
and
rapid prototyping.

This document describes version \Version\ of the Lua programming language
and the API that allows interaction between Lua programs and their
host C programs.
\end{abstract}

\def\abstractname{Resumo}
\begin{abstract}
\noindent
Lua \'e uma linguagem de programa\c{c}\~ao
poderosa e leve,
projetada para extender aplica\c{c}\~oes.
Lua tamb\'em \'e frequentemente usada como uma linguagem de prop\'osito geral.
Lua combina programa\c{c}\~ao procedural
(com sintaxe semelhante \`a de Pascal)
com
poderosas constru\c{c}\~oes para descri\c{c}\~ao de dados,
baseadas em tabelas associativas e sem\^antica extens\'\i vel.
Lua \'e
tipada dinamicamente,
interpretada a partir de \emph{bytecodes},
e tem gerenciamento autom\'atico de mem\'oria com coleta de lixo.
Essas caracter\'{\i}sticas fazem de Lua uma linguagem ideal para
configura\c{c}\~ao,
automa\c{c}\~ao (\emph{scripting})
e prototipagem r\'apida.

Este documento descreve a vers\~ao \Version\ da linguagem de
programa\c{c}\~ao Lua e a Interface de Programa\c{c}\~ao (API) que permite
a intera\c{c}\~ao entre programas Lua e programas C hospedeiros.
\end{abstract}

\newpage
\null
\newpage
\tableofcontents

\newpage
\setcounter{page}{1}
\pagestyle{plain}


\section{Introduction}

Lua is an extension programming language designed to support
general procedural programming with data description
facilities.
Lua is intended to be used as a powerful, light-weight
configuration language for any program that needs one.

Lua is implemented as a library, written in C.
Being an extension language, Lua has no notion of a ``main'' program:
it only works \emph{embedded} in a host client,
called the \emph{embedding} program.
This host program can invoke functions to execute a piece of
code in Lua, can write and read Lua variables,
and can register C~functions to be called by Lua code.
Through the use of C~functions, Lua can be augmented to cope with
a wide range of different domains,
thus creating customized programming languages sharing a syntactical framework.

Lua is free-distribution software,
and provided as usual with no guarantees,
as stated in the copyright notice.
The implementation described in this manual is available
at the following URL's:
\begin{verbatim}
   http://www.tecgraf.puc-rio.br/lua/
   ftp://ftp.tecgraf.puc-rio.br/pub/lua/
\end{verbatim}

Like any other reference manual,
this document is dry in places.
For a discussion of the decisions behind the design of Lua,
see the papers below,
which are available at the web site above.
\begin{itemize}
\item
R.~Ierusalimschy, L.~H.~de Figueiredo, and W.~Celes.
Lua---an extensible extension language.
\emph{Software: Practice \& Experience} {\bf 26} \#6 (1996) 635--652.
\item
L.~H.~de Figueiredo, R.~Ierusalimschy, and W.~Celes.
The design and implementation of a language for extending applications.
\emph{Proceedings of XXI Brazilian Seminar on Software and Hardware} (1994) 273--283.
\item
L.~H.~de Figueiredo, R.~Ierusalimschy, and W.~Celes.
Lua: an extensible embedded language.
\emph{Dr. Dobb's Journal} {\bf  21} \#12 (Dec 1996) 26--33.
\end{itemize}

\section{Environment and Chunks}

All statements in Lua are executed in a \Def{global environment}.
This environment, which keeps all global variables,
is initialized with a call from the embedding program to
\verb|lua_newstate| and
persists until a call to \verb|lua_close|,
or the end of the embedding program.
Optionally, a user can create multiple independent global
environments, and freely switch between them \see{mangstate}.

The global environment can be manipulated by Lua code or
by the embedding program,
which can read and write global variables
using API functions from the library that implements Lua.

\Index{Global variables} do not need declaration.
Any variable is assumed to be global unless explicitly declared local
\see{localvar}.
Before the first assignment, the value of a global variable is \nil;
this default can be changed \see{tag-method}.

The unit of execution of Lua is called a \Def{chunk}.
A chunk is simply a sequence of statements:
\begin{Produc}
\produc{chunk}{\rep{stat} \opt{ret}}
\end{Produc}%
Statements are described in \See{stats}.
(The notation above is the usual extended BNF,
in which
\rep{\emph{a}} means 0 or more \emph{a}'s,
\opt{\emph{a}} means an optional \emph{a}, and
\oneormore{\emph{a}} means one or more \emph{a}'s.)

A chunk may be in a file or in a string inside the host program.
A chunk may optionally end with a \verb|return| statement \see{return}.
When a chunk is executed, first all its code is pre-compiled,
and then the statements are executed in sequential order.
All modifications a chunk effects on the global environment persist
after the chunk ends.

Chunks may also be pre-compiled into binary form;
see program \IndexVerb{luac} for details.
Text files with chunks and their binary pre-compiled forms
are interchangeable.
Lua automatically detects the file type and acts accordingly.
\index{pre-compilation}

\section{\Index{Types and Tags}} \label{TypesSec}

Lua is a \emph{dynamically typed language}.
This means that
variables do not have types; only values do.
Therefore, there are no type definitions in the language.
All values carry their own type.
Besides a type, all values also have a \IndexEmph{tag}.

There are six \Index{basic types} in Lua: \Def{nil}, \Def{number},
\Def{string}, \Def{function}, \Def{userdata}, and \Def{table}.
\emph{Nil} is the type of the value \nil,
whose main property is to be different from any other value.
\emph{Number} represents real (double-precision floating-point) numbers,
while \emph{string} has the usual meaning.
Lua is \Index{eight-bit clean},
and so strings may contain any 8-bit character,
\emph{including} embedded zeros (\verb|'\0'|) \see{lexical}.
The \verb|type| function returns a string describing the type
of a given value \see{pdf-type}.

Functions are considered \emph{first-class values} in Lua.
This means that functions can be stored in variables,
passed as arguments to other functions, and returned as results.
Lua can call (and manipulate) functions written in Lua and
functions written in C.
The kinds of functions can be distinguished by their tags:
all Lua functions have the same tag,
and all C~functions have the same tag,
which is different from the tag of Lua functions.
The \verb|tag| function returns the tag
of a given value \see{pdf-tag}.

The type \emph{userdata} is provided to allow
arbitrary \Index{C pointers} to be stored in Lua variables.
It corresponds to a \verb|void*| and has no pre-defined operations in Lua,
besides assignment and equality test.
However, by using \emph{tag methods},
the programmer can define operations for \emph{userdata} values
\see{tag-method}.

The type \emph{table} implements \Index{associative arrays},
that is, \Index{arrays} that can be indexed not only with numbers,
but with any value (except \nil).
Therefore, this type may be used not only to represent ordinary arrays,
but also symbol tables, sets, records, etc.
Tables are the main data structuring mechanism in Lua.
To represent \Index{records}, Lua uses the field name as an index.
The language supports this representation by
providing \verb|a.name| as syntactic sugar for \verb|a["name"]|.
Tables may also carry \emph{methods}:
Because functions are first class values,
table fields may contain functions.
The form \verb|t:f(x)| is syntactic sugar for \verb|t.f(t,x)|,
which calls the method \verb|f| from the table \verb|t| passing
itself as the first parameter \see{func-def}.

Note that tables are \emph{objects}, and not values.
Variables cannot contain tables, only \emph{references} to them.
Assignment, parameter passing, and returns always manipulate references
to tables, and do not imply any kind of copy.
Moreover, tables must be explicitly created before used
\see{tableconstructor}.

Tags are mainly used to select \emph{tag methods} when
some events occur.
Tag methods are the main mechanism for extending the
semantics of Lua \see{tag-method}.
Each of the types \M{nil}, \M{number}, and \M{string} has a different tag.
All values of each of these types have the same pre-defined tag.
Values of type \M{function} can have two different tags,
depending on whether they are Lua functions or C~functions.
Finally,
values of type \M{userdata} and \M{table} have
variable tags, assigned by the program \see{tag-method}.
Tags are created with the function \verb|newtag|,
and the function \verb|tag| returns the tag of a given value.
To change the tag of a given table,
there is the function \verb|settag| \see{pdf-newtag}.


\section{The Language}

This section describes the lexis, the syntax, and the semantics of Lua.


\subsection{Lexical Conventions} \label{lexical}

\IndexEmph{Identifiers} in Lua can be any string of letters,
digits, and underscores,
not beginning with a digit.
This coincides with the definition of identifiers in most languages,
except that
the definition of letter depends on the current locale:
Any character considered alphabetic by the current locale
can be used in an identifier.
The following words are \emph{reserved}, and cannot be used as identifiers:
\index{reserved words}
\begin{verbatim}
   and       break     do        else
   elseif    end       for       function
   if        local     nil       not
   or        repeat    return    then
   until     while
\end{verbatim}
Lua is a case-sensitive language:
\T{and} is a reserved word, but \T{And} and \T{\'and}
(if the locale permits) are two different, valid identifiers.
As a convention, identifiers starting with underscore followed by
uppercase letters (such as \verb|_INPUT|)
are reserved for internal variables.

The following strings denote other \Index{tokens}:
\begin{verbatim}
   ~=  <=  >=  <   >   ==  =   +   -   *   /   %
   (   )   {   }   [   ]   ;   ,   .   ..  ...
\end{verbatim}

\IndexEmph{Literal strings} can be delimited by matching single or double quotes,
and can contain the C-like escape sequences
\verb|'\a'| (bell),
\verb|'\b'| (backspace),
\verb|'\f'| (form feed),
\verb|'\n'| (newline),
\verb|'\r'| (carriage return),
\verb|'\t'| (horizontal tab),
\verb|'\v'| (vertical tab),
\verb|'\\'|, (backslash),
\verb|'\"'|, (double quote),
\verb|'\''| (single quote),
and \verb|'\|\emph{newline}\verb|'| (that is, a backslash followed by a real newline,
which  results in a newline in the string).
A character in a string may also be specified by its numerical value,
through the escape sequence \verb|'\ddd'|,
where \verb|ddd| is a sequence of up to three \emph{decimal} digits.
Strings in Lua may contain any 8-bit value, including embedded zeros,
which can be specified as \verb|'\000'|.

Literal strings can also be delimited by matching \verb|[[| \dots\ \verb|]]|.
Literals in this bracketed form may run for several lines,
may contain nested \verb|[[ ... ]]| pairs,
and do not interpret escape sequences.
This form is specially convenient for
writing strings that contain program pieces or
other quoted strings.
As an example, in a system using ASCII,
the following three literals are equivalent:
\begin{verbatim}
1) "alo\n123\""
2) '\97lo\10\04923"'
3) [[alo
   123"]]
\end{verbatim}


\Index{Comments} start anywhere outside a string with a
double hyphen (\verb|--|) and run until the end of the line.
Moreover,
the first line of a chunk is skipped if it starts with \verb|#|.
This facility allows the use of Lua as a script interpreter
in Unix systems \see{lua-sa}.

\Index{Numerical constants} may be written with an optional decimal part,
and an optional decimal exponent.
Examples of valid numerical constants are
\begin{verbatim}
   3     3.0     3.1416  314.16e-2   0.31416E1
\end{verbatim}

\subsection{The \Index{Pre-processor}} \label{pre-processor}

All lines that start with a \verb|$| sign are handled by a pre-processor.
The following directives are understood by the pre-processor:
\begin{description}
\item[\T{\$debug}] --- turn on debugging facilities \see{pragma}.
\item[\T{\$nodebug}] --- turn off debugging facilities \see{pragma}.
\item[\T{\$if \M{cond}}] --- start a conditional part.
If \M{cond} is false, then this part is skipped by the lexical analyzer.
\item[\T{\$ifnot \M{cond}}] --- start a conditional part.
If \M{cond} is true, then this part is skipped by the lexical analyzer.
\item[\T{\$end}] --- end a conditional part.
\item[\T{\$else}] --- start an ``else'' conditional part,
flipping the ``skip'' status.
\item[\T{\$endinput}] --- end the lexical parse of the chunk.
For all purposes,
it is as if the chunk physically ended at this point.
\end{description}

Directives may be freely nested.
In particular, a \verb|$endinput| may occur inside a \verb|$if|;
in that case, even the matching \verb|$end| is not parsed.

A \M{cond} part may be
\begin{description}
\item[\T{nil}] --- always false.
\item[\T{1}] --- always true.
\item[\T{\M{name}}] --- true if the value of the
global variable \M{name} is different from \nil.
Note that \M{name} is evaluated \emph{before} the chunk starts its execution.
Therefore, actions in a chunk do not affect its own conditional directives.
\end{description}

\subsection{\Index{Coercion}} \label{coercion}

Lua provides some automatic conversions between values at run time.
Any arithmetic operation applied to a string tries to convert
that string to a number, following the usual rules.
Conversely, whenever a number is used when a string is expected,
that number is converted to a string, in a reasonable format.
The format is chosen so that
a conversion from number to string then back to number
reproduces the original number \emph{exactly}.
Thus,
the conversion does not necessarily produces nice-looking text for some numbers.
For complete control on how numbers are converted to strings,
use the \verb|format| function \see{format}.


\subsection{\Index{Adjustment}} \label{adjust}

Functions in Lua can return many values.
Because there are no type declarations,
when a function is called
the system does not know how many values the function will return,
or how many parameters it needs.
Therefore, sometimes, a list of values must be \emph{adjusted}, at run time,
to a given length.
If there are more values than are needed,
then the excess values are thrown away.
If there are less values than are needed,
then the list is extended with as many  \nil's as needed.
This adjustment occurs in multiple assignments \see{assignment}
and function calls \see{functioncall}.


\subsection{Statements}\label{stats}

Lua supports an almost conventional set of \Index{statements},
similar to those in Pascal or C.
The conventional commands include
assignment, control structures, and procedure calls.
Non-conventional commands include table constructors
\see{tableconstructor}
and local variable declarations \see{localvar}.

\subsubsection{Blocks}
A \Index{block} is a list of statements, which are executed sequentially.
A statement may be have an optional \Index{label},
which is syntactically an identifier,
and can be optionally followed by a semicolon:
\begin{Produc}
\produc{block}{\opt{label} \rep{stat \opt{\ter{;}}}}
\produc{label}{\ter{$\vert$} name \ter{$\vert$}}
\end{Produc}%
\NOTE
For syntactic reasons, the \rwd{return} and
\rwd{break} statements can only be written
as the last statement of a block.

A block may be explicitly delimited:
\begin{Produc}
\produc{stat}{\rwd{do} block \rwd{end}}
\end{Produc}%
This is useful to control the scope of local variables \see{localvar},
and to add a \rwd{return} or \rwd{break} statement in the middle
of another block:
\begin{verbatim}
  do return end        -- return is the last statement in this block
\end{verbatim}

\subsubsection{\Index{Assignment}} \label{assignment}
The language allows \Index{multiple assignment}.
Therefore, the syntax for assignment
defines a list of variables on the left side
and a list of expressions on the right side.
Both lists have their elements separated by commas:
\begin{Produc}
\produc{stat}{varlist1 \ter{=} explist1}
\produc{varlist1}{var \rep{\ter{,} var}}
\end{Produc}%
This statement first evaluates all values on the right side
and eventual indices on the left side,
and then makes the assignments.
So
\begin{verbatim}
   i = 3
   i, a[i] = 4, 20
\end{verbatim}
sets \verb|a[3]| to 20, but does not affect \verb|a[4]|.

Multiple assignment can be used to exchange two values, as in
\begin{verbatim}
   x, y = y, x
\end{verbatim}

The two lists in a multiple assignment may have different lengths.
Before the assignment, the list of values is adjusted to
the length of the list of variables \see{adjust}.

A single name can denote a global variable, a local variable,
or a formal parameter:
\begin{Produc}
\produc{var}{name}
\end{Produc}%
Square brackets are used to index a table:
\begin{Produc}
\produc{var}{simpleexp \ter{[} exp1 \ter{]}}
\end{Produc}%
The \M{simpleexp} should result in a table value,
from where the field indexed by the expression \M{exp1}
value gets the assigned value.

The syntax \verb|var.NAME| is just syntactic sugar for
\verb|var["NAME"]|:
\begin{Produc}
\produc{var}{simpleexp \ter{.} name}
\end{Produc}%

The meaning of assignments and evaluations of global variables and
indexed variables can be changed by tag methods \see{tag-method}.
Actually,
an assignment \verb|x = val|, where \verb|x| is a global variable,
is equivalent to a call \verb|setglobal("x", val)|;
an assignment \verb|t[i] = val| is equivalent to
\verb|settable_event(t,i,val)|.
See \See{tag-method} for a complete description of these functions.
(The function \verb|setglobal| is pre-defined in Lua.
The function \T{settable\_event} is used only for explanatory purposes.)

\subsubsection{Control Structures}
The control structures 
\index{while-do}\index{repeat-until}\index{if-then-else}%
\T{if}, \T{while}, and \T{repeat} have the usual meaning and
familiar syntax:
\begin{Produc}
\produc{stat}{\rwd{while} exp1 \rwd{do} block \rwd{end}}
\produc{stat}{\rwd{repeat} block \rwd{until} exp1}
\produc{stat}{\rwd{if} exp1 \rwd{then} block
  \rep{\rwd{elseif} exp1 \rwd{then} block}
   \opt{\rwd{else} block} \rwd{end}}
\end{Produc}%
The \Index{condition expression} \M{exp1} of a control structure may return any value.
All values different from \nil\ are considered true;
only \nil\ is considered false.

\index{return}
The \rwd{return} statement is used to return values from a function or from a chunk.
\label{return}
Because functions or chunks may return more than one value,
the syntax for a \Index{return statement} is
\begin{Produc}
\produc{stat}{\rwd{return} \opt{explist1}}
\end{Produc}%

\index{break}
The \rwd{break} statement can be used to terminate the execution of a block,
skipping to the next statement after the block:
\begin{Produc}
\produc{stat}{\rwd{break} \opt{name}}
\end{Produc}%
A \rwd{break} without a label ends the innermost enclosing loop
(while, repeat, or for).
A \rwd{break} with a label breaks the innermost enclosing
statement with that label.
Thus,
labels do not have to be unique.

For syntactic reasons, \rwd{return} and \rwd{break}
statements can only be written as the last statement of a block.

\subsubsection{For Statement} \label{for}\index{for}

The \rwd{for} statement has the following syntax:
\begin{Produc}
\produc{stat}{\rwd{for} name \ter{=} exp1 \ter{,} exp1 \opt{\ter{,} exp1}
                    \rwd{do} block \rwd{end}}
\end{Produc}%
A \rwd{for} statement like
\begin{verbatim}
   for var=e1,e2,e3 do block end
\end{verbatim}
is equivalent to the following code:
\begin{verbatim}
   do
     local var, _limit, _step = tonumber(e1), tonumber(e2), tonumber(e3)
     if not (var and _limit and _step) then error() end
     while (_step>0 and var<=_limit) or (_step<=0 and var>=_limit) do
       block
       var = var+_step
     end
   end
\end{verbatim}
Notice the following:
\begin{itemize}\itemsep=0pt
\item \verb|_limit| and \verb|_step| are invisible variables.
The names are here for explanatory purposes only.
\item The behavior is \emph{undefined} if you assign to \verb|var| inside
the block.
\item If the third expression (the step) is absent, then a step of 1 is used.
\item Both the limit and the step are evaluated only once,
before the loop starts.
\item The variable \verb|var| is local to the statement;
you cannot use its value after the \rwd{for} ends.
\item You can use \rwd{break} to exit a \rwd{for}.
If you need the value of the index,
then assign it to another variable before breaking.
\end{itemize}

\subsubsection{Function Calls as Statements} \label{funcstat}
Because of possible side-effects,
function calls can be executed as statements:
\begin{Produc}
\produc{stat}{functioncall}
\end{Produc}%
In this case, all returned values are thrown away.
Function calls are explained in \See{functioncall}.

\subsubsection{Local Declarations} \label{localvar}
\Index{Local variables} may be declared anywhere inside a block.
The declaration may include an initial assignment:
\begin{Produc}
\produc{stat}{\rwd{local} declist \opt{init}}
\produc{declist}{name \rep{\ter{,} name}}
\produc{init}{\ter{=} explist1}
\end{Produc}%
If present, an initial assignment has the same semantics
of a multiple assignment.
Otherwise, all variables are initialized with \nil.

The scope of local variables begins \emph{after}
the declaration and lasts until the end of the block.
Thus, the code
\verb|local print=print|
creates a local variable called \verb|print| whose
initial value is that of the \emph{global} variable of the same name.


\subsection{\Index{Expressions}}

\subsubsection{\Index{Basic Expressions}}
The basic expressions in Lua are
\begin{Produc}
\produc{exp}{\ter{(} exp \ter{)}}
\produc{exp}{\rwd{nil}}
\produc{exp}{number}
\produc{exp}{literal}
\produc{exp}{function}
\produc{exp}{simpleexp}
\end{Produc}%
\begin{Produc}
\produc{simpleexp}{var}
\produc{simpleexp}{upvalue}
\produc{simpleexp}{functioncall}
\produc{simpleexp}{tableconstructor}
\end{Produc}%

Numbers (numerical constants) and
literal strings are explained in \See{lexical};
variables are explained in \See{assignment};
upvalues are explained in \See{upvalue};
function definitions (\M{function}) are explained in \See{func-def};
function calls are explained in \See{functioncall}.
Table constructors are explained in \See{tableconstructor}.

An access to a global variable \verb|x| is equivalent to a
call \verb|getglobal("x")|;
an access to an indexed variable \verb|t[i]| is equivalent to
a call \verb|gettable_event(t,i)|.
See \See{tag-method} for a description of these functions.
(Function \verb|getglobal| is pre-defined in Lua.
Function \T{gettable\_event} is used only for explanatory purposes.)

The non-terminal \M{exp1} is used to indicate that the values
returned by an expression must be adjusted to one single value:
\begin{Produc}
\produc{exp1}{exp}
\end{Produc}%

\subsubsection{Arithmetic Operators}
Lua supports the usual \Index{arithmetic operators}:
the binary \verb|+| (addition),
\verb|-| (subtraction), \verb|*| (multiplication),
\verb|/| (division) and \verb|^| (exponentiation),
and unary \verb|-| (negation).
If the operands are numbers, or strings that can be converted to
numbers (according to the rules given in \See{coercion}),
then all operations except exponentiation have the usual meaning.
Otherwise, an appropriate tag method is called \see{tag-method}.
An exponentiation always calls a tag method.
The standard mathematical library redefines this method for numbers,
giving the expected meaning to \Index{exponentiation}
\see{mathlib}.

\subsubsection{Relational Operators}
Lua provides the following \Index{relational operators}:
\begin{verbatim}
   ==  ~=  <   >   <=  >=
\end{verbatim}
All these return \nil\ as false and a value different from \nil\ as true.

Equality first compares the tags of its operands.
If they are different, then the result is \nil.
Otherwise, their values are compared.
Numbers and strings are compared in the usual way.
Tables, userdata, and functions are compared by reference,
that is, two tables are considered equal only if they are the \emph{same} table.
The operator \verb|~=| is exactly the negation of equality (\verb|==|).

\NOTE
The conversion rules of \See{coercion}
\emph{do not} apply to equality comparisons.
Thus, \verb|"0"==0| evaluates to \emph{false},
and \verb|t[0]| and \verb|t["0"]| denote different
entries in a table.

The order operators work as follows.
If both arguments are numbers, then they are compared as such.
Otherwise, if both arguments are strings,
then their values are compared using lexicographical order.
Otherwise, the ``lt'' tag method is called \see{tag-method}.

\subsubsection{Logical Operators}
The \Index{logical operators} are
\index{and}\index{or}\index{not}
\begin{verbatim}
   and   or   not
\end{verbatim}
Like control structures, all logical operators
consider \nil\ as false and anything else as true.
The conjunction operator \verb|and| returns \nil\ if its first argument is \nil;
otherwise, it returns its second argument.
The disjunction operator \verb|or| returns its first argument
if it is different from \nil;
otherwise, it returns its second argument.
Both \verb|and| and \verb|or| use \Index{short-cut evaluation},
that is,
the second operand is evaluated only when necessary.

There are two useful Lua idioms with logical operators.
The first idiom is \verb|x = x or v|,
which is equivalent to
\begin{verbatim}
      if x == nil then x = v end
\end{verbatim}
i.e., it sets \verb|x| to a default value \verb|v| when
\verb|x| is not set.
The other idiom is \verb|x = a and b or c|,
which should be read as \verb|x = a and (b or c)|,
is equivalent to
\begin{verbatim}
   if a then x = b else x = c end
\end{verbatim}
provided that \verb|b| is not \nil.

\subsubsection{Concatenation}
The string \Index{concatenation} operator in Lua is
denoted by ``\IndexVerb{..}''.
If both operands are strings or numbers, they are converted to
strings according to the rules in \See{coercion}.
Otherwise, the ``concat'' tag method is called \see{tag-method}.

\subsubsection{Precedence}
\Index{Operator precedence} follows the table below,
from the lower to the higher priority:
\begin{verbatim}
   and   or
   <   >   <=  >=  ~=  ==
   ..
   +   -
   *   /
   not  - (unary)
   ^
\end{verbatim}
All binary operators are left associative,
except for \verb|^| (exponentiation),
which is right associative.
\NOTE
The pre-compiler may rearrange the order of evaluation of
associative operators (such as~\verb|..| or~\verb|+|),
as long as these optimizations do not change normal results.
However, these optimizations may change some results
if you define non-associative
tag methods for these operators.

\subsubsection{Table Constructors} \label{tableconstructor}
Table \Index{constructors} are expressions that create tables;
every time a constructor is evaluated, a new table is created.
Constructors can be used to create empty tables,
or to create a table and initialize some fields.
The general syntax for constructors is
\begin{Produc}
\produc{tableconstructor}{\ter{\{} fieldlist \ter{\}}}
\produc{fieldlist}{lfieldlist \Or ffieldlist \Or lfieldlist \ter{;} ffieldlist
	\Or ffieldlist \ter{;} lfieldlist}
\produc{lfieldlist}{\opt{lfieldlist1}}
\produc{ffieldlist}{\opt{ffieldlist1}}
\end{Produc}%

The form \emph{lfieldlist1} is used to initialize lists:
\begin{Produc}
\produc{lfieldlist1}{exp \rep{\ter{,} exp} \opt{\ter{,}}}
\end{Produc}%
The expressions in the list are assigned to consecutive numerical indices,
starting with 1.
For example,
\begin{verbatim}
   a = {"v1", "v2", 34}
\end{verbatim}
is equivalent to
\begin{verbatim}
  do
    local temp = {}
    temp[1] = "v1"
    temp[2] = "v2"
    temp[3] = 34
    a = temp
  end
\end{verbatim}

The form \emph{ffieldlist1} initializes other fields in a table:
\begin{Produc}
\produc{ffieldlist1}{ffield \rep{\ter{,} ffield} \opt{\ter{,}}}
\produc{ffield}{\ter{[} exp \ter{]} \ter{=} exp \Or name \ter{=} exp}
\end{Produc}%
For example,
\begin{verbatim}
   a = {[f(k)] = g(y), x = 1, y = 3, [0] = b+c}
\end{verbatim}
is equivalent to
\begin{verbatim}
  do
    local temp = {}
    temp[f(k)] = g(y)
    temp.x = 1    -- or temp["x"] = 1
    temp.y = 3    -- or temp["y"] = 3
    temp[0] = b+c
    a = temp
  end
\end{verbatim}
An expression like \verb|{x = 1, y = 4}| is
in fact syntactic sugar for \verb|{["x"] = 1, ["y"] = 4}|.

Both forms may have an optional trailing comma,
and can be used in the same constructor separated by
a semi-collon.
For example, all forms below are correct.
\begin{verbatim}
   x = {;}
   x = {"a", "b",}
   x = {type="list"; "a", "b"}
   x = {f(0), f(1), f(2),; n=3,}
\end{verbatim}

\subsubsection{Function Calls}  \label{functioncall}
A \Index{function call} has the following syntax:
\begin{Produc}
\produc{functioncall}{simpleexp args}
\end{Produc}%
First, \M{simpleexp} is evaluated.
If its value has type \emph{function},
then this function is called,
with the given arguments.
Otherwise, the ``function'' tag method is called,
having as first parameter the value of \M{simpleexp},
and then the original call arguments.

The form
\begin{Produc}
\produc{functioncall}{simpleexp \ter{:} name args}
\end{Produc}%
can be used to call ``methods''.
A call \verb|simpleexp:name(...)|
is syntactic sugar for
\begin{verbatim}
  simpleexp.name(simpleexp, ...)
\end{verbatim}
except that \verb|simpleexp| is evaluated only once.

Arguments have the following syntax:
\begin{Produc}
\produc{args}{\ter{(} \opt{explist1} \ter{)}}
\produc{args}{tableconstructor}
\produc{args}{\ter{literal}}
\produc{explist1}{\rep{exp1 \ter{,}} exp}
\end{Produc}%
All argument expressions are evaluated before the call.
A call of the form \verb|f{...}| is syntactic sugar for
\verb|f({...})|, that is,
the parameter list is a single new table.
A call of the form \verb|f'...'|
(or \verb|f"..."| or \verb|f[[...]]|) is syntactic sugar for
\verb|f('...')|, that is,
the parameter list is a single literal string.

Because a function can return any number of results
\see{return},
the number of results must be adjusted before used.
If the function is called as a statement \see{funcstat},
then its return list is adjusted to~0,
thus discarding all returned values.
If the function is called in a place that needs a single value
(syntactically denoted by the non-terminal \M{exp1}),
then its return list is adjusted to~1,
thus discarding all returned values but the first one.
If the function is called in a place that can hold many values
(syntactically denoted by the non-terminal \M{exp}),
then no adjustment is made.
The only places that can hold many values
is the last (or the only) expression in an assignment,
in an argument list, or in a return statement.
Here are some examples.
\begin{verbatim}
   f();               -- adjusted to 0
   g(f(), x);         -- f() is adjusted to 1 result
   g(x, f());         -- g gets x plus all values returned by f()
   a,b,c = f(), x;    -- f() is adjusted to 1 result (and c gets nil)
   a,b,c = x, f();    -- f() is adjusted to 2
   a,b,c = f();       -- f() is adjusted to 3
   return f();        -- returns all values returned by f()
   return x,y,f();    -- returns a, b, and all values returned by f()
\end{verbatim}

\subsubsection{\Index{Function Definitions}} \label{func-def}

The syntax for function definition is
\begin{Produc}
\produc{function}{\rwd{function} \ter{(} \opt{parlist1} \ter{)}
  block \rwd{end}}
\produc{stat}{\rwd{function} funcname \ter{(} \opt{parlist1} \ter{)}
  block \rwd{end}}
\produc{funcname}{name \Or name \ter{.} name \Or name \ter{:} name}
\end{Produc}%
The statement
\begin{verbatim}
      function f ()
        ...
      end
\end{verbatim}
is just syntactic sugar for
\begin{verbatim}
      f = function ()
            ...
          end
\end{verbatim}
and the statement
\begin{verbatim}
      function o.f ()
        ...
      end
\end{verbatim}
is syntactic sugar for
\begin{verbatim}
      o.f = function ()
              ...
            end
\end{verbatim}

A function definition is an executable expression,
whose value has type \emph{function}.
When Lua pre-compiles a chunk,
all its function bodies are pre-compiled, too.
Then, whenever Lua executes the function definition,
its upvalues are fixed \see{upvalue},
and the function is \emph{instantiated} (or \emph{closed}).
This function instance (or \emph{closure})
is the final value of the expression.
Different instances of the same function
may have different upvalues.

Parameters act as local variables,
initialized with the argument values:
\begin{Produc}
\produc{parlist1}{\ter{\ldots}}
\produc{parlist1}{name \rep{\ter{,} name} \opt{\ter{,} \ter{\ldots}}}
\end{Produc}%
\label{vararg}
When a function is called,
the list of \Index{arguments} is adjusted to
the length of the list of parameters \see{adjust},
unless the function is a \Def{vararg} function,
which is
indicated by the dots (\ldots) at the end of its parameter list.
A vararg function does not adjust its argument list;
instead, it collects all extra arguments into an implicit parameter,
called \IndexVerb{arg}.
The value of \verb|arg| is a table,
with a field~\verb|n| whose value is the number of extra arguments,
and the extra arguments at positions 1,~2,~\ldots,\M{n}.

As an example, consider the following definitions:
\begin{verbatim}
   function f(a, b) end
   function g(a, b, ...) end
   function r() return 1,2,3 end
\end{verbatim}
Then, we have the following mapping from arguments to parameters:
\begin{verbatim}
   CALL            PARAMETERS

   f(3)             a=3, b=nil
   f(3, 4)          a=3, b=4
   f(3, 4, 5)       a=3, b=4
   f(r(), 10)       a=1, b=10
   f(r())           a=1, b=2

   g(3)             a=3, b=nil, arg={n=0}
   g(3, 4)          a=3, b=4, arg={n=0}
   g(3, 4, 5, 8)    a=3, b=4, arg={5, 8; n=2}
   g(5, r())        a=5, b=1, arg={2, 3; n=2}
\end{verbatim}

Results are returned using the \verb|return| statement \see{return}.
If control reaches the end of a function
without encountering a \rwd{return} statement,
then the function returns with no results.

The syntax
\begin{Produc}
\produc{funcname}{name \ter{:} name}
\end{Produc}%
is used for defining \Index{methods},
that is, functions that have an implicit extra parameter \IndexVerb{self}:
Thus, the statement
\begin{verbatim}
      function v:f (...)
        ...
      end
\end{verbatim}
is equivalent to
\begin{verbatim}
      v.f = function (self, ...)
        ...
      end
\end{verbatim}
that is, the function gets an extra formal parameter called \verb|self|.
Note that the variable \verb|v| must have been
previously initialized with a table value.


\subsection{Visibility and Upvalues} \label{upvalue}
\index{Visibility} \index{Upvalues}

A function body may refer to its own local variables
(which include its parameters) and to global variables,
as long as they are not \emph{shadowed} by local
variables from enclosing functions.
A function \emph{cannot} access a local
variable from an enclosing function,
since such variables may no longer exist when the function is called.
However, a function may access the \emph{value} of a local variable
from an enclosing function, using \emph{upvalues},
whose syntax is
\begin{Produc}
\produc{upvalue}{\ter{\%} name}
\end{Produc}%
An upvalue is somewhat similar to a variable expression,
but whose value is \emph{frozen} when the function wherein it
appears is instantiated.
The name used in an upvalue may be the name of any variable visible
at the point where the function is defined,
that is
global variables and local variables from the immediately enclosing function.

Here are some examples:
\begin{verbatim}
      a,b,c = 1,2,3   -- global variables
      local d
      function f (x)
        local b       -- x and b are local to f; b shadows the global b
        local g = function (a)
          local y     -- a and y are local to g
          p = a       -- OK, access local 'a'
          p = c       -- OK, access global 'c'
          p = b       -- ERROR: cannot access a variable in outer scope
          p = %b      -- OK, access frozen value of 'b' (local to 'f')
          p = %c      -- OK, access frozen value of global 'c'
          p = %y      -- ERROR: 'y' is not visible where 'g' is defined
          p = %d      -- ERROR: 'd' is not visible where 'g' is defined
        end           -- g
      end             -- f
\end{verbatim}


\subsection{Error Handling} \label{error}

Because Lua is an extension language,
all Lua actions start from C~code in the host program
calling a function from the Lua library.
Whenever an error occurs during Lua compilation or execution,
the function \verb|_ERRORMESSAGE| is called \Deffunc{_ERRORMESSAGE}
(provided it is different from \nil),
and then the corresponding function from the library
(\verb|lua_dofile|, \verb|lua_dostring|,
\verb|lua_dobuffer|, or \verb|lua_callfunction|)
is terminated, returning an error condition.

The only argument to \verb|_ERRORMESSAGE| is a string
describing the error.
The default definition for
this function calls \verb|_ALERT|, \Deffunc{_ALERT}
which prints the message to \verb|stderr| \see{alert}.
The standard I/O library redefines \verb|_ERRORMESSAGE|,
and uses the debug facilities \see{debugI}
to print some extra information,
such as a call stack traceback.

To provide more information about errors,
Lua programs should include the compilation pragma \verb|$debug|,
\index{debug pragma}\label{pragma}
or be loaded from the host after calling \verb|lua_setdebug(1)|
\see{debugI}.
When an error occurs in a chunk compiled with this option,
the I/O error-message routine is able to print the number of the
lines where the calls (and the error) were made.

Lua code can explicitly generate an error by calling the built-in
function \verb|error| \see{pdf-error}.
Lua code can ``catch'' an error using the built-in function
\verb|call| \see{pdf-call}.


\subsection{Tag Methods} \label{tag-method}

Lua provides a powerful mechanism to extend its semantics,
called \Def{tag methods}.
A tag method is a programmer-defined function
that is called at specific key points during the evaluation of a program,
allowing the programmer to change the standard Lua behavior at these points.
Each of these points is called an \Def{event}.

The tag method called for any specific event is selected
according to the tag of the values involved
in the event \see{TypesSec}.
The function \IndexVerb{settagmethod} changes the tag method
associated with a given pair \M{(tag, event)}.
Its first parameter is the tag, the second parameter is the event name
(a string; see below),
and the third parameter is the new method (a function),
or \nil\ to restore the default behavior for the pair.
The \verb|settagmethod| function returns the previous tag method for that pair.
Another function, \IndexVerb{gettagmethod},
receives a tag and an event name and returns the
current method associated with the pair.

Tag methods are called in the following events,
identified by the given names.
The semantics of tag methods is better explained by a Lua function
describing the behavior of the interpreter at each event.
This function not only shows when a tag method is called,
but also its arguments, its results, and the default behavior.
The code shown here is only \emph{illustrative};
the real behavior is hard coded in the interpreter,
and it is much more efficient than this simulation.
All functions used in these descriptions
(\verb|rawgetglobal|, \verb|tonumber|, \verb|call|, etc.)
are described in \See{predefined}.

\begin{description}

\item[``add'':]\index{add event}
called when a \verb|+| operation is applied to non numerical operands.

The function \verb|getbinmethod| defines how Lua chooses a tag method
for a binary operation.
First, Lua tries the first operand.
If its tag does not define a tag method for the operation,
then Lua tries the second operand.
If it also fails, then it gets a tag method from tag~0:
\begin{verbatim}
      function getbinmethod (op1, op2, event)
        return gettagmethod(tag(op1), event) or
               gettagmethod(tag(op2), event) or
               gettagmethod(0, event)
      end
\end{verbatim}
Using this function,
the tag method for the ``add' event is
\begin{verbatim}
      function add_event (op1, op2)
        local o1, o2 = tonumber(op1), tonumber(op2)
        if o1 and o2 then  -- both operands are numeric
          return o1+o2  -- '+' here is the primitive 'add'
        else  -- at least one of the operands is not numeric
          local tm = getbinmethod(op1, op2, "add")
          if tm then
            -- call the method with both operands and an extra
            -- argument with the event name
            return tm(op1, op2, "add")
          else  -- no tag method available: default behavior
            error("unexpected type at arithmetic operation")
          end
        end
      end
\end{verbatim}

\item[``sub'':]\index{sub event}
called when a \verb|-| operation is applied to non numerical operands.
Behavior similar to the ``add'' event.

\item[``mul'':]\index{mul event}
called when a \verb|*| operation is applied to non numerical operands.
Behavior similar to the ``add'' event.

\item[``div'':]\index{div event}
called when a \verb|/| operation is applied to non numerical operands.
Behavior similar to the ``add'' event.

\item[``pow'':]\index{pow event}
called when a \verb|^| operation (exponentiation) is applied.
\begin{verbatim}
      function pow_event (op1, op2)
        local tm = getbinmethod(op1, op2, "pow")
        if tm then
          -- call the method with both operands and an extra
          -- argument with the event name
          return tm(op1, op2, "pow")
        else  -- no tag method available: default behavior
          error("unexpected type at arithmetic operation")
        end
      end
\end{verbatim}

\item[``unm'':]\index{unm event}
called when a unary \verb|-| operation is applied to a non numerical operand.
\begin{verbatim}
      function unm_event (op)
        local o = tonumber(op)
        if o then  -- operand is numeric
          return -o  -- '-' here is the primitive 'unm'
        else  -- the operand is not numeric.
          -- Try to get a tag method from the operand;
          --  if it does not have one, try a "global" one (tag 0)
          local tm = gettagmethod(tag(op), "unm") or
                     gettagmethod(0, "unm")
          if tm then
            -- call the method with the operand, nil, and an extra
            -- argument with the event name
            return tm(op, nil, "unm")
          else  -- no tag method available: default behavior
            error("unexpected type at arithmetic operation")
          end
        end
      end
\end{verbatim}

\item[``lt'':]\index{lt event}
called when an order operation is applied to non-numerical
or non-string operands.
It corresponds to the \verb|<| operator.
\begin{verbatim}
      function lt_event (op1, op2)
        if type(op1) == "number" and type(op2) == "number" then
          return op1 < op2   -- numeric comparison
        elseif type(op1) == "string" and type(op2) == "string" then
          return op1 < op2   -- lexicographic comparison
        else
          local tm = getbinmethod(op1, op2, "lt")
          if tm then
            return tm(op1, op2, "lt")
          else
            error("unexpected type at comparison");
          end
        end
      end
\end{verbatim}
The other order operators use this tag method according to the
usual equivalences:
\begin{verbatim}
   a>b    <=>  b<a
   a<=b   <=>  not (b<a)
   a>=b   <=>  not (a<b)
\end{verbatim}

\item[``concat'':]\index{concatenation event}
called when a concatenation is applied to non string operands.
\begin{verbatim}
      function concat_event (op1, op2)
        if (type(op1) == "string" or type(op1) == "number") and
           (type(op2) == "string" or type(op2) == "number") then
          return op1..op2  -- primitive string concatenation
        else
          local tm = getbinmethod(op1, op2, "concat")
          if tm then
            return tm(op1, op2, "concat")
          else
            error("unexpected type for concatenation")
          end
        end
      end
\end{verbatim}

\item[``index'':]\index{index event}
called when Lua tries to retrieve the value of an index
not present in a table.
See event ``gettable'' for its semantics.

\item[``getglobal'':]\index{getglobal event}
called whenever Lua needs the value of a global variable.
This method can only be set for \nil\ and for tags
created by \verb|newtag|.
Note that
the tag is that of the \emph{current value} of the global variable.
\begin{verbatim}
      function getglobal (varname)
        local value = rawgetglobal(varname)
        local tm = gettagmethod(tag(value), "getglobal")
        if not tm then
          return value
        else
          return tm(varname, value)
        end
      end
\end{verbatim}
The function \verb|getglobal| is pre-defined in Lua \see{predefined}.

\item[``setglobal'':]\index{setglobal event}
called whenever Lua assigns to a global variable.
This method cannot be set for numbers, strings, and tables and
userdata with default tags.
\begin{verbatim}
      function setglobal (varname, newvalue)
        local oldvalue = rawgetglobal(varname)
        local tm = gettagmethod(tag(oldvalue), "setglobal")
        if not tm then
          rawsetglobal(varname, newvalue)
        else
          tm(varname, oldvalue, newvalue)
        end
      end
\end{verbatim}
The function \verb|setglobal| is pre-defined in Lua \see{predefined}.

\item[``gettable'':]\index{gettable event}
called whenever Lua accesses an indexed variable.
This method cannot be set for tables with default tag.
\begin{verbatim}
      function gettable_event (table, index)
        local tm = gettagmethod(tag(table), "gettable")
        if tm then
          return tm(table, index)
        elseif type(table) ~= "table" then
          error("indexed expression not a table");
        else
          local v = rawgettable(table, index)
          tm = gettagmethod(tag(table), "index")
          if v == nil and tm then
            return tm(table, index)
          else
            return v
          end
        end
      end
\end{verbatim}

\item[``settable'':]\index{settable event}
called when Lua assigns to an indexed variable.
This method cannot be set for tables with default tag.
\begin{verbatim}
      function settable_event (table, index, value)
        local tm = gettagmethod(tag(table), "settable")
        if tm then
          tm(table, index, value)
        elseif type(table) ~= "table" then
          error("indexed expression not a table")
        else
          rawsettable(table, index, value)
        end
      end
\end{verbatim}

\item[``function'':]\index{function event}
called when Lua tries to call a non function value.
\begin{verbatim}
      function function_event (func, ...)
        if type(func) == "function" then
          return call(func, arg)
        else
          local tm = gettagmethod(tag(func), "function")
          if tm then
            for i=arg.n,1,-1 do
              arg[i+1] = arg[i]
            end
            arg.n = arg.n+1
            arg[1] = func
            return call(tm, arg)
          else
            error("call expression not a function")
          end
        end
      end
\end{verbatim}

\item[``gc'':]\index{gc event}
called when Lua is ``garbage collecting'' a userdata.
This tag method can be set only from~C,
and cannot be set for a userdata with default tag.
For each userdata to be collected,
Lua does the equivalent of the following function:
\begin{verbatim}
      function gc_event (obj)
        local tm = gettagmethod(tag(obj), "gc")
        if tm then
          tm(obj)
        end
      end
\end{verbatim}
Moreover, at the end of a garbage collection cycle,
Lua does the equivalent of the call \verb|gc_event(nil)|.

\end{description}




\section{The Application Program Interface}

This section describes the API for Lua, that is,
the set of C~functions available to the host program to communicate
with the Lua library.
The API functions can be classified into the following categories:
\begin{enumerate}
\item managing states;
\item exchanging values between C and Lua;
\item executing Lua code;
\item manipulating (reading and writing) Lua objects;
\item calling Lua functions;
\item defining C~functions to be called by Lua;
\item manipulating references to Lua Objects.
\end{enumerate}
All API functions and related types and constants
are declared in the header file \verb|lua.h|.

\NOTE
Even when we use the term \emph{function},
\emph{any facility in the API may be provided as a macro instead}.
All such macros use each of its arguments exactly once,
and so do not generate hidden side-effects.


\subsection{States} \label{mangstate}

The Lua library is reentrant:
it does not have any global variable.
The whole state of the Lua interpreter
(global variables, stack, tag methods, etc.)
is stored in a dynamic structure; \Deffunc{lua_State}
this state must be passed as the first argument to almost
every function in the library.

Before calling any API function,
you must create a state.
This is done by calling\Deffunc{lua_newstate}
\begin{verbatim}
lua_State *lua_newstate (const char *s, ...);
\end{verbatim}
The arguments to this function form a list of name-value options,
terminated with \verb|NULL|.
Currently, the function accepts the following options:
\begin{itemize}
\item \verb|"stack"| --- the stack size.
Each function call needs one stack position for each local variable
and temporary variables, plus one position for book-keeping.
The stack must also have at least ten extra positions available.
For very small implementations, without recursive functions,
a stack size of 100 should be enough.
The default stack size is 1024.

\item \verb|"builtin"| --- the value is a boolean (0 or 1) that
indicates whether the predefined functions should be loaded or not.
The default is to load those functions.
\end{itemize}
For instance, the call
\begin{verbatim}
lua_State *L = lua_newstate(NULL);
\end{verbatim}
creates a new state with a stack of 1024 positions
and with the predefined functions loaded;
the call
\begin{verbatim}
lua_State *L = lua_newstate("builtin", 0, "stack", 100, NULL);
\end{verbatim}
creates a new state with a stack of 100 positions,
without the predefined functions.

To release a state, you call
\begin{verbatim}
void lua_close (lua_State *L);
\end{verbatim}
This function destroys all objects in the current Lua environment
(calling the corresponding garbage collection tag methods)
and frees all dynamic memory used by the state.
Usually, you do not need to call this function,
because all resources are naturally released when the program ends.
On the other hand,
long-running programs ---
like a daemon or web server, for example ---
might need to release states as soon as they are not needed,
to avoid growing too big.

With the exception of \verb|lua_newstate|,
all functions in the API need a state as their first argument.
However, most applications use a single state.
To avoid the burden of passing this only state explicitly to all
functions, and also to keep compatibility with old versions of Lua,
the API provides a set of macros and one global variable that
take care of this state argument for single-state applications:
\begin{verbatim}
#ifndef LUA_REENTRANT
\end{verbatim}
\begin{verbatim}
extern lua_State *lua_state;
\end{verbatim}
\begin{verbatim}
#define lua_close()             (lua_close)(lua_state)
#define lua_dofile(filename)    (lua_dofile)(lua_state, filename)
#define lua_dostring(str)       (lua_dostring)(lua_state, str)
   ...
\end{verbatim}
\begin{verbatim}
#endif
\end{verbatim}
For each function in the API, there is a macro with the same name
that supplies \verb|lua_state| as the first argument to the call.
(The parentheses around the function name avoid it being expanded
again as a macro.)
The only exception is \verb|lua_newstate|;
in this case, the corresponding macro is
\begin{verbatim}
#define lua_open()      ((void)(lua_state?0:(lua_state=lua_newstate(0))))
\end{verbatim}
This code checks whether the global state has been initialized;
if not, it creates a new state with default settings and
assigns it to \verb|lua_newstate|.

By default, the single-state macros are all active.
If you need to use multiple states,
and therefore will provide the state argument explicitly in each call,
you should define \IndexVerb{LUA_REENTRANT} before
including \verb|lua.h| in your code:
\begin{verbatim}
#define LUA_REENTRANT
#include "lua.h"
\end{verbatim}

In the sequel, we will show all functions in the single-state form
(therefore, they are actually macros).
When you define \verb|LUA_REENTRANT|,
all of them get a state as the first parameter.


\subsection{Exchanging Values between C and Lua} \label{valuesCLua}
Because Lua has no static type system,
all values passed between Lua and C have type
\verb|lua_Object|\Deffunc{lua_Object},
which works like an abstract type in C that can hold any Lua value.
Values of type \verb|lua_Object| have no meaning outside Lua;
for instance,
you cannot compare two \verb|lua_Object's| directly.
Instead, you should use the following function:
\Deffunc{lua_equal}
\begin{verbatim}
int lua_equal       (lua_Object o1, lua_Object o2);
\end{verbatim}

To check the type of a \verb|lua_Object|,
the following functions are available:
\Deffunc{lua_isnil}\Deffunc{lua_isnumber}\Deffunc{lua_isstring}
\Deffunc{lua_istable}\Deffunc{lua_iscfunction}\Deffunc{lua_isuserdata}
\Deffunc{lua_isfunction}
\Deffunc{lua_type}
\begin{verbatim}
int lua_isnil        (lua_Object object);
int lua_isnumber     (lua_Object object);
int lua_isstring     (lua_Object object);
int lua_istable      (lua_Object object);
int lua_isfunction   (lua_Object object);
int lua_iscfunction  (lua_Object object);
int lua_isuserdata   (lua_Object object);
const char *lua_type (lua_Object object);
\end{verbatim}
The \verb|lua_is*| functions return 1 if the object is compatible
with the given type, and 0 otherwise.
The function \verb|lua_isnumber| accepts numbers and numerical strings,
\verb|lua_isstring| accepts strings and numbers \see{coercion},
and \verb|lua_isfunction| accepts Lua functions and C~functions.
To distinguish between Lua functions and C~functions,
you should use \verb|lua_iscfunction|.
To distinguish between numbers and numerical strings,
you can use \verb|lua_type|.
The \verb|lua_type| returns one of the following strings,
describing the type of the given object:
\verb|"nil"|, \verb|"number"|, \verb|"string"|, \verb|"table"|,
\verb|"function"|, \verb|"userdata"|, or \verb|"NOOBJECT"|.

To get the tag of a \verb|lua_Object|,
use the following function:
\Deffunc{lua_tag}
\begin{verbatim}
int lua_tag (lua_Object object);
\end{verbatim}

To translate a value from type \verb|lua_Object| to a specific C type,
you can use the following conversion functions:
\Deffunc{lua_getnumber}\Deffunc{lua_getstring}\Deffunc{lua_strlen}
\Deffunc{lua_getcfunction}\Deffunc{lua_getuserdata}
\begin{verbatim}
double         lua_getnumber    (lua_Object object);
const char    *lua_getstring    (lua_Object object);
long           lua_strlen       (lua_Object object);
lua_CFunction  lua_getcfunction (lua_Object object);
void          *lua_getuserdata  (lua_Object object);
\end{verbatim}

\verb|lua_getnumber| converts a \verb|lua_Object| to a floating-point number.
This \verb|lua_Object| must be a number or a string convertible to number
\see{coercion}; otherwise, \verb|lua_getnumber| returns~0.

\verb|lua_getstring| converts a \verb|lua_Object| to a string
(\verb|const char*|).
This \verb|lua_Object| must be a string or a number;
otherwise, the function returns \verb|NULL|.
This function does not create a new string,
but returns a pointer to a string inside the Lua environment.
Those strings always have a 0 after their last character (like in C),
but may contain other zeros in their body.
If you do not know whether a string may contain zeros,
you can use \verb|lua_strlen| to get the actual length.
Because Lua has garbage collection,
there is no guarantee that the pointer returned by \verb|lua_getstring|
will be valid after the block ends
\see{GC}.
So,
if you need the string later on,
you should duplicate it with something like
\verb|memcpy(malloc(lua_strlen(o),lua_getstring(o)))|.

\verb|lua_getcfunction| converts a \verb|lua_Object| to a C~function.
This \verb|lua_Object| must be a C~function;
otherwise, \verb|lua_getcfunction| returns \verb|NULL|.
The type \verb|lua_CFunction| is explained in \See{LuacallC}.

\verb|lua_getuserdata| converts a \verb|lua_Object| to \verb|void*|.
This \verb|lua_Object| must have type \emph{userdata};
otherwise, \verb|lua_getuserdata| returns \verb|NULL|.

\subsection{Communication between Lua and C}\label{Lua-C-protocol}

All communication between Lua and C is done through two
abstract data types, called \Def{lua2C} and \Def{C2lua}.
The first one, as the name implies, is used to pass values
from Lua to C:
parameters when Lua calls C and results when C calls Lua.
The structure C2lua is used in the reverse direction:
parameters when C calls Lua and results when Lua calls C.

The structure lua2C is an \emph{abstract array}
that can be indexed with the function:
\Deffunc{lua_lua2C}
\begin{verbatim}
lua_Object lua_lua2C (int number);
\end{verbatim}
where \verb|number| starts with 1.
When called with a number larger than the array size,
this function returns \verb|LUA_NOOBJECT|\Deffunc{LUA_NOOBJECT}.
In this way, it is possible to write C~functions that receive
a variable number of parameters,
and to call Lua functions that return a variable number of results.
Note that the structure lua2C cannot be directly modified by C code.

The structure C2lua is an \emph{abstract stack}.
Pushing elements into this stack
is done with the following functions:
\Deffunc{lua_pushnumber}\Deffunc{lua_pushlstring}\Deffunc{lua_pushstring}
\Deffunc{lua_pushcfunction}\Deffunc{lua_pushusertag}
\Deffunc{lua_pushnil}\Deffunc{lua_pushobject}
\Deffunc{lua_pushuserdata}\label{pushing}
\begin{verbatim}
void lua_pushnumber    (double n);
void lua_pushlstring   (const char *s, long len);
void lua_pushstring    (const char *s);
void lua_pushusertag   (void *u, int tag);
void lua_pushnil       (void);
void lua_pushobject    (lua_Object object);
void lua_pushcfunction (lua_CFunction f);  /* macro */
\end{verbatim}
All of them receive a C value,
convert it to a corresponding \verb|lua_Object|,
and leave the result on the top of C2lua.
In particular, functions \verb|lua_pushlstring| and \verb|lua_pushstring|
make an internal copy of the given string.
Function \verb|lua_pushstring| can only be used to push proper C strings
(that is, strings that end with a zero and do not contain embedded zeros);
otherwise you should use the more general \verb|lua_pushlstring|.
The function
\Deffunc{lua_pop}
\begin{verbatim}
lua_Object lua_pop (void);
\end{verbatim}
returns a reference to the object at the top of the C2lua stack,
and pops it.

As a general rule, all API functions pop from the stack
all elements they use.

When C code calls Lua repeatedly, as in a loop,
objects returned by these calls can accumulate,
and may cause a stack overflow.
To avoid this,
nested blocks can be defined with the functions
\begin{verbatim}
void lua_beginblock (void);
void lua_endblock   (void);
\end{verbatim}
After the end of the block,
all \verb|lua_Object|'s created inside it are released.
The use of explicit nested blocks is good programming practice
and is strongly encouraged.

\subsection{Garbage Collection}\label{GC}
Because Lua has automatic memory management and garbage collection,
a \verb|lua_Object| has a limited scope,
and is only valid inside the \emph{block} where it has been created.
A C~function called from Lua is a block,
and its parameters are valid only until its end.
It is good programming practice to convert Lua objects to C values
as soon as they are available,
and never to store \verb|lua_Object|s in C global variables.

A garbage collection cycle can be forced by:
\Deffunc{lua_collectgarbage}
\begin{verbatim}
long lua_collectgarbage (long limit);
\end{verbatim}
This function returns the number of objects collected.
The argument \verb|limit| makes the next cycle occur only
after that number of new objects have been created.
If \verb|limit| is 0,
then Lua uses an adaptive heuristics to set this limit.


\subsection{Userdata and Tags}\label{C-tags}

Because userdata are objects,
the function \verb|lua_pushusertag| may create a new userdata.
If Lua has a userdata with the given value (\verb|void*|) and tag,
then that userdata is pushed.
Otherwise, a new userdata is created, with the given value and tag.
If this function is called with
\verb|tag| equal to \verb|LUA_ANYTAG|\Deffunc{LUA_ANYTAG},
then Lua will try to find any userdata with the given value,
regardless of its tag.
If there is no userdata with that value, then a new one is created,
with tag equal to 0.

Userdata can have different tags,
whose semantics are only known to the host program.
Tags are created with the function
\Deffunc{lua_newtag}
\begin{verbatim}
int lua_newtag (void);
\end{verbatim}
The function \verb|lua_settag| changes the tag of
the object on the top of C2lua (and pops it);
the object must be a userdata or a table:
\Deffunc{lua_settag}
\begin{verbatim}
void lua_settag (int tag);
\end{verbatim}
The given \verb|tag| must be a value created with \verb|lua_newtag|.

\subsection{Executing Lua Code}
A host program can execute Lua chunks written in a file or in a string
using the following functions:%
\Deffunc{lua_dofile}\Deffunc{lua_dostring}\Deffunc{lua_dobuffer}
\begin{verbatim}
int lua_dofile   (const char *filename);
int lua_dostring (const char *string);
int lua_dobuffer (const char *buff, int size, const char *name);
\end{verbatim}
All these functions return an error code:
0, in case of success; non zero, in case of errors.
More specifically, \verb|lua_dofile| returns 2 if for any reason
it could not open the file.
(In this case,
you may want to
check \verb|errno|,
call \verb|strerror|,
or call \verb|perror| to tell the user what went wrong.)
When called with argument \verb|NULL|,
\verb|lua_dofile| executes the \verb|stdin| stream.
Functions \verb|lua_dofile| and \verb|lua_dobuffer|
are both able to execute pre-compiled chunks.
They automatically detect whether the chunk is text or binary,
and load it accordingly (see program \IndexVerb{luac}).
Function \verb|lua_dostring| executes only source code,
given in textual form.

The third parameter to \verb|lua_dobuffer| (\verb|name|)
is the ``name of the chunk'',
used in error messages and debug information.
If \verb|name| is \verb|NULL|,
then Lua gives a default name to the chunk.

These functions return, in structure lua2C,
any values eventually returned by the chunks.
They also empty the stack C2lua.


\subsection{Manipulating Lua Objects}
To read the value of any global Lua variable,
one uses the function
\Deffunc{lua_getglobal}
\begin{verbatim}
lua_Object lua_getglobal (const char *varname);
\end{verbatim}
As in Lua, this function may trigger a tag method
for the ``getglobal'' event.
To read the real value of any global variable,
without invoking any tag method,
use the \emph{raw} version:
\Deffunc{lua_rawgetglobal}
\begin{verbatim}
lua_Object lua_rawgetglobal (const char *varname);
\end{verbatim}

To store a value previously pushed onto C2lua in a global variable,
there is the function
\Deffunc{lua_setglobal}
\begin{verbatim}
void lua_setglobal (const char *varname);
\end{verbatim}
As in Lua, this function may trigger a tag method
for the ``setglobal'' event.
To set the real value of any global variable,
without invoking any tag method,
use the \emph{raw} version:
\Deffunc{lua_rawgetglobal}
\begin{verbatim}
void lua_rawsetglobal (const char *varname);
\end{verbatim}

Tables can also be manipulated via the API.
The function
\Deffunc{lua_gettable}
\begin{verbatim}
lua_Object lua_gettable (void);
\end{verbatim}
pops a table and an index from the stack C2lua,
and returns the contents of the table at that index.
As in Lua, this operation may trigger a tag method
for the ``gettable'' event.
To get the real value of any table index,
without invoking any tag method,
use the \emph{raw} version:
\Deffunc{lua_rawgetglobal}
\begin{verbatim}
lua_Object lua_rawgettable (void);
\end{verbatim}

To store a value in an index,
the program must push the table, the index, and the value onto C2lua
(in this order),
and then call the function
\Deffunc{lua_settable}
\begin{verbatim}
void lua_settable (void);
\end{verbatim}
As in Lua, this operation may trigger a tag method
for the ``settable'' event.
To set the real value of any table index,
without invoking any tag method,
use the \emph{raw} version:
\Deffunc{lua_rawsettable}
\begin{verbatim}
void lua_rawsettable (void);
\end{verbatim}

Finally, the function
\Deffunc{lua_createtable}
\begin{verbatim}
lua_Object lua_createtable (void);
\end{verbatim}
creates and returns a new, empty table.


\subsection{Calling Lua Functions}
Functions defined in Lua by a chunk
can be called from the host program.
This is done using the following protocol:
first, the arguments to the function are pushed onto C2lua
\see{pushing}, in direct order, i.e., the first argument is pushed first.
Then, the function is called using
\Deffunc{lua_callfunction}
\begin{verbatim}
int lua_callfunction (lua_Object function);
\end{verbatim}
This function returns an error code:
0, in case of success; non zero, in case of errors.
Finally, the results are returned in structure lua2C
(recall that a Lua function may return many values),
and can be retrieved with the macro \verb|lua_getresult|,
\Deffunc{lua_getresult}
which is just another name for the function \verb|lua_lua2C|.
Note that \verb|lua_callfunction|
pops all elements from the C2lua stack.

The following example shows how the host program may do the
equivalent to the Lua code:
\begin{verbatim}
      a,b = f("how", t.x, 4)
\end{verbatim}
\begin{verbatim}
  lua_pushstring("how");                               /* 1st argument */
  lua_pushobject(lua_getglobal("t"));      /* push value of global 't' */
  lua_pushstring("x");                          /* push the string 'x' */
  lua_pushobject(lua_gettable());      /* push result of t.x (2nd arg) */
  lua_pushnumber(4);                                   /* 3rd argument */
  lua_callfunction(lua_getglobal("f"));           /* call Lua function */
  lua_pushobject(lua_getresult(1));   /* push first result of the call */
  lua_setglobal("a");                       /* set global variable 'a' */
  lua_pushobject(lua_getresult(2));  /* push second result of the call */
  lua_setglobal("b");                       /* set global variable 'b' */
\end{verbatim}

Some special Lua functions have exclusive interfaces.
The host program can generate a Lua error calling the function
\Deffunc{lua_error}
\begin{verbatim}
void lua_error (const char *message);
\end{verbatim}
This function never returns.
If \verb|lua_error| is called from a C~function that has been called from Lua,
then the corresponding Lua execution terminates,
as if an error had occurred inside Lua code.
Otherwise, the whole host program terminates with a call to \verb|exit(1)|.
Before terminating execution,
the \verb|message| is passed to the error handler function,
\verb|_ERRORMESSAGE| \see{error}.
If \verb|message| is \verb|NULL|,
then \verb|_ERRORMESSAGE| is not called.

Tag methods can be changed with: \Deffunc{lua_settagmethod}
\begin{verbatim}
lua_Object lua_settagmethod (int tag, const char *event);
\end{verbatim}
The first parameter is the tag,
and the second is the event name \see{tag-method};
the new method is pushed from C2lua.
This function returns a \verb|lua_Object|,
which is the old tag method value.
To get just the current value of a tag method,
use the function \Deffunc{lua_gettagmethod}
\begin{verbatim}
lua_Object lua_gettagmethod (int tag, const char *event);
\end{verbatim}

It is also possible to copy all tag methods from one tag
to another: \Deffunc{lua_copytagmethods}
\begin{verbatim}
int lua_copytagmethods (int tagto, int tagfrom);
\end{verbatim}
This function returns \verb|tagto|.

You can traverse a table with the function \Deffunc{lua_next}
\begin{verbatim}
int lua_next (lua_Object t, int i);
\end{verbatim}
Its first argument is the table to be traversed,
and the second is a \emph{cursor};
this cursor starts in 0,
and for each call the function returns a value to
be used in the next call,
or 0 to signal the end of the traversal.
The function also returns, in the Lua2C array,
a key-value pair from the table.
A typical traversal looks like the following code:
\begin{verbatim}
  int i;
  lua_Object t;
  ...   /* gets the table at `t' */
  i = 0;
  lua_beginblock();
  while ((i = lua_next(t, i)) != 0) {
    lua_Object key = lua_getresult(1);
    lua_Object value = lua_getresult(2);
    ...  /* uses `key' and `value' */
    lua_endblock();
    lua_beginblock();  /* reopens a block */
  }
  lua_endblock();
\end{verbatim}
The pairs of \verb|lua_beginblock|/\verb|lua_endblock| remove the
results of each iteration from the stack.
Without them, a traversal of a large table may overflow the stack.

To traverse the global variables, use \Deffunc{lua_nextvar}
\begin{verbatim}
const char *lua_nextvar (const char *varname);
\end{verbatim}
Here, the cursor is a string;
in the first call you set it to \verb|NULL|;
for each call the function returns the name of a global variable,
to be used in the next call,
or \verb|NULL| to signal the end of the traverse.
The function also returns, in the Lua2C array,
the name (again) and the value of the global variable.
A typical traversal looks like the following code:
\begin{verbatim}
  const char *name = NULL;
  lua_beginblock();
  while ((name = lua_nextvar(name)) != NULL) {
    lua_Object value = lua_getresult(2);
    ...  /* uses `name' and `value' */
    lua_endblock();
    lua_beginblock();  /* reopens a block */
  }
  lua_endblock();
\end{verbatim}


\subsection{Defining C Functions} \label{LuacallC}
To register a C~function to Lua,
there is the following convenience macro:
\Deffunc{lua_register}
\begin{verbatim}
#define lua_register(n,f)       (lua_pushcfunction(f), lua_setglobal(n))
/* const char *n;   */
/* lua_CFunction f; */
\end{verbatim}
which receives the name the function will have in Lua,
and a pointer to the function.
This pointer must have type \verb|lua_CFunction|,
which is defined as
\Deffunc{lua_CFunction}
\begin{verbatim}
typedef void (*lua_CFunction) (void);
\end{verbatim}
that is, a pointer to a function with no parameters and no results.

In order to communicate properly with Lua,
a C~function must follow a protocol,
which defines the way parameters and results are passed.

A C~function receives its arguments in structure lua2C;
to access them, it uses the macro \verb|lua_getparam|, \Deffunc{lua_getparam}
again just another name for \verb|lua_lua2C|.
To return values, a C~function just pushes them onto the stack C2lua,
in direct order \see{valuesCLua}.
Like a Lua function, a C~function called by Lua can also return
many results.

When a C~function is created,
it is possible to associate some \emph{upvalues} to it
\see{upvalue},
thus creating a C closure;
these values are passed to the function whenever it is called,
as common arguments.
To associate upvalues to a C~function,
first these values must be pushed on C2lua.
Then the function \Deffunc{lua_pushcclosure}
\begin{verbatim}
void lua_pushcclosure (lua_CFunction fn, int n);
\end{verbatim}
is used to put the C~function on C2lua,
with the argument \verb|n| telling how many upvalues must be
associated with the function;
in fact, the macro \verb|lua_pushcfunction| is defined as
\verb|lua_pushcclosure| with \verb|n| set to 0.
Then, whenever the C~function is called,
these upvalues are inserted as the first arguments \M{n} to the function,
before the actual arguments provided in the call.

For some examples of C~functions, see files \verb|lstrlib.c|,
\verb|liolib.c| and \verb|lmathlib.c| in the official Lua distribution.
In particular,
\verb|liolib.c| defines C~closures with file handles are upvalues.

\subsection{References to Lua Objects}

As noted in \See{GC}, \verb|lua_Object|s are volatile.
If the C code needs to keep a \verb|lua_Object|
outside block boundaries,
then it must create a \Def{reference} to the object.
The routines to manipulate references are the following:
\Deffunc{lua_ref}\Deffunc{lua_getref}
\Deffunc{lua_unref}
\begin{verbatim}
int        lua_ref    (int lock);
lua_Object lua_getref (int ref);
void       lua_unref  (int ref);
\end{verbatim}
The function \verb|lua_ref| creates a reference
to the object that is on the top of the stack,
and returns this reference.
For a \nil\ object,
the reference is always \verb|LUA_REFNIL|;\Deffunc{LUA_REFNIL}
otherwise, it is a non-negative integer.
The constant \verb|LUA_NOREF| \Deffunc{LUA_NOREF}
is different from any valid reference.
If \verb|lock| is true, then the object is \emph{locked}:
this means the object will not be garbage collected.
\emph{Unlocked references may be garbage collected}.
Whenever the referenced object is needed in~C,
a call to \verb|lua_getref|
returns a handle to it;
if the object has been collected,
\verb|lua_getref| returns \verb|LUA_NOOBJECT|.

When a reference is no longer needed,
it can be released with a call to \verb|lua_unref|.



\section{Predefined Functions and Libraries}

The set of \Index{predefined functions} in Lua is small but powerful.
Most of them provide features that allow some degree of
\Index{reflexivity} in the language.
Some of these features cannot be simulated with the rest of the
language nor with the standard Lua API.
Others are just convenient interfaces to common API functions.

The libraries, on the other hand, provide useful routines
that are implemented directly through the standard API.
Therefore, they are not necessary to the language,
and are provided as separate C modules.
Currently, there are three standard libraries:
\begin{itemize}
\item string manipulation;
\item mathematical functions (sin, log, etc);
\item input and output (plus some system facilities).
\end{itemize}
To have access to these libraries,
the C host program must call the functions
\verb|lua_strlibopen|, \verb|lua_mathlibopen|,
and \verb|lua_iolibopen|, declared in \verb|lualib.h|.
\Deffunc{lua_strlibopen}\Deffunc{lua_mathlibopen}\Deffunc{lua_iolibopen}


\subsection{Predefined Functions} \label{predefined}

\subsubsection*{\ff \T{_ALERT (message)}}\Deffunc{alert}\label{alert}
Prints its only string argument to \IndexVerb{stderr}.
All error messages in Lua are printed through the function stored
in the \verb|_ALERT| global variable
\see{error}.
Therefore, a program may assign another function to this variable
to change the way such messages are shown
(for instance, for systems without \verb|stderr|).

\subsubsection*{\ff \T{assert (v [, message])}}\Deffunc{assert}
Issues an \emph{``assertion failed!''} error
when its argument \verb|v| is \nil.
This function is equivalent to the following Lua function:
\begin{verbatim}
      function assert (v, m)
        if not v then
          m = m or ""
          error("assertion failed!  " .. m)
        end
      end
\end{verbatim}

\subsubsection*{\ff \T{call (func, arg [, mode [, errhandler]])}}\Deffunc{call}
\label{pdf-call}
Calls function \verb|func| with
the arguments given by the table \verb|arg|.
The call is equivalent to
\begin{verbatim}
      func(arg[1], arg[2], ..., arg[n])
\end{verbatim}
where \verb|n| is the result of \verb|getn(arg)| \see{getn}.

By default,
all results from \verb|func| are simply returned by \verb|call|.
If the string \verb|mode| contains \verb|"p"|,
then the results are \emph{packed} in a single table.\index{packed results}
That is, \verb|call| returns just one table;
at index \verb|n|, the table has the total number of results
from the call;
the first result is at index 1, etc.
For instance, the following calls produce the following results:
\begin{verbatim}
   a = call(sin, {5})                --> a = 0.0871557 = sin(5)
   a = call(max, {1,4,5; n=2})       --> a = 4 (only 1 and 4 are arguments)
   a = call(max, {1,4,5; n=2}, "p")  --> a = {4; n=1}
   t = {x=1}
   a = call(next, {t,nil;n=2}, "p")  --> a={"x", 1; n=2}
\end{verbatim}

By default,
if an error occurs during the call to \verb|func|,
the error is propagated.
If the string \verb|mode| contains \verb|"x"|,
then the call is \emph{protected}.\index{protected calls}
In this mode, function \verb|call| does not propagate an error,
regardless of what happens during the call.
Instead, it returns \nil\ to signal the error
(besides calling the appropriated error handler).

If \verb|errhandler| is provided,
the error function \verb|_ERRORMESSAGE| is temporarily set \verb|errhandler|,
while \verb|func| runs.
In particular, if \verb|errhandler| is \nil,
no error messages will be issued during the execution of the called function.

\subsubsection*{\ff \T{collectgarbage ([limit])}}\Deffunc{collectgarbage}
Forces a garbage collection cycle.
Returns the number of objects collected.
The optional argument \verb|limit| is a number that
makes the next cycle occur only after that number of new
objects have been created.
If \verb|limit| is absent or equal to 0,
then Lua uses an adaptive algorithm to set this limit.
\verb|collectgarbage| is equivalent to
the API function \verb|lua_collectgarbage|.

\subsubsection*{\ff \T{copytagmethods (tagto, tagfrom)}}
\Deffunc{copytagmethods}
Copies all tag methods from one tag to another;
it returns \verb|tagto|.

\subsubsection*{\ff \T{dofile (filename)}}\Deffunc{dofile}
Receives a file name,
opens the named file, and executes its contents as a Lua chunk,
or as pre-compiled chunks.
When called without arguments,
\verb|dofile| executes the contents of the standard input (\verb|stdin|).
If there is any error executing the file,
then \verb|dofile| returns \nil.
Otherwise, it returns the values returned by the chunk,
or a non \nil\ value if the chunk returns no values.
It issues an error when called with a non string argument.
\verb|dofile| is equivalent to the API function \verb|lua_dofile|.

\subsubsection*{\ff \T{dostring (string [, chunkname])}}\Deffunc{dostring}
Executes a given string as a Lua chunk.
If there is any error executing the string,
then \verb|dostring| returns \nil.
Otherwise, it returns the values returned by the chunk,
or a non \nil\ value if the chunk returns no values.
The optional parameter \verb|chunkname|
is the ``name of the chunk'',
used in error messages and debug information.
\verb|dostring| is equivalent to the API function \verb|lua_dostring|.

\subsubsection*{\ff \T{error (message)}}\Deffunc{error}\label{pdf-error}
Calls the error handler \see{error} and then terminates
the last protected function called
(in~C: \verb|lua_dofile|, \verb|lua_dostring|,
\verb|lua_dobuffer|, or \verb|lua_callfunction|;
in Lua: \verb|dofile|, \verb|dostring|, or \verb|call| in protected mode).
If \verb|message| is \nil, then the error handler is not called.
Function \verb|error| never returns.
\verb|error| is equivalent to the API function \verb|lua_error|.

\subsubsection*{\ff \T{foreach (table, function)}}\Deffunc{foreach}
Executes the given \verb|function| over all elements of \verb|table|.
For each element, the function is called with the index and
respective value as arguments.
If the function returns any non-\nil\ value,
then the loop is broken, and this value is returned
as the final value of \verb|foreach|.

This function could be defined in Lua:
\begin{verbatim}
      function foreach (t, f)
        local i, v = nil
        while 1 do
          i, v = next(t, i)
          if not i then break end
          local res = f(i, v)
          if res then return res end
        end
      end
\end{verbatim}

You may change the \emph{values} of existing fields in the table during the traversal,
but
if you create new indices,
then
the semantics of \verb|foreach| is undefined.


\subsubsection*{\ff \T{foreachi (table, function)}}\Deffunc{foreachi}
Executes the given \verb|function| over the
numerical indices of \verb|table|.
For each index, the function is called with the index and
respective value as arguments.
Indices are visited in sequential order,
from 1 to \verb|n|,
where \verb|n| is the result of \verb|getn(table)| \see{getn}.
If the function returns any non-\nil\ value,
then the loop is broken, and this value is returned
as the final value of \verb|foreachi|.

This function could be defined in Lua:
\begin{verbatim}
      function foreachi (t, f)
        for i=1,getn(t) do
          local res = f(i, t[i])
          if res then return res end
        end
      end
\end{verbatim}

You may change the \emph{values} of existing fields in the table during the traversal,
but
if you create new indices (even non-numeric),
then
the semantics of \verb|foreachi| is undefined.

\subsubsection*{\ff \T{foreachvar (function)}}\Deffunc{foreachvar}
Executes \verb|function| over all global variables.
For each variable,
the function is called with its name and its value as arguments.
If the function returns any non-nil value,
then the loop is broken, and this value is returned
as the final value of \verb|foreachvar|.

This function could be defined in Lua:
\begin{verbatim}
      function foreachvar (f)
        local n, v = nil
        while 1 do
          n, v = nextvar(n)
          if not n then break end
          local res = f(n, v)
          if res then return res end
        end
      end
\end{verbatim}

You may change the values of existing global variables during the traversal,
but
if you create new global variables,
then
the semantics of \verb|foreachvar| is undefined.


\subsubsection*{\ff \T{getglobal (name)}}\Deffunc{getglobal}
Gets the value of a global variable,
or calls a tag method for ``getgloball''.
Its full semantics is explained in \See{tag-method}.
The string \verb|name| does not need to be a
syntactically valid variable name.

\subsubsection*{\ff \T{getn (table)}}\Deffunc{getn}\label{getn}
Returns the ``size'' of a table, when seen as a list.
If the table has an \verb|n| field with a numeric value,
this value is its ``size''.
Otherwise, the size is the largest numerical index with a non-nil
value in the table.
This function could be defined in Lua:
\begin{verbatim}
      function getn (t)
        if type(t.n) == 'number' then return t.n end
        local max, i = 0, nil
        while 1 do
          i = next(t, i)
          if not i then break end
          if type(i) == 'number' and i>max then max=i end
        end
        return max
      end
\end{verbatim}

\subsubsection*{\ff \T{gettagmethod (tag, event)}}
\Deffunc{gettagmethod}
Returns the current tag method
for a given pair \M{(tag, event)}.

\subsubsection*{\ff \T{newtag ()}}\Deffunc{newtag}\label{pdf-newtag}
Returns a new tag.
\verb|newtag| is equivalent to the API function \verb|lua_newtag|.

\subsubsection*{\ff \T{next (table, [index])}}\Deffunc{next}
Allows a program to traverse all fields of a table.
Its first argument is a table and its second argument
is an index in this table.
It returns the next index of the table and the
value associated with the index.
When called with \nil\ as its second argument,
\verb|next| returns the first index
of the table and its associated value.
When called with the last index,
or with \nil\ in an empty table,
it returns \nil.
If the second argument is absent, then it is interpreted as \nil.

Lua has no declaration of fields;
semantically, there is no difference between a
field not present in a table or a field with value \nil.
Therefore, \verb|next| only considers fields with non \nil\ values.
The order in which the indices are enumerated is not specified,
\emph{even for numeric indices}
(to traverse a table in numeric order,
use a counter or the function \verb|foreachi|).

You may change the \emph{values} of existing fields in the table during the traversal,
but
if you create new indices,
then
the semantics of \verb|next| is undefined.

\subsubsection*{\ff \T{nextvar (name)}}\Deffunc{nextvar}
This function is similar to the function \verb|next|,
but iterates instead over the global variables.
Its single argument is the name of a global variable,
or \nil\ to get a first name.
If this argument is absent, then it is interpreted as \nil.
Like \verb|next|, \verb|nextvar| returns the name of another variable
and its value,
or \nil\ if there are no more variables.

You may change the \emph{values} of existing global variables during the traversal,
but
if you create new global variables,
then
the semantics of \verb|nextvar| is undefined.

\subsubsection*{\ff \T{print (e1, e2, ...)}}\Deffunc{print}
Receives any number of arguments,
and prints their values using the strings returned by \verb|tostring|.
This function is not intended for formatted output,
but only as a quick way to show a value,
for instance for debugging.
See \See{libio} for functions for formatted output.

\subsubsection*{\ff \T{rawgetglobal (name)}}\Deffunc{rawgetglobal}
Gets the value of a global variable,
without invoking any tag method.
The string \verb|name| does not need to be a
syntactically valid variable name.

\subsubsection*{\ff \T{rawgettable (table, index)}}\Deffunc{rawgettable}
Gets the real value of \verb|table[index]|,
without invoking any tag method.
\verb|table| must be a table,
and \verb|index| is any value different from \nil.

\subsubsection*{\ff \T{rawsetglobal (name, value)}}\Deffunc{rawsetglobal}
Sets the named global variable to the given value,
without invoking any tag method.
The string \verb|name| does not need to be a
syntactically valid variable name.
Therefore,
this function can be used to set global variables with strange names like
\verb|"m v 1"| or \verb|"34"|.

\subsubsection*{\ff \T{rawsettable (table, index, value)}}\Deffunc{rawsettable}
Sets the real value of \verb|table[index]| to \verb|value|,
without invoking any tag method.
\verb|table| must be a table,
\verb|index| is any value different from \nil,
and \verb|value| is any Lua value.

\subsubsection*{\ff \T{setglobal (name, value)}}\Deffunc{setglobal}
Sets the named global variable to the given value,
or calls a tag method for ``setgloball''.
Its full semantics is explained in \See{tag-method}.
The string \verb|name| does not need to be a
syntactically valid variable name.

\subsubsection*{\ff \T{settag (t, tag)}}\Deffunc{settag}
Sets the tag of a given table \see{TypesSec}.
\verb|tag| must be a value created with \verb|newtag|
\see{pdf-newtag}.
It returns the value of its first argument (the table).
For the safety of host programs,
it is impossible to change the tag of a userdata from Lua.

\subsubsection*{\ff \T{settagmethod (tag, event, newmethod)}}
\Deffunc{settagmethod}
Sets a new tag method to the given pair \M{(tag, event)}.
It returns the old method.
If \verb|newmethod| is \nil,
then \verb|settagmethod| restores the default behavior for the given event.

\subsubsection*{\ff \T{sort (table [, comp])}}\Deffunc{sort}
Sorts table elements in a given order, \emph{in-place},
from \verb|table[1]| to \verb|table[n]|,
where \verb|n| is the result of \verb|getn(table)| \see{getn}.
If \verb|comp| is given,
it must be a function that receives two table elements,
and returns true when the first is less than the second
(so that \verb|not comp(a[i+1], a[i])| will be true after the sort).
If \verb|comp| is not given,
the standard Lua operator \verb|<| is used instead.

\subsubsection*{\ff \T{tag (v)}}\Deffunc{tag}\label{pdf-tag}
Allows Lua programs to test the tag of a value \see{TypesSec}.
It receives one argument, and returns its tag (a number).
\verb|tag| is equivalent to the API function \verb|lua_tag|.

\subsubsection*{\ff \T{tonumber (e [, base])}}\Deffunc{tonumber}
Receives one argument,
and tries to convert it to a number.
If the argument is already a number or a string convertible
to a number, then \verb|tonumber| returns that number;
otherwise, it returns \nil.

An optional argument specifies the base to interpret the numeral.
The base may be any integer between 2 and 36, inclusive.
In bases above~10, the letter `A' (either upper or lower case)
represents~10, `B' represents~11, and so forth, with `Z' representing 35.

In base 10 (the default), the number may have a decimal part,
as well as an optional exponent part \see{coercion}.
In other bases, only unsigned integers are accepted.

\subsubsection*{\ff \T{tostring (e)}}\Deffunc{tostring}
Receives an argument of any type and
converts it to a string in a reasonable format.
For complete control on how numbers are converted,
use function \verb|format|.



\subsubsection*{\ff \T{tinsert (table [, pos] , value)}}\Deffunc{tinsert}

Inserts element \verb|value| at table position \verb|pos|,
shifting other elements to open space, if necessary.
The default value for \verb|pos| is \verb|n+1|,
where \verb|n| is the result of \verb|getn(table)| \see{getn},
so that a call \verb|tinsert(t,x)| inserts \verb|x| at the end
of table \verb|t|.
This function also sets or increments the field \verb|n| of the table
to \verb|n+1|.

This function is equivalent to the following Lua function,
except that the table accesses are all \emph{raw} (that is, without tag methods):
\begin{verbatim}
      function tinsert (t, ...)
        local pos, value
        local n = getn(t)
        if arg.n == 1 then
          pos, value = n+1, arg[1]
        else
          pos, value = arg[1], arg[2]
        end
        t.n = n+1;
        for i=n,pos,-1 do
          t[i+1] = t[i]
        end
        t[pos] = value
      end
\end{verbatim}

\subsubsection*{\ff \T{tremove (table [, pos])}}\Deffunc{tremove}

Removes from \verb|table| the element at position \verb|pos|,
shifting other elements to close the space, if necessary.
Returns the value of the removed element.
The default value for \verb|pos| is \verb|n|,
where \verb|n| is the result of \verb|getn(table)| \see{getn},
so that a call \verb|tremove(t)| removes the last element
of table \verb|t|.
This function also sets or decrements the field \verb|n| of the table
to \verb|n-1|.

This function is equivalent to the following Lua function,
except that the table accesses are all \emph{raw} (that is, without tag methods):
\begin{verbatim}
      function tremove (t, pos)
        local n = getn(t)
        if n<=0 then return end
        pos = pos or n
        local value = t[pos]
        for i=pos,n-1 do
          t[i] = t[i+1]
        end
        t[n] = nil
        t.n = n-1
        return value
      end
\end{verbatim}

\subsubsection*{\ff \T{type (v)}}\Deffunc{type}\label{pdf-type}
Allows Lua programs to test the type of a value.
It receives one argument, and returns its type, coded as a string.
The possible results of this function are
\verb|"nil"| (a string, not the value \nil),
\verb|"number"|,
\verb|"string"|,
\verb|"table"|,
\verb|"function"|,
and \verb|"userdata"|.
\verb|type| is equivalent to the API function \verb|lua_type|.


\subsection{String Manipulation}
This library provides generic functions for string manipulation,
such as finding and extracting substrings and pattern matching.
When indexing a string, the first character is at position~1
(not at~0, as in C).

\subsubsection*{\ff \T{strbyte (s [, i])}}\Deffunc{strbyte}
Returns the internal numerical code of the character \verb|s[i]|.
If \verb|i| is absent, then it is assumed to be 1.
If \verb|i| is negative,
it is replaced by the length of the string minus its
absolute value plus 1.
Therefore, \Math{-1} points to the last character of \verb|s|.

\NOTE
\emph{numerical codes are not necessarily portable across platforms}.

\subsubsection*{\ff \T{strchar (i1, i2, \ldots)}}\Deffunc{strchar}
Receives 0 or more integers.
Returns a string with length equal to the number of arguments,
wherein each character has the internal numerical code equal
to its correspondent argument.

\NOTE
\emph{numerical codes are not necessarily portable across platforms}.

\subsubsection*{\ff \T{strfind (str, pattern [, init [, plain]])}}
\Deffunc{strfind}
Looks for the first \emph{match} of
\verb|pattern| in \verb|str|.
If it finds one, then it returns the indices on \verb|str|
where this occurrence starts and ends;
otherwise, it returns \nil.
If the pattern specifies captures (see \verb|gsub| below),
the captured strings are returned as extra results.
A third optional numerical argument specifies where to start the search;
its default value is 1.
If \verb|init| is negative,
it is replaced by the length of the string minus its
absolute value plus 1.
Therefore, \Math{-1} points to the last character of \verb|str|.
A value of 1 as a fourth optional argument
turns off the pattern matching facilities,
so the function does a plain ``find substring'' operation,
with no characters in \verb|pattern| being considered ``magic''.

\subsubsection*{\ff \T{strlen (s)}}\Deffunc{strlen}
Receives a string and returns its length.
The empty string \verb|""| has length 0.
Embedded zeros are counted.

\subsubsection*{\ff \T{strlower (s)}}\Deffunc{strlower}
Receives a string and returns a copy of that string with all
upper case letters changed to lower case.
All other characters are left unchanged.
The definition of what is an upper-case
letter depends on the current locale.

\subsubsection*{\ff \T{strrep (s, n)}}\Deffunc{strrep}
Returns a string that is the concatenation of \verb|n| copies of
the string \verb|s|.

\subsubsection*{\ff \T{strsub (s, i [, j])}}\Deffunc{strsub}
Returns another string, which is a substring of \verb|s|,
starting at \verb|i|  and running until \verb|j|.
If \verb|i| or \verb|j| are negative,
they are replaced by the length of the string minus their
absolute value plus 1.
Therefore, \Math{-1} points to the last character of \verb|s|
and \Math{-2} to the previous one.
If \verb|j| is absent, it is assumed to be equal to \Math{-1}
(which is the same as the string length).
In particular,
the call \verb|strsub(s,1,j)| returns a prefix of \verb|s|
with length \verb|j|,
and the call \verb|strsub(s, -i)| returns a suffix of \verb|s|
with length \verb|i|.

\subsubsection*{\ff \T{strupper (s)}}\Deffunc{strupper}
Receives a string and returns a copy of that string with all
lower case letters changed to upper case.
All other characters are left unchanged.
The definition of what is a lower case
letter depends on the current locale.

\subsubsection*{\ff \T{format (formatstring, e1, e2, \ldots)}}\Deffunc{format}
\label{format}
Returns a formatted version of its variable number of arguments
following the description given in its first argument (which must be a string).
The format string follows the same rules as the \verb|printf| family of
standard C~functions.
The only differences are that the options/modifiers
\verb|*|, \verb|l|, \verb|L|, \verb|n|, \verb|p|,
and \verb|h| are not supported,
and there is an extra option, \verb|q|.
The \verb|q| option formats a string in a form suitable to be safely read
back by the Lua interpreter:
The string is written between double quotes,
and all double quotes, returns, and backslashes in the string
are correctly escaped when written.
For instance, the call
\begin{verbatim}
format('%q', 'a string with "quotes" and \n new line')
\end{verbatim}
will produce the string:
\begin{verbatim}
"a string with \"quotes\" and \
 new line"
\end{verbatim}

Conversions can be applied to the \M{n}-th argument in the argument list,
rather than the next unused argument.
In this case, the conversion character \verb|%| is replaced
by the sequence \verb|%d$|, where \verb|d| is a
decimal digit in the range [1,9],
giving the position of the argument in the argument list.
For instance, the call \verb|format("%2$d -> %1$03d", 1, 34)| will
result in \verb|"34 -> 001"|.
The same argument can be used in more than one conversion.

The options \verb|c|, \verb|d|, \verb|E|, \verb|e|, \verb|f|,
\verb|g|, \verb|G|, \verb|i|, \verb|o|, \verb|u|, \verb|X|, and \verb|x| all
expect a number as argument,
whereas \verb|q| and \verb|s| expect a string.
The \verb|*| modifier can be simulated by building
the appropriate format string.
For example, \verb|"%*g"| can be simulated with
\verb|"%"..width.."g"|.

\NOTE
\emph{Neither the format string nor the string values to be formatted with
\T{format} can contain embedded zeros.}

\subsubsection*{\ff \T{gsub (s, pat, repl [, n])}}
\Deffunc{gsub}
Returns a copy of \verb|s|,
in which all occurrences of the pattern \verb|pat| have been
replaced by a replacement string specified by \verb|repl|.
This function also returns, as a second value,
the total number of substitutions made.

If \verb|repl| is a string, then its value is used for replacement.
Any sequence in \verb|repl| of the form \verb|%n|
with \verb|n| between 1 and 9
stands for the value of the \M{n}-th captured substring.

If \verb|repl| is a function, then this function is called every time a
match occurs, with all captured substrings passed as arguments,
in order (see below).
If the value returned by this function is a string,
then it is used as the replacement string;
otherwise, the replacement string is the empty string.

The last, optional parameter \verb|n| limits
the maximum number of substitutions to occur.
For instance, when \verb|n| is 1 only the first occurrence of
\verb|pat| is replaced.

Here are some examples:
\begin{verbatim}
  x = gsub("hello world", "(%w+)", "%1 %1")
  --> x="hello hello world world"

  x = gsub("hello world", "(%w+)", "%1 %1", 1)
  --> x="hello hello world"

  x = gsub("hello world from Lua", "(%w+)%s*(%w+)", "%2 %1")
  --> x="world hello Lua from"

  x = gsub("home = $HOME, user = $USER", "%$(%w+)", getenv)
  --> x="home = /home/roberto, user = roberto"  (for instance)

  x = gsub("4+5 = $return 4+5$", "%$(.-)%$", dostring)
  --> x="4+5 = 9"

  local t = {name="lua", version="3.2"}
  x = gsub("$name - $version", "%$(%w+)", function (v) return %t[v] end)
  --> x="lua - 3.2"

  t = {n=0}
  gsub("first second word", "(%w+)", function (w) tinsert(%t, w) end)
  --> t={"first", "second", "word"; n=3}
\end{verbatim}


\subsubsection*{Patterns} \label{pm}

\paragraph{Character Class:}
a \Def{character class} is used to represent a set of characters.
The following combinations are allowed in describing a character class:
\begin{description}
\item[\emph{x}] (where \emph{x} is any character not in the list
\verb|^$()%.[]*+-?|)
--- represents the character \emph{x} itself.
\item[\T{.}] --- (a dot) represents all characters.
\item[\T{\%a}] --- represents all letters.
\item[\T{\%c}] --- represents all control characters.
\item[\T{\%d}] --- represents all digits.
\item[\T{\%l}] --- represents all lower case letters.
\item[\T{\%p}] --- represents all punctuation characters.
\item[\T{\%s}] --- represents all space characters.
\item[\T{\%u}] --- represents all upper case letters.
\item[\T{\%w}] --- represents all alphanumeric characters.
\item[\T{\%x}] --- represents all hexadecimal digits.
\item[\T{\%z}] --- represents the character with representation 0.
\item[\T{\%\M{x}}] (where \M{x} is any non alphanumeric character)  ---
represents the character \M{x}.
This is the standard way to escape the magic characters \verb|()%.[]*-?|.
It is strongly recommended that any control character (even the non magic)
should be preceded by a \verb|%|
when used to represent itself in a pattern,

\item[\T{[char-set]}] ---
represents the class which is the union of all
characters in char-set.
A range of characters may be specified by
separating the end characters of the range with a \verb|-|.
All classes \verb|%|\emph{x} described above may also be used as
components in a char-set.
All other characters in char-set represent themselves.
For example, \verb|[%w_]| (or \verb|[_%w]|)
represents all alphanumeric characters plus the underscore,
\verb|[0-7]| represents the octal digits,
and \verb|[0-7%l%-]| represents the octal digits plus
the lower case letters plus the \verb|-| character.

The interaction between ranges and classes is not defined.
Therefore, patterns like \verb|[%a-z]| or \verb|[a-%%]|
have no meaning.

\item[\T{[\^{ }char-set]}] ---
represents the complement of char-set,
where char-set is interpreted as above.
\end{description}
For all classes represented by single letters (\verb|%a|, \verb|%c|, \ldots),
the corresponding upper-case letter represents the complement of the class.
For instance, \verb|%S| represents all non-space characters.

The definitions of letter, space, etc. depend on the current locale.
In particular, the class \verb|[a-z]| may not be equivalent to \verb|%l|.
The second form should be preferred for portability.

\paragraph{Pattern Item:}
a \Def{pattern item} may be
\begin{itemize}
\item
a single character class,
which matches any single character in the class;
\item
a single character class followed by \verb|*|,
which matches 0 or more repetitions of characters in the class.
These repetition items will always match the longest possible sequence;
\item
a single character class followed by \verb|+|,
which matches 1 or more repetitions of characters in the class.
These repetition items will always match the longest possible sequence;
\item
a single character class followed by \verb|-|,
which also matches 0 or more repetitions of characters in the class.
Unlike \verb|*|,
these repetition items will always match the shortest possible sequence;
\item
a single character class followed by \verb|?|,
which matches 0 or 1 occurrence of a character in the class;
\item
\T{\%\M{n}}, for \M{n} between 1 and 9;
such item matches a sub-string equal to the \M{n}-th captured string
(see below);
\item
\T{\%b\M{xy}}, where \M{x} and \M{y} are two distinct characters;
such item matches strings that start with~\M{x}, end with~\M{y},
and where the \M{x} and \M{y} are \emph{balanced}.
This means that, if one reads the string from left to right,
counting \Math{+1} for an \M{x} and \Math{-1} for a \M{y},
the ending \M{y} is the first where the count reaches 0.
For instance, the item \verb|%b()| matches expressions with
balanced parentheses.
\end{itemize}

\paragraph{Pattern:}
a \Def{pattern} is a sequence of pattern items.
A \verb|^| at the beginning of a pattern anchors the match at the
beginning of the subject string.
A \verb|$| at the end of a pattern anchors the match at the
end of the subject string.
At other positions,
\verb|^| and \verb|$| have no special meaning and represent themselves.

\paragraph{Captures:}
A pattern may contain sub-patterns enclosed in parentheses,
that describe \Def{captures}.
When a match succeeds, the sub-strings of the subject string
that match captures are stored (\emph{captured}) for future use.
Captures are numbered according to their left parentheses.
For instance, in the pattern \verb|"(a*(.)%w(%s*))"|,
the part of the string matching \verb|"a*(.)%w(%s*)"| is
stored as the first capture (and therefore has number~1);
the character matching \verb|.| is captured with number~2,
and the part matching \verb|%s*| has number~3.

\NOTE
{\em A pattern cannot contain embedded zeros.
Use \verb|%z| instead.}


\subsection{Mathematical Functions} \label{mathlib}

This library is an interface to some functions of the standard C math library.
In addition, it registers a tag method for the binary operator \verb|^| that
returns \Math{x^y} when applied to numbers \verb|x^y|.

The library provides the following functions:
\Deffunc{abs}\Deffunc{acos}\Deffunc{asin}\Deffunc{atan}
\Deffunc{atan2}\Deffunc{ceil}\Deffunc{cos}\Deffunc{floor}
\Deffunc{log}\Deffunc{log10}\Deffunc{max}\Deffunc{min}
\Deffunc{mod}\Deffunc{sin}\Deffunc{sqrt}\Deffunc{tan}
\Deffunc{frexp}\Deffunc{ldexp}
\Deffunc{random}\Deffunc{randomseed}
\begin{verbatim}
   abs  acos  asin  atan  atan2  ceil  cos  deg     floor  log  log10
   max  min   mod   rad   sin    sqrt  tan  frexp   ldexp
   random     randomseed
\end{verbatim}
plus a global variable \IndexVerb{PI}.
Most of them
are only interfaces to the homonymous functions in the C~library,
except that, for the trigonometric functions,
all angles are expressed in \emph{degrees}, not radians.
Functions \IndexVerb{deg} and \IndexVerb{rad} can be used to convert
between radians and degrees.

The function \verb|max| returns the maximum
value of its numeric arguments.
Similarly, \verb|min| computes the minimum.
Both can be used with 1, 2, or more arguments.

The functions \verb|random| and \verb|randomseed| are interfaces to
the simple random generator functions \verb|rand| and \verb|srand|,
provided by ANSI C.
(No guarantees can be given for their statistical properties.)
The function \verb|random|, when called without arguments,
returns a pseudo-random real number in the range \Math{[0,1)}.
When called with a number \Math{n},
\verb|random| returns a pseudo-random integer in the range \Math{[1,n]}.
When called with two arguments, \Math{l} and \Math{u},
\verb|random| returns a pseudo-random integer in the range \Math{[l,u]}.


\subsection{I/O Facilities} \label{libio}

All input and output operations in Lua are done, by default,
over two \Def{file handles}, one for reading and one for writing.
These handles are stored in two Lua global variables,
called \verb|_INPUT| and \verb|_OUTPUT|.
The global variables
\verb|_STDIN|, \verb|_STDOUT|, and \verb|_STDERR|
are initialized with file descriptors for
\verb|stdin|, \verb|stdout| and \verb|stderr|.
Initially, \verb|_INPUT=_STDIN| and \verb|_OUTPUT=_STDOUT|.
\Deffunc{_INPUT}\Deffunc{_OUTPUT}
\Deffunc{_STDIN}\Deffunc{_STDOUT}\Deffunc{_STDERR}

A file handle is a userdata containing the file stream \verb|FILE*|,
and with a distinctive tag created by the I/O library.

Unless otherwise stated,
all I/O functions return \nil\ on failure and
some value different from \nil\ on success.

\subsubsection*{\ff \T{openfile (filename, mode)}}\Deffunc{openfile}

This function opens a file,
in the mode specified in the string \verb|mode|.
It returns a new file handle,
or, in case of errors, \nil\ plus a string describing the error.
This function does not modify either \verb|_INPUT| or \verb|_OUTPUT|.

The \verb|mode| string can be any of the following:
\begin{description}
\item[``r''] read mode;
\item[``w''] write mode;
\item[``a''] append mode;
\item[``r+''] update mode, all previous data is preserved;
\item[``w+''] update mode, all previous data is erased;
\item[``a+''] append update mode, previous data is preserved,
  writing is only allowed at the end of file.
\end{description}
The \verb|mode| string may also have a \verb|b| at the end,
which is needed in some systems to open the file in binary mode.
This string is exactlty what is used in the standard~C function \verb|fopen|.

\subsubsection*{\ff \T{closefile (handle)}}\Deffunc{closefile}

This function closes the given file.
It does not modify either \verb|_INPUT| or \verb|_OUTPUT|.

\subsubsection*{\ff \T{readfrom (filename)}}\Deffunc{readfrom}

This function may be called in two ways.
When called with a file name, it opens the named file,
sets its handle as the value of \verb|_INPUT|,
and returns this value.
It does not close the current input file.
When called without parameters,
it closes the \verb|_INPUT| file,
and restores \verb|stdin| as the value of \verb|_INPUT|.

If this function fails, it returns \nil,
plus a string describing the error.

\begin{quotation}
\noindent
\emph{System dependent}: if \verb|filename| starts with a \verb-|-,
then a \Index{piped input} is opened, via function \IndexVerb{popen}.
Not all systems implement pipes.
Moreover,
the number of files that can be open at the same time is
usually limited and depends on the system.
\end{quotation}

\subsubsection*{\ff \T{writeto (filename)}}\Deffunc{writeto}

This function may be called in two ways.
When called with a file name,
it opens the named file,
sets its handle as the value of \verb|_OUTPUT|,
and returns this value.
It does not close the current output file.
Note that, if the file already exists,
then it will be \emph{completely erased} with this operation.
When called without parameters,
this function closes the \verb|_OUTPUT| file,
and restores \verb|stdout| as the value of \verb|_OUTPUT|.
\index{closing a file}

If this function fails, it returns \nil,
plus a string describing the error.

\begin{quotation}
\noindent
\emph{System dependent}: if \verb|filename| starts with a \verb-|-,
then a \Index{piped output} is opened, via function \IndexVerb{popen}.
Not all systems implement pipes.
Moreover,
the number of files that can be open at the same time is
usually limited and depends on the system.
\end{quotation}

\subsubsection*{\ff \T{appendto (filename)}}\Deffunc{appendto}

Opens a file named \verb|filename| and sets it as the
value of \verb|_OUTPUT|.
Unlike the \verb|writeto| operation,
this function does not erase any previous contents of the file.
If this function fails, it returns \nil,
plus a string describing the error.

\subsubsection*{\ff \T{remove (filename)}}\Deffunc{remove}

Deletes the file with the given name.
If this function fails, it returns \nil,
plus a string describing the error.

\subsubsection*{\ff \T{rename (name1, name2)}}\Deffunc{rename}

Renames file named \verb|name1| to \verb|name2|.
If this function fails, it returns \nil,
plus a string describing the error.

\subsubsection*{\ff \T{flush ([filehandle])}}\Deffunc{flush}

Saves any written data to the given file.
If \verb|filehandle| is not specified,
then \verb|flush| flushes all open files.
If this function fails, it returns \nil,
plus a string describing the error.

\subsubsection*{\ff \T{seek (filehandle [, whence] [, offset])}}\Deffunc{seek}

Sets and gets the file position,
measured in bytes from the beginning of the file,
to the position given by \verb|offset| plus a base
specified by the string \verb|whence|, as follows:
\begin{description}
\item[``set''] base is position 0 (beginning of the file);
\item[``cur''] base is current position;
\item[``end''] base is end of file;
\end{description}
In case of success, function \verb|seek| returns the final file position,
measured in bytes from the beginning of the file.
If the call fails, it returns \nil,
plus a string describing the error.

The default value for \verb|whence| is \verb|"cur"|,
and for \verb|offset| is 0.
Therefore, the call \verb|seek(file)| returns the current
file position, without changing it;
the call \verb|seek(file, "set")| sets the position to the
beginning of the file (and returns 0);
and the call \verb|seek(file, "end")| sets the position to the
end of the file, and returns its size.

\subsubsection*{\ff \T{tmpname ()}}\Deffunc{tmpname}

Returns a string with a file name that can safely
be used for a temporary file.
The file must be explicitly opened before its use
and removed when no longer needed.

\subsubsection*{\ff \T{read ([filehandle,] format1, ...)}}\Deffunc{read}

Reads file \verb|_INPUT|,
or \verb|filehandle| if this argument is given,
according to the given formats, which specify what to read.
For each format,
the function returns a string (or a number) with the characters read,
or \nil\ if it cannot read data with the specified format.
When called without formats,
it uses a default format that reads the next line
(see below).

The available formats are
\begin{description}
\item[``*n''] reads a number;
this is the only format that returns a number instead of a string.
\item[``*l''] reads the next line
(skipping the end of line), or \nil\ on end of file.
This is the default format.
\item[``*a''] reads the whole file, starting at the current position.
On end of file, it returns the empty string.
\item[``*w''] reads the next word
(maximal sequence of non white-space characters),
skipping spaces if necessary, or \nil\ on end of file.
\item[\emph{number}] reads a string with up to that number of characters,
or \nil\ on end of file.
\end{description}

\subsubsection*{\ff \T{write ([filehandle, ] value1, ...)}}\Deffunc{write}

Writes the value of each of its arguments to
file \verb|_OUTPUT|,
or to \verb|filehandle| if this argument is given.
The arguments must be strings or numbers.
To write other values,
use \verb|tostring| or \verb|format| before \verb|write|.
If this function fails, it returns \nil,
plus a string describing the error.

\subsubsection*{\ff \T{date ([format])}}\Deffunc{date}

Returns a string containing date and time
formatted according to the given string \verb|format|,
following the same rules of the ANSI~C function \verb|strftime|.
When called without arguments,
it returns a reasonable date and time representation that depends on
the host system and on the current locale.

\subsubsection*{\ff \T{clock ()}}\Deffunc{clock}

Returns an approximation of the amount of CPU time
used by the program, in seconds.

\subsubsection*{\ff \T{exit ([code])}}\Deffunc{exit}

Calls the C~function \verb|exit|,
with an optional \verb|code|,
to terminate the program.
The default value for \verb|code| is the success code.

\subsubsection*{\ff \T{getenv (varname)}}\Deffunc{getenv}

Returns the value of the process environment variable \verb|varname|,
or \nil\ if the variable is not defined.

\subsubsection*{\ff \T{execute (command)}}\Deffunc{execute}

This function is equivalent to the C~function \verb|system|.
It passes \verb|command| to be executed by an operating system shell.
It returns a status code, which is system-dependent.

\subsubsection*{\ff \T{setlocale (locale [, category])}}\Deffunc{setlocale}

This function is an interface to the ANSI~C function \verb|setlocale|.
\verb|locale| is a string specifying a locale;
\verb|category| is an optional string describing which category to change:
\verb|"all"|, \verb|"collate"|, \verb|"ctype"|,
\verb|"monetary"|, \verb|"numeric"|, or \verb|"time"|;
the default category is \verb|"all"|.
The function returns the name of the new locale,
or \nil\ if the request cannot be honored.


\section{The Debug Interface} \label{debugI}

Lua has no built-in debugging facilities.
Instead, it offers a special interface,
by means of functions and \emph{hooks},
which allows the construction of different
kinds of debuggers, profilers, and other tools
that need ``inside information'' from the interpreter.
This interface is declared in the header file \verb|luadebug.h|,
and has \emph{no} single-state variant.

\subsection{Stack and Function Information}

\Deffunc{lua_getstack}
The main function to get information about the interpreter stack is
\begin{verbatim}
int lua_getstack (lua_State *L, int level, lua_Debug *ar);
\end{verbatim}
It fills parts of a \verb|lua_Debug| structure with
an identification of the \emph{activation record}
of the function executing at a given level.
Level~0 is the current running function,
whereas level \Math{n+1} is the function that has called level \Math{n}.
Usually, \verb|lua_getstack| returns 1;
when called with a level greater than the stack depth,
it returns 0.

\Deffunc{lua_Debug}
The structure \verb|lua_Debug| is used to carry different pieces of information
about an active function:
\begin{verbatim}
struct lua_Debug {
  const char *event;     /* "call", "return" */
  const char *source;    /* (S) */
  int linedefined;       /* (S) */
  const char *what;      /* (S) "Lua" function, "C" function, Lua "main" */
  int currentline;       /* (l) */
  const char *name;      /* (n) */
  const char *namewhat;  /* (n) global, tag method, local, field */
  int nups;              /* (u) number of upvalues */
  lua_Object func;       /* (f) function being executed */
  /* private part */
  ...
};
\end{verbatim}
The \verb|lua_getstack| function fills only the private part
of this structure, for future use.
To fill in the other fields of \verb|lua_Debug| with useful information,
call \Deffunc{lua_getinfo}
\begin{verbatim}
int lua_getinfo (lua_State *L, const char *what, lua_Debug *ar);
\end{verbatim}
This function returns 0 on error
(e.g., an invalid option in \verb|what|).
Each character in the string \verb|what|
selects some fields of \verb|ar| to be filled,
as indicated by the letter in parentheses in the definition of \verb|lua_Debug|;
that is, an \verb|S| fills the fields \verb|source| and \verb|linedefined|,
and \verb|l| fills the field \verb|currentline|, etc.
We describe each field below:
\begin{description}

\item[source]
If the function was defined in a string,
\verb|source| is that string;
if the function was defined in a file,
\verb|source| starts with a \verb|@| followed by the file name.

\item[linedefined]
the line number where starts the definition of the function.

\item[what] the string \verb|"Lua"| if this is a Lua function,
\verb|"C"| if this is a C~function,
or \verb|"main"| if this is the main part of a chunk.

\item[currentline]
the current line where the given function is executing.
It only works if the function has been compiled with debug
information.
When no line information is available,
\verb|currentline| is set to \Math{-1}.

\item[name]
a reasonable name for the given function.
Because functions in Lua are first class values,
they do not have a fixed name:
Some functions may be the value of many global variables,
while others may be stored only in a table field.
The \verb|lua_getinfo| function checks whether the given
function is a tag method or the value of a global variable.
If the given function is a tag method,
then \verb|name| points to the event name.
If the given function is the value of a global variable,
then \verb|name| points to the variable name.
If the given function is neither a tag method nor a global variable,
then \verb|name| is set to \verb|NULL|.

\item[namewhat]
Explains the previous field.
If the function is a global variable,
\verb|namewhat| is \verb|"global"|;
if the function is a tag method,
\verb|namewhat| is \verb|"tag-method"|;
otherwise \verb|namewhat| is \verb|""| (the empty string).

\item[nups]
Number of upvalues of a C~function.
If the function is not a C~function,
\verb|nups| is set to 0.

\item[func]
The function being executed, as a \verb|lua_Object|.

\end{description}

The generation of debug information is controlled by an internal flag,
which can be switched with
\begin{verbatim}
int lua_setdebug (lua_State *L, int debug);
\end{verbatim}
This function sets the flag and returns its previous value.
This flag can also be set from Lua~\see{pragma}.
Setting the flag using \verb|lua_setdebug| affects all chunks that are
compiled afterwards.
Individual functions may still control the generation of debug information
by using \verb|$debug| or \verb|$nodebug|.

\subsection{Manipulating Local Variables}

For the manipulation of local variables,
\verb|luadebug.h| defines the following record:
\begin{verbatim}
struct lua_Localvar {
  int index;
  const char *name;
  lua_Object value;
};
\end{verbatim}
where \verb|index| is an index for local variables
(the first parameter has index 1, and so on,
until the last active local variable).

\Deffunc{lua_getlocal}\Deffunc{lua_setlocal}
The following functions allow the manipulation of the
local variables of a given activation record.
They only work if the function has been compiled with debug
information \see{pragma}.
For these functions, a local variable becomes
visible in the line after its definition.
\begin{verbatim}
int lua_getlocal (lua_State *L, const lua_Debug *ar, lua_Localvar *v);
int lua_setlocal (lua_State *L, const lua_Debug *ar, lua_Localvar *v);
\end{verbatim}
The parameter \verb|ar| must be a valid activation record,
filled by a previous call to \verb|lua_getstack| or
given as argument to a hook \see{sub-hooks}.
To use \verb|lua_getlocal|,
you fill the \verb|index| field of \verb|v| with the index
of a local variable; then the function fills the fields
\verb|name| and \verb|value| with the name and the current
value of that variable.
For \verb|lua_setlocal|,
you fill the \verb|index| and the \verb|value| fields of \verb|v|,
and the function assigns that value to the variable.
Both functions return 0 on failure, that happens
if the index is greater than the number of active local variables,
or if the activation record has no debug information.

As an example, the following function lists the names of all
local variables for a function in a given level of the stack:
\begin{verbatim}
int listvars (lua_State *L, int level) {
  lua_Debug ar;
  int i;
  if (lua_getstack(L, level, &ar) == 0)
    return 0;  /* failure: no such level on the stack */
  for (i=1; ; i++) {
    lua_Localvar v;
    v.index = i;
    if (lua_getlocal(L, &ar, &v) == 0)
      return 1;  /* no more locals, or no debug information */
    printf("%s\n", v.name);
  }
}
\end{verbatim}


\subsection{Hooks}\label{sub-hooks}

The Lua interpreter offers two hooks for debugging purposes:
a \emph{call} hook and a \emph{line} hook.
Both have the same type,
\begin{verbatim}
typedef void (*lua_Hook) (lua_State *L, lua_Debug *ar);
\end{verbatim}
and you can set them with the following functions:
\Deffunc{lua_Hook}\Deffunc{lua_setcallhook}\Deffunc{lua_setlinehook}
\begin{verbatim}
lua_Hook lua_setcallhook (lua_State *L, lua_Hook func);
lua_Hook lua_setlinehook (lua_State *L, lua_Hook func);
\end{verbatim}
A hook is disabled when its value is \verb|NULL|,
which is the initial value of both hooks.
The functions \verb|lua_setcallhook| and \verb|lua_setlinehook|
set their corresponding hooks and return their previous values.

The call hook is called whenever the
interpreter enters or leaves a function.
The \verb|event| field of \verb|ar| has the strings \verb|"call"|
or \verb|"return"|.
This \verb|ar| can then be used in calls to \verb|lua_getinfo|,
\verb|lua_getlocal|, and \verb|lua_setlocal|,
to get more information about the function and to manipulate its
local variables.

The line hook is called every time the interpreter changes
the line of code it is executing.
The \verb|event| field of \verb|ar| has the string \verb|"line"|,
and the \verb|currentline| field has the line number.
Again, you can use this \verb|ar| in other calls to the debug API.
This hook is called only if the active function
has been compiled with debug information~\see{pragma}.

While Lua is running a hook, it disables other calls to hooks.
Therefore, if a hook calls Lua to execute a function or a chunk,
this execution ocurrs without any calls to hooks.

A hook cannot call \T{lua_error}.
It must return to Lua through a regular return.
(There is no problem if the error is inside a chunk or a Lua function
called by the hook, because those errors are protected;
the control returns to the hook anyway.)


\subsection{The Reflexive Debug Interface}

The library \verb|ldblib| provides
the functionality of the debug interface to Lua programs.
If you want to use this library,
your host application must open it,
by calling \verb|lua_dblibopen|.

You should exert great care when using this library.
The functions provided here should be used exclusively for debugging
and similar tasks (e.g., profiling).
Please resist the temptation to use them as a
usual programming tool.
They are slow and violate some (otherwise) secure aspects of the
language (e.g., privacy of local variables).
As a general rule, if your program does not need this library,
do not open it.


\subsubsection*{\ff \T{getstack (level, [what])}}\Deffunc{getstack}

This function returns a table with information about the function
running at level \verb|level| of the stack.
Level 0 is the current function (\verb|getstack| itself);
level 1 is the function that called \verb|getstack|.
If \verb|level| is larger than the number of active functions,
the function returns \nil.
The table contains all the fields returned by \verb|lua_getinfo|,
with the string \verb|what| describing what to get.
The default for \rerb|what| is to get all information available.

For instance, the expression \verb|getstack(1,"n").name| returns
the name of the current function,
if a reasonable name can be found.


\subsubsection*{\ff \T{getlocal (level, local)}}\Deffunc{getlocal}

This function returns the name and the value of the local variable
with index \verb|local| of the function at level \verb|level| of the stack.
(The first parameter has index 1, and so on,
until the last active local variable.)
The function returns \nil\ if there is no local
variable with the given index,
and raises an error when called with a \verb|level| out of range.
(You can call \verb|getstack| to check wheter the level is valid.)

\subsubsection*{\ff \T{setlocal (level, local, value)}}\Deffunc{setlocal}

This function assigns the value \verb|value| to the local variable
with index \verb|local| of the function at level \verb|level| of the stack.
The function returns \nil\ if there is no local
variable with the given index,
and raises an error when called with a \verb|level| out of range.

\subsubsection*{\ff \T{setcallhook (hook)}}\Deffunc{setcallhook}

Sets the function \verb|hook| as the call hook;
this hook will be called every time the interpreter starts and
exits the execution of a function.
The only argument to this hook is the event name (\verb|"call"| or
\verb|"return"|).
You can call \verb|getstack| with level 2 to get more information about
the function being called or returning
(level 0 is the \verb|getstack| function,
and level 1 is the hook function).

When called without arguments,
this function turns off call hooks.

\subsubsection*{\ff \T{setlinehook (hook)}}\Deffunc{setlinehook}

Sets the function \verb|hook| as the line hook;
this hook will be called every time the interpreter changes
the line of code it is executing.
The only argument to the hook is the line number the interpreter
is about to execute.
This hook is called only if the active function
has been compiled with debug information~\see{pragma}.

When called without arguments,
this function turns off line hooks.


\section{\Index{Lua Stand-alone}} \label{lua-sa}

Although Lua has been designed as an extension language,
the language is frequently used as a stand-alone interpreter.
An implementation of such an interpreter,
called simply \verb|lua|,
is provided with the standard distribution.

This program can be called with any sequence of the following arguments:
\begin{description}
\item[\T{-}] executes \verb|stdin| as a file;
\item[\T{-c}] calls \verb|lua_close| after running all arguments;
\item[\T{-d}] turns on debug information;
\item[\T{-e} \rm\emph{stat}] executes string \verb|stat|;
\item[\T{-f filename}] executes file \verb|filename| with the
remaining arguments in table \verb|arg|;
\item[\T{-i}] enters interactive mode with prompt;
\item[\T{-q}] enters interactive mode without prompt;
\item[\T{-v}] prints version information;
\item[\T{var=value}] sets global \verb|var| to string \verb|"value"|;
\item[\T{filename}] executes file \verb|filename|.
\end{description}
When called without arguments,
Lua behaves as \verb|lua -v -i| when \verb|stdin| is a terminal,
and as \verb|lua -| otherwise.

All arguments are handled in order.
For instance, an invocation like
\begin{verbatim}
$ lua -i a=test prog.lua
\end{verbatim}
will first interact with the user until an \verb|EOF| in \verb|stdin|,
then will set \verb|a| to \verb|"test"|,
and finally will run the file \verb|prog.lua|.

When the option \T{-f filename} is used,
all following arguments from the command line
are passed to the Lua program in a table called \verb|arg|.
The field \verb|n| gets the index of the last argument,
and the field 0 gets the \T{filename}.
For instance, in the call
\begin{verbatim}
$ lua a.lua -f b.lua t1 t3
\end{verbatim}
the interpreter first runs the file \T{a.lua},
then creates a table \T{arg},
\begin{verbatim}
  arg = {"t1", "t3";  n = 2, [0] = "b.lua"}
\end{verbatim}
and then runs the file \T{b.lua}.
The stand-alone interpreter also provides a \verb|getarg| function that
can be used to access \emph{all} command line arguments.

In interactive mode,
a multi-line statement can be written finishing intermediate
lines with a backslash (\verb|\|).
If the global variable \verb|_PROMPT| is defined as a string,
its value is used as the prompt. \index{_PROMPT}
Therefore, the prompt can be changed like below:
\begin{verbatim}
$ lua _PROMPT='myprompt> ' -i
\end{verbatim}

In Unix systems, Lua scripts can be made into executable programs
by using \verb|chmod +x| and the~\verb|#!| form,
as in \verb|#!/usr/local/bin/lua|,
or \verb|#!/usr/local/bin/lua -f| to get other arguments.


\section*{Acknowledgments}

The authors would like to thank CENPES/PETROBRAS which,
jointly with \tecgraf, used extensively early versions of
this system and gave valuable comments.
The authors would also like to thank Carlos Henrique Levy,
who found the name of the game.
Lua means \emph{moon} in Portuguese.


\appendix

\section*{Incompatibilities with Previous Versions}

Although great care has been taken to avoid incompatibilities with
the previous public versions of Lua,
some differences had to be introduced.
Here is a list of all these incompatibilities.

\subsection*{Incompatibilities with \Index{version 3.2}}
\begin{itemize}

\item
General read patterns are now deprecated.
\item
Garbage-collection tag methods for tables is now deprecated.
\item
\verb|setglobal|, \verb|rawsetglobal|, and \verb|sort| no longer return a value;
\verb|type| no longer return a second value.
\item
In nested function calls like \verb|f(g(x))|
\emph{all} return values from \verb|g| are passed as arguments to \verb|f|.
(This only happens when \verb|g| is the last
[or the only] argument to \verb|f|.)
\item
There is now only one tag method for order operators.
\item
The debug API has been completely rewritten.
\item
The pre-compiler may use the fact that some operators are associative,
for optimizations.
This may cause problems if these operators
have non-associative tag methods.
\item
All functions from the old API are now macros.
\item
A \verb|const| qualifier has been added to \verb|char *|
in all API functions that handle C~strings.
\item
\verb|luaL_openlib| no longer automatically calls \verb|lua_open|.
So,
you must now explicitly call \verb|lua_open| before opening
the standard libraries.
\item
\verb|lua_type| now returns a string describing the type,
and is no longer a synonym for \verb|lua_tag|.
\item Old pre-compiled code is obsolete, and must be re-compiled.

\end{itemize}

%{===============================================================
\section*{The complete syntax of Lua}

\renewenvironment{Produc}{\vspace{0.8ex}\par\noindent\hspace{3ex}\it\begin{tabular}{rrl}}{\end{tabular}\vspace{0.8ex}\par\noindent}

\renewcommand{\OrNL}{\\ & \Or & }

\begin{Produc}

\produc{chunk}{\rep{stat} \opt{ret}}

\produc{block}{\opt{label} \rep{stat \opt{\ter{;}}}}

\produc{label}{\ter{$\vert$} name \ter{$\vert$}}

\produc{stat}{%
	varlist1 \ter{=} explist1
\OrNL	functioncall
\OrNL	\rwd{do} block \rwd{end}
\OrNL	\rwd{while} exp1 \rwd{do} block \rwd{end}
\OrNL	\rwd{repeat} block \rwd{until} exp1
\OrNL	\rwd{if} exp1 \rwd{then} block
	\rep{\rwd{elseif} exp1 \rwd{then} block}
	\opt{\rwd{else} block} \rwd{end}
\OrNL	\rwd{return} \opt{explist1}
\OrNL	\rwd{break} \opt{name}
\OrNL	\rwd{for} name \ter{=} exp1 \ter{,} exp1 \opt{\ter{,} exp1}
	\rwd{do} block \rwd{end}
\OrNL	\rwd{function} funcname \ter{(} \opt{parlist1} \ter{)} block \rwd{end}
\OrNL	\rwd{local} declist \opt{init}
}

\produc{var}{%
	name
\OrNL	simpleexp \ter{[} exp1 \ter{]}
\OrNL	simpleexp \ter{.} name
}

\produc{varlist1}{var \rep{\ter{,} var}}

\produc{declist}{name \rep{\ter{,} name}}

\produc{init}{\ter{=} explist1}

\produc{exp}{%
	\rwd{nil}
\Or	number
\Or	literal
\Or	function
\Or	simpleexp
\Or	\ter{(} exp \ter{)}
}

\produc{exp1}{exp}

\produc{explist1}{\rep{exp1 \ter{,}} exp}

\produc{simpleexp}{%
	var
\Or	upvalue
\Or	functioncall
\Or	tableconstructor
}

\produc{tableconstructor}{\ter{\{} fieldlist \ter{\}}}
\produc{fieldlist}{%
	lfieldlist
\Or	ffieldlist
\Or	lfieldlist \ter{;} ffieldlist
\Or	ffieldlist \ter{;} lfieldlist
}
\produc{lfieldlist}{\opt{lfieldlist1}}
\produc{ffieldlist}{\opt{ffieldlist1}}
\produc{lfieldlist1}{exp \rep{\ter{,} exp} \opt{\ter{,}}}
\produc{ffieldlist1}{ffield \rep{\ter{,} ffield} \opt{\ter{,}}}
\produc{ffield}{%
	\ter{[} exp \ter{]} \ter{=} exp
\Or	name \ter{=} exp
}

\produc{functioncall}{%
	simpleexp args
\Or	simpleexp \ter{:} name args
}

\produc{args}{%
	\ter{(} \opt{explist1} \ter{)}
\Or	tableconstructor
\Or	\ter{literal}
}

\produc{function}{\rwd{function} \ter{(} \opt{parlist1} \ter{)} block \rwd{end}}

\produc{funcname}{%
	name
\OrNL	name \ter{.} name
\OrNL	name \ter{:} name
}

\produc{parlist1} name}

\end{Produc}
%}===============================================================

% restore underscore to usual meaning
\catcode`\_=8

\newcommand{\indexentry}[2]{\item {#1} #2}
\begin{theindex}
\input{manual.id}
\end{theindex}


\end{document}

\end{theindex}


\end{document}

\end{theindex}


\end{document}

\end{theindex}

\pagebreak
\tableofcontents

\end{document}
